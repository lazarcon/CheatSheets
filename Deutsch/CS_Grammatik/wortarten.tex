\section{Wortarten}
Interpretation des Wortes \enquote{Wort} als:
\begin{description}\itemsep0em
	\item [Lexem] Wörterbucheinheit; so wie es im Wörterbuch steht (Nominativ singular bzw. Infinitiv)
	\item [syntaktische Einheit] Wortform, wie sie in Sätzen vorkommt
\end{description}
Bildung der syntaktischen Wörter erfolgt durch Flexion des Lexems.

\subsection{Flexionsmerkmale}
\settowidth{\MyLenA}{Komparation~~}
\begin{tabular}{@{}p{\the\MyLenA}%
				@{}p{\linewidth-\the\MyLenA}}
Numerus 	& Zahl: Singular, Plural\\
Genus 		& Geschlecht: Maskulinum, Femininum, Neutrum \\
Person 		& 1te, 2te, 3te Person\\
Kasus 		& Fall: Nominativ, Dativ, Akkusativ, Genitiv\\
Tempus 		& Zeit: Präsens, Perfekt, Präteritum, Plusquamperfekt, Futur I, Futur II\\
Modus 		& Aussageweise: Indikativ, Imperativ, Konjunktiv I, Konjunktiv II\\
Diathese 	& Handlungsrichtung/Genus Verbi: Aktiv, Passiv\\
Komperation	& Steigerung: Positiv, Komparativ, Superlativ\\
\end{tabular}

\subsection{Systematik}
\subsubsection{Veränderbare -- Tempus (Verben)}
\settowidth{\MyLenA}{Infinite Form~~}
\begin{tabular}{@{}p{\the\MyLenA}%
				@{}p{\linewidth-\the\MyLenA}}
Finite Form 	& nach Person und Numerus bestimmt\\
Infinite Form	& Person nicht veränderbar (Infinitv, Partizip I/II)\\
\end{tabular}

\subsubsection{Veränderbare -- Kasus (Nomen, Pronomen, Adjektive)}
\settowidth{\MyLenA}{Adjektive~~}
\begin{tabular}{@{}p{\the\MyLenA}%
				@{}p{\linewidth-\the\MyLenA}}
Nomen 	& werden dekliniert: Genus, Numerus, Kasus\\
Pronomen	& Begleiter oder Stellvertreter für Nomen: Kasus\\
Adjektive	& zwischen Begleiter und Nomen: Kasus, Komparation\\
\end{tabular}

\subsubsection{Unveränderbare (Partikel)}
\settowidth{\MyLenA}{Unterordnende Konjunktionen~~}
\begin{tabular}{@{}p{\the\MyLenA}%
				@{}p{\linewidth-\the\MyLenA}}
	Präpositionen & bestimmen den Kasus - \textit{ohne}: Akkusativ, \textit{mit}: Dativ, \textit{wegen}: Genitiv\\
	Beiordnende Konjunktionen & verbinden gleichrangiges: \textit{und}, \textit{sowie}, \textit{oder}, \textit{aber}, \textit{sondern}\\
	Unterordnende Konjunktionen & leiten Nebensätze ein: \textit{dass}, \textit{ob}, \textit{wenn}, \textit{weil}, \textit{solange}\\
	Interjektionen & Ausrufe ausserhalb des Satzes: \textit{ja}, \textit{nein}, \textit{danke}, \textit{pfui}, \textit{miau}\\
	Adverbien & alle übrigen: \textit{oben}, \textit{heute}, \textit{fast}, \textit{deshalb}, \textit{sehr}\\
\end{tabular}

\subsection{Verben (Konjugierbar)}
Flexionsmöglichkeiten:
\begin{itemize}\itemsep0em
	\item Person und Numerus: \textit{ich}, \textit{du}, \dots, \textit{ihr}, \textit{sie}
	\item Tempus: Plusquamperfekt, Präteritum, \dots, Futur II
	\item Modus: \textit{Geh!}, \textit{wir gingen}, \dots
	\item Diathese: \textit{ich sehe}, \textit{ich werde gesehen}
\end{itemize}

\subsubsection{Gebrauchsweisen}
\settowidth{\MyLenA}{intransitive Verben~~}
\begin{tabular}{@{}p{\the\MyLenA}%
				@{}p{\linewidth-\the\MyLenA}}
	Hilfsverben & \textit{sein}, \textit{haben}, \textit{werden} wenn zur Bildung von Tempi oder Passiv gebraucht\\
	Modalverben & \textit{wollen}, \textit{sollen}, \textit{können}, \textit{müssen}, \textit{dürfen}, \textit{mögen} wenn ein Infinitiv von ihnen abhängt\\
	modifizerende Verben & wie Modalverben, nur das der abhängige Infinitiv mit \textit{zu} erweitert ist (z.\,B. \textit{versuchen})\\
	transitive Verben & verlangen ein Akkusativobjekt (z.\,B. \textit{treffen})\\
	intransitive Verben & benötigen kein Akkusativobjekt (z.\,B. \textit{gehen})\\
	reflexive Verben & benötigen ein Reflexivpronomen (z.\,B. \textit{sich erinnern})\\
\end{tabular}

\subsubsection{Infinite Verbformen}
\settowidth{\MyLenA}{Partizip II~~}
\begin{tabular}{@{}p{\the\MyLenA}%
				@{}p{\linewidth-\the\MyLenA}}
	Infinitiv & Grundform, endet auf \textit{-en} oder nur auf \textit{-n}\\
	Partizip I & adjektivischer Gebrauch (ggf. mit Partikel \textit{zu}), endet auf \textit{-end} oder nur \textit{-nd}\\
	Partizip II & adjektivischer Gebrauch oder mit Hilfsverb für zusammengesetzte Verbformen, endet auf \textit{-t} oder \textit{-en}, meist mit Präfix \textit{ge-}. Mit Modalverben hat das Partizip II die gleiche Form wie der Infinitiv.\\
\end{tabular}
\textbf{Nota bene} Partizipien II von intransitiven und reflexiven Verben, die das Perfekt mit \textit{haben} bilden, können nicht adjektivisch gebraucht werden.\\

\subsubsection{Verbzusätze}
\settowidth{\MyLenA}{Unfest zusammengesetzt~~}
\begin{tabular}{@{}p{\the\MyLenA}%
				@{}p{\linewidth-\the\MyLenA}}
	einfache Verben & haben keinen Zusatz (z.\,B. \textit{fragen})\\
	Verben mit Präfix & haben eine Zusatz (z.\,B. \textit{\underline{be}fragen}\\
	Fest zusammengesetzt & untrennbarer Zusatz (z.\,B. \textit{\underline{voll}bringen})\\
	Unfest zusammengesetzt & trennbarer Zusatz (z.\,B. \textit{\underline{aus}fragen})\\
\end{tabular}

\subsubsection{Tempus}
Die einzelnen Tempus lassen sich unterschiedlich verwenden:
\settowidth{\MyLenA}{Plusquamperfekt~~}
\begin{tabular}{@{}p{\the\MyLenA}%
				@{}p{\linewidth-\the\MyLenA}}
	Futur I & offen: zukünfig, gegenwärtig, zeitlos, vergangen\\
	Präsens & offen: zukünfig, gegenwärtig, zeitlos, vergangen\\
	Präteritum & offen: vergangen\\
	Futur II & abgeschlossen: zukünfig, gegenwärtig, zeitlos, vergangen\\
	Perfekt & offen: vergangen, abgeschlossen: zukünfig, gegenwärtig, zeitlos, vergangen\\
	Plusquamperfekt & abgeschlossen, vergangen\\
\end{tabular}

\subsubsection{Modus}
\settowidth{\MyLenA}{Konjunktiv II~~}
\begin{tabular}{@{}p{\the\MyLenA}%
				@{}p{\linewidth-\the\MyLenA}}
	Indikativ & neutral, \enquote{normale} Aussageweise (Wirklichkeit) (z.\,B. Heute \textit{scheint} die Sonne.)\\
	Imperativ & Aufforderung etwas zu tun (z.\,B. \textit{Geh} nach draussen.)\\
	Konjunktiv I & Indirekte Rede, Anweisungen, Vergleiche, Einräumungen (z.\,B. \dots und \textit{sei} es noch so kalt)\\
	Konjunktiv II & Unwirkliche Aussagen, bei unregelmässig Verben Bildung mit \textit{würde} + Infinitiv (z.\,B. \dots auch wenn es \textit{regnen würde}) 
	\textit{Würde} auch verwenden, wenn Indikativ Präteritum und Konjunktiv II Präteritum die gleiche Form haben.\\
\end{tabular}

\subsubsection{Diathese}
Passiv: Hilfsverb \textit{werden} und Partizip II. Passiv ist nur möglich für Verben, deren Subjekt eine Tat darstellt.
Also beispielsweise nicht \textit{schlafen} $\rightarrow$ Ich \textit{werde geschlafen} geht nicht
\begin{itemize}\itemsep0em
	\item transitiven Verben: Subjekt und Objekt werden vertauscht
	\item intransitive Verben: sind im passiv subjektlos; Das ursprüngliche Subjekt wird zur Präpositionalgruppe.
\end{itemize}

Es gibt auch Passivvarianten (Das Geschäft \textit{liefert} frische Ware $\rightarrow$ Die Ware \textit{ist geliefert})
Diese drücken Möglichkeiten, Notwendigkeiten, Beginn oder Abschluss eines Vorgangs aus. Man spricht auch von Zustandspassiv.

\subsubsection{Konjugatinsarten}
\settowidth{\MyLenA}{regelmässige (schwache)~~}
\begin{tabular}{@{}p{\the\MyLenA}%
				@{}p{\linewidth-\the\MyLenA}}
	regelmässige (schwache) & Präteritum und Partizip II mit \textit{-t}-Endung\\
	unregelmässige & Präteritum und Partizip II unterscheiden sich im Stammvokal\\
	starke Verben & innere Abwandlung genügt für Präteritum, Partizip II endet auf \textit{-en}\\
	gemischte Verben & zusätzlich Endung \textit{-te} für Präteritum, Partizip II endet auf \textit{-t}
\end{tabular}


\subsection{Pronomen}
\settowidth{\MyLenA}{bestimmte Zahlprononem~~}
\begin{tabular}{@{}p{\the\MyLenA}%
				@{}p{\linewidth-\the\MyLenA}}
	Personalpronomen & \textit{ich}, \textit{mich}, \textit{mir}, \dots\\
	Reflexivprononem & \textit{mich}, \textit{mir}, \dots, \textit{einander}\\
	Possesivprononem & \textit{mein}, \textit{dein}, \dots\\
	Demonstrativprononem & \textit{dieser}, \textit{jener}, \textit{derselbe}, \dots\\
	Interrogativprononem & \textit{wer}, \textit{was}, \textit{welcher}, \textit{was für ein}\\
	Relativprononem & \textit{wer}, \textit{was}, \textit{welcher}\\
	bestimmte Zahlprononem & \textit{ein}, \textit{zwei}, \dots\\
	Indefinitprononem & \textit{ein}, \textit{eine}, \textit{man}, \textit{etliche}, \textit{nichts} \dots\\
	bestimmter Artikel & \textit{der}, \textit{die}, \textit{das}\\
	unbestimmter Artikel & \textit{ein}, \textit{eine}\\
\end{tabular}


