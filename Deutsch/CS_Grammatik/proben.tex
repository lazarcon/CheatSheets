\section{Proben}

\begin{description}\itemsep0em
	\item [Ersatzprobe] Einen Ausdruck durch einen anderen ersetzen, zum
	\begin{itemize}\itemsep0em
		\item Kasus bestimmen: Ersetzen durch \textit{wer}, \textit{wen}, \textit{wem} oder \textit{wessen}
		\item Wortart bestimmen: Erst ersetzen, dann Pronomenart aus den Pronomenliste ableiten
	\end{itemize}

	\item[Ablese- oder Listenprobe] Einen Ausdruck aufgrund eines Listenvergleichs bestimmen.
	\begin{itemize}\itemsep0em
		\item Kasus bestimmen:
		\begin{tabular}{lllll}
		Nominativ & wer & der & ein & dieser\\
		Akkusativ & wen & den & einen & diesen\\
		Dativ & wem & dem & einem & diesem\\
		Genitiv & wessen & des & eines & dieses\\
		\end{tabular}
		
		\item Wortart bestimmen: Aus Pronomenliste ableiten
	\end{itemize}

	\item [Einsetzprobe] Zur Unterscheidung zwischen Adjektiven und Adverbien: lässt sich das Wort zwischen
	Begleiter und Nomen stellen, ist es ein Adjektiv, sonst ein Adverb.
	Das Buch ist \textit{gratis} $\rightarrow$ Das \textit{gratise} Buch $\rightarrow$ geht nicht $\rightarrow$ Adverb

	\item [Flexionsprobe] Einen Ausdruck flexieren:
	\begin{itemize}\itemsep0em
		\item Verb oder Adjektiv: Konjugieren und Komparieren
		\item Kasus von Nominalgruppen bestimmen: Subjekt und Verb stehen im Numerus überein
	\end{itemize}

	\item [Erweiterungsprobe] Bestimmen, ob ein Infintiv nominalisiert ist (einfügen von \textit{das} vor dem Infinitiv.
	Er hasst \textit{warten/Warten} (?) $\rightarrow$ Er hasst das Warten $\rightarrow$ richtig\\
	Er muss \textit{warten/Warten} (?) $\rightarrow$ Er muss das Warten $\rightarrow$ falsch\\

	\item [Weglassprobe] Weglassen von Satzteilen um den Satzkern zu bestimmen.

	\item [Verschiebeprobe] Bestimmen von Satzgliedern durch verschieben einzelner Satzteile ohne 
	den Inhalt ausser der Gewichtung zu ändern.

	\item [Umformungsprobe] Umfassender Umbau des Satzes ohne den Sinn zu verändern.
	\begin{itemize}\itemsep0em
		\item Subjekt bestimmen $\rightarrow$ Infintivprobe: Die Lärche ist ein Nadelbaum $\rightarrow$ Nadelbaum sein / die Lärche $/rightarrow$ Lärche ist Subjekt
		\item Inhaltliche Bestimmung von Nebensätzen: Es würde mich freuen, wenn \textit{du mitkämest} $\rightarrow$ \textit{Dein Mitkommen} würde mich freuen
		$\rightarrow$ keine Bedingung
		\item Prädikativer oder adverbialer Gebrauch: Bezug auf das Verb: adverbial, sonst prädikativ 
	\end{itemize}

\end{description}

