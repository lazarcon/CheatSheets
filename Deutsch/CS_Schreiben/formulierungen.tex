\section{Formulierungen}
$\Rightarrow$ Kurz und prägnant formulieren $\Rightarrow$ Inhalt muss schnell und leicht verständlich sein: So ausführlich wie nötig, so kompakt und prägnant wie möglich.

\subsection{Does}
\begin{itemize}\itemsep0em
	\item[$+$] syntaktisch und orthographisch korrekt formulieren
	\item[$+$] sachbezogen bleiben; nur zur Fragestellung relevantes aufführen
	\item[$+$] unmissverständlich formulieren
	\item[$+$] kurz und prägnant formulieren
	\item[$+$] Fachbegriffe richtig verwenden
	\item[$+$] exakte Angaben machen
	\item[$+$] objektiv formulieren; Erkenntnisse müssen nachprüfbar sein
	\item[$+$] Quellen vollständig und korrekt angeben
	\item[$+$] als Verfasser unsichtbar bleiben
	\item[$+$] im Präsenz formulieren
	\item[$+$] Unauffällige Metaphern verwenden 
	\item[$+$] Mathematische Aussagen als Ziffern darstellen
	\item[$+$] Symbole (\%, \S) mit Leerschlag absetzen
	\item[$+$] Abkürzungen mit halbem Leerschlag trennen
	\item[$+$] Mehrteilige Verben dicht beieinander halten
	\item[$+$] Das Wesentliche zuerst nennen (z.\,B. \underline{Sprache} ist nach Keller (2003, 11) ein Folge von \dots
	\item[$+$] Regieanweisung mit Inhalt verbinden (z.\,B. Zusammenfassend kann festgestellt werden, dass \dots)
\end{itemize}

\subsection{Don'ts}
\begin{itemize}\itemsep0em
	\item[$-$] Worthülsen benutzen (\textit{-bezogen, -gerichtet, -gestützt, -basiert} statt \textit{-orientiert})
	\item[$-$] hohe Fremdwortdichte; alles was den Lesefluss behindert
	\item[$-$] Pleonasmen (z.\,B. \textit{Einzelindividuum}) verwenden
	\item[$-$] ungenaue Angaben (z.\,B. \textit{grösser als}) machen
	\item[$-$] Fachausdrücke falsch oder falsche Fachausdrücke verwenden
	\item[$-$] subjektive Aussagen einfliessen lassen, Wertungen vornehmen
	\item[$-$] Antrophomorphismen (Vermenschlichungen) verwenden
	\item[$-$] \enquote{ich} durch 3te Person ersetzen (z.\,B. Der Autor denkt \dots)
	\item[$-$] im Präteritum formulieren
	\item[$-$] einen \enquote{blumigen} Stil pflegen
	\item[$-$] Zahlen unter 13 als Ziffern darstellen
	\item[$-$] mehrere Zahlen direkt aufeinander folgen lassen
	\item[$-$] Sätze, die länger als drei Zeilen sind, formulieren
	\item[$-$] Genitiv-Bandwürmer erzeugen (z.\,B. Der Sohn der Cousine des Beamten der Stadverwaltung \dots)
	\item[$-$] Mehr als eine Partizipialgruppe verwenden
	\item[$-$] Inhaltsleere Sätze und Floskeln verwenden (z.\,B. Regieanweisungen: Ich komme jetzt zum nächsten Punkt \dots)
\end{itemize}
