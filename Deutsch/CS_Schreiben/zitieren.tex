\section{Zitieren}
\begin{itemize}\itemsep0em
	\item Zitat darf ursprüngliche Aussage nicht verfälschen
	\item Nichts belangloses wörtlich zitieren
	\item Zitieren ersetzt eigene Formulierungen nicht, es sei denn die Aussage ist besonders prägnant
\end{itemize}

\subsection{Direkte Zitate}
\begin{itemize}\itemsep0em
	\item Buchstabengetreu übernehmen
	\item stehen in Anführungszeichen
	\item der Zitatbeleg steht am Ende z.\,B. \enquote{Bla bla} (Keller, 2003, 11)
	\item Auslassungen werden mit [\dots] markiert
	\item Eigene grammatikalische Anpassung sind ebenfalls in eckige Klammern zu setzen
	\item Hervorhebungen sind zu übernehmen, oder der Text ist als \enquote{ohne Hervorhebungen} zu kennzeichnen
	\item Rechtschreibfehler auch übernehmen und mit dem Hinweis \textit{[sic]} (= so, wie es da steht) versehen
\end{itemize}
 
\subsection{Indirekte Zitate}
\begin{itemize}\itemsep0em
	\item werden im Indikativ formuliert
	\item stehen nicht in Anführungszeichen
	\item Zitatbeleg steht am Ende mit vgl. (vergleiche) z.\,B. (vgl. Keller 2003, 11)
\end{itemize}
