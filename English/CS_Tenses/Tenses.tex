\section{Tenses (Zeiten)}
\subsection{Simple Present}
% Präsens, einfache Gegenwart:
\begin{itemize}\itemsep0em
	\item einmalige/wiederholte Handlung in Gegenwart
	\item von allgemeiner Gültigkeit
	\item für aufeinander folgende Handlungen
	\item für festgelegte Handlungen in der Zukunft (Fahrplan)
\end{itemize}
\textit{always, every \dots, never, normally, often, seldom, sometimes, usually}

Bsp: \textit{He speaks. He does not speak. Does he speak? If he speaks \dots}

\subsection{Present progressive}
% Verlaufsform des Präsens/der Gegenwart
\begin{itemize}\itemsep0em
	\item im Ablauf befindliche Handlung
	\item auf bestimmten Zeitraum begrenzte Handlung
	\item bereits abgesprochene Handlung in der Zukunft
\end{itemize}
\textit{at the moment, just, just now, now, right now, Listen!, Look!}

Bsp: \textit{He is speaking. He is not speaking. Is he speaking?}

\subsection{Simple past}
% Präteritum, einfache Vergangenheit
\begin{itemize}\itemsep0em
	\item einmalige/wiederholte Handlung in Vergangenheit
	\item aufeinander folgende Handlungen in Vergangenheit
	\item neue eintretende Handlung, die eine im Ablauf befindliche Handlung unterbricht
\end{itemize}
\textit{yesterday, 2 minutes ago, in 1990, the other day, last Friday/\dots day}

Bsp: \textit{He spoke. He did not speak. Did he speak? If he spoke \dots}

\subsection{Past progressive}
% Verlaufsform des Präteritum/der Vergangenheit
\begin{itemize}\itemsep0em
	\item in Vergangenheit im Ablauf befindliche Handlung
	\item gleichzeitig ablaufende Handlungen
	\item im Ablauf befindliche Handlung, die durch eine neue Handlung unterbrochen wird
\end{itemize}
\textit{when, while, as long as}

Bsp: \textit{He  was speaking. He was not speaking. Was he speaking?}

\subsection{Present perfect simple}
% Perfekt, vollendete Gegenwart
\begin{itemize}\itemsep0em
	\item das Ergebnis -- nicht die Handlung -- wird betont
	\item bis in Gegenwart dauernde Handlung
	\item eben abgeschlossene Handlungen
	\item abgeschlossene Handlung, die Einfluss auf die Gegenwart hat
	\item bis zum Zeitpunkt des Sprechens nie, ein- oder mehrmals stattgefundene Handlung
\end{itemize}
\textit{already, ever, just, never, not yet, so far, till now, up to now}

Bsp: \textit{He  has spoken. He has not spoken. Has he spoken?}

\subsection{Present perfect progressive}
% Verlaufsform des Perfekt/der vollendeten Gegenwart
\begin{itemize}\itemsep0em
	\item die Handlung -- nicht das Ergebnis -- wird betont
	\item bis in Gegenwart dauernde Handlung
	\item eben abgeschlossene Handlungen
	\item abgeschlossene Handlung, die Einfluss auf die Gegenwart hat
\end{itemize}
\textit{all day, for x years, since 19\dots, how long?, the whole week}

Bsp: \textit{He  has been speaking. He has not been speaking. Has he been speaking?}

\subsection{Past perfect simple}
% Plusquamperfekt, Vorvergangenheit
\begin{itemize}\itemsep0em
	\item Handlung vor einem Zeitpunkt der Vergangenheit
	\item manchmal mit Past Perfect Progr. austauschbar
	\item betont nur die Tatsache, dass etwas vor einem Zeitpunkt in der Vergangenheit stattfand
\end{itemize}
\textit{already, just, never, not yet, once, until that day}

Bsp: \textit{He had spoken. He had not spoken. Had he spoken? If he had spoken \dots}

\subsection{Past perfect progressive}
% Verlaufs des Plusquamperfekt/der Vorvergangenheit
\begin{itemize}\itemsep0em
	\item Handlung vor einem Zeitpunkt der Vergangenheit
	\item manchmal mit Past Perfect Simple austauschbar
	\item betont die Handlung bzw. Dauer der Handlung
\end{itemize}
\textit{for, since, the whole day, all day}

Bsp: \textit{He had been speaking. He had not been speaking. Had he been speaking?}

\subsection{Future I Simple (will)}
% Futur I, Zukunft
\begin{itemize}\itemsep0em
	\item nicht beeinflussbares Geschehen in der Zukunft
	\item spontaner Entschluss
	\item Vermutungen hinsichtlich der Zukunft
\end{itemize}
\textit{in a year, next \dots, tomorrow}\\
Vermutung: \textit{I think, probably, perhaps}

Bsp: \textit{He  will speak. He will not speak. Will he speak? If he speaks, they will listen.}

\subsection{Future I Simple (going to)}
% Futur I, Zukunft
\begin{itemize}\itemsep0em
	\item bereits bestehende Absicht hinsichtlich der Zukunft
	\item logische Schlussfolgerung hinsichtlich der Zukunft
\end{itemize}
\textit{in one year, next week, tomorrow}

Bsp: \textit{He is going to speak. He is not going to speak. Is he going to speak?}

\subsection{Future I progressive}
% Verlaufsform des Futur/der Zukunft
\begin{itemize}\itemsep0em
	\item zu einem zukünftigen Zeitpunkt im Ablauf befindliche Handlungen
	\item sichere oder selbstverständliche Handlungen
\end{itemize}
\textit{in one year, next week, tomorrow}

Bsp: \textit{He will be speaking. He will not be speaking. Will he be speaking?}

\subsection{Future II simple}
% Futur II, vollendete Zukunft
\begin{itemize}\itemsep0em
	\item Handlung, die zu einem zukünftigen Zeitpunkt abgeschlossen sein wird
\end{itemize}
\textit{by Monday, in a week}

Bsp: \textit{He will have spoken. He will not have spoken. Will haven spoken?}

\subsection{Future II progressive}
% Verlaufsform des Futur II/der vollendeten Zukunft
\begin{itemize}\itemsep0em
	\item Handlung, die zu einem zukünftigen Zeitpunkt abgeschlossen sein wird
	\item betont die Dauer der Handlung
\end{itemize}
\textit{for \dots, the last couple of hours, all day long}

Bsp: \textit{He will have been speaking. He will not have been speaking. Will he have been speaking?}

\subsection{Conditional I simple}
% Konjunktiv II (Gegenwart)/Möglichkeitsform
\begin{itemize}\itemsep0em
	\item Handlung, die möglicherweise eintreten könnte
\end{itemize}

Bsp: \textit{He would speak. He would not speak. Would he speak? If I were you, I would speak.}

\subsection{Conditional I progressive}
% Verlaufsform des Konjunktiv II/ der Möglichkeitsform
\begin{itemize}\itemsep0em
	\item Handlung, die möglicherweise eintreten könnte
	\item betont die Handlung bzw. deren Dauer
\end{itemize}

Bsp: \textit{He would be speaking. He would not be speaking. Would he be speaking?}

\subsection{Conditional II simple}
% Konjunktiv II (Vergangenheit)/Möglichkeitsform
\begin{itemize}\itemsep0em
	\item Handlung, die möglicherweise in der Vergangenheit eingetreten wäre
\end{itemize}

Bsp: \textit{He would have spoken. He would not haven spoken. Would he have spoken? If I had heard that, I would have spoken.}

\subsection{Conditional II progressive}
% Verlaufsform des Konjunktiv II/ der Möglichkeitsform
\begin{itemize}\itemsep0em
	\item Handlung, die möglicherweise in der Vergangenheit eingetreten wäre
	\item betont die Handlung bzw. deren Dauer
\end{itemize}

Bsp: \textit{He would have been speaking. He would not have been speaking. Would he have been speaking?}

\subsection{Simple oder progressive?}
\begin{description}\itemsep0em
	\item [simple] für:
		\begin{itemize}\itemsep0em
			\item einmalige, wiederholte, plötzliche oder spontane Handlungen
			\item die Handlung steht im Vordergrund
		\end{itemize}
	\item [progressive] für:
		\begin{itemize}
			\item zu einer bestimmten Zeit oder parallel ablaufende Handlungen
			\item das Ergebnis der Handlung steht im Vordergrund
		\end{itemize}
\end{description}
