\documentclass[10pt,landscape]{scrartcl}
\usepackage{multicol}
\usepackage{calc}
\usepackage{ifthen}
\usepackage[landscape]{geometry}
% Umlaute ermöglichen
\usepackage[T1]{fontenc}
% Format dieses Dokuments
\usepackage[utf8]{inputenc}
% Deutsche Trennungsregeln
\usepackage[ngerman]{babel}
\usepackage[babel,german=quotes]{csquotes}
\usepackage{siunitx}

%\usepackage{color}
%\definecolor{lightgrey}{gray}{0.75}
%\definecolor{darkgrey}{gray}{0.25}

\usepackage{amssymb,amsmath,amsthm,mathtools}
\newtheorem*{satz}{Satz}
\usepackage{wasysym}

\usepackage{tikz}

\usepackage{ulem}

\usepackage{hyperref}
%\pdfoutput=1
\hypersetup{
	pdfauthor   = {Lazari, Constantin},
	pdftitle    = {Cheat Sheet Elektrotechnik},
	pdfsubject  = {Physik},
	pdfkeywords = {},
	pdfcreator  = {Kile},		% Texnic Center oder Kile z.B.
	pdfproducer = {pdflatex},
	colorlinks  = false		% Links nicht farbig hervorheben (sieht Scheisse aus).
} 

% To make this come out properly in landscape mode, do one of the following
% 1.
%  pdflatex latexsheet.tex
%
% 2.
%  latex latexsheet.tex
%  dvips -P pdf  -t landscape latexsheet.dvi
%  ps2pdf latexsheet.ps

% This sets page margins to .5 inch if using letter paper, and to 1cm
% if using A4 paper. (This probably isn't strictly necessary.)
% If using another size paper, use default 1cm margins.
\ifthenelse{\lengthtest { \paperwidth = 11in}}
	{ \geometry{top=.5in,left=.5in,right=.5in,bottom=.5in} }
	{\ifthenelse{ \lengthtest{ \paperwidth = 297mm}}
		{\geometry{top=1cm,left=1cm,right=1cm,bottom=1cm} }
		{\geometry{top=1cm,left=1cm,right=1cm,bottom=1cm} }
	}

% Turn off header and footer
\pagestyle{empty}
 
% Redefine section commands to use less space
\makeatletter
\renewcommand{\section}{\@startsection{section}{1}{0mm}%
                                {-1ex plus -.5ex minus -.2ex}%
                                {0.5ex plus .2ex}%x
                                {\normalfont\large\bfseries}}
\renewcommand{\subsection}{\@startsection{subsection}{2}{0mm}%
                                {-1explus -.5ex minus -.2ex}%
                                {0.5ex plus .2ex}%
                                {\normalfont\normalsize\bfseries}}
\renewcommand{\subsubsection}{\@startsection{subsubsection}{3}{0mm}%
                                {-1ex plus -.5ex minus -.2ex}%
                                {1ex plus .2ex}%
                                {\normalfont\small\bfseries}}
\newcommand{\msout}[1]{\text{\sout{\ensuremath{#1}}}}

\newcommand{\N}{\mathbb{N}} % Natürliche Zahlen
\newcommand{\Z}{\mathbb{Z}}	 % Ganze Zahlen

\makeatother

% Define BibTeX command
\def\BibTeX{{\rm B\kern-.05em{\sc i\kern-.025em b}\kern-.08em
    T\kern-.1667em\lower.7ex\hbox{E}\kern-.125emX}}

% Don't print section numbers
\setcounter{secnumdepth}{0}
\setlength{\parindent}{0pt}

\setlength{\parskip}{0pt plus 0.5ex}


% -----------------------------------------------------------------------

\begin{document}

	\raggedright
	\footnotesize
	\begin{multicols}{3}


	% multicol parameters
	% These lengths are set only within the two main columns
	%\setlength{\columnseprule}{0.25pt}
	\setlength{\premulticols}{1pt}
	\setlength{\postmulticols}{1pt}
	\setlength{\multicolsep}{1pt}
	\setlength{\columnsep}{2pt}
	\newlength{\MyLenA}
	\newlength{\MyLenB}

	\begin{center}
	\large{\textbf{Cheat Sheet Elektrotechnik}} \\
	\end{center}

	\section{Definitionen}

\subsection{Atomare Einheiten}
Masse, Zeit, Länge, Leuchtstärke, Stoffmenge, Stromstärke, 
\si{\kilogram\metre\per\second}

\subsection{Grössenordnungen}


	\section{Elektrische Ladung}
\begin{description}
 \item [Elektrische Ladung] ist eine intrinsische Eigenschaft elementare Bauelemente
 \item [Positive Ladung ($q+$)] bezeichnet einen Überschuss an positiv geladenen Teilen (Elektronenunterschuss)
 \item [Negative Ladung ($q-$)] bezeichnet eine Überschuss an negativ geladenen Teilen (Elektronenüberschuss)
 \item [Leiter] bezeichnet Stoffe, in denen sich Elektronen (relativ) frei bewegen können
 \item [Nichtleiter] sind Stoffe, in denen sich Elektronen nicht bewegen können
\end{description}
Ladungen mit gleichem Vorzeichen stossen einander ab, Ladungen entgegengesetzten Vorzeichens ziehen einander an

\subsection{Coulomb}
Ein Coulomb (C) ist die Ladungsmenge, die durch einen Draht, in dem Strom der Stärke 1 \si{\ampere} in 1 \si{\second} fliesst: 
\begin{align*}
 \mathrm{d}q& = i \mathrm{d}t & [q]& = \si{\coulomb} = \si{\ampere}\si{\second} 
\end{align*}

Die Kraft zwischen zwei Ladungen $q_1$  und $q_2$ mit dem Abstand $d$ ist definiert als:
\begin{align*}
 F& = \frac{1}{4 \pi \varepsilon_0} \cdot \frac{|q_1| \cdot |q_2|}{r^2} & [F]& = \si{\newton} = \frac{\si{\kilogram}\si{\metre}}{\si{\second^2}}
\end{align*}
Die Richtung der Kraft ist entlang der Verbindungslinie von $q_1$ und $q_2$. 

Die Kraft zwischen $n$ Ladungen ist die Summe der Kräfte zwischen allen Ladungspaaren.

\subsection{Kugelschalentheorem}
\begin{itemize}
 \item Eine homogen geladene Kugelschale verhält sich so, als sei die gesamte Ladung im Zentrum vereinigt
 \item Innerhalb einer homogen geladenen Kugelschale ist die elektrostatische Kraft null
\end{itemize}



	\section{Elektrische Felder}
Elektrische Felder sind Vektorfelder. Es wird definiert durch die Kraft $\overrightarrow{F}$, die auf eine positive Punktladung $q_0$ wirkt:
\begin{align*}
 \overrightarrow{E}& = \frac{\overrightarrow{F}}{q_0} & [E]& = \frac{\si{\newton}}{\si{\coulomb}}
\end{align*}
Elektrische Feldlinien (aka Kraftvektoren im Feld) zeigen in Richtung der negativen Ladung!

Das elektrische Feld einer Ladung $q$ ist:
\begin{equation*}
 E = \frac{F}{q} = \frac{1}{4 \pi \varepsilon_0} \cdot \frac{|q|}{d^2}
\end{equation*}
Das elektrische Feld von $n$ Ladungen ist die Summe der elektrischen Felder jeder Einzelladung.

\subsection{Dipol}
Für einen Dipol mit dem Ladungsabstand $d$ zweier Ladungen vom Betrag $q$ ist das elektrische Feld $\overrightarrow{E}$ an einem Punkt mit dem Abstand $z$ vom Mittelpunkt des Dipols gilt:
\begin{equation*}
 \overrightarrow{E} = \frac{1}{2 \pi \varepsilon_0} \cdot \frac{dq}{z^3}
\end{equation*}
Dabei ist $dq$ das Dipolmoment $\overrightarrow{p}$ des Dipols.
	
	\section{Elektrisches Potenzial}

Die Änderung $\Delta U$ der elektrischen potentiellen Energie einer Punktladung, die sich von $a$ nach $b$ bewegt ist:
\begin{align*}
 \Delta U &= U_b - U_a = -W & [U]& = [W] = \si{\joule}
\end{align*}

Die Potentialdifferenz ($U$) ist entsprechend
\begin{align*}
 \Delta V& = V_b - V_a = - \frac{W}{q}& [V]&= \frac{\si{\joule}}{\si{\coulomb}} = \si{\volt}
\end{align*}

In Bezug auf eine Ladung $q: U = V \cdot q$

\subsection{Äquipotenzialfläche} 
Ein elektrisches Feld ist in jedem Punkt senkrecht zu einer Äquipotenzialfläche



	
	\rule{0.3\linewidth}{0.25pt}\\
	\scriptsize
	Copyright \copyright\ 2013 Constantin Lazari\\
	% Should change this to be date of file, not current date.
	Revision: 1.0, Datum: \today\\
	\end{multicols}
\end{document}