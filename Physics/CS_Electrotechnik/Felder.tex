\section{Elektrische Felder}
Elektrische Felder sind Vektorfelder. Es wird definiert durch die Kraft $\overrightarrow{F}$, die auf eine positive Punktladung $q_0$ wirkt:
\begin{align*}
 \overrightarrow{E}& = \frac{\overrightarrow{F}}{q_0} & [E]& = \frac{\si{\newton}}{\si{\coulomb}}
\end{align*}
Elektrische Feldlinien (aka Kraftvektoren im Feld) zeigen in Richtung der negativen Ladung!

Das elektrische Feld einer Ladung $q$ ist:
\begin{equation*}
 E = \frac{F}{q} = \frac{1}{4 \pi \varepsilon_0} \cdot \frac{|q|}{d^2}
\end{equation*}
Das elektrische Feld von $n$ Ladungen ist die Summe der elektrischen Felder jeder Einzelladung.

\subsection{Dipol}
Für einen Dipol mit dem Ladungsabstand $d$ zweier Ladungen vom Betrag $q$ ist das elektrische Feld $\overrightarrow{E}$ an einem Punkt mit dem Abstand $z$ vom Mittelpunkt des Dipols gilt:
\begin{equation*}
 \overrightarrow{E} = \frac{1}{2 \pi \varepsilon_0} \cdot \frac{dq}{z^3}
\end{equation*}
Dabei ist $dq$ das Dipolmoment $\overrightarrow{p}$ des Dipols.