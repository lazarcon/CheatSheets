\section{Elektrische Ladung}
\begin{description}
 \item [Elektrische Ladung] ist eine intrinsische Eigenschaft elementare Bauelemente
 \item [Positive Ladung ($q+$)] bezeichnet einen Überschuss an positiv geladenen Teilen (Elektronenunterschuss)
 \item [Negative Ladung ($q-$)] bezeichnet eine Überschuss an negativ geladenen Teilen (Elektronenüberschuss)
 \item [Leiter] bezeichnet Stoffe, in denen sich Elektronen (relativ) frei bewegen können
 \item [Nichtleiter] sind Stoffe, in denen sich Elektronen nicht bewegen können
\end{description}
Ladungen mit gleichem Vorzeichen stossen einander ab, Ladungen entgegengesetzten Vorzeichens ziehen einander an

\subsection{Coulomb}
Ein Coulomb (C) ist die Ladungsmenge, die durch einen Draht, in dem Strom der Stärke 1 \si{\ampere} in 1 \si{\second} fliesst: 
\begin{align*}
 \mathrm{d}q& = i \mathrm{d}t & [q]& = \si{\coulomb} = \si{\ampere}\si{\second} 
\end{align*}

Die Kraft zwischen zwei Ladungen $q_1$  und $q_2$ mit dem Abstand $d$ ist definiert als:
\begin{align*}
 F& = \frac{1}{4 \pi \varepsilon_0} \cdot \frac{|q_1| \cdot |q_2|}{r^2} & [F]& = \si{\newton} = \frac{\si{\kilogram}\si{\metre}}{\si{\second^2}}
\end{align*}
Die Richtung der Kraft ist entlang der Verbindungslinie von $q_1$ und $q_2$. 

Die Kraft zwischen $n$ Ladungen ist die Summe der Kräfte zwischen allen Ladungspaaren.

\subsection{Kugelschalentheorem}
\begin{itemize}
 \item Eine homogen geladene Kugelschale verhält sich so, als sei die gesamte Ladung im Zentrum vereinigt
 \item Innerhalb einer homogen geladenen Kugelschale ist die elektrostatische Kraft null
\end{itemize}

