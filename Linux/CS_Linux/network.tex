\section{Network Management}
\settowidth{\MyLen}{\texttt{whois}~\textit{domain}~~}
\begin{tabular}{@{}p{\the\MyLen}%
				@{}p{\linewidth-\the\MyLen}}
	\verb!ping !\textit{host}					& Ping host \textit{host}\\
	\verb!whois !\textit{domain}				& Get who is information about \textit{domain}\\
	\verb!dig !\textit{domain}					& Get DNS for \textit{domain}\\
	\verb!dig -x !\textit{host}					& Reverse lookup \textit{host}\\
	\verb!ifconfig!								& Show network address(es)\\
	\verb!netstat!								& Show usage of ports and protocols\\
	\verb!nmap!									& Scan ports (see \verb!man nmap!)\\
	\verb!/etc/hosts/!							& Stores local dns lookup table\\
\end{tabular}

\subsection{Firewall}
\settowidth{\MyLen}{\texttt{ufw disable}~~}
\begin{tabular}{@{}p{\the\MyLen}%
				@{}p{\linewidth-\the\MyLen}}
	\verb!ufw enable!				& Enable Linux firewall (now and forever)\\
	\verb!ufw disable!				& Disable Linux firewall (now and forever)\\
	\verb!ufw status!				& Show status of Linux firewall\\
\end{tabular}

\subsubsection{iptables}
\verb!iptables! contains the rules for the unix firewall.\\
It is advisable to create a script setting up the rules, which should be loaded at startup. Create two executable scripts:
\verb!/etc/network/if-pre-up.d/iptablesload! and \verb!/etc/network/if-post-down.d/iptablessave! (see scripts)\\

Flags:\\
\settowidth{\MyLen}{\texttt{-m conntrack --cstate}~~}
\begin{tabular}{@{}p{\the\MyLen}%
				@{}p{\linewidth-\the\MyLen}}
	\verb!-A!						& Append rule \\
	\verb!-i!						& Apply rule to interace: \verb![eth0|lo]!\\
	\verb!-p!						& Connection protocol: \verb![tcp|udp|udplite|icmp|esp|ah|sctp|all]!\\
	\verb!-d!						& Destination address[/mask] e.g. \verb!192.168.1.0/16!\\
	\verb!-s!						& Source address[/mask] e.g. \verb!192.168.1.0/16!\\
	\verb!--dport!					& Destination ports: \textit{start}\verb![:!\textit{end}\verb!]!\\
	\verb!--sport!					& Source ports: \textit{start}\verb![:!\textit{end}\verb!]!\\
	\verb!-m conntrack --cstate!	& Base rule on connection state: \verb![NEW|RELATED|ESTABLISHED|INVALID]!\\
	\verb!-j!						& What to do: \verb![ACCEPT|REJECT|DROP|LOG]!\\
\end{tabular}

\subsection{wget}
General usage to download a file: \verb!wget !\textit{url}
\settowidth{\MyLen}{\texttt{-P }\textit{directory}~~}
\begin{tabular}{@{}p{\the\MyLen}%
				@{}p{\linewidth-\the\MyLen}}
	\verb!-i !\textit{file}				& Get urls from file\\
	\verb!-nv!							& Non verbose output\\
	\verb!-q!							& Don't print anything on console\\
	\verb!-t!\textit{n}					& Try maximal \textit{n} times\\
	\verb!-N !							& Download only if newer than stored file\\
	\verb!-O !\textit{file}				& Store downloaded file as \textit{file}\\
	%\verb!--user=!\textit{user}			& Username for ftp-downloads\\
	%\verb!--password=!\textit{password}	& Password for ftp-downloads\\
	\verb!-P !\textit{directory}		& Store files to directory \textit{directory}\\
	\verb!-nd!							& Don't make directories as in source\\
	\verb!-x!							& Make same structure as in source\\
	\verb!-r!							& Recursive download\\
	\verb!-l!\textit{n}					& Go \textit{n} levels deep\\
	\verb!-A !\textit{list}				& Comma seperated list of file extensions to download\\
	\verb!-R !\textit{list}				& Comma seperated list of file extensions to not download\\
	\verb!-D !\textit{list}				& Comma seperated list of domains to search\\
	\verb!-np!							& Ignore parent directories\\
	%\verb!--useragent="!\textit{user agent}\verb!"! & Act as user agent \textit{user agent}\\ 
	\verb!-p!							& Pretend download of whole page\\
\end{tabular}

e.g.\\
\verb!wget -r -l1 -A mp3,aac !\textit{url} - Download all direct linked mp3 or aac files from \textit{url}\\ 

