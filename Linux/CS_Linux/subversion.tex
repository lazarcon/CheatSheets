\section{Subversion}
\settowidth{\MyLen}{List files in project~~}
\begin{tabular}{@{}p{\the\MyLen}%
				@{}p{\linewidth-\the\MyLen}}
	Install subversion			& \verb!sudo apt-get install subversion! \\
	Create repository			& \verb!svnadmin create /!\textit{path}\verb!/!\textit{to}\verb!/!\textit{repo}\verb!/!\\
	Make a project folder		& \verb!svn mkdir file:///!\textit{path}\verb!/!\textit{to}\verb!/!\textit{repo}\verb!/!\textit{project} \\
	Import project				& Inside the project directory:\\
								& \verb!svn import file:///!\textit{path}\verb!/!\textit{to}\verb!/!\textit{repo}\verb!/!\textit{project} \\
	Checkout					& \verb!svn checkout file:///!\textit{path}\verb!/!\textit{to}\verb!/!\textit{repo}\verb!/!\textit{project}\\
	Commit changes				& Inside the project directory:\\ 
								& \verb!svn commit -m "!\textit{commit message}\verb!"!\\
	Get latest revision			& Inside the project directory: \verb!svn update!\\
	Revert						& Inside the project directory: \\
								&\verb!svn revert <filename>!\\
	Revert to version			& Inside the project directory: \\
								& \verb!svn update -r !\textit{number}~\verb!<filename>!\\
	Add files 					& Inside the project directory:\\
								& \verb!svn add (!\textit{file}\verb!|!\textit{directory}\verb!)!\\
	Remove files				& Inside the project directory:\\
								& \verb!svn delete (!\textit{file}\verb!|!\textit{directory}\verb!)!\\
	List files in project		& \verb!svn list --verbose!\\
								& \verb!file:///!\textit{path}\verb!/!\textit{to}\verb!/!\textit{repo}\verb!/!\textit{project}\\
	Creating releases			& \verb!svn copy !\\
								& \verb!file:///!\textit{path}\verb!/!\textit{to}\verb!/!\textit{repo}\verb!/!\textit{project}\verb!/trunk! \\
								& \verb!file:///!\textit{path}\verb!/!\textit{to}\verb!/!\textit{repo}\verb!/!\textit{project}\verb!/tags/!\textit{version} \\
								& \verb!-m "Tagging the !\textit{version}\verb! release"!\\
	Listing releases			& \verb!svn list !\\
								& \verb!file:///!\textit{path}\verb!/!\textit{to}\verb!/!\textit{repo}\verb!/!\textit{project}\verb!/tags!\\
	Checkout release			& \verb!svn checkout !\\
								& \verb!file:///!\textit{path}\verb!/!\textit{to}\verb!/!\textit{repo}\verb!/!\textit{project}\verb!/tags/!\textit{version}\\
\end{tabular}

\verb!<filename>! is an optional name of a file.

Before you can use subversion checkout a working-copy of your projects first. After importing a project you can remove it and than check it out from the projects parent directory.
Sometimes subversions opens your default text editor. You should enter comments there.

\textbf{Don't forget} to make the repository accessable (\verb!chmod 666!)
