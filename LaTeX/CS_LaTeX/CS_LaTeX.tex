\documentclass[10pt,landscape]{scrartcl}
\usepackage{multicol}
\usepackage{calc}
\usepackage{ifthen}
\usepackage[landscape]{geometry}
% Umlaute ermöglichen
\usepackage[T1]{fontenc}
% Format dieses Dokuments
\usepackage[utf8]{inputenc}
% Deutsche Trennungsregeln
\usepackage[ngerman]{babel}
\usepackage[babel,german=quotes]{csquotes}

\usepackage{listings}
\usepackage{color}
\definecolor{lightgrey}{gray}{0.75}
\definecolor{darkgrey}{gray}{0.25}
\lstloadlanguages{TeX} 
\lstset{% general command to set parameter(s)
basicstyle=\footnotesize, 										% print whole listing small
keywordstyle=\color{black}\bfseries, 	% bold black keywords
identifierstyle=, 										% nothing happens
commentstyle=\color{darkgrey},				% grey comments
stringstyle=\ttfamily, 								% typewriter type for strings
showstringspaces=false								% no special string spaces
%numbers=left, 												% show line numbers
%numberstyle=\tiny,										% make line numbers tiny (or small, or large), 
%stepnumber=1, 												% counter for line numbers
%numbersep=5pt												% offset of linenumbers
%backgroundcolor=\color{lightgrey}			% backgroundcolor
}
\lstset{tabsize=4, breaklines=true}
\lstset{
  literate=	{ö}{{\"o}}1
           	{ä}{{\"a}}1
           	{ü}{{\"u}}1
		   	{Ö}{{\"O}}1
		  	{Ä}{{\"A}}1
		   	{Ü}{{\"U}}1
}

\usepackage{hyperref}
%\pdfoutput=1
\hypersetup{
	pdfauthor   = {Lazari, Constantin},
	pdftitle    = {LaTeX Cheat Sheet},
	pdfsubject  = {LaTeX},
	pdfkeywords = {},
	pdfcreator  = {Kile},		% Texnic Center oder Kile z.B.
	pdfproducer = {pdflatex},
	colorlinks  = false		% Links nicht farbig hervorheben (sieht Scheisse aus).
} 

% To make this come out properly in landscape mode, do one of the following
% 1.
%  pdflatex latexsheet.tex
%
% 2.
%  latex latexsheet.tex
%  dvips -P pdf  -t landscape latexsheet.dvi
%  ps2pdf latexsheet.ps


% If you're reading this, be prepared for confusion.  Making this was
% a learning experience for me, and it shows.  Much of the placement
% was hacked in; if you make it better, let me know...


% 2008-04
% Changed page margin code to use the geometry package. Also added code for
% conditional page margins, depending on paper size. Thanks to Uwe Ziegenhagen
% for the suggestions.

% 2006-08
% Made changes based on suggestions from Gene Cooperman. <gene at ccs.neu.edu>


% To Do:
% \listoffigures \listoftables
% \setcounter{secnumdepth}{0}


% This sets page margins to .5 inch if using letter paper, and to 1cm
% if using A4 paper. (This probably isn't strictly necessary.)
% If using another size paper, use default 1cm margins.
\ifthenelse{\lengthtest { \paperwidth = 11in}}
	{ \geometry{top=.5in,left=.5in,right=.5in,bottom=.5in} }
	{\ifthenelse{ \lengthtest{ \paperwidth = 297mm}}
		{\geometry{top=1cm,left=1cm,right=1cm,bottom=1cm} }
		{\geometry{top=1cm,left=1cm,right=1cm,bottom=1cm} }
	}

% Turn off header and footer
\pagestyle{empty}
 

% Redefine section commands to use less space
\makeatletter
\renewcommand{\section}{\@startsection{section}{1}{0mm}%
                                {-1ex plus -.5ex minus -.2ex}%
                                {0.5ex plus .2ex}%x
                                {\normalfont\large\bfseries}}
\renewcommand{\subsection}{\@startsection{subsection}{2}{0mm}%
                                {-1explus -.5ex minus -.2ex}%
                                {0.5ex plus .2ex}%
                                {\normalfont\normalsize\bfseries}}
\renewcommand{\subsubsection}{\@startsection{subsubsection}{3}{0mm}%
                                {-1ex plus -.5ex minus -.2ex}%
                                {1ex plus .2ex}%
                                {\normalfont\small\bfseries}}
\makeatother

% Define BibTeX command
\def\BibTeX{{\rm B\kern-.05em{\sc i\kern-.025em b}\kern-.08em
    T\kern-.1667em\lower.7ex\hbox{E}\kern-.125emX}}

% Don't print section numbers
\setcounter{secnumdepth}{0}


\setlength{\parindent}{0pt}
\setlength{\parskip}{0pt plus 0.5ex}


% -----------------------------------------------------------------------

\begin{document}

\raggedright
\footnotesize
\begin{multicols}{3}


% multicol parameters
% These lengths are set only within the two main columns
%\setlength{\columnseprule}{0.25pt}
\setlength{\premulticols}{1pt}
\setlength{\postmulticols}{1pt}
\setlength{\multicolsep}{1pt}
\setlength{\columnsep}{2pt}

\begin{center}
  \Large{\textbf{\LaTeXe\ Cheat Sheet}} \\
\end{center}

%\section{Beispiel \LaTeX-Dokument}
%\lstinputlisting[language=TeX]{SampleDocument.tex}

\section{Allgemeiner Aufbau}
\begin{lstlisting}
\documentclass{dokumentklasse}
\begin{document} 
  % Hier steht text ...
\end{document} 
\end{lstlisting}

\section{Wichtige Dokumentklassen}
\begin{tabular}{@{}ll@{}}
\verb!scrbook!   & Zweiseitig, Vor- und Nachwort\\
\verb!scrreprt!  & Einseitig, ohne Vor- und Nachwort\\
\verb!scrartcl!  & Ohne \verb!\chapter! Gliederung \\
%\verb!chletter!  & Brief im Schweizer Format\\
%\verb!beamer!    & Präsentationen
\end{tabular}

\section{Klassenoptionen der \enquote{scr}-Klassen}
Allgmeine Syntax:\\
\lstinline$\documentclass[option, option, ...]{documentklasse}$

\newlength{\MyLen}
\settowidth{\MyLen}{\texttt{a}$x$\texttt{paper}/\texttt{b}$x$\texttt{paper}~}
\begin{tabular}{@{}p{\the\MyLen}%
                @{}p{\linewidth-\the\MyLen}@{}}
\texttt{10pt}/\texttt{11pt}/\texttt{12pt} & Schriftgrösse \\
\texttt{a}$x$\texttt{paper}/\texttt{b}$x$\texttt{paper} & Papiergrösse, $x=0\dots6$ \\
\texttt{BCOR}$x$\texttt{cm}&Bindekorrektur $x$ cm, $x=0,0\dots20$\\
\texttt{DIV=}$x$&Satzspiegel berechnen, $x$=1,2\dots,\textbf{14}, 15\\
\texttt{fleqn} & Formeln linksbündig, statt zentriert \\
\texttt{leqno} & Formeln links nummerieren statt rechts\\
\texttt{twocolumn} & In zwei Spalten drucken \\
\texttt{twoside}   & Vorder- Rückseiten bedrucken \\
\texttt{landscape} & Ausgabe im Querformat. Benötigt zusätzlich \lstinline$\usepackage[landscape]{geometry}$\\
\texttt{draft}     & Rahmen statt Bildern, Satzprobleme hervorgehoben\\
\texttt{titlepage} & Eigenständiges Titelblatt (\texttt{scrartcl})\\
\texttt{notitlepage} & Kein eigenes Titelblatt (\texttt{scrreprt}, \texttt{scrbook})\\
\texttt{openright} & Kapitel immer auf rechter Seite beginnen (\texttt{scrreprt})\\ 
\texttt{openany} & Kapitel überall beginnen (\texttt{scrbook})
\end{tabular}

\section{Titel(seite)}
\settowidth{\MyLen}{\texttt{.publishers.text.} }
\begin{tabular}{@{}p{\the\MyLen}%
                @{}p{\linewidth-\the\MyLen}@{}}
\verb!\author{!\textit{text}\verb!}! & Autor\\
\verb!\title{!\textit{text}\verb!}!  & Titel\\
\verb!\date{!\textit{text}\verb!}!   & Datum\\
\verb!\subtitle{!\textit{text}\verb!}! & Untertitel\\
\verb!\subject{!\textit{text}\verb!}! & Betreff/Thema\\
\verb!\titlehead{!\textit{text}\verb!}! & Titelkopf/Serie\\
\verb!\publishers{!\textit{text}\verb!}! & Herausgeber\\
\end{tabular}

Diese Angaben stehen vor \verb!\begin{document}!\\
Mit \verb!\maketitle! nach \verb!\begin{document}! Titelei erzeugen

Mehrere Autoren oder Herausgeber nach dem Muster:\\
\verb!\author{ich \and{du} \and{er}}!

\section{Dokumentstruktur}
\subsection{Gliederungsebenen}
\begin{multicols}{2}
\verb!\part{!\textit{title}\verb!}!  \\
\verb!\chapter{!\textit{title}\verb!}!  \\
\verb!\section{!\textit{title}\verb!}!  \\
\verb!\subsection{!\textit{title}\verb!}!  \\
\verb!\subsubsection{!\textit{title}\verb!}!  \\
\verb!\paragraph{!\textit{title}\verb!}!  \\
\verb!\subparagraph{!\textit{title}\verb!}!
\end{multicols}
{\raggedright
Strukturkommandos kann ein \texttt{*} folgen, wie z.B.
\verb!\section*{!\textit{title}\verb!}! um die Nummerierung zu unterdrücken. \verb!\setcounter{secnumdepth}{!$x$\verb!}!
unterdrückt alle Strukturnummern für die gilt: $>x$. \verb!chapter! hat die \verb!secnumdepth! 0

Die Option \verb![!\textit{text}\verb!]! erzeugt einen Kurztitel für das Inhaltsverzeichnis. Beispiel:
\lstinline$\section[Kurz]{Langer Titel}$
}

\subsection{Umgebungen}
\settowidth{\MyLen}{\texttt{.begin.quotation.}}
\begin{tabular}{@{}p{\the\MyLen}%
                @{}p{\linewidth-\the\MyLen}@{}}
\verb!\begin{comment}!    &  Kommentarblock -- wird nicht gedruckt \\
\verb!\begin{quote}!      &  Eingerückter Zitat--Block \\
\verb!\begin{quotation}!  &  Wie \texttt{quote} mit eingerückten Absätzen \\
\verb!\begin{verse}!      &  Zitatblock für Verse
\end{tabular}

Umgebungen enden immer mit \verb!\end{!\textit{umgebung}\verb!}!.

\subsubsection{Listen}
\settowidth{\MyLen}{\texttt{.begin.description.}}
\begin{tabular}{@{}p{\the\MyLen}%
                @{}p{\linewidth-\the\MyLen}@{}}
\verb!\begin{enumerate}!        &  Nummerierte Liste \\
\verb!\begin{itemize}!          &  Unnumerierte Liste \\
\verb!\begin{description}!      &  Liste von Beschreibungen \\
\verb!\item! \textit{text}      &  Listenpunkt \\
\verb!\item[!\textit{x}\verb!]! \textit{text}
                                &  Ersetzt die das normale Zeichen mit \textit{x}. Erforderlich bei Beschreibungen \\
\end{tabular}

\subsubsection{Fliessumgebungen}
\settowidth{\MyLen}{\texttt{.begin.equation..place.}}
\begin{tabular}{@{}p{\the\MyLen}%
                @{}p{\linewidth-\the\MyLen}@{}}
\verb!\begin{table}[!\textit{ort}\verb!]!     &  Nummerierte Tabelle \\
\verb!\begin{figure}[!\textit{ort}\verb!]!    &  Nummerierte Abbildung \\
\verb!\begin{equation}[!\textit{ort}\verb!]!  &  Nummerierte Gleichung \\
\verb!\caption{!\textit{text}\verb!}!           &  Unterschrift der Umgebung \\
\end{tabular}

Der \textit{ort} ist eine Liste gültiger Platzierungen.  \texttt{t}=top/oben,
\texttt{h}=hier, \texttt{b}=bottom/unten, \texttt{p}=page/separate Seite, \texttt{!}=Setzen, auch wenn
\enquote{hässlich}

Die Marker \verb!\caption{!\textit{text}\verb!}! und \verb!\label{!\textit{text}\verb!}! müssen vor dem Ende der
Umgebung kommen

\subsection{Referenzen}
\settowidth{\MyLen}{\texttt{.nameref.marker..}}
\begin{tabular}{@{}p{\the\MyLen}%
                @{}p{\linewidth-\the\MyLen}@{}}
\verb!\label{!\textit{marker}\verb!}!   &  Setzt einen Marker für Referenzen. 
                          Meist in der Form: \verb!\label{sec:item}!. \\
\verb!\ref{!\textit{marker}\verb!}!   &  Gibt die Nummer des Markers zurück \\
\verb!\pageref{!\textit{marker}\verb!}! &  Gibt die Seitennummer des Markers \\
\verb!\nameref{!\textit{marker}\verb!}! &  Gibt den Text des Markers zurück\\
\verb!\footnote{!\textit{text}\verb!}!  &  Erzeugt eine Fussnote am Ende der Seite \\
\end{tabular}
\verb!\newcommand{\see}[2]{#1 \ref{#2}: \nameref{#2},!\\\verb!Seite \pageref{#2}}! erzeugt neuen Befehl für
Referenzen. \verb!\see{Abbildung}{fig:Bild}! erzeugt die Ausgabe: Abbildung 1: Bildunterschrift, Seite 1

%---------------------------------------------------------------------------
\subsection{Layouteinstellungen}

\subsection{Seitenstile}
\settowidth{\MyLen}{\texttt{.pagestyle.headings.} }
\begin{tabular}{@{}p{\the\MyLen}%
                @{}p{\linewidth-\the\MyLen}@{}}
\verb!\pagestyle{empty}!     &   Keine Kopf- oder Fusszeile \\
\verb!\pagestyle{headings}!     &   Erzeugt Kopf- und Fusszeilen \\

\end{tabular}

\subsection{Textausrichtung}
\begin{tabular}{@{}ll@{}}
\textit{Umgebung}  &  \textit{Deklaration}  \\
\verb!\begin{center}!      & \verb!\centering!  \\
\verb!\begin{flushleft}!  & \verb!\raggedright! \\
\verb!\begin{flushright}! & \verb!\raggedleft!  \\
\end{tabular}

\subsection{Absatzformatierung}
\begin{tabular}{@{}ll@{}}
\verb!\setlength{\parindent}{!$x$\verb!}!	& Absätze um $x$ einrücken\\
\verb!\setlength{\parskip}{!$x$\verb!}!	& Absatzabstand $x$\\
\end{tabular}

\subsection{Zeilenabstand}
\verb!\linespread{!$x$\verb!}! Ändert den Zeilenabstand auf $x$

\subsection{Andere Abstände und Linien}
\settowidth{\MyLen}{\texttt{.rule.w..h.} }
\begin{tabular}{@{}p{\the\MyLen}%
                @{}p{\linewidth-\the\MyLen}@{}}
\verb!\hspace{!$l$\verb!}! & Horizontaler Abstand der Länge $l$
                                (Bsp: $l=\mathtt{20pt}$). \\
\verb!\vspace{!$l$\verb!}! & Vertikaler Abstand der Länge $l$. \\
\verb!\rule{!$w$\verb!}{!$h$\verb!}! & Linie der Breite $w$ und Höhe $h$. \\
\end{tabular}

\subsection{Zeilen- und Seitenumbruch}
\settowidth{\MyLen}{\texttt{.pagebreak} }
\begin{tabular}{@{}p{\the\MyLen}%
                @{}p{\linewidth-\the\MyLen}@{}}
\verb!\\!          &  Neue Zeile, aber kein neuer Absatz  \\
\verb!\\*!         &  Kein Seitenumbruch nach Zeilenumbruch \\
\verb!\kill!       &  Zeile nicht drucken \\
\verb!\pagebreak!  &  Seitenumbruch \\
\verb!\noindent!   &  Zeile nicht einrücken \\
\verb!~!       		&  Abstand aber kein Zeilenumbruch (\verb!D.E.~Knuth!).
\end{tabular}

\section{Schrift}

\subsection{Schriftstil}
\newcommand{\FontCmd}[3]{\PBS\verb!\#1{!\textit{text}\verb!}!  \> %
                         \verb!{\#2 !\textit{text}\verb!}! \> %
                         \#1{#3}}
\begin{tabular}{@{}l@{}l@{}l@{}}
\textit{Befehl} & \textit{Deklaration} & \textit{Effekt} \\
\verb!\textrm{!\textit{text}\verb!}!                    & %
        \verb!{\rmfamily !\textit{text}\verb!}!               & %
        \textrm{Roman} \\
\verb!\textsf{!\textit{text}\verb!}!                    & %
        \verb!{\sffamily !\textit{text}\verb!}!               & %
        \textsf{Sans serif} \\
\verb!\texttt{!\textit{text}\verb!}!                    & %
        \verb!{\ttfamily !\textit{text}\verb!}!               & %
        \texttt{Typewriter} \\
\verb!\textmd{!\textit{text}\verb!}!                    & %
        \verb!{\mdseries !\textit{text}\verb!}!               & %
        \textmd{Medium Serie} \\
\verb!\textbf{!\textit{text}\verb!}!                    & %
        \verb!{\bfseries !\textit{text}\verb!}!               & %
        \textbf{Fett} \\
\verb!\textup{!\textit{text}\verb!}!                    & %
        \verb!{\upshape !\textit{text}\verb!}!               & %
        \textup{Aufrecht} \\
\verb!\textit{!\textit{text}\verb!}!                    & %
        \verb!{\itshape !\textit{text}\verb!}!               & %
        \textit{Kursiv} \\
\verb!\textsl{!\textit{text}\verb!}!                    & %
        \verb!{\slshape !\textit{text}\verb!}!               & %
        \textsl{Schräg} \\
\verb!\textsc{!\textit{text}\verb!}!                    & %
        \verb!{\scshape !\textit{text}\verb!}!               & %
        \textsc{Kapitälchen} \\
\verb!\emph{!\textit{text}\verb!}!                      & %
        \verb!{\em !\textit{text}\verb!}!               & %
        \emph{Betont} \\
\verb!\textnormal{!\textit{text}\verb!}!                & %
        \verb!{\normalfont !\textit{text}\verb!}!       & %
        \textnormal{Dokumentschrift} \\
\verb!\underline{!\textit{text}\verb!}!                 & %
                                                        & %
        \underline{Unterstrichen}
\end{tabular}

\subsection{Schriftgrösse}
\setlength{\columnsep}{14pt} % Need to move columns apart a little
\begin{multicols}{2}
\begin{tabbing}
\verb!\footnotesize!          \= \kill
\verb!\tiny!                  \>  \tiny{tiny} \\
\verb!\scriptsize!            \>  \scriptsize{scriptsize} \\
\verb!\footnotesize!          \>  \footnotesize{footnotesize} \\
\verb!\small!                 \>  \small{small} \\
\verb!\normalsize!            \>  \normalsize{normalsize} \\
\verb!\large!                 \>  \large{large} \\
\verb!\Large!                 \=  \Large{Large} \\  % Tab hack for new column
\verb!\LARGE!                 \>  \LARGE{LARGE} \\
\verb!\huge!                  \>  \huge{huge} \\
\verb!\Huge!                  \>  \Huge{Huge}
\end{tabbing}
\end{multicols}
\setlength{\columnsep}{1pt} % Set column separation back

Zum gezielten Verwenden diese Befehle in der Form 
\verb!{\small! \ldots\verb!}! einsetzen


\section{Sonderzeichen}

\subsection{Bindestriche}
\begin{tabular}{@{}llll@{}}
\textit{Name} & \textit{Code} & \textit{Beispiel} & \textit{Verwendung} \\
hyphen  & \verb!-!   & Dipl.-Ing.     & In Wörtern \\
en-dash & \verb!--!  & 1--5           & Zwischen Zahlen \\
em-dash & \verb!---! & Ja---oder nein?    & Gedankenstrich
\end{tabular}

\subsection{\LaTeX-Symbole}
\begin{tabular}{@{}l@{\hspace{1em}}l@{\hspace{2em}}l@{\hspace{1em}}l@{\hspace{2em}}l@{\hspace{1em}}l@{\hspace{2em}}l@{
\hspace{1em}}l@{}}
\&              &  \verb!\&! &
\_              &  \verb!\_! &
\ldots          &  \verb!\ldots! &
\textbullet     &  \verb!\textbullet! \\
\$              &  \verb!\$! &
\^{}            &  \verb!\^{}! &
\textbar        &  \verb!\textbar! &
\textbackslash  &  \verb!\textbackslash! \\
\%              &  \verb!\%! &
\~{}            &  \verb!\~{}! &
\#              &  \verb!\#! &
\S              &  \verb!\S! \\
\end{tabular}

\subsection{Trennzeichen}
\begin{tabular}{@{}l@{\ }ll@{\ }ll@{\ }ll@{\ }ll@{\ }ll@{\ }l@{}}
`       & \verb!`!  &
``      & \verb!``! &
\{      & \verb!\{! &
\lbrack & \verb!\lbrack! &
(       & \verb!(! &
\textless  &  \verb!\textless! \\
'       & \verb!'!  &
''      & \verb!''! &
\}      & \verb!\}! &
\rbrack & \verb!\rbrack! &
)       & \verb!)! &
\textgreater  &  \verb!\textgreater! \\
\end{tabular}


\section{Tabellen}

\subsection{\texttt{tabbing} Tabulatoren Umgebung}
\begin{tabular}{@{}l@{\hspace{1.5ex}}l@{\hspace{10ex}}l@{\hspace{1.5ex}}l@{}}
\verb!\=!  &   Tabulator setzen &
\verb!\>!  &   Zum Tabulator springen
\end{tabular}

\verb!\\! zum Abschluss der Zeilen verwenden

\subsection{\texttt{tabular} Tabellen Umgebung}
\verb!\begin{array}[!\textit{pos}\verb!]{!\textit{cols}\verb!}!   \\
\verb!\begin{tabular}[!\textit{pos}\verb!]{!\textit{cols}\verb!}! \\
\verb!\begin{tabular*}{!\textit{breite}\verb!}[!\textit{pos}\verb!]{!\textit{spalten}\verb!}!


\subsubsection{\texttt{tabular} Spaltenspezifikation}
\settowidth{\MyLen}{\texttt{p}\{\textit{width}\} \ }
\begin{tabular}{@{}p{\the\MyLen}@{}p{\linewidth-\the\MyLen}@{}}
\texttt{l}    &   Links ausgerichtete Spalte\\
\texttt{c}    &   Zentrierte Spalte\\
\texttt{r}    &   Rechts ausgerichte Spalte\\
\verb!p{!\textit{width}\verb!}!  &  Spalte mit fixer Breite\\
\verb!|!      &   Fügt eine vertikale Linie zwischen Spalten ein 
\end{tabular}

\subsubsection{\texttt{tabular} Elemente}
\settowidth{\MyLen}{\texttt{.cline.x-y..}}
\begin{tabular}{@{}p{\the\MyLen}@{}p{\linewidth-\the\MyLen}@{}}
\verb!\hline!           &  Horizontale Linie zwischen Zeilen \\
\verb!\cline{!$x$\verb!-!$y$\verb!}!  &
                        Horizontal Linie von Spalte $x$ bis $y$ \\
\verb!\multicolumn{!\textit{n}\verb!}{!\textit{spec}\verb!}{!\textit{text}\verb!}! \\
        &  Vereinigt \textit{n} Spalten, mit der \textit{spec} Spaltenspezifikation und dem Text \textit{text}
\end{tabular}

% \subsubsection{\texttt{tabular} \enquote{Hübsche Linien}}
% \verb!\usepackage{booktabs}!
% \settowidth{\MyLen}{\texttt{bottomrule}~~}
% \begin{tabular}{@{}p{\the\MyLen}@{}p{\linewidth-\the\MyLen}@{}}
% \verb!\toprule!           &  Breite horizontale Linie \\
% \verb!\midrule!  & Dünnere horizontale Linie \\
% \verb!\bottomrule! & Breite horizontale Linie\\
% \end{tabular}

\subsubsection{Beispielcode Tabelle}

\begin{lstlisting}
\begin{table}[ht!]
 \begin{tabular}{rrr}
  \multicolumn{1}{c}{\textbf{A}} &
  \multicolumn{1}{c}{\textbf{B}} &
  \multicolumn{1}{c}{\textbf{C}} \\
  \hline
  0 & 0 & 7 \\
 \end{tabular}
 \caption{Beispiel Tabelle}
 \label{tab:Beispiel}
\end{table}
\end{lstlisting}
% 
% 
% \begin{center}
%  \begin{tabular}{rrr}
%   \multicolumn{1}{c}{\textbf{A}} &
%   \multicolumn{1}{c}{\textbf{B}} &
%   \multicolumn{1}{c}{\textbf{C}} \\
%   \hline
%   0 & 0 & 7 \\
%  \end{tabular}
% 
% Tabelle 1: Beispiel
% \end{center}

% \subsection{\texttt{longtable} Mehrseitige Tabelle}
% \verb!\usepackage{booktabs}! verteilt eine Tabelle auf mehrere Seiten. Die Spaltenüberschriften werden übernommen
% werden. Beispiel:
% \begin{lstlisting}
% \begin{center}
%   \begin{longtable}{|l|l|}
% 	\caption[Beispiel 2]{Lange Tabelle} 
% 	\label{tab:longtable} \\
% 	\hline 
% 	\multicolumn{1}{|c|}{\textbf{Zeit}} & 
% 	\multicolumn{1}{c|}{\textbf{Strecke}} \\
% 	\hline \endfirsthead
% 	\multicolumn{2}{c}
% 	{{\bfseries \tablename\ \thetable{} -- Fortsetzung von vorheriger Seite}} \\
% 	\hline 
% 	\multicolumn{1}{|c|}{\textbf{Zeit}} & 
% 	\multicolumn{1}{c|}{\textbf{Strecke}} \\
% 	\hline \endhead
% 	\hline 
% 	\multicolumn{3}{|r|}{{Fortsetzung nächste Seite}} \\ 
% 	\hline \endfoot
% 	\hline \hline \endlastfoot
% 	% Tabellendaten ab hier
% 	10 s & 100 m \\
% 	% usw.
%   \end{longtable}
% \end{center}
% \end{lstlisting}

\section{Listings}
\verb!\usepackage{listings}! Erlaubt es Programmiercode direkt einzubinden. Syntax wird automatisch hervorgehoben

\subsection{Formatierung}
\verb!\lstset{!\textit{einstellung}\verb!}! Definiert die Formatierung:

\settowidth{\MyLen}{\texttt{showstringspace=\lbrack true\textbar false\rbrack~}}
\begin{tabular}{@{}p{\the\MyLen}@{}p{\linewidth-\the\MyLen}@{}}
\verb!basicstyle=!\textit{schriftcmd} & Allgemeiner Stil \\
\verb!keywortstyle=!\textit{schriftcmd} & Schlüsselwörter Stil \\
\verb!commentstyle=!\textit{schriftcmd} & Kommentar Stil \\
\verb!stringstyle=!\textit{schriftcmd} & String Stil \\
\verb!showstringspace=[true|false]! & Leerzeichen anzeigen \\
\verb!numbers=[left|right]! & Zeilennummern links oder rechts\\
\verb!numberstyle=!\textit{schriftcmd} & Stil Zeilennummern\\
\verb!stepnumber=!$x$ & $x$ Differenz zwischen Zeilennummern\\
\verb!tabsize=!$x$ & $x$ Leerzeichen pro Tabulator\\
\verb!numbers=[true|false]! & Zeilenumbrechen\\
\end{tabular}

\subsubsection{Umlaute in Listings}
%Mit Hilfe von:
\begin{lstlisting}
\lstset{
  literate=	
	{ä}{{\"a}}1 {ö}{{\"o}}1 {ü}{{\"u}}1
	{Ä}{{\"A}}1 {Ö}{{\"O}}1 {Ü}{{\"U}}1
}
\end{lstlisting}

\subsection{Sprachen}
\verb!\lstloadlanguages{!\textit{sprache}\verb!}! lädt die Sprachdateien (Komma getrennt)
Dialekte stehen in eckigen Klammern.

\begin{multicols}{3}
  ABAP\\
  ACSL\\
  Ada\\
  Algol\\
  Ant\\
  Assembler\\
  Awk\\
  bash\\
  Basic\\
  \lbrack ANSI, Handel, Objective, Sharp\rbrack~C\\
  \lbrack ANSI, GNU, ISO, Visual\rbrack~C++\\ 
  Caml\\
  Clean\\
  Cobol\\
  Comal\\
  command.com.\\
  csh\\
  Delphi\\
  Eiffel\\
  Elan\\
  erlang\\
  Euphoria\\
  Fortran\\
  GCL\\
  Gnuplot\\
  Haskell\\
  HTML\\
  IDL\\
  inform\\
  Java\\
  JVMIS\\
  ksh\\
  Lisp\\
  Logo\\
  make\\
  Mathematica1\\
  Matlab\\
  Mercury\\
  MetaPost\\
  Miranda\\
  Mizar\\
  ML\\
  Modula-2\\
  MuPAD\\
  NASTRAN\\
  Oberon-2\\
  OCL\\
  Octave\\
  Oz\\
  Pascal\\
  Perl\\
  PHP\\
  PL/I\\
  Plasm\\
  POV\\
  Prolog\\
  Promela\\
  Python\\
  R\\
  Reduce\\
  Rexx\\
  RSL\\
  Ruby\\
  S\\
  SAS\\
  Scilab\\
  sh\\
  SHELXL\\
  Simula\\
  SQL\\
  tcl\\
  \lbrack AlLaTeX, common, LaTeX, plain, primitive\rbrack~TeX\\ 
  VBScript\\
  Verilog\\
  VHDL\\
  VRML\\
  XML\\
  XSLT
\end{multicols}


\subsection{Einbinden von Listings}
\settowidth{\MyLen}{\texttt{\textbackslash lstinputlisting.relativerPfad...}~}
\begin{tabular}{@{}p{\the\MyLen}@{}p{\linewidth-\the\MyLen}@{}}
\verb!\begin{lstlisting}! & Listing Umgebung \\
\verb!\lstinline{!\textit{code}\verb!}! & Innerhalb einer Textzeile \\
\verb!\lstinputlisting{!\textit{relativerPfad}\verb!}! & Externe Datei \\
\end{tabular}

\subsubsection{Listing Optionen}
\settowidth{\MyLen}{\texttt{language=\textit{sprache}}~}
\begin{tabular}{@{}p{\the\MyLen}@{}p{\linewidth-\the\MyLen}@{}}
\verb!language=!\textit{sprache} & Sprache des Listings\\
\verb!caption={!\textit{text}\verb!}! & Überschrift des Listings \\
\verb!label=!\textit{label} & Referenz Marker \\
\verb!firstline=!$x$ & $x$ Erste darzustellende Zeile \\
\verb!lastline=!$x$ & $x$ Letzte darzustellende Zeile \\
\end{tabular}
Verwendung im Stil: \verb!\lstinputlisting[language=C, caption={Code}]{datei}!


\section{Mathematik}
Mathematische Ausdrücke entweder mit \texttt{\$} oder mit
\verb!\begin{equation}! umschliessen

\begin{tabular}{@{}l@{\hspace{1em}}l@{\hspace{2em}}l@{\hspace{1em}}l@{}}
Superscript$^{x}$       &
\verb!^{x}!             &  
Subscript$_{x}$         &
\verb!_{x}!             \\  
$\frac{x}{y}$           &
\verb!\frac{x}{y}!      &  
$\sum_{k=1}^n$          &
\verb!\sum_{k=1}^n!     \\  
$\sqrt[n]{x}$           &
\verb!\sqrt[n]{x}!      &  
$\prod_{k=1}^n$         &
\verb!\prod_{k=1}^n!    \\ 
\end{tabular}

\subsection{Mathematische Symbole}

% The ordering of these symbols is slightly odd.  This is because I had to put all the
% long pieces of text in the same column (the right) for it all to fit properly.
% Otherwise, it wouldn't be possible to fit four columns of symbols here.

\begin{tabular}{@{}l@{\hspace{1ex}}l@{\hspace{1em}}l@{\hspace{1ex}}l@{\hspace{1em}}l@{\hspace{1ex}}
l@{\hspace{1em}}l@{\hspace{1ex}}l@{}}
$\leq$          &  \verb!\leq!  &
$\geq$          &  \verb!\geq!  &
$\neq$          &  \verb!\neq!  &
$\approx$       &  \verb!\approx!  \\
$\times$        &  \verb!\times!  &
$\div$          &  \verb!\div!  &
$\pm$           & \verb!\pm!  &
$\cdot$         &  \verb!\cdot!  \\
$^{\circ}$      & \verb!^{\circ}! &
$\circ$         &  \verb!\circ!  &
$\prime$        & \verb!\prime!  &
$\cdots$        &  \verb!\cdots!  \\
$\infty$        & \verb!\infty!  &
$\neg$          & \verb!\neg!  &
$\wedge$        & \verb!\wedge!  &
$\vee$          & \verb!\vee!  \\
$\supset$       & \verb!\supset!  &
$\forall$       & \verb!\forall!  &
$\in$           & \verb!\in!  &
$\rightarrow$   &  \verb!\rightarrow! \\
$\subset$       & \verb!\subset!  &
$\exists$       & \verb!\exists!  &
$\notin$        & \verb!\notin!  &
$\Rightarrow$   &  \verb!\Rightarrow! \\
$\cup$          & \verb!\cup!  &
$\cap$          & \verb!\cap!  &
$\mid$          & \verb!\mid!  &
$\Leftrightarrow$   &  \verb!\Leftrightarrow! \\
$\dot a$        & \verb!\dot a!  &
$\hat a$        & \verb!\hat a!  &
$\bar a$        & \verb!\bar a!  &
$\tilde a$      & \verb!\tilde a!  \\

$\alpha$        &  \verb!\alpha!  &
$\beta$         &  \verb!\beta!  &
$\gamma$        &  \verb!\gamma!  &
$\delta$        &  \verb!\delta!  \\
$\epsilon$      &  \verb!\epsilon!  &
$\zeta$         &  \verb!\zeta!  &
$\eta$          &  \verb!\eta!  &
$\varepsilon$   &  \verb!\varepsilon!  \\
$\theta$        &  \verb!\theta!  &
$\iota$         &  \verb!\iota!  &
$\kappa$        &  \verb!\kappa!  &
$\vartheta$     &  \verb!\vartheta!  \\
$\lambda$       &  \verb!\lambda!  &
$\mu$           &  \verb!\mu!  &
$\nu$           &  \verb!\nu!  &
$\xi$           &  \verb!\xi!  \\
$\pi$           &  \verb!\pi!  &
$\rho$          &  \verb!\rho!  &
$\sigma$        &  \verb!\sigma!  &
$\tau$          &  \verb!\tau!  \\
$\upsilon$      &  \verb!\upsilon!  &
$\phi$          &  \verb!\phi!  &
$\chi$          &  \verb!\chi!  &
$\psi$          &  \verb!\psi!  \\
$\omega$        &  \verb!\omega!  &
$\Gamma$        &  \verb!\Gamma!  &
$\Delta$        &  \verb!\Delta!  &
$\Theta$        &  \verb!\Theta!  \\
$\Lambda$       &  \verb!\Lambda!  &
$\Xi$           &  \verb!\Xi!  &
$\Pi$           &  \verb!\Pi!  &
$\Sigma$        &  \verb!\Sigma!  \\
$\Upsilon$      &  \verb!\Upsilon!  &
$\Phi$          &  \verb!\Phi!  &
$\Psi$          &  \verb!\Psi!  &
$\Omega$        &  \verb!\Omega!  
\end{tabular}
\footnotesize

\section{Nützliche Packages}
\settowidth{\MyLen}{\texttt{microtype}\hspace{2pt}}
\begin{tabular}{@{}p{\the\MyLen}%
                @{}p{\linewidth-\the\MyLen}@{}}
\texttt{inputenc} & Eingabecodierung. Option: \texttt{utf8} oder \texttt{ansinew} \\
\texttt{fontenc} & Schriftcodierung. Option: \texttt{T1} für Umlaute \\
\texttt{lmodern} & \enquote{Moderne} Schrift \\
\texttt{babel} & Mit der Option \texttt{ngerman} für deutsche Trennungsregeln \\
\texttt{csquotes} & Anführungszeichen. Optionen: \verb!babel,german=quotes!\\
\texttt{microtype} & Optischer Randausgleich \\
\texttt{graphicx}  &  Bilder verwenden:
                        \verb!\includegraphics[width=!%
                        \textit{x}\verb!]{!%
                        \textit{datei}\verb!}!. Optionen: \texttt{dvips, pdftex}\\
\texttt{subfigure} & Abbildung unterteilen in Unterabbildungen\\
\texttt{wrapfig} & Abbildungen mit umfliessenden Text\\
\texttt{pdfpages} & PDFs einbinden. Option: \texttt{final}. Einbinden über:
  \verb!\includepdf[pages=-]{!\textit{pdfDatei}\verb!}!\\
\texttt{hyperref}  & PDF mit klickbarem Inhaltsverzeichnis. Optionen: 
  \verb!pdfborder={0 0 0},plainpages=false,pdfpagelabels!. 
  Bsps.: \verb!\url{!%
                        \textit{http://\ldots}%
                        \verb!}!, \verb!\href{!\textit{http://\dots}\verb!}{!\textit{text}\verb!}!.\\
\texttt{longtable} & Mehrseitige Tabellen\\
\texttt{booktabs} & Verschönert Tabellen\\
\texttt{tikz-er2} & Zeichnungen in \LaTeX\ erstellen\\
\end{tabular}

Pakete vor \verb!\begin{document}! einbinden.\\
Syntax: \verb!\usepackage[!\textit{optionen}\verb!]{!\textit{package}\verb!}!

\subsection{PDF-Ausgabe}
%\verb!\pdfoutput=1! Erzeugt ein PDF
Mit \verb!\hypersetup{!\textit{einstellung}\verb!}! werden der PDF-Datei Metadaten hinzugefügt. Einstellungen:

\settowidth{\MyLen}{\texttt{colorlinks=\lbrack true\textbar false\rbrack}}
\begin{tabular}{@{}p{\the\MyLen}%
                @{}p{\linewidth-\the\MyLen}@{}}
\texttt{pdfauthor=\{}\textit{text}\texttt{\}} & Author(en) \\
\texttt{pdftitle=\{}\textit{text}\texttt{\}} & Titel \\
\texttt{pdfsubject=\{}\textit{text}\texttt{\}} & Thema \\
\texttt{pdfcreator=\{}\textit{text}\texttt{\}} & Erzeugender Editor \\
\texttt{pdfproducer=\{}\textit{text}\texttt{\}} & Erzeugendes Programm \\
\texttt{colorlinks=\lbrack true\textbar false\rbrack} & Links farblich hervorheben\\
\end{tabular}


%\subsection{Special symbols}
%\begin{tabular}{@{}ll@{}}
%$^{\circ}$  &  \verb!^{\circ}! Ex: $22^{\circ}\mathrm{C}$: \verb!$22^{\circ}\mathrm{C}$!.
%\end{tabular}


\rule{0.3\linewidth}{0.25pt}\\
\scriptsize
Copyright \copyright\ 2011 Constantin Lazari\\
% Should change this to be date of file, not current date.
Revision: 1.0, Datum: \today\\
\end{multicols}
\end{document}
