\section{Prozessor}
Eigenschaften:
\begin{description}\itemsep0em
	\item [Befehlssatz] Die Befehle, die hardwareseitig implementiert sind. Je grösser deren Anzahl, desto kürzere und damit schnellere Programme sind möglich. Ein Compiler macht aus einem Programmbefehl $n$ Prozessorbefehle (Befehlsarchitektur)
	\item [Taktzyklus] beschreibt die Dauer eines Zykluses (normalerweise in Hz also Taktzyklen pro Sekunde angegeben)
	\item [Taktzyklen pro Befehl (CPI)] (\textit{clock cyles per instruction}) gibt an, wieviele Taktzyklen ein Befehl im Durchschnitt benötigt (Einzelne Befehle gehen schneller 1 Taktzyklus, andere langsamer $>$ 1 Taktzyklus)
\end{description}

\subsection{Befehl laden}
Befehlszähler addressiert Speicherzelle, der Inhalt wird ins Befehlsregister übertragen
(für arithmetisch-logische Operationen). Dazu sind Schaltnetze und Schaltwerke erforderlich.
\begin{description}\itemsep0em
	\item [Schaltnetz] besteht ausschliesslich aus logischen Bauelementen (AND, ADD, \dots)
	\begin{itemize}\itemsep0em
		\item besteht aus $2^n$ Dateneingängen, $n$ Steuereingängen und einem Ausgang. 	Mit Hilfe der Steuereingänge wird bestimmt, welches der $n$-Eingangssignale durchgeleitet wird
		\item Kombinationen von Schaltnetzen sind wieder ein Schaltnetz
	\end{itemize}

	\item [Schaltwerk] speichert intern Daten
	\begin{itemize}\itemsep0em
		\item mindestens ein Dateneingang, ein Datenausgang und ein Takteingang
		\item Takteingang bestimmt, wann ein Speicherelement überschrieben wird (gelesen werden kann es immer)
		\item Unterscheidung zwischen pegel- und flankengesteuerten Schaltwerken (heute: Flanken), d.\,h. die Spannung muss einen bestimmten Wert übersteigen, damit Schreibvorgänge stattfinden
	\end{itemize}	
\end{description}

Allgemein: Schaltwerk $\rightarrow$ Schaltnetz $\rightarrow$ Schaltwerk
Bei flankengesteuerten Schaltwerken können die Daten wieder ins selbe Schaltwerk geschrieben werden

\subsection{Operanden laden, speichern}
Mit Hilfe von Schaltnetzen und -werken werden Daten aufgrund der von der ALU berechneten  absolute Speicheradresse manipuliert

\subsection{Ablaufzeit eines Programms $P$}
\begin{equation*}
	t_P = \mbox{(Anzahl Befehle)} \cdot \mbox{CPI} \cdot \mbox{Zeit pro Zyklus}
\end{equation*}

Die minimale Zykluszeit hängt davon ab, wie lange der Strom entlang des längsten möglichen Datenpfades für einen Befehl braucht. Optimierung muss sich immer nach diesem Befehl richten,
auch wenn der Befehl nur selten gebraucht wird. Alle anderen Befehle müssen immer auf den langsamsten aller möglichen Befehle warten.

Load/Store unter Einbezug des Speichers ist wohl immer am langsamsten.

Befehle können gruppiert werden, so dass alle Befehle einer Gruppe etwa gleich lange Pfade haben. Langsamere Befehle beanspruchen dann einfach $n$ Zyklen (so kommt dann der CPI-Wert zustande). Allerdings wird dann die Steuerung und das Design deutlich komplexer.

\subsection{Pipelining}
Mehrere Befehle werden gleichzeitig (überlappend ausgeführt)


