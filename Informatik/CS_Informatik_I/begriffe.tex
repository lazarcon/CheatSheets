\section{Begriffe}
\settowidth{\MyLenA}{Word~~}
\begin{tabular}{@{}p{\the\MyLenA}%
				@{}p{\linewidth-\the\MyLenA}}
	Bit & [0,1], Masseinheit für übertragene Datenmengen und Informationsgehalt\\
	Byte & 8-Tupel von Bit, Masseinheit für Speicherung\\
	Word & Grundverarbeitungsdatengrösse in einem Rechner. Immer eine 2er Potenz\\
\end{tabular}

\subsection{Algorithmus}
Ein Algorithmus ist eine Verarbeitungsvorschrift, die entweder durch einen Menschen oder eine Maschine (mechanisch/elektronisch)
effizient durchgeführt werden kann, und bei der die Abfolge der endlichen Verarbeitungsschritte eindeutig festgelegt und ist und
und bei welcher die einzelnen Verarbeitungsschritte eindeutig spezifiziert sind. Für den gleichen Input liefert ein Algorithmus 
also immer den gleichen Output. 

Bsp. Euklid GGT von $a$ und $b$:\\
Widerhole:
\begin{enumerate}
	\item $r = a mod b$ Bestimme den Rest von $a/b$ 
	\item $a = b$ Weise a den Wert von $b$ zu
	\item $b = r$ Weise b den Wert von $r$ zu
\end{enumerate}
bis ($r=0$)

\subsection{Information}
Gegeben ist ein Alphabet $Z = \{z_1, z_2, \dots z_n\}$, jedes Zeichen $z_i$ habe die Auftrittswahrscheinlichkeit $p_z$.

\settowidth{\MyLenA}{Informationsgehalt~~}
\begin{tabular}{@{}p{\the\MyLenA}%
				@{}p{\linewidth-\the\MyLenA}}
	Informationsgehalt & $I(z) = - \log_2{p_{z}}$ Bit\\
	Entropie			& $H = \sum_{z\in Z} p_z \cdot I(p_z) = - \sum_{z\in Z} p_z \cdot \log_2 p_z$ Bit\\
	Redundanz			& $R = \frac{\mbox{Bits gesamt}}{\mbox{Bits genutzt}} - 1$\\
\end{tabular}

