\section{Zahlensysteme}

\subsection{Darstellung allgemein}
Jede gebrochene Zahl $z$ mit $n$-Stellen vor und $m$ Stellen nach dem Komma ($n, m \in \mathbb{N}\textbackslash\{0\}$) 
mit einer Basis $B$ und den Ziffern $b \in \{0, 1, \dots, B-1\}$ lässt sich als Summe darstellen:
\begin{equation*}
	z = \sum_{i=-m}^{n-1} b_i \cdot B^i
\end{equation*}

\subsection{Konvertierung}
\subsubsection{Zahlensystem zur Basis $B$ ins Dezimalsystem}
$n$ = Vorkommastellen, $m$ = Nachkommastellen
\settowidth{\MyLenA}{Nach~~}
\begin{tabular}{@{}p{\the\MyLenA}%
				@{}p{\linewidth-\the\MyLenA}}
	Vor  & $z = (\dots((b_n \cdot B + b_{n-1}) \cdot B + b_{n-2}) \cdot B + \dots + b_1) \cdot B + b_0$\\
	Nach  & $z = (\dots((b_m \cdot 1/B + b_{-m+1}) \cdot 1/B + b_{-m+2}) \cdot 1/B + \dots + b_{-1}) \cdot 1/B$\\
\end{tabular}

\subsubsection{Dezimalsystem in System zur Basis $B$}

\textbf{Vorkommastellen} Eingabe Vorkommastellen $n$, Basis $B$\\
Wiederhole
\begin{enumerate}
	\item $y = n : B$ Rest $z$ (Dividiere durch $B$, behalte $z$)
	\item $n = y$ (Rechne mit Divisionsergebnis ohne Rest weiter)
\end{enumerate}
bis ($y = 0)$\\
Die Reste $z$ von unten nach oben ergeben die die gesuchte Zahl.

Beispiel: $n = 122$, gesucht Darstellung zur Basis 8 (Oktal)
\begin{align*}
	122 \div 8 = 15&~\mbox{Rest}~2\\
	15 \div 8 = 1&~\mbox{Rest}~7\\
	1 \div 8 = 0&~\mbox{Rest}~1 \rightarrow 122_{10} = 172_8
\end{align*}

\textbf{Nachkommastellen} Eingabe Nachkommastellen $m$, Basis $B$\\
Wiederhole
\begin{enumerate}
	\item $y = m \cdot B$ (Multipliziere $m$ mit $B$)
	\item $m = \mbox{frac}(y)$ (Setze den Nachkommateil von $y$ zum neuen $m$)
\end{enumerate}
bis ($m = 0$)\\
Die ganzzahligen Anteile von oben nach unten ergeben die gesuchte Zahl.

Beispiel: $n = 0{,}3984375$, gesucht Darstellung zur Basis 8 (Oktal)
\begin{align*}
	0{,}3984375 \cdot 8 = 3{,}1875&~\mbox{Ganzzahlanteil:}~3\\
	0{,}1875 \cdot 8 = 1{,}5&~\mbox{Ganzzahlanteil:}~1\\
	0{,}5 \cdot 8 = 4{,}0&~\mbox{Ganzzahlanteil:}~4 \rightarrow 0{,}3984375_{10} = 0{,}314_8
\end{align*}

\subsection{Einerkomplement}
Erstes Bit = Vorzeichen (1 für negativ), Zahl = Invertierung der positiven Zahl.
Überlauf ist, wenn zu einer 1 ganz links, eine 1 addiert wird. Diese wird der Ziffer ganz rechts hinzugefügt.
Nachteile: Redundante Darstellung der 0, separate Betrachtung des Überlaufs $\rightarrow$ 2 Additionen

\subsection{Zweierkomplement}
Beim Zweierkomplement werden negativen Zahlen des Einerkomplements um 1 erhöht. Beim Überlauf wird ein 
Warnmeldungsflag (Overflow) auf wahr gesetzt.

\subsection{Festpunktzahlen}
Haben eine festgelegte Länge von Vorkommazahlen ($n$) und Nachkommazahlen ($m$).
Sie decken jeweils nur einen bestimmten Wertebereich ab, gewisse Zahlen lassen sich exakt darstellen, andere nicht.
Rechnen mit Werten sehr unterschiedlicher Grösse ist schwierig bis unmöglich. Festkommazahlen werden daher nur
für Spezialanwendungen gebraucht.

\subsection{Fliesskommazahlen}
(IEEE 754): $s$ = Vorzeichen (0 positiv, 1 negativ), $M$ = Mantisse, $B$ = Basis und $E$ = Exponent.
\begin{equation*}
x = s  \cdot B^E \cdot M
\end{equation*}
Die Mantisse ist 23 bzw. 52 Bit lang und sie ist normiert (Dargestellt als 1.xxx), wobei die führende 1 weggelassen wird 
um die Präzision um ein Bit zu erhöhen, also zu verdoppeln.

	\begin{tabular}{ll}
							& \textbf{Single Precision}\\ 
	Vorzeichen				& 1 Bit\\
	Exponent				& 8 Bit\\
	Mantisse				& 23 Bit\\
	Gesamtlänge				& 32 Bit\\
	Biaswert				& 127\\
	Wertebereich Exponent 	& $-126 \leq e \leq 127$\\
	Dezimalstellen 			& 7 \dots\ 8\\
	Genauigkeit				& $2^{-(23 + 1)} \approx 5{,}96 \cdot 10^{-8}$\\
	Minimum					& $2^{-(23 + 126)} \approx 1 \cdot 10^{-45}$\\
	Maximum					& $(1 - 2^{-24}) \cdot 2^{128} \approx 3{,}4 \cdot 10^{38}$\\
	\end{tabular}
	\begin{tabular}{ll}
							& \textbf{Double Precision} \\
	Vorzeichen				& 1 Bit	\\
	Exponent				& 11 Bit\\	
	Mantisse				& 52 Bit\\
	Gesamtlänge				& 64 Bit\\
	Biaswert				& 1\,023\\
	Wertebereich Exponent	& $-1\,022 \leq e \leq 1\,023$\\
	Dezimalstellen			& 15 \dots\ 16\\
	Genauigkeit				& $2^{-(52 + 1)} \approx 1{,}1\cdot 10^{16}$\\
	Minimum					& $2^{-(52 + 1\,022)} \approx 5 \cdot 10^{-324}$\\
	Maximum					& $(1 - 2^{-52}) \cdot 2^{1\,024} \approx 1{,}8 \cdot 10^{308}$\\
	\end{tabular}

Beispiel: Darstellung von $18{,}4_{10}$ als IEEE 754 Gleitpunktzahl:
\begin{enumerate}
	\item Vorzeichen: positiv $\rightarrow 0$ 
	\item Mantisse:
		\begin{enumerate}
			\item Umwandlung in duale Festkommazahl ohne Vorzeichen:
			\begin{align*}
				18_{10}& = 1\,0010_2\\
				0.4_{10}& = 0.01100\,\overline{1100}_2\\
				18.4_{10}& = 1\,0010.01100\overline{1100}_2
			\end{align*}
			\item Normalisierung
			\begin{equation*}
				18.4_{10} = 1\,0010.01100\,1100\,\dots_2 \cdot 2^0 = 
1.0010\,01100\,\overline{1100}_2 \cdot 2^4
			\end{equation*}
			Aus $2^4 \rightarrow e = 4$
		\end{enumerate}
	\item Exponent:
		\begin{enumerate}
			\item Exponent $e=4$, Biaswert = 127;
			\item $E = e + Biaswert = 4_{10} + 127_{10} = 131_{10} = 1000\,0011_2$
		\end{enumerate}
\end{enumerate}
Die zusammengesetzte Gleitpunktzahl ist: $0\,10000011\,00101100110011001100110$