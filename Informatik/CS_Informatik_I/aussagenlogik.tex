\section{Logik}
\subsection{Aussagenlogik}
\textbf{Aussage} Eine Aussage ist eine sprachliche Äusserung, 
die wahr ($w$) oder falsch($f$) sein kann.\\
Bsp: \enquote{Es gibt unendlich viele natürliche Zahlen.}

\subsubsection{Junktoren (Verknüpfungsoperatoren)}
\begin{tabular}{cl}
	$\neg$				& Negation, \enquote{nicht \dots}\\
	$\wedge$ 			& Konjunktion, \enquote{\dots\ und \dots}\\
	$\vee$ 				& Disjunktion, \enquote{\dots\ oder \dots}\\
	$\Rightarrow$ 		& Implikation, \enquote{wenn \dots, dann \dots}\\
	$\Leftrightarrow$	& Äquivalenz, \enquote{\dots\ genau dann, wenn \dots} \\
	$\bigoplus$			& Antivalenz, \enquote{entweder \dots oder  \dots}\\
\end{tabular}\\

Für eine einzelne Aussage $A$ gilt:\\
\begin{tabular}{c||c}
	$A$ & $\neg A$ \\\hline
	$f$ & $w$ \\
	$w$ & $f$ \\
\end{tabular}

Für zwei Aussagen $A$ und $B$ gilt:\\
\begin{tabular}{c|c||c|c|c|c|c}
	$A$ & $B$ 	& $A \wedge B$ 	& $A \vee B$ 	& $A \Rightarrow B$ 	& $A \Leftrightarrow B$ 	& $A \bigoplus B$ \\\hline
	$f$ & $f$ 	& $f$				& $f$			& $w$				& $w$					 	& $f$\\
	$f$ & $w$ 	& $f$				& $w$			& $w$				& $f$					 	& $w$\\	
	$w$ & $f$ 	& $f$				& $w$			& $f$				& $f$						& $w$\\
	$w$ & $w$ 	& $w$				& $w$			& $w$				& $w$						& $f$\\
\end{tabular}\\
Bindungsstärke: $\neg$ vor $\wedge$ vor $\vee$ vor $\Rightarrow$ vor $\Leftrightarrow$.\\

\subsubsection{Rechenregeln}
\settowidth{\MyLenA}{Doppelte Negation~~}
\begin{equation*}
\begin{array}{lrl}
	\mbox{Duplizität}			& A \wedge A				& \Leftrightarrow A\\
								& A \vee A					& \Leftrightarrow A\\
	\mbox{Doppelte Negation}	& \neg\neg A 				& \Leftrightarrow A\\
	\mbox{Kommutativität}		& A \wedge B 				& \Leftrightarrow B \wedge A\\
								& A \vee B					& \Leftrightarrow B \vee A\\
	\mbox{Assoziativität} 		& (A \wedge B) \wedge C 	& \Leftrightarrow A \wedge (B \wedge C)\\
								& (A \vee B) \vee C 		& \Leftrightarrow A \vee (B \vee C)\\
	\mbox{Distributiviät}		& A \wedge (B \vee C) 		& \Leftrightarrow (A \wedge B) \vee (A \wedge C)\\
								& A \vee (B \wedge C) 		& \Leftrightarrow (A \vee B) \wedge (A \vee C)\\
	\mbox{De Morgan Regeln}		& \neg (A \wedge B) 		& \Leftrightarrow \neg A \vee \neg B\\
								& \neg (A \vee B) 			& \Leftrightarrow \neg A \wedge \neg B\\
	\mbox{Implikation}			& (A \Rightarrow B) 		& \Leftrightarrow (\neg A \vee B)\\
	\mbox{Kontraposition}		& (A \Rightarrow B) 		& \Leftrightarrow (\neg B \Rightarrow \neg A)\\
	\mbox{Äquivalenz}			& (A \Leftrightarrow B)		& \Leftrightarrow (A \Rightarrow B) \wedge (B \Rightarrow A)\\
	\mbox{Absorbtion}			& A \wedge (A \vee B)		& \Leftrightarrow A\\
								& A \vee (A \wedge B)		& \Leftrightarrow A\\
\end{array}
\end{equation*}
Die Rechenregel der Aussagenlogik werden mit Hilfe von Wahrheitstafeln bewiesen.


\settowidth{\MyLenA}{Kanonisch konjunktive Normalform (KKNF)~~}
\begin{description}
	\item [Erfüllbarkeit] Es gibt eine Interpretation der Variablen, so dass die Formel wahr wird.
	\item [Allgemeingültigkeit] Ein Formel wird mit jeder Interpretation der Variablen wahr.
	\item [Widerlegbarkeit] Es gibt mindestens eine Interpretation der Variablen, so dass die Formel falsch wird.
	\item [Entscheidbarkeit] Man kann wissen, bei welcher Variablen-Kombination eine Formel wahr wird.
	\item [Konjunktive Normalform (KNF)] Konjunktion von Disjunktionstermen $(a \wedge b) \vee (c \wedge b)$
	\item [Kanonisch konjunktive Normalform (KKNF)] (Minterm) Jede Variable tritt pro KNF-Term genau einmal auf.
	\item [Disjunktive Normalform (DNF)] Disjunktion von Konjunktionstermen $(a \vee b) \wedge (c \vee d)$
	\item [Kanonisch disjunktive Normalform (KDNF)] (Maxterm) Jede Variable tritt pro DNF-Term genau einmal auf.\\	
\end{description}

Normalformen werden über Wahrheitstafeln oder durch Term-Umformung konstruiert.

Für die KDNF alle Belegungen einer Wahrheitstafel, die $w$ ergeben konjunktiv verknüpfen. Einzelterme disjunktiv verknüpfen.
Für die KKNF alle Belegungen einer Wahrheitstafel, die $f$ ergeben disjunktiv verknüpfen. Einzelterme konjunktiv verknüpfen.

\subsubsection{Beispiel Formelumwandlung zu DNF/KDNF}
Gegeben $F = ((a_1 \Rightarrow a_2) \Rightarrow \neg a_3) \vee \neg a_2$\\

1. Umrechung in DNF:\\
	\begin{align*}
		F& = ((a_1 \Rightarrow a_2) \Rightarrow \neg a_3) \vee \neg a_2\\
		(a_1 \Rightarrow a_2) & \Leftrightarrow \neg a_1 \vee a_2	& \mbox{Impl.}\\
		((\neg a_1 \vee a_2) \Rightarrow \neg a_3) & \Leftrightarrow (\neg (\neg a_1 \vee a_2)) \vee \neg a_3) & \mbox{Impl.}\\
		& \Leftrightarrow \neg(a_2 \vee \neg a_1) \vee \neg a_3 & \mbox{Assoz.}\\
		\neg(a_2 \vee \neg a_1) & \Leftrightarrow \neg a_2 \wedge \neg \neg a_1 & \mbox{De Morgan}\\
		& \Leftrightarrow \neg a_2 \wedge a_1 & \mbox {D. Vernein.}\\
		((\neg a_2 \wedge a_1) \vee \neg a_3) \vee \neg a_2 & \Leftrightarrow (\neg a_2 \wedge a_1) \vee \neg a_3 \vee \neg a_2 & \mbox{Absorbtion}\\
		& \Leftrightarrow \neg a_2 \wedge \neg a_3 
	\end{align*}

2. KDNF erstellen:\\
\begin{center}
	\begin{tabular}{c|c|c||c||c}
		$a_1$ 	&$a_2$ 	& $a_3$	 	& $F_2$ & Ausdruck\\\hline
		$w$ 	& $w$	& $w$		& $f$	& \\ 
		$w$ 	& $w$	& $f$		& $w$	& $a_1 \wedge a_2 \wedge \neg a_3$\\ 
		$w$ 	& $f$	& $w$		& $w$	& $a_1 \wedge \neg a_2 \wedge a_3$\\ 
		$w$ 	& $f$	& $f$		& $w$	& $a_1 \wedge \neg a_2 \wedge \neg a_3$\\ 
		$f$ 	& $w$	& $w$		& $f$	& \\
		$f$ 	& $w$	& $f$		& $w$	& $\neg a_1 \wedge a_2 \wedge \neg a_3$\\ 
		$f$ 	& $f$	& $w$		& $w$	& $\neg a_1 \wedge \neg a_2 \wedge a_3$\\ 
		$f$ 	& $f$	& $f$		& $w$	& $\neg a_1 \wedge \neg a_2 \wedge \neg a_3$\\ 
	\end{tabular}
\end{center}

3. Ergebnis zusammensetzen:
\begin{align*} F_2:& (a_1 \wedge a_2 \wedge \neg a_3) \vee (a_1 \wedge \neg a_2 \wedge a_3) \vee (a_1 \wedge \neg a_2 \wedge \neg a_3)\\ 
& \vee (\neg a_1 \wedge a_2 \wedge \neg a_3) \vee (\neg a_1 \wedge \neg a_2 \wedge a_3) \vee (\neg a_1 \wedge \neg a_2 \wedge \neg a_3)
\end{align*}

\subsection{Prädikatenlogik}
\textbf{Aussageform} Eine Aussageform ist eine sprachliche Äusserung, in 
der Variablen vorkommen und die in Abhängigkeit der Variablenwerte wahr ($w$) oder falsch ($f$) 
sein kann -- Aussageformen sind manchmal wahr, manchmal falsch.\\
Bsp: \enquote{Die Zahl $x$ ist eine gerade Zahl.}

Es wird unterschieden zwischen Objekt und Prädikat (Eigenschaft): für oberes Beispiel ist \enquote{ist gerade}
das Prädikat, während $x$ das Objekt ist.

\subsubsection{Quantoren (Variablenbinder)}
$A(x)$ sei eine Aussageform, $M$ eine Menge von Objekten.
\settowidth{\MyLenA}{$\forall x \in M A(x))$~~}
\begin{tabular}{@{}p{\the\MyLenA}%
				@{}p{\linewidth-\the\MyLenA}}
	$\forall x \in M A(x)$ & Für alle $x$ der Menge $M$ gilt $A(x)$\\
	$\exists x \in M A(x)$ & Für mindestens ein $x$ der Menge $M$ gilt $A(x)$\\
\end{tabular}

\subsubsection{Besondere Ausdrücke}
\begin{tabular}{ll}
	Für mindestens zwei gilt~~ & $\exists x\, \exists y ((A(x) \wedge A(y)) \Rightarrow (x \neq y))$\\
	Es gibt höchstens ein & $\forall x\, \forall y\,((A(x) \wedge A(y)) \Rightarrow (x = y))$\\
\end{tabular}

