\section{Fehler}
\settowidth{\MyLenA}{Synchronistationsfehler~~}
\begin{tabular}{@{}p{\the\MyLenA}%
				@{}p{\linewidth-\the\MyLenA}}
	Fehler & Ein Fehler liegt vor, wenn sich die gesendeten und empfangen bzw. abgespeicherte und ausgelesene $n$-Bits
	an mindestens einer Stelle unterscheiden (Bitweises XOR $\neq$ 0)\\
	Mehrfachfehler & Ein Mehrfachfehler von $x$-Bit liegt vor, wenn sich die gesendeten und empfangenen bzw. diesem
	abgespeicherten und ausgelesenen $n$-Bits an mindestens $x, x > 1$ Stellen unterscheiden.\\
	Einzelbitfehler & statistisch voneinander unabhängig -- das Auftreten eines Fehlers hängt nicht von
	anderen Fehlern ab.\\
	Bündelfehler  & (Blockfehler, Bursts) mehrere direkt aufeinanderfolgende Bits fehlerhaft
	sind -- das Auftreten hängt von anderen/früheren Fehlern ab.\\
	Synchronistationsfehler & lange Bündelfehler
\end{tabular}

\subsection{Fehlerursachen}
Typische physikalische Ursachen für Fehler sind:
\begin {itemize}
	\item Rauschen (Wärme- oder Impulsrauschen) verursacht langfristig gleichmässig verteilte Einzelbitfehler. 
	\item Dämpfung, Dispersion (Aufgrund von Leitungswiderstand, Dämpfung, Reflexion) verusacht Einzelbitfehler oder Bündelfehler.
	\item Strahlung (kosmische, radioaktive) verusacht langfristig gleichmässig verteilte Einzelbitfehler. 
	\item Störungen (elektr. Funken, Kratzer) führen zu Bündelfehlern. Die Fehler sind ungleichmässig verteilt.
	\item Nebensprechen (kapazitive Kopplung benachbarter Datenleitungen) verursacht ungleichmässig verteilte Bündelfehler.
\end{itemize}

\subsection{Einfache Parität}
Die einfache Parität wird auch einfach gebildet:
\begin{enumerate}
	\item Die Code-Wörter werden um ein zusätzliches Bit, das Paritäts-Bit erweitert.
	\item Der Wert dieses Bits wird bei:
		\begin{description}
			\item [gerader Parität (even parity)] so gesetzt, dass die Summe der Einsen im Code-Wort inkl. des Paritäts-Bits gerade ist.
			(Beispiel: Das Code-Wort sei $0100.1110 \rightarrow $ 4 Einsen sind gerade $\implies$ Paritätsbit = 0) 
			\item [ungerader Parität (ungerade)] so gesetzt, dass die Summe der Einsen im Code-Wort inkl. des Paritäts-Bits ungerade ist.
			(Beispiel: Das Code-Wort sei $0100.1110 \rightarrow $ 4 Einsen sind gerade $\implies$ Paritätsbit = 1) 
		\end{description}
\end{enumerate}

\begin{itemize}
	\item Alle ungeraden-Bit-Fehler können erkannt werden
	\item Gerade-Bit-Fehler können nicht erkannt werden
	\item Bündelfehler können nur zu 50\% erkannt werden
	\item Fehler können nicht korrigiert werden.
\end{itemize}

\subsection{Zweidimensionale Parität}
Ansatz:
\begin{itemize}
	\item Ein Block-Code aus $n$-Bits wird in Wörter gleicher Länge $l$ aufgeteilt.
	\item Jedes Wort wird um ein Paritätsbit ergänzt
	\item Jedes $i$-te Bit (inkl. Paritätsbit) jedes Wortes wird ebenfalls um ein Paritätsbit ergänzt.
\end{itemize}

Im Ergebnis:
\begin{itemize}
	\item 1-Bit-Fehler werden doppelt erkannt (Zeile und Spalte) und können korrigiert werden
	\item ungerade-Bit-Fehler werden erkannt, können aber nicht korrigiert werden (Maskierung)
	\item 2, 6, 10, \dots-Bit-Fehler werden erkannt (Zeile oder Spalte), können aber nicht korrigiert werden
	\item Vielfache von 4-Bit-Fehler werden in aller Regel erkannt. Es sei denn, sie maskieren sich doppelt, was nur extrem selten 
	(Bsp. bei einer Anordnung im Viereck) der Fall ist
	\item Bündelfehler werden nur Ausnahmsweise nicht erkannt.
\end{itemize}

\subsection{Blockcode - CRC}
$(n, k)$-Block Code bedeutet, dass für $k$ übertragene Nutzdatenbits $n-k$ Prüfbits und somit insgesamt $n$-Bits übertragen werden.

\subsubsection{Algorithmus}
Eine Bitfolge von $n$-Bits wirt als Nachrichtenpolynom vom Grad $n-1$ dargestellt:
\begin{equation*}
 I(x) = a_{n-1}x^{n-1} + a_{n-2}x^{n-2} + \dots + a_1x^1 + a_0x^0 = \sum_{i=0}^{n-1} a_i \cdot x^i
\end{equation*}

Ein Generatorpolynom $C(x)$ mit Grad $n-k, n > k, n \in \mathbb{N}$ wird zwischen Sender und Empfänger
vereinbart.

Der genaue Algorithmus zur Übermittlung einer Nachricht $I(x)$ und dem Generatorpolynom $C(x)$ vom Grad $n-k$ 
liesst sich folgendermassen:
\begin{enumerate}
 \item Sender erzeugt $N(x)$:
    \begin{enumerate}
      \item $M(x) = I(x)\cdot x^{n-k}$
      \item $R_s(x) = I(x) \mod C(x)$
      \item $N(x) = (M(x) + R_s(x))$
    \end{enumerate}
  \item Sender sendet $N(x)$
  \item Empfänger berechnet: $R_e = N(x) \mod C(x)$ und entscheidet:
    \begin{itemize}
     \item Falls $R_e = 0$: Kein Fehler, also $I(x) = M(x) + R_s(x)$
     \item Falls $R_e \neq 0$: Ein Fehler ist aufgetreten, also nochmal senden
    \end{itemize}
\end{enumerate}

\begin{description}
 \item [Ungerade-Bit-Fehler] können sicher erkannt werden, wenn das Generatorpolynom den Term $x-1$ enthält
 \item [Gerade-Bit-Fehler]  werden erkannt, deren Polynomdarstellung einen kleineren Grad als das Generatorpolynom hat,
Sie werden nur dann nicht erkannt, wenn sie ein Vielfaches des Generatorpolynoms sind
  \item [Bündelfehler] werden erkannt, wenn ihre Länge $l \leq n-k$
\end{description}