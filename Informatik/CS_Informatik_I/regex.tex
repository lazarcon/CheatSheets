\section{Reguläre Ausdrücke}
\begin{enumerate}\itemsep0em
	\item $\epsilon$ und $\emptyset$ sind reguläre Ausdrücke und beschreiben
	die Sprache $L(\epsilon) = \epsilon$ bzw. $L(\emptyset) = \emptyset$
	
	\item Wenn $a$ ein Symbol ist, dann ist $a$ auch ein regulärer Ausdruck und
	beschreibt die Sprache $L(a) = \{a\}$.
	
	\item Wenn $R$ ein regulärer Ausdruck ist, dann ist auch $(R)$ ein regulärer
	Ausdruck, der die gleiche Sprache spezifiziert: $L((R)) = L(R)$ 
	d.\,h. Klammern sind optional und dienen der Lesbarkeit.

	\item Wenn $R_1$ und $R_2$ reguläre Ausdrücke sind, dann ist auch $R_1 + R_2$ ein
	regulärer Ausdruck und es gilt: 
	\begin{enumerate}\itemsep0em
		\item $L(R_1 + R_2) = L(R_1) \cup L(R_2)$
		\item $L(R_1R_2) = L(R_1)L(R_2)$
	\end{enumerate}

	\item Wenn $R$ ein regulärer Ausdruck ist, dann ist auch $R^*$ ein regulärer Ausdruck
	und es gilt: $L(R^*) = (L(R))^*$

\end{enumerate}

Reihenfolge Auswertung: $R^* > R_1R_2 > R_1 + R_2$\\
$R? = R + \epsilon$ (Auftreten: ein oder keinmal)
\subsection{Konventionen}
\settowidth{\MyLenA}{$[a_1, a_2, \dots, a_n]$~~}
\begin{tabular}{@{}p{\the\MyLenA}%
				@{}p{\linewidth-\the\MyLenA}}
. & beliebiges Zeichen\\
$[a_1, a_2, \dots, a_n]$ & Folgen\\
$[\dots]$ & Bereichsangaben wie [0-9] oder Schlüsselwörter
\end{tabular}

\subsection{Gesetze}
\settowidth{\MyLenA}{Kommutativgesetz~~}
\begin{tabular}{@{}p{\the\MyLenA}%
				@{}p{\linewidth-\the\MyLenA}}
Kommutativgesetz & $L + M = M + L$\\
Assoziativgesetz & $(L + M) + N = L + (M + N)$\\
Verkettung & $(LM)N = L(MN)$\\
Distributivgesetz & $L(M + N) = LM + LN$\\ 
Idempotenzgesetz & $L + L = L$
\end{tabular}
\subsection{Rechenregeln}
\begin{align*}
	\emptyset + L &= L = L + \emptyset 		&  \epsilon^* &= \epsilon\\
	\epsilon L &= L = L\epsilon 				& LL^* &= L^*L = L^+\\
	\emptyset L & = \emptyset = L \emptyset 	& (L^*)^* &= L^*\\
	\emptyset^* &= \epsilon & L^* 				&= L + \epsilon
\end{align*}

Anwendungen:
\begin{enumerate}\itemsep0em
	\item Mustersuche in Texten
	\item Lexikalische Analyse (Compiler), Erkennung von Schlüsselwörtern (\enquote{Token})
	\item Darstellung von Symbolmengen
	\item Nachweis der Gültigkeit von gültigen Gesetzen (Gleichheit regulärer Ausdrücke) wird
	auf die Frage der Gleichheit der Sprachen reduziert (via endliche Automaten, Minimierung und Vergleich)
\end{enumerate}


