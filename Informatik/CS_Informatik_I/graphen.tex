\section{Graphen}
Ein Graph ist ein Tupel $G = \{V, E\}$ mit der Knotenmenge $V$ und der Kantenmenge $E$, wobei $e_i = \{V_j, V_k\}$.
Di-Graphen haben gerichtete Kanten.

\settowidth{\MyLenA}{Zusammenhängend~~}
\begin{tabular}{@{}p{\the\MyLenA}%
				@{}p{\linewidth-\the\MyLenA}}
	Vollständigkeit & Von jedem Knoten existiert ein Kante zu jedem anderen Knoten ($\frac{|V| \cdot (|V| - 1)}{2}$)\\
	Verbunden & Zwei Knoten sind verbunden, wenn eine Folge von Knoten und Kanten existiert, mit der man von einem
	Konten zum anderen gelangt\\
	Zusammenhängend & Jeder Knoten ist mit jedem anderen Knoten verbunden\\
	Schleife & Sind Knoten über mehr als eine nicht-identische Kantenfolge verbunden, liegt eine Schleife vor\\
\end{tabular}

\subsection{Bäume}
Haben eine Wurzel-Knoten, der nur gerichtete Kanten hat. Jeder handere Knoten hat nur eine einlaufende Kante.
Binäre Bäume haben pro Knoten höchstens zwei auslaufende Kanten.
\settowidth{\MyLenA}{Minimale Höhe~~}
\begin{tabular}{@{}p{\the\MyLenA}%
				@{}p{\linewidth-\the\MyLenA}}
	Blatt & Knoten ohne auslaufende Kanten\\
	Höhe & Maximum der Kanten zwischen Wurzel und Blättern\\
	Minimale Höhe & Es gibt keinen anderen Baum mit geringerer Höhe\\
	Ausgewogenheit & Es existiert höchstens ein Knoten mit nur einer auslaufenden Kante\\
\end{tabular}
