\subsection{Kompression}
\settowidth{\MyLenA}{Hamming-Abstand~~}
\begin{tabular}{@{}p{\the\MyLenA}%
				@{}p{\linewidth-\the\MyLenA}}
	Präfixfreiheit & Kein Wort ist der Anfang eines anderen Wortes (Fano-Bedingung)\\
	Zyklisch & Wenn aus einem Code-Wort alle anderen gültigen Code-Wörter abgeleitet werden können. (Bsp: abc, bca, cab)\\
	Hamming-Abstand & Anzahl unterschiedlicher Zeichen zweier gleichlanger Wörter\\
	Hamming-Distanz & Minimum des Hamming-Abstandes aller Wörter eines Codes\\
	Gray-Code		& Benachbarte Wörter haben den Hamming-Abstand 1\\
	Gleichverteilung & Alle Wörter gleich $n$ Bits lang: $\lceil n\rceil = \log_2 |Z|$\\
\end{tabular}

Umwandlung einer Dezimalzahl ($d$) in Graycode:
\begin{enumerate}
	\item Setze $s$ = Rechts-Shift von dezimal um 1 Bit
	\item Setze $g$ = XOR($d$, $s$)
	\item Gebe $g$ zurück
\end{enumerate}

\subsection{Shannon-Fano-Algorithmus}
Betrachte alle Zeichen als eine Gruppe.\\
Erstelle einen neuen Knoten auf der aktuellen Ebene.\\
Solange (Die Gruppe aus mehr als einem Zeichen besteht):
\begin{enumerate}
	\item Sortiere die Zeichen nach ihrer Häufigkeit.
	\item Teile die Zeichen entlang dieser Reihenfolge so in 2 Gruppen, dass die Summe der Häufigkeiten in den beiden Gruppen möglichst gleich ist.
	\item Führe diesen Algorithmus mit jeder Gruppe aus.
\end{enumerate}
Das Ergebnis weisst eine unnötig hohe Redundanz auf ($R = \frac{\mbox{mittlere Länge}}{\mbox{Entropie}} - 1$).
Die Anzahl der Ebenen unterhalb der Wurzel, gibt die Anzahl zu verwendender Bits an.

\subsection{Huffmann-Algorithmus}
Erstelle eine Wald mit einem Baum für jedes Zeichen.\\
Jeder dieser Bäume enthält nur einen einzigen Knoten: das Zeichen.\\ 
Schreibe die Häufigkeit an die Kante.\\
Wiederhole:
\begin{enumerate}
	\item Suche die beiden Bäume, die die geringste Häufigkeit haben.
	\item Fasse beide Bäume zusammen, indem sie die Teilbäume einer neuen Wurzel werden.
	\item Benutze die Summe der Häufigkeiten dieses neuen Baumes zur weiteren Analyse.
\end{enumerate}
bis nur noch ein Baum übrig ist.\\
Die Anzahl der Ebenen unterhalb der Wurzel, gibt die Anzahl zu verwendender Bits an.

\subsection{Arithmetische Codierung}
Kodierer und Dekodierer einigen sich auf ein Intervall. Normalerweise wird hier das halboffene Intervall $[0,1)$ verwendet.
Kodierer und Dekodierer bei der De- bzw. Kodierung eines Zeichens immer identische Tabellen 
Wahrscheinlichkeitstabellen aller dekodierbaren Zeichen. Der Algorithmus für die Codierung sieht dann wie folgt aus:
\begin{enumerate}
	\item Initialisiere das aktuelle Intervall mit dem vereinbarten Startintervall
	\item Zerlege das aktuelle Intervall auf die identische Art wie der Dekodierer in Subintervalle, 
		und weise jedem Subintervall ein Zeichen zu.
	\item Das Subintervall, das dem nächsten Zeichen der Eingabe entspricht, wird zum aktuellen Intervall.
	\item Sind noch weitere Zeichen zu kodieren, dann weiter bei Punkt 2. Sonst weiter beim nächsten Punkt.
	\item Gib eine beliebige Zahl aus dem aktuellen Intervall und zusätzlich die Anzahl der kodierten Zeichen aus. 
	Dieses ist die Zahl $x$, die vom Dekodierer wie oben beschrieben entschlüsselt werden kann. 
	Diese Zahl wird so gewählt, dass sie möglichst wenig Nachkommastellen hat, also möglichst „rund“ ist 
	und sich daher mit relativ wenigen Bits darstellen lässt.
\end{enumerate}

Beispiel: Gegeben sei das Wort \enquote{bad}.\\
\begin{tabular}{lclcl}
	$p(a) = 0{,}65$ & ~ & 
	$p(b) = 0{,}15$ & ~ &
	$p(c) = 0{,}10$ \\
	$p(d) = 0{,}04$ & ~ & 
	$p(e) = 0{,}04$ & ~ &
	$p(f) = 0{,}02$ 
\end{tabular}
\begin{enumerate}
	\item Startintervall sei $I_0 = [0, 1)$.
	\item Aufgrund der Häufigkeiten gelten folgende Teil-Intervalle:\\
		\begin{tabular}{lll}
			$I_a = [0, 0{,}65)$  & $I_b = [0{,}65, 0{,}80)$  & $I_c = [0{,}80, 0{,}90)$ \\
			$I_d = [0{,}90, 0{,}94)$  & $I_e = [0{,}94, 0{,}98)$  & $I_f = [0{,}98, 1{,}0)$ \\			
		\end{tabular}
	\item Erstes Zeichen = $b \rightarrow $ neues Startintervall = $I_0 = [0{,}65, 0{,}80)$
	\item Aufgrund der Häufigkeiten gelten folgende Teil-Intervalle:\\
		\begin{tabular}{lll}
			$I_a = [0{,}65, 0{,}7475)$  & $I_b = [0{,}7475, 0{,}77)$  & $I_c = [0{,}77, 0{,}785)$ \\
			$I_d = [0{,}785, 0{,}791)$  & $I_e = [0{,}791, 0{,}797)$  & $I_f = [0{,}797, 0{,}80)$ \\			
		\end{tabular}
	\item Zweites Zeichen = $a \rightarrow$ neues Startintervall = $I_0 = [0{,}65, 0{,}7475)$
	\item Aufgrund der Häufigkeiten gelten folgende Teil-Intervalle:\\
		\begin{tabular}{ll}
			$I_a = [0{,}65, 0{,}713375)$  & $I_b = [0{,}713375, 0{,}728)$\\  
			$I_c = [0{,}728, 0{,}73775)$  & $I_d = [0{,}73775, 0{,}74165)$\\
			$I_e = [0{,}74165, 0{,}74555)$  & $I_f = [0{,}74555, 0{,}7475)$ \\			
		\end{tabular}
	\item Drittes Zeichen = $d \rightarrow $ neues Startintervall = $[0{,}73775, 0{,}74165)$
	\item Da keine weiteren Zeichen vorhanden sind, eine möglichst kleine Zahl aus $[0{,}73775, 0{,}74165)$ wählen: $0{,}74 = \frac{74}{100}$
\end{enumerate}

Dekodierung:
Beispiel: Gegeben sei die Zahl $0{,}74$ für ein Wort bestehend aus drei Zeichen. Welches Wort verbirgt sich darin?
\begin{enumerate}
	\item Startintervall sei $I_0 = [0, 1)$.
	\item Aufgrund der Häufigkeiten gelten folgende Teil-Intervalle:\\
		\begin{tabular}{lll}
			$I_a = [0, 0{,}65)$  & $I_b = [0{,}65, 0{,}80)$  & $I_c = [0{,}80, 0{,}90)$ \\
			$I_d = [0{,}90, 0{,}94)$ & $I_e = [0{,}94, 0{,}98)$ & $I_f = [0{,}98, 1{,}0)$ \\			
		\end{tabular}
	\item $0{,}74$ liegt im Teil-Intervall von $I_b = [0{,}65, 0{,}80) \rightarrow$ das erste Zeichen ist ein $b$.
	\item Aufgrund der Häufigkeiten gelten folgende Teil-Intervalle:\\
		\begin{tabular}{lll}
			$I_a = [0{,}65, 0{,}7475)$ & $I_b = [0{,}7475, 0{,}77)$ & $I_c = [0{,}77, 0{,}785)$ \\
			$I_d = [0{,}785, 0{,}791)$ & $I_e = [0{,}791, 0{,}797)$ & $I_f = [0{,}797, 0{,}80)$ \\			
		\end{tabular}
	\item $0{,}74$ liegt im Teil-Intervall von $I_a = [0{,}65, 0{,}7475) \rightarrow$ das zweite Zeichen ist ein $a$.
	\item Aufgrund der Häufigkeiten gelten folgende Teil-Intervalle:\\
		\begin{tabular}{ll}
			$I_a = [0{,}65, 0{,}713375)$ & $I_b = [0{,}713375, 0{,}728)$\\
			$I_c = [0{,}728, 0{,}73775)$ & $I_d = [0{,}73775, 0{,}74165)$ \\
			$I_e = [0{,}74165, 0{,}74555)$ & $I_f = [0{,}74555, 0{,}7475)$ \\			
		\end{tabular}
	\item $0{,}74$ liegt im Teil-Intervall von $I_d = [0{,}73775, 0{,}74165) \rightarrow$ das dritte Zeichen ist ein $d$.
	\item Das gesuchte Wort ist \enquote{bad}
\end{enumerate}
Die Länge des gesuchten Wortes muss bekannt sein!
