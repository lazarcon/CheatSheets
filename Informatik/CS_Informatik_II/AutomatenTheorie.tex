\columnbreak
\section{Automatentheorie}
(Stephan Kleene, Michael Rabin, Dana Scott, 40er/50er Jahre)

\subsection{Endliche Automaten (EA)}
5-Tupel: $\mathcal{A} = (Q, \Sigma, \delta, q_0, F)$
\begin{align*}
	Q &= \{q_0, \dots, q_n\} & \mbox{endliche Menge von Zuständen}\\
	\Sigma &= \{a_0, \dots, a_m\} & \mbox{Eingabe-Alphabet}\\
	\delta &: Q \times \Sigma \rightarrow Q & \mbox{Übergangsfunktion}\\
	q_0 &\in Q & \mbox {Startzustand}\\
	F &= \{q_o, \dots, q_p\} & \mbox{Akzeptierende Endzustände}
\end{align*}
\begin{description}
	\item [Deterministischer endlicher Automat (DEA)]
	Geht auf Grund einer Eingabe von einem Zustand in genau einen Zustand über.
	\item [Nichtdeterministischer endlicher Automat (NEA)]
	Geht auf Grund einer Eingabe in einen von mehreren Zuständen über.
	\begin{equation*}
		\delta: Q \times \Sigma \rightarrow \mathcal{P}(Q) 
	\end{equation*}
	\item[$\epsilon$-NEA] wie NEA, erlaubt $\epsilon$ als Eingabe $\Rightarrow$ kann spontan wechseln.
	\item [Erweiterte Übergangsfunktion] 
	Ein Automat verarbeitet einem Reihe von Eingabesymbolen (das Eingabewort) in dem er mit Hilfe der Übergangsfunktion $\delta$ den jeweiligen Folgezustand ermittelt: $\delta(q_0, a_0)= q_1, \delta(q_1, a_1) = q_2, \dots$\\
	Dies wird als erweiterte Übergangsfunktion $\hat{\delta}: Q \times \Sigma \rightarrow Q$ (d.\,h. $\hat{\delta}(q_0, w) = q_n$) definiert
	
	\item[Akzeptieren] 
	Ein Eingabewort $w$ wird dann akzeptiert, wenn $\hat{\delta}$ für $w$ in einen akzeptierenden Endzustand $q_n \in F$ führt.

	\item[Sprache]
	bzeichnet die Menge aller Zeichenreihen, die der Automat akzeptiert:
		$L(A) = \{w | \hat{\delta(q_0, w)} \rightarrow q_n \in F\}$
\end{description}

\subsection{Beispiel}
Gesucht wird ein endlicher Automat $\mathcal{A}$ der die Sprache $L = \{x01y\ | x, y \in  \Sigma^* \mbox{ mit } \Sigma = \{0, 1\}\}$ akzeptiert.\\
Lösung (DEA):
\begin{align*}
\mathcal{A} = 		& (\{q_0, q_1, q_2\}, \{0, 1\}, \delta, q_0, \{q_2\})\\
\delta = 	& \{\delta(q_0, 1) = q_0, \delta(q_0, 0) = q_1, \delta(q_1, 0) = q_1 ,\\
			& \delta(q_1, 1) = q_2, \delta(q_2, 0) = q_2, \delta(q_2, 1) = q_2\}
\end{align*}
\begin{center}
\begin{tikzpicture}[->,>=stealth',shorten >=1pt,auto,node distance=3cm, semithick]
	\tikzstyle{every state}=[fill=black!10]
	\node[state, initial] 	(q0)				{$q_0$};
	\node[state]			(q1) [right of=q0]	{$q_1$};
	\node[state, accepting]	(q2) [right of=q1]  {$q_2$};
	\path 	(q0) 	edge [loop above] 	node {$1$} (q0)
					edge				node {$0$} (q1)
			(q1)	edge [loop above]	node {$0$} (q1)
					edge				node {$1$} (q2)
			(q2)	edge [loop above]	node {$0, 1$} (q2);
\end{tikzpicture}
\end{center}
Lösung (NEA):
\begin{align*}
\mathcal{A} = 		& (\{q_0, q_1, q_2\}, \{0, 1\}, \delta, q_0, \{q_2\})\\
\delta = 	& \{\delta(q_0, 1) = \{q_0\}, \delta(q_0, 0) = \{q_0, q_1\}, \delta(q_1, 1) = \{q_2\}\\
\end{align*}
\begin{center}
\begin{tikzpicture}[->,>=stealth',shorten >=1pt,auto,node distance=3cm, semithick]
	\tikzstyle{every state}=[fill=black!10]
	\node[state, initial] 	(q0)				{$q_0$};
	\node[state]			(q1) [right of=q0]	{$q_1$};
	\node[state, accepting]	(q2) [right of=q1]  {$q_2$};
	\path 	(q0) 	edge [loop above] 	node {$0, 1$} (q0)
					edge				node {$0$} (q1)
			(q1)	edge				node {$1$} (q2);
\end{tikzpicture}
\end{center}
Es gilt: $\mbox{RA} \Leftrightarrow \mbox{DEA} \Leftrightarrow \mbox{NEA} \Leftrightarrow \epsilon\mbox{-NEA}$

\subsection{Eigenschaften regulärer Sprachen}
Sind $L_1$ und $L_2$ reguläre Sprachen, dann gilt:
\settowidth{\MyLenA}{ Homomorphismus~~}
\begin{tabular}{@{}p{\the\MyLenA}%
				@{}p{\linewidth-\the\MyLenA}}
	Vereinigung & $L_1 \cup L_2$ ist regulär\\
	Durchschnitt & $L_1 \cap L_2$ ist regulär\\
	Komplement & ${L_1}^C$ ist regulär\\
	Verkettung & $L_1 + L_2$ ist regulär\\
	Differenz & $L_1 - L_2$ ist regulär\\
	Hülle & $L_1^*$ und $ L_1^+$ sind regulär\\
	Homomorphismus & $h(L_1)$ und $h^{-1}(L_1)$ sind regulär\\
	Spiegelung & ${L_1}^R$ ist regulär\\
\end{tabular}

\subsection{Entscheidbarkeit regulärer Sprachen}
Gebeben sei eine reguläre Sprache $L$
\settowidth{\MyLenA}{Ist $L$ inhärent mehrdeutig?~~}
\begin{tabular}{@{}p{\the\MyLenA}%
				@{}p{\linewidth-\the\MyLenA}}
	Ist $L$ leer? & entscheidbar\\
	$w \in L$? & entscheidbar\\
	$L_1 = L_2$ ? & entscheidbar\\
\end{tabular}

\subsection{Pumping-Lemma}
Sei $L$ eine reguläre Sprache. Dann gibt es eine Konstante $n$, so dass jder Zeichenreihe $w \in L, |w| > n,$ in drei Teilzeichenreihen $w = xyz$ wie folgt
zerlegt werden kann:
\begin{align*}
	y & \neq \epsilon\\
	|xy| & \leq n\\
	xy^kz &\in L, \mbox{ für alle }k \geq 0 
\end{align*}
\begin{center}
\begin{tikzpicture}[->,>=stealth',shorten >=1pt,auto,node distance=3cm, semithick]
	\tikzstyle{every state}=[fill=black!10]
	\node[state, initial] 	(q0)				{$q_0$};
	\node[state]			(q1) [right of=q0]	{$q_1$};
	\node[state, accepting]	(q2) [right of=q1]  {$q_2$};
	\path 	(q0) 	edge				node {$x = a_1 \dots a_i$} (q1)
			(q1)	edge [loop above]	node {$y = a_{i+1} \dots a_j$} (q1)
					edge				node {$z = a_{j+1} \dots a_m$} (q2);
\end{tikzpicture}
\end{center}
Kann für eine Sprache gezeigt werden, dass sie das Pumping-Lemma nicht erfüllt,
ist sie nicht regulär.

\subsection{Pro Memoria: Moore- \&  Mealy-Automat}
Beide Typen sind logisch äquivalent und lassen sich ineinander überführen. Beide unterscheiden sich von den endlichen Automaten dadurch, dass sie neben finalen Zuständen auch eine Ausgabe erzeugen.
Sie werden als 7-Tupel definiert:
\begin{equation*}
	\mathcal{A} = ( Q, \Sigma, \Omega, \delta, \lambda, q_0, F )	
\end{equation*}
\begin{itemize}\itemsep0em
		\item [$Q$] endliche Menge von Zuständen
		\item [$\Sigma$] endliches Eingabealphabet
		\item [$\Omega$] endliches Ausgabealphabet
		\item [$\delta$] Übergangsfunktion $\delta : Q \times \Sigma \rightarrow Q$
		\item [$\lambda$] Ausgabefunktion $\lambda : Q \rightarrow \Omega$
		\item [$q_0$] Startzustand
		\item [$F$] endliche Menge finaler Zustände ($F \subset Q$)
\end{itemize}


\begin{description}
	\item [Moore-Automat] 
	Die Ausgabe wird mit dem Zustand assoziiert und hängt somit nur vom Zustand ab.	

	\item [Mealy-Automat] 
	Die Ausgabe wird mit der Transition assoziiert, dass heisst, sie hängt vom Zustand und der Eingabe ab. Beim Zustandswechsel wird zusätzlich etwas ausgegeben.

\end{description}
