\section{Berechenbarkeit/Rekursionstheorie}
(Kurt Gödel, Alonzo Church, Alan Turing, 30er Jahre)
\begin{equation*}
	\mbox{Computer} \Leftrightarrow \mbox{Turing-Maschine}
\end{equation*}

% \begin{proof}
% 	~\\
% 	\begin{itemize}\itemsep0em
% 		\item [\enquote{$\Rightarrow$}] Elemente der Turing-Maschine $\mathcal{M}$ werden simuliert:
% 		\begin{itemize}\itemsep0em
% 			\item $\mathcal{M}$ hat eine endliche Zahl von Zuständen\\
% 			$\mapsto$ Zeichenreihen fester Länge im Speicher
% 			\item $\mathcal{M}$ hat eine endliche Zahl von Zustandsübergängen\\
% 			$\mapsto$ Tabellen im Speicher
% 			\item $\mathcal{M}$ hat eine endliche Zahl von Bandsymbolen\\ 
% 			$\mapsto$ Zeichenreihen fester Länge
% 			\item $\mathcal{M}$ hat ein unendliches Band\\
% 			$\mapsto$ einer unendlichen Menge von Wechseldatenspeichern
% 		\end{itemize}
% 		$\Rightarrow$ ein Computer kann alles, was $\mathcal{M}$ kann
% 		\item [\enquote{$\Leftarrow$}] 	Gegeben sei eine Turing-Maschine mit $4+n$ Bändern:
% 		\begin{itemize}\itemsep0em
% 			\item Band 1 simuliere den Speicher eines Computers
% 			\item Band 2 simuliere den Programmzähler
% 			\item Band 3 simuliere die Speicheradressen
% 			\item Band 4 simuliere die Eingabedaten
% 			\item $n$ weitere Bänder für notwendigen Berechnungen
% 		\end{itemize}
% 		unter diesen Bedingungen kann gezeigt werden, dass eine Turing-Maschine jede grundlegende Operation eines Computers simulieren kann $\Rightarrow$ Eine Turing-Maschine kann alles, was ein Computer kann\qedhere
% 	\end{itemize}
% \end{proof}

\begin{description}\itemsep0em
	\item [Berechenbarkeit] 
	Eine Funktion $f$ ist berechenbar, wenn es eine TM $\mathcal{M}$ gibt, die $f$ berechnet. D.\,h. es kann ein Algorithmus gefunden werden, der $f$ berechnet

	Jede TM kann \textit{partiell rekursive Funktionen}/rekursive aufzählbare Sprachen berechnen/erkennen

	\item [Primitiv rekursive Funktion] Eine Funktion, die über mindestens ein Abbruchkriterium verfügt und sich selbst aufruft. Für primitiv rekursive Funktionen gilt: Eine TM hält für alle Eingaben. Beispiele: Addition, Multiplikation, Potenz, Fibonacci-Zahlen
	
	\item [Partiell rekursive Funktion ($\mu$-rekursive Funktion)]
	Eine Funktion, welche die primitiv rekursiven Funktionen um den $\mu$-Operator erweitert. Der $\mu$ bildet dabei eine $k + 1$ stellige Funktion auf ein $k$ stellige ab. (Beispiele: Nachfolgerfunktion $\sigma(n) = n + 1$, Ackermannfunktion (s.\,u.)). Somit gilt:
	\begin{equation*}
		\mbox{primitiv rekursive Funktionen} \subsetneq \mu-\mbox{rekursive Funktionen}
	\end{equation*}
	
	\item [Rekursiv aufzählbar] 
	Es gibt einen Algorithmus, um allen Elementen einer Menge eine natürliche Zahl $n$ zuzuordnen

	\item [Church'sche These]
	Die Klasse der Turing-berechenbaren Funktionen ist genau die Klasse der \enquote{intuitiv} berechenbaren Funktionen	

	\item[Rekursive Sprache]
	Eine Sprache $L$ heisst rekursiv, wenn $L = L(\mathcal{M})$ für eine TM $\mathcal{M}$ ist und für jedes $w$ gilt:
	\begin{itemize}\itemsep0em
		\item $w \in L \Rightarrow \mathcal{M}$ hält und akzeptiert
		\item $w \notin L \Rightarrow \mathcal{M}$ hält und akzeptiert nicht
	\end{itemize}
\end{description}

\subsection{Programm-Typen}
Als Beispiel ist jeweils die Addition von $x$ + $y$ aufgeführt.
\subsubsection{GOTO-Programme}
Endliche Anzahl von Variablen $a_1, \dots a_n$ und Konstanten $K_1, \dots K_n$\\
Jede Anweisung (Zeile) ist nummeriert\\
Fünf verschiedene Anweisungen:
\lstset{numbers=left, stepnumber=1, firstnumber=1, numberfirstline=true, language=ada}
\begin{tabular}{lcl}
\begin{minipage}{0.55\linewidth}
	\begin{enumerate}\itemsep0em
		\item Zuweisung \& \enquote{$+$}: $a = a + K$
		\item Zuweisung \& \enquote{$-$}: $a = a - K$
		\item Sprunganw. (GOTO Zeile $n$)
		\item Verzweigung (\texttt{IF} \dots\ \texttt{ELSE})
		\item Stopp-Anweisung
	\end{enumerate}
\end{minipage}
& ~
&
\begin{minipage}{0.35\linewidth}
\begin{lstlisting}
k := x
l := y
k := k + 1
l := l - 1
if (l > 0) goto 3
end
\end{lstlisting}
\end{minipage}
\end{tabular}
GOTO-Programme stehen für unstrukturierten Code (Programmiersprache BASIC). Es gilt:
\begin{equation*}
	\mbox{GOTO-Programme} \Leftrightarrow \mbox{Turing-Maschine}
\end{equation*}


\subsubsection{WHILE-Programme}
\lstset{numbers=none, language=ada}
Endliche Anzahl Variablen $a_1, \dots a_n$ und Konstanten $K_1, \dots K_n$\\
Drei verschiedene Anweisungen:\\
\begin{tabular}{lcl}
\begin{minipage}{0.55\linewidth}
	\begin{enumerate}\itemsep0em
		\item Zuweisung \& \enquote{$+$}: $a = a + K$
		\item Zuweisung \& \enquote{$-$}: $a = a - K$
		\item Bedingte Schleife\\ (\texttt{WHILE} \dots\ \texttt{DO} \dots\ \texttt{END})
	\end{enumerate}
~\\
\end{minipage}
& ~
&
\begin{minipage}{0.35\linewidth}
\begin{lstlisting}
k := x
l := y
while (l > 0) do
k := k + 1
l := l - 1
end
\end{lstlisting}
\end{minipage}
\end{tabular}
Jedes beliebige WHILE-Programm kann durch eine WHILE-Programm mit genau einer Schleife ersetzt werden!

WHILE-Programmen stehen für strukturierten Code (Programmiersprachen Pascal, C, Java, \dots). Es gilt:
\begin{equation*}
	\mbox{WHILE-Programme} \Leftrightarrow \mbox{GOTO-Programme}
\end{equation*}

\subsubsection{LOOP-Programme}
Endliche Anzahl Variablen $a_1, \dots a_n$ und Konstanten $K_1, \dots K_n$\\
Drei verschiedene Anweisungen:
\begin{tabular}{lcl}
\begin{minipage}{0.6\linewidth}
	\begin{enumerate}\itemsep0em
		\item Zuweisung \& \enquote{$+$}: $a = a + K$
		\item Zuweisung \& \enquote{$-$}: $a = a - K$
		\item Feste Schleife (\texttt{LOOP DO} \dots\ \texttt{END})\\
		Die Anzahl der Schleifen ist zu Beginn festgelegt (\texttt{FOR NEXT})
	\end{enumerate}
\end{minipage}
& ~
&
\begin{minipage}{0.35\linewidth}
\begin{lstlisting}
k := x
l := y
loop l do
k := k + 1
end
\end{lstlisting}
\end{minipage}
\end{tabular}
Es gilt:
\begin{itemize}\itemsep0em
	\item [$\to$] Jedes \texttt{LOOP}-Programm kann durch ein \texttt{WHILE}- oder \texttt{GOTO}-Programm ersetzt werden
	\item [$\to$] Nicht jedes \texttt{WHILE}-/\texttt{GOTO}-Programm kann durch ein \texttt{LOOP}-Programm ersetzt werden
	\item [$\to$] \texttt{LOOP}-Programme entsprechen den primitiv rekursiven Funktionen und terminieren immer
	\item [$\to$] \texttt{GOTO}-/\texttt{WHILE}-Programme entsprechen den $\mu$-rekursiven Funktionen und terminieren nicht immer
\end{itemize}

\subsection{Partiell rekursive Funktionen }
Mit Hilfe von struktureller Induktion und Widerspruch kann gezeigt werden, dass es Funktionen (z.\,B. Ackermannfunktion) gibt, die schneller wachsen als primitiv rekursive Funktionen. 
Die Ackermannfunktion ist rekursiv, aber nicht primitiv rekursiv.
\begin{align*}
	a(0, m) & = m + 1\\
	a(n + 1, 0)& = a(n, 1)\\
	a(n + 1, m + 1) & = a(n, a(n + 1, m))
\end{align*}

\section{Entscheidbarkeit}
Die Entscheibarkeit eines Problems $L$ lässt sich unterteilen:
\begin{description}\itemsep0em
	\item [Entscheibar] sind alle Probleme/Sprachen, die rekursiv sind. D.\,h. eine TM hält für alle rekursiven Eingaben.
	\item [Semi-entscheidbar] sind alle Probleme/Sprachen, die zwar rekursiv aufzählbar, aber nicht rekursiv sind.
	D.\,h. eine TM hält für alle rekursiv aufzählbaren Eingaben.
	\item [Unentscheidbar] sind alle Probleme/Sprachen, die nicht rekursiv aufzählbar sind. 
\end{description}

% \begin{center}
% 	\begin{tabular}{lp{0.35\linewidth}|p{0.35\linewidth}}
% 		& \multicolumn{1}{c|}{\textbf{Entscheidbar}} & \multicolumn{1}{c}{\textbf{Unentscheidbar}}\\\cline{2-3}
% 	$L$ = Sprachtyp & rekursiv & nicht rekursiv \\
% 	TM		& hält für alle Eingaben & hält für mindestens eine Eingabe nicht
% 	\end{tabular}
% \end{center}
% Semi-entscheidbar heisst, 

\subsection{Unentscheidbare Probleme}
\subsubsection{Diagonalisierungssprache $L_d$ (codiert eine TM)}
Bildung:

%Für eine TM existieren i.\,d.\,R. viele Codierungen. (Reihenfolge der $\delta$). Es gibt Zahlen, die ergeben keine oder keine sinnvolle TM. Für diese wird angenommen, dass die TM nur einen Zustand hat und für jede Eingabe hält, ohne zu akzeptieren.

Jede $\mathcal{M}_i$ lässt sich als Zeichenreihe $w_i$ darstellen (Codierung TM). In einer Tabelle mit allen $w_i$ als Zeilen und $w_i$ als Spalten steht in Zelle $z, s$ ob $\mathcal{M}_z \mathcal{M}_s$ als Eingabe akzeptiert (Wert: $1$) oder nicht (Wert: $0$) $L_d$ ist die Sprache aus allen denen Wörtern, die Turing-Maschinen codieren, die sich selbst nicht als Eingabewort akzeptieren. 
\begin{equation*}
L_d = \{w_i | w_i \notin L(\mathcal{M}_i)\}
\end{equation*}
$\Rightarrow L_d$ ist nicht rekursiv aufzählbar $\Rightarrow$ nicht entscheidbar.\\

%$L_d$ kann nicht rekursiv aufzählbar sein, weil $\mathcal{M}_i$ nicht zu gleich $w_i$ akzeptieren (d.\,h. es wäre aufzählbar) und nicht akzeptieren ($w_i$ ist genau so definiert) kann.

\subsection{Semi-entscheidbare Probleme}
\subsubsection{Halteproblem}
\begin{satz}
	Es ist unentscheidbar, ob eine TM für jedes codierte Paar $(c(\mathcal{M}), w)$ als Eingabe hält oder nicht.
\end{satz}
$\mathcal{M}$ hält nur für Paare bei denen $w \in L_U$ ist.
% 
% Sei $L_U$ die Menge aller Zeichenreihen, die sich aus der Konkatenation einer codierten TM $\mathcal{M}$, \enquote{111} und einem von $\mathcal{M}$ akzeptierten Eingabewort ergeben.
% (z.\,B. $c(\mathcal{M}) = 0100100010100\,11\,000101010010, w=11$\\$ \Rightarrow 0100100010100\,11\,000101010010\,111\,11 \in L_U$). Folgendes lässt sich dann beweisen:
% \begin{enumerate}\itemsep0em
% 	\item Es existiert eine \textit{universelle} TM $U$, die $L_U$ akzeptiert d.\,h. $L_U \in L(U)$.
% 	\item $L_U$ ist rekursiv aufzählbar.
% 	\item $L_U$ ist nicht rekursiv.
% \end{enumerate}
\begin{equation*}
	\Rightarrow \{L_{\mbox{rekursiv}}\} \subsetneq \{L_{\mbox{rekursiv aufzählbar}}\} \not\subset \{L_{\mbox{nicht rekursiv aufzählbar}}\}
\end{equation*}	

\subsubsection{Game-of-Life}
% Eine Ebene ist in einheitliche Quadrate aufgeteilt. Jedem Quadrat ist ein Automat zugeordnet, der den Folgezustand (lebendig/tot) in Abhängigkeit des Zustandes der bis zu acht Nachbarquadraten berechnet. Die Regeln sind:
% \begin{enumerate}\itemsep0em
% 	\item lebendig, wenn selbst lebendig und 2 oder 3 Nachbarn lebendig sind
% 	\item lebendig, wenn selbst tod, aber genau 3 Nachbarn lebendig sind
% 	\item tot, wenn selbst lebendig und mehr als 3 Nachbarn lebendig sind
% 	\item tot, wenn selbst lebendig und weniger als zwei Nachbarn lebendig sind
% \end{enumerate}
Die Frage, ob eine gegebene Anfangskonfiguration ($w$) zu ein stabilen oder periodischen Muster, oder dem kompletten Aussterben führt, kann nicht immer beantwortet werden.

\subsubsection{Collatz-Zahlen}
Gegeben sei die Collatz-Funktion col:
\begin{equation*}
	 \mbox{col}(n) = \begin{cases} n/2 &\text{falls } n \equiv 0 \pmod{2}\\ 3n+1 & \text{sonst}\end{cases}
\end{equation*}
Wendet man die Funktion auf das Ergebnis einer ursprüngliche Eingabe $n \in \mathbb{N}$ wiederholt an, entsteht die Collatz-Folge. Sie endet bei $n = 1$. Es ist unbekannt, ob die Menge der Collatz-Zahlen rekursiv ist.
%Die Collatz-Vermutung besagt, dass sich die Collatz-Folge für jede beliebige Zahl $n \in \mathbb{N}$ irgendwann die 1 erreicht. Es ist nur eine Vermutung, weil unbekannt ist, ob die Menge der Collatz-Zahlen rekursiv ist.

\subsubsection{Fleissige Biber}
Ein fleissiger Biber ist eine Turingmaschine mit dem Alphabet $\Sigma = \{0, 1\}$ und $n$ Zuständen, die hält und zuvor auf ein leeres (aus Nullen bestehendes) Band die maximale Anzahl $k_n$ von Einsen schreibt, verglichen mit allen anderen haltenden Turingmaschinen mit ebenfalls $n$ Zuständen (Nur Nicht-haltende-Turingmaschinen schreiben mehr einsen).
Die Fleissige-Biber-Funktion ist definiert als $\sum(n) = k_n$. Es ist nicht entscheidbar, ob eine gegebene TM tatsächlich eine Kette maximaler Länge schreibt.

\subsection{Reduktion}
Technik um zu zeigen, dass ein Problem $P_2$ mindestens so schwer ist, wie ein bekanntes Problem $P_1$.
Dazu wird jede Instanz von $P_1$ wird auf eine Instanz von $P_2$ mit gleicher Antwort überführt. $\Rightarrow P_1 \subset P_2$
\begin{enumerate}\itemsep0em
	\item $P_1$ ist unentscheidbar $\Leftrightarrow P_2$ ist unentscheidbar
	\item $P_1$ ist nicht rekursiv aufzählbar $\Leftrightarrow P_2$ ist nicht rekursiv aufzählbar
\end{enumerate}

\subsection{Unentscheidbarkeit}
\begin{satz}
	Jede nicht triviale Eigenschaft der rekursiv aufzählbaren Sprachen ist unentscheidbar.
\end{satz}
\begin{description}\item
	\item [Eigenschaft] einer rekursiv aufzählbaren Sprache ist eine Menge von rekursiv aufzählbaren Sprachen
	\item [trivial] Entweder leer (d.\,h. kann keiner Sprache zugeordnet werden) oder aus allen rekursiv aufzählbaren Sprachen bestehend
\end{description}

% Die Sprachen $L_e$ und $L_{ne}$ enthalten als Binärzeichenreihe codierte TMs. Dann ist definiert:
% \begin{enumerate}\itemsep0em
% 	\item $L_e = \{\mathcal{M} | L(\mathcal{M}) = \emptyset\}$ d.\,h. $L_e$ enthält TMs, die keine Eingabe akzeptieren
% 	\item $L_{ne} = \{\mathcal{M} | L(\mathcal{M}) \neq \emptyset\}$ d.\,h. enthält TMs, die mindestens eine Eingabe akzeptieren. $L_{ne}$ ist rekursiv aufzählbar aber nicht rekursiv.
% \end{enumerate}

Unentscheidbar sind u.\,a. folgende Fragen:
\begin{itemize}\itemsep0em
	\item ist eine von $\mathcal{M}$ akzeptierte Sprache endlich?
	\item ist eine von $\mathcal{M}$ akzeptierte Sprache kontextfrei?
	\item ist eine von $\mathcal{M}$ akzeptierte Sprache regulär?
	\item berechnet eine $\mathcal{M}$ eine gegebene Funktion?
	\item sind zwei gegebene Programme äquivalent?
	\item ist eine berechnete Funktion injektiv, surjektiv oder monoton?
	\item ist eine kfG eindeutig?
\end{itemize}


Eine Eigenschaft .
