\section{Kontextfreie Grammatik (kfG)}
Eine kontextfreie Grammatik wird als 4-Tupel definiert:
\begin{equation*}
	G = (V, T, P, S)
\end{equation*}
\begin{itemize}\itemsep0em
	\item [V] Endliche Menge von Zeichenreihen ($A, B, C, \dots$)
	\item [T] Endliche Menge von Terminalsymbolen ($a, b, c, \dots$)
	\item [P] Endliche Menge von Produktionen/Regeln, definieren die Sprache rekursiv ($A \rightarrow AA$)
	\item [S] Das Startsymbol ($S \in V$)
\end{itemize}

\subsection{Beispiel: Palindrome ($\Sigma = \{0, 1\}$)}
\begin{gather*}
	G_P = (\{A\}, \{0,1\}, P, A)\\
	P: A \rightarrow \epsilon, A \rightarrow 0, A \rightarrow 1, 
	A \rightarrow 0A0,  A \rightarrow 1A1\\
	\mbox{oder }P: A \rightarrow \epsilon|0|1|0A0|1A1
\end{gather*}

\begin{description}\itemsep0em
	\item [Ableitungsschritt] 
	Gegeben sei eine kfG $G$ und eine Wort $\alpha A\beta$, mit $\alpha$ und $\beta$ sind Wörter aus $V \cup T$ und $A \in V$ sowie einer Produktion $A \rightarrow \gamma \in P$. Dann ist die Relation $\alpha A\beta \Rightarrow \alpha\gamma\beta$ ein Ableitungsschritt in $G$.\\
	Beispiel: $0A0 \Rightarrow 00A00$ ist ein Ableitungsschritt.\\

	\item[Ableitung]
	Ein Ableitungsschritt kann zu einer Ableitung aus $0-n$ Ableitungsschritten erweitert werden: $\alpha \overset{\centerdot}{\underset{G}{\Rightarrow}} \gamma$ ist eine Ableitung in $G$.\\
	Beispiel: $A \overset{\centerdot}{\underset{G}{\Rightarrow}} 1101011: A \underset{G}{\Rightarrow} 1A1 \underset{G}{\Rightarrow} 11A11 \underset{G}{\Rightarrow} 110A011 \underset{G}{\Rightarrow} 1101011$
\end{description}

Um Ableitungsschritte zu definieren empfiehlt sich:
\begin{enumerate}\itemsep0em
	\item Ausdrucksproduktionen ($A$) bestimmen (z.\,B. $A \rightarrow A + A$)
	\item Bezeichnerproduktionen ($B$) bestimmen (z.\,B. $B \rightarrow a|b|\dots|z$)
	\item Ausdrücke ableiten (von links oder von rechts: $A \underset{G}{\Rightarrow} A + A$) 
	\item Bezeichner ableiten (von links oder rechts, aber gleich wie im vorherigen Schritt: $A + A \underset{G}{\Rightarrow} a + A \underset{G}{\Rightarrow} a + b$)
\end{enumerate}

\begin{description}\itemsep0em
	\item [Linksseitige Ableitung] in jedem Schritt wird die äusserste linke Variable durch eine Produktion ersetzt: $\alpha \overset{\centerdot}{\underset{lm}{\Rightarrow}} \gamma$.
	\item [Rechtsseitige Ableitung] in jedem Schritt wird die äusserste rechte
	Variable durch eine Produktion ersetzt: $\alpha \overset{\centerdot}{\underset{rm}{\Rightarrow}} \gamma$.
	\item[Sprache der kfG] Menge der Zeichenreihen aus terminalen Symbolen,
	die sich vom Startsymbol ableiten lassen:
	\begin{equation*}
		L(G) = \{w | S \overset{\centerdot}{\underset{G}{\Rightarrow}} w \wedge w \in T^*\}
	\end{equation*}
\end{description} 

\subsection{Parsebäume}
\begin{itemize}
	\item stellen Ableitungen als Bäume dar
	\item verdeutlichen wie terminale Wörter aus Teilwörtern strukturiert sind
	\item werden von Compilern erzeugt: Datenstruktur, die Quellprogramme repräsentiert
\end{itemize}

Gegeben sei eine kfG $G = (V, T, P, S)$. Ein Baum ist ein Parsebaum von $G$ wenn:
\begin{enumerate}
	\item Alle inneren Knoten mit einer in $V$ enthaltenen Variable bezeichnet sind
	\item Für jedes Blatt gilt, dass es entweder
	\begin{itemize}
		\item mit einer Variablen aus $V$
		\item einem terminalen Symbol aus $T$ oder
		\item dem leeren Wort $\epsilon$ bezeichnet ist, dann muss das Blatt aber das einzige des Vorgängerknotens sein
	\end{itemize}
	\item Für alle innere Knoten gilt ($A =$ Bezeichnung des Knotens und $X_1, \dots, X_n$ Bezeichnung der Nachfolgerknoten von links nach rechts): $(A \rightarrow X_1, \dots, X_n) \in P$
\end{enumerate}
Beispiel Programm-Ausdruck: $A \overset{\centerdot}{\underset{G}{\Rightarrow}} a + b \cdot c$:\\
\begin{tabular}{ll}
\settowidth{\MyLenA}{(Linksseitige) Ableitung~}
\begin{minipage}{\the\MyLenA}
	(Linksseitige) Ableitung:	
	\begin{tabular}{ll}
	$A$ & $\underset{G}{\Rightarrow} A + A$\\
	& $\underset{G}{\Rightarrow} A + A \cdot A$\\
	& $\underset{G}{\Rightarrow} B + A \cdot A$\\
	& $\underset{G}{\Rightarrow} B + B \cdot A$\\
	& $\underset{G}{\Rightarrow} B + B \cdot B$\\
	& $\underset{G}{\Rightarrow} a + B \cdot B$\\
	& $\underset{G}{\Rightarrow} a + b \cdot B$\\
	& $\underset{G}{\Rightarrow} a + b \cdot c$\\
	\end{tabular}
\end{minipage}
&
\begin{minipage}{\linewidth-\the\MyLenA}
\begin{tikzpicture}[->,>=stealth',shorten >=1pt,auto,level distance=1.25cm, sibling distance = 1.25cm, semithick]
	\tikzstyle{every state}=[fill=black!10]
	\node[state] {$A$}
		child {node [state] {$A$}
			child {node [state] {$B$}
				child {node [state, accepting] {$a$}}
			}
		}
		child {node [state, accepting] {$+$}}
		child {node [state] {$A$}
			child {node [state] {$A$}
				child {node [state] {$B$}
					child {node [state, accepting] {$b$}}
				}
			}
			child {node [state, accepting] {$\cdot$}}
			child {node [state] {$A$}
				child {node [state] {$B$}
					child {node [state, accepting] {$c$}}
				}
			}
		}
	;
\end{tikzpicture}
\end{minipage}
\end{tabular}
Die Blätter eines Parsebaums ergeben von links nach rechts gelesen eine
Zeichenreihe einer kfG $G = (V, T, P, S)$ wenn:
\begin{enumerate}\itemsep0em
	\item Die Wurzel mit dem Startsymbol bezeichnet ist
	\item Alle Blätter mit terminalen Symbolen oder $\epsilon$ bezeichnet sind
\end{enumerate}
Die Sprache $L(G)$ ist genau die Menge an Zeichenreihen, die sich aus der
von links nach rechts gelesenen Zeichen der Blätter aller Parsebäume ergeben.

\subsubsection{Mehrdeutigkeit}
\begin{itemize}\itemsep0em
	\item Anwendungen (Programmiersprachen) erfordern, dass die mit einer kfG erzeugten Strukturen eindeutig sind
	\item Nicht jede kfG ist eindeutig
	\item Es gibt Sprachen, für die jede kfG immer mindestens
	einer Zeichenkette mehr als eine Struktur zuordnen. Diese Sprachen heissen
	\enquote{inhärent mehrdeutig}. Im Beispiel ($A \overset{\centerdot}{\underset{G}{\Rightarrow}} a + b \cdot c$) kann nicht erkannt werden ob
	$a + (b \cdot c)$ oder $(a + b) \cdot c$ zu rechnen ist (Aus der linksseitigen Ableitung ergibt sich $a + (b \cdot c) $, aus der rechtsseitigen ergäbe sich das andere).
\end{itemize}
Ein Sprache ist mehrdeutig, wenn für mindestens eine Zeichenreihe mehr als ein Parsebaum exisitiert.

Häufig (aber nicht immer) lässt sich eine kfG so anpassen, dass sie eindeutig wird.

\subsection{Mehrdeutigkeit eliminieren}
Das Beispiel $A \overset{\centerdot}{\underset{G}{\Rightarrow}} a + b \cdot c$ wird eindeutig durch:
\begin{itemize}\itemsep0em
	\item erzwungene Klammern oder
	\item angepasste Produktionen, so dass $\cdot > +$
	\item Linksableitung vorgeben $A \rightarrow A + A | A \cdot A | (A)$ (Addition wird vor der Multiplikation, die wiederum vor der Klammer aufgelöst)
\end{itemize}

\subsection{Anwendungen}
\begin{itemize}\itemsep0em
	\item Beschreibung natürlicher Sprachen (N. Chomsky): allerdings sind kfG nicht für natürliche Sprachen geeignet, weil Mehrdeutigkeiten häufig vom Kontext abhängen
	\item Compiler: Klammern, Blöcke, Bedingungen erfordern KfG, für andere Komponenten reichen reguläre Ausdrücke (z.\,B. $A \rightarrow \epsilon | if (A) else | if (A) | AA$)
	\item Parsergeneratoren (Quellcode $\rightarrow$ Parsebaum):
	\begin{itemize}
		\item Bezeichner werden über lexikalische Analyse erkannt
		\item Strukturen werden über kfG erkannt
	\end{itemize}
	\item Markupsprachen (z.\,B. HTML): wie Parsegeneratoren
	\item XML: (kfG beschreibt die Semantik über die DTD(Dokumentendefinition))
\end{itemize}
