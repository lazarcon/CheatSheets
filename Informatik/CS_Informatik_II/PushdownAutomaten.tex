\section{Pushdown-Automaten (=Kellerautomaten)}
Sind im Gegensatz zu endlichen Automaten unendliche Automaten.
\begin{description}\itemsep0em
	\item [Pushdown-Automat (PDA)] 
	Ein (nichtdeterministischer) Pushdown-Automat $P$ wird als 7 Tupel definiert: $P = (Q, \Sigma, \Gamma, \delta, q_0, s_0, F)$
	\begin{itemize}\itemsep0em
		\item [$Q$] endliche Menge von Zuständen
		\item [$\Sigma$] endliches Eingabealphabet
		\item [$\Gamma$] endliches Stackalphabet
		\item [$\delta$] Übergangsfunktion $\delta \colon Q \times (\Sigma \cup \{ \varepsilon \} ) \times \Gamma \rightarrow \mathcal P(Q \times \Gamma^{*})$
		\item [$q_0$] Startzustand ($q_0 \in Q$)
		\item [$s_o$] Startsymbol des Stacks ($s_0 \in \Gamma$) 
		\item [$F$] Menge der akzeptieren Zustände ($F \subset Q$)
	\end{itemize}

	Die Übergangsfunktion $\delta$ bildet das Tripel $\delta(q_v \in Q, e \in \Sigma, s_v \in \Gamma)$  auf eine endliche Menge von Paaren $(q_n \in Q, s_n \in \Gamma)$ ab. Die Indexe $n$ und $v$ stehen dabei für vorher ($v$) respektive nachher ($n$).

	\item [Stack]
	Ein Stack ist ein Last In -- First Out (LIFO) Speicher. Ein PDA liest das oberste Elemente $x \in \Gamma$ und ersetzt es durch $\gamma \in \Gamma$.
	\begin{itemize}\itemsep0em
		\item $\gamma = \varepsilon$ (\enquote{pop}, entfernt oberstes Element)
		\item $\gamma = x$ (\enquote{void}, Stack unverändert)
		\item $\gamma \neq x \wedge \gamma \neq \varepsilon$ (\enquote{pop/push}, ersetzt oberstes Element)
		\item Ist $|\gamma| > 1$ entspricht einem \enquote{pop} und $|\gamma|$ \enquote{pushs}.
	\end{itemize}

	\item [Konfiguration] Das Tripel $(q, w, \gamma)$ wird als Konfiguration $K$ von $P$ bezeichnet.
	\begin{itemize}\itemsep0em
		\item [$q$] aktueller Zustand
		\item [$w$] noch nicht gelesener Teil der Eingabealphabet
		\item [$\gamma$] Inhalt des Stacks
	\end{itemize}

	\item[Bewegung]
	Sei $(p, \alpha)$ ein Element von $\delta(q, e, s)$, dann stellt ein Wechsel von Konfiguration $(q_v, ew, s\beta)$ nach $(q_n, w, \alpha\beta)$ eine Bewegung dar: $(q_v, ew, s\beta) \underset{P}{\vdash} (q_n, w, \alpha\beta)$

	\item[Berechnung]
	Gegeben sei ein PDA und eine Folge von Konfigurationen $K_1, \dots, K_n$ für die paarweise gilt: $K_i \vdash K_{i+1}$ ist eine Bewegung, dann stellt die Folge der Konfigurationen eine Berechnung vom PDA dar (analog $\hat{\delta}$): $K_1 \overset{\centerdot}{\underset{P}{\vdash}} K_n$

	\item[Sprache (Endzustand)] 
	Die Menge aller Zeichenreihen $w$, für die in $P$ ausgehend von der Anfangskonfiguration $K_0$ eine Berechnung exisitiert, so dass $P$ in einen akzeptierenden Endzustand wechselt wird mit $L(P)$ bezeichnet.
	\begin{equation*}
		L(P) = \{w | (q_0, w, s_0) \overset{\centerdot}{\underset{P}{\vdash}} (f, \varepsilon, \beta) \wedge f \in F\}
	\end{equation*}
	
	$P$ akzeptiert durch seinen Endzustand. Der Stack spielt keine Rolle.

	\item[Sprache (leerer Stack)] Die Menge aller Zeichenreihen $w$ für die in $P_N$ ausgehend von einer Anfangskonfiguration $K_0$ eine Berechnung existiert, so dass $P$ nach dem Einlesen von $w$ einen leeren Stack aufweist wird mit $N(P)$ bezeichnet.
	\begin{equation*}
		N(P) = \{w | (q_0, w, s_0) \overset{\centerdot}{\underset{P}{\vdash}} (f, \varepsilon, \varepsilon)\}
	\end{equation*}

	$P$ akzeptiert durch leeren Stack. Der Zustand spielt keine Rolle d.\,h. es gibt keine Endzustände, d.\,h. $P = (Q, \Sigma, \Gamma, \delta, q_0, s_0$) (6-Tupel!)

	Der Automat schreibt also ein $\varepsilon$, wenn er $s_0$ im Keller hat und die Eingabe $\varepsilon$ liest.

\end{description}

\subsection{Beispiel (akzeptiert durch Endzustand)}
Ein Automat, der die Sprache $L = \{a^{n}b^{n} | n > 0\}$ erkennt:
\begin{equation*}
	P = (\{q_0, q_1, q_2\}, \{a, b\}, \{a, s_0\}, \{(q_0, w, s_0) \overset{\centerdot}{\vdash} (q_3, \varepsilon, \varepsilon)\}, q_0, s_0, q_2) 
\end{equation*}
\begin{center}
	\begin{tikzpicture}[->,>=stealth',shorten >=1pt,auto, node distance=2.5cm, semithick]
	\tikzstyle{every state}=[fill=black!10, align=center]
	\tikzstyle{every node}=[align=center]

	\node[state, initial]		(A) 				{$q_0$};
	\node[state] 				(B)	[right of=A]	{$q_1$};
	\node[state, accepting]		(C) [right of=B]	{$q_2$};

	\path 	(A) edge [loop above] 	node {$a, s_0/as_0$\\$a, a/aa$} (A)
				edge              	node {$b, a/\varepsilon$} (B)
			(B) edge [loop above]	node {$b, a/\varepsilon$} (B)
				edge 				node {$\varepsilon, s_0/s_0$} (C);
	\end{tikzpicture}
\end{center}
Der Automat führt beispielsweise die Berechnung $(q_0, aabb, s_0) \overset{\centerdot}{\vdash} (q_3, \varepsilon, s_0)$ durch. Die Bewegungen dabei sind:
\begin{multline*}
	(q_0, aabb, s_0) \vdash (q_0, abb, as_0) \vdash (q_0, bb, aas_0) \vdash (q_1, b, as_0)\\ \vdash (q_1, \varepsilon, s_0) \vdash (q_2, \varepsilon, s_0)
\end{multline*}

\subsection{Sätze}
Gegeben sei ein PDA $P = P = (Q, \Sigma, \Gamma, \delta, q_0, s_0, F)$:
\begin{itemize}\itemsep0em
	\item Wenn $(q, x, \alpha) \overset{\centerdot}{\underset{P}{\vdash}} (p, y, \beta)$ eine Berechnung in $P$ ist, dann gilt für alle $w \in \Sigma^*$ und alle $\gamma \in \Gamma^*$, dass auch $(q, xw, \alpha\gamma) \overset{\centerdot}{\underset{P}{\vdash}} (p, yw, \beta\gamma)$ eine Berechnung in $P$ ist.

	\item Wenn $(q, xw, \alpha) \overset{\centerdot}{\underset{P}{\vdash}} (p, yw, \beta)$ eine Berechnung in $P$ ist, dann ist auch $(q, x, \alpha) \overset{\centerdot}{\underset{P}{\vdash}} (p, y, \beta)$ eine Berechnung in $P$ (Umkehrung von 1)

	\item Aber aus $(q, x, \alpha\gamma) \overset{\centerdot}{\underset{P}{\vdash}} (p, \gamma, \beta\gamma)$ folgt nicht, dass auch $(q, x, \alpha) \overset{\centerdot}{\underset{P}{\vdash}} (p, y, \beta)$ eine Berechnung ist, weil die Eingabe \enquote{verbraucht} wird.

	\item Für jeden Automat $P_N$, der durch leeren Stack akzeptiert, gibt es auch einen Automat $P_L$, der durch den Endzustand akzeptiert: $L = N(P_N) = L(P_L)$
		
	\item Für jeden Automat $P_L$, der durch Endzustand akzeptiert, gibt es auch einen Automat $P_N$, der durch leeren Stack akzeptiert: $L = L(P_L) = N(P_N)$. 

	\item $L(G) \Leftrightarrow N(P) \Leftrightarrow L(P)$ (mit $G$ ist eine kfG).
\end{itemize}

\subsection{Deterministischer PDA (DPDA)}
Ein PDA ist deterministisch wenn:
\begin{enumerate}\itemsep0em
	\item $\delta(q, a, s)$ höchstens ein Element enthält
	\item $\delta(q, a, \gamma) \neq \emptyset \Rightarrow \delta(q, \varepsilon, \gamma) = \emptyset$
\end{enumerate}
Ein DPDA $P$ kann in jeder Konfiguration höchstens eine Bewegung ausführen. Er kann reguläre Sprachen erkennen, aber nicht alles, was ein PDA erkennen kann. %Weil er beispielsweise bei Palindromen die Mitte der Zeichenkette nicht ermitteln kann ohne den Stack zu leeren und dabei alle gespeicherten Symbole zu \enquote{vergessen}.
\begin{equation*}
	L^*(\mbox{DEA}) = L^*(\mbox{NEA}) = L^*(\varepsilon\mbox{NEA}) \subsetneq L^*(\mbox{DPDA}_E) \subsetneq L^*(\mbox{PDA})
\end{equation*}
DPDA$_E$ steht für einen Automaten, der durch Endzustand akzeptiert.
	
Für eine Sprache $L$ und einen DPDA $P$, der über leeren Stack akzeptiert, gilt $N(P) = L$ genau dann wenn:
\begin{enumerate}\itemsep0em
	\item $L$ präfixfrei (kein Wort ist der Anfang eines anderen Wortes) ist
	\item Es einen DPDA $P'$, der über leeren Stack akzeptiert, gibt mit $L(P) = L$ 
\end{enumerate}

Ein DPDA der durch leeren Stack akzeptiert erkennt nicht mal alle reguläre Sprachen (z.\,B.: $L = \{0\}^*$), allerdings kann er auch nicht-reguläre Sprachen (z.\,B. $L=\{wcw^R | w \in \{0, 1\}\}$) erkennen.

\subsubsection{DPDA und kfG}
Gegeben sei eine Sprache $L$ und ein DPDA $P$.
\begin{itemize}\itemsep0em
	\item $L = N(P) \Rightarrow L$ hat eine eindeutige kontextfreie Grammatik.
	\item $L = L(P) \Rightarrow L$ hat eine eindeutige kontextfreie Grammatik.
	\item $L$ hat eine eindeutige kontextfreie Grammatik heisst aber nicht, dass es einen passenden DPDA gibt. (z.\,B. Palindrome)
\end{itemize}

\subsection{Eigenschaften kontextfreier Sprachen}
Seien $L_1$ und $L_2$ kontextfreie Sprachen über $\Sigma$. Dann gelten:
\settowidth{\MyLenA}{ Homomorphismus~~}
\begin{tabular}{@{}p{\the\MyLenA}%
				@{}p{\linewidth-\the\MyLenA}}
	Vereinigung & $L_1 \cup L_2$ ist kontextfrei\\
	Durchschnitt & $L_1 \cap L_2$ ist nicht kontextfrei\\
	Komplement & ${L_1}^C$ kann kontextfrei sein\\
	Verkettung & $L_1 + L_2$ ist kontextfrei\\
	Differenz & $L_1 - L_2$ kann kontextfrei sein\\
	Hülle & $L_1^*$ und $ L_1^+$ sind kontextfrei\\
	Homomorphismus & $h(L_1)$ und $h^{-1}(L_1)$ sind kontextfrei\\
	Spiegelung & ${L_1}^R$ ist eine kontextfrei\\
\end{tabular}

\subsubsection{Entscheidbarkeit kontextfreier Sprachen}
Gebeben sei ein kontextfreie Sprache $L$ und eine kontextfreie Grammatik $G$
\settowidth{\MyLenA}{Ist $L$ inhärent mehrdeutig?~~}
\begin{tabular}{@{}p{\the\MyLenA}%
				@{}p{\linewidth-\the\MyLenA}}
	Ist $L$ leer? & entscheidbar\\
	$w \in L$? & entscheidbar\\
	Ist $L$ inhärent mehrdeutig? & nicht entscheidbar\\
	Ist $G$ mehrdeutig & nicht entscheibar\\
	$L_1 \cap L_2 = \emptyset$? & nicht entscheidbar\\
	$L_1 = L_2$ ? & nicht entscheidbar\\
\end{tabular}

\subsection{Nicht kontextfreie Sprachen}
\subsubsection{Pumping-Lemma}
Sei $L$ eine kontextfreie Sprache. Dann gibt es eine eine Konstante $n$, so dass jede Zeichenreihe $w \in L$ und die Länge von $z \geq 0$ in fünf Teilzeichenreihen $w = uvxyz$ derart zerlegt werden kann, dass:
\begin{enumerate}\itemsep0em
	\item $|vxy| \leq n$
	\item $vx \neq \varepsilon$
	\item $uv^kxy^kz \in L$ für alle $k \geq 0$
\end{enumerate}
Eine mittlere Teilzeichenreihe ($vxy$) wird beliebig oft wiederholt. 

Beispiel: $L = \{0^m1^m2^m | m > 0\}$\\
Ist $L$ kontextfrei, dann gibt es eine Konstante $n$, so dass $0^n1^n2^n$ ebenfalls in $L$ ist:
Nun muss $w$ in $uvxyz$ so zerlegt werden, dass $vx \neq \varepsilon$ und $|vxy| \leq n$ gilt.
$vxy$ kann aber nicht zugleich $0$ und $2$ enthalten (sonst wäre der Abstand zwischen der letzen $0$ und der ersten $2 \neq n + 1$)\\
Fall I: $2 \notin vxy \Rightarrow 0 \vee 1 \in uwz$: Widerspruch zur Annahme $uwz \in L$\\
Fall II: $0 \notin vxy \Rightarrow 1 \vee 2 \in uwz$: Widerspruch zur Annahme $uwz \in L$



