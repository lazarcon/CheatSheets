\section{Komplexitätstheorie}
(Kleene, Rabin, Scott, 70er Jahre)
\begin{description}\itemsep0em
	\item [Effizienz] Die Effizienz eines Algorithmus bemisst sich nach Laufzeit und Speicherplatzbedarf
	\item [Laufzeiteffizienz] 
	wird über eine Funktion $f(n)$ in Abhängigkeit der Eingabewerte $n$ beschreiben und bestimmt die obere Schranke
\end{description}

\subsection{Laufzeitvergleiche}
\begin{itemize}\itemsep0em
	\item müssen unabhängig von Soft- und Hardware sein
	\item betrachten besonders das Verhalten für alle grosse Eingaben
	\item ausschlaggebend ist der schlechteste Fall -- die obere Schranke
	\item erfolgen mit Hilfe der 
	\begin{itemize}
		\item $\Omega$-Notation: untere Schranke (d.\,h. mindestens so lang)
		\item O-Notation: obere Schranke (d.\,h. höchstens so lang) 
	\end{itemize}
\end{itemize}

\subsubsection{$\Omega$-Notation}
Die untere Schranke lässt sich nur schwer bestimmen, weil gezeigt werden muss, dass alle Algorithmen mindestens diese Komplexität besitzen. Formell:
%Die $\Omega$-Notation ermöglicht eine Abschätzung der minimalen Laufzeit. Formell:
\begin{equation*}
	\Omega(f(n)) = \{g: \mathbb{N} \to \mathbb{N} | \exists\, n_0 > 0 \wedge \exists\, c > 0 : \forall\, n \geq n_0 : |g(n)| \geq c \cdot |f(n)|\}
\end{equation*}


\subsubsection{O-Notation} 
(\textit{asymptotische Laufzeitkomplexität}) ermöglicht eine Abschätzung der Laufzeit bei unendlich grossen Eingaben. Formell:
\begin{equation*}
	\mathcal{O}(f(n)) = \{g: \mathbb{N} \to \mathbb{N} | \exists\, n_0 > 0 \wedge \exists\, c > 0 : \forall\, n \geq n_0 : |g(n)| \leq c \cdot |f(n)|\}
\end{equation*}
$\mathcal{O}(f(n))$ ist eine Menge von Funktionen, für die gilt: Es gibt zwei Konstanten $c$ und $n_0$, so dass für alle $n > n_0$ gilt, $g(n)$ und $f(n)$ unterscheiden sich im Wachstum um den Faktor $c$.
\vfill
\settowidth{\MyLenA}{$g(n) \in \mathcal{O}(n \cdot \log n)$~~}
\settowidth{\MyLenB}{Wurzelfunktion~~}
\begin{tabular}{
	@{}p{\the\MyLenA}%
	@{}p{\the\MyLenB}%
	@{}p{\linewidth - \the\MyLenA - \the\MyLenB}
}
	\multicolumn{1}{c}{\textbf{Notation}} & \multicolumn{1}{c}{\textbf{Bedeutung}} & \multicolumn{1}{c}{\textbf{Beispiel}}\\
	$g(n) \in \mathcal{O}(1)$ & beschränkt & Array-Zugriff, Zuweisung\\
	$g(n) \in \mathcal{O}(\log_x n)$ & logarithmisch & binäre Suche\\
	$g(n) \in \mathcal{O}(\sqrt[x]{n})$ & Wurzelfunktion & \enquote{naiver} Primzahlentest\\
	$g(n) \in \mathcal{O}(n)$ & linear & sequentielle Suche\\
	$g(n) \in \mathcal{O}(n \cdot \log n)$ & beschränkt & Mergesort\\
	$g(n) \in \mathcal{O}(n^x)$ & polynominell & Bubblesort\\
	$g(n) \in \mathcal{O}(x^n)$ & exponentiell & Erfüllbarkeit der Aussagenlogik\\
	$g(n) \in \mathcal{O}(n!)$ & faktoriell & Handlungsreisender\\
\end{tabular}
~\\
Algorithmen mit mindestens exponentiellen gelten als nicht praktikabel. Wenig praktikabel sind
Algorithmen der Klasse $\mathcal{O}(n^x)$ für $x > 3$. 

\paragraph{Rechenregeln}
\begin{align*}
	\mathcal{O}(\mathcal{O}(f(n))) &\in \mathcal{O}(f(n))&
	\mbox{Polynom vom Grad } x &\in \mathcal{O}(n^x)\\
	c + \mathcal{O}(f(n)) &\in \mathcal{O}(f(n))&
	\mathcal{O}(f(n)) + \mathcal{O}(g(n)) &\in \mathcal{O}(\mbox{max}(f(n), g(n)))\\
	c \cdot \mathcal{O}(f(n)) &\in \mathcal{O}(f(n))&
	\mathcal{O}(f(n)) \cdot \mathcal{O}(g(n)) &\in \mathcal{O}((f(n) \cdot g(n)))
\end{align*}

\paragraph{Beispiel Bestimmung O-Notation}
\settowidth{\MyLenA}{~~~~\texttt{// vier Zuweisung/Berechnungen}~~}
\begin{tabular}{
	@{}p{\the\MyLenA}
	@{}p{\linewidth - \the\MyLenA}%
}
\texttt{function(int a, int b)} & $\Rightarrow 2 \cdot c_0 \Rightarrow \underline{\mathcal{O}(1)}$\\
~~\texttt{int x, y, z} & $\Rightarrow 3 \cdot c_1 \Rightarrow \underline{\mathcal{O}(1)}$\\
~~\texttt{for x = a to 0 do} & $\Rightarrow a \cdot (4 \cdot c_2) \Rightarrow \underline{\mathcal{O}(n)}$\\
~~~~\texttt{// vier Zuweisung/Berechnungen} & $\Rightarrow 4 \cdot c_2 \Rightarrow \mathcal{O}(1)$\\
~~\texttt{end for} & \\
~~\texttt{for x = b to a} & 
$\Rightarrow (b - a) \cdot (a \cdot (1 \cdot c_3) + 2 \cdot c_4 + 4 \cdot c_5) \Rightarrow \underline{\mathcal{O}(n^2)}$\\
~~~~\texttt{for y = a to 0} & $\Rightarrow a \cdot (1 \cdot c_3)\Rightarrow \mathcal{O}(n)$\\
~~~~~~\texttt{z = z + 1} & $\Rightarrow 1 \cdot c_3 \Rightarrow \mathcal{O}(1)$\\
~~~~\texttt{end for} & \\
~~~~\texttt{// zwei Dateizugriffe} & $\Rightarrow 2 \cdot c_4 \Rightarrow \mathcal{O}(1)$\\
~~~~\texttt{// vier Ausgabebefehle} & $\Rightarrow 4 \cdot c_5 \Rightarrow \mathcal{O}(1)$\\
~~\texttt{end for} & \\
\texttt{end function} & \\
\end{tabular}
$\Rightarrow \mathcal{O}(1) + \mathcal{O}(1) + \mathcal{O}(n) + \mathcal{O}(n^2) \in \mathcal{O}(n^2)$

% \textit{Beispiel:} $g(n) = n^2 + n + a$\\
% $a$ ist eine Konstante und kann vernachlässigt werden:
% \begin{align*}
% 	n^2 + n &\leq 2 \cdot n^2 &(c = 2\mbox{ eingeführt, Klasse }n^2\\
% 	n^2 + n &\leq n^2 + n^2 &&| - n^2\\
% 	n &\leq n^2 &&| \div n^2\\
% 	1 &\leq n && \Rightarrow n_0 = 1, c = 2
% \end{align*}
% Für $n_0 = 1$ und $c = 2$ ist $g(n) \in \mathcal{O}(n^2)$

%Die Bestimmung der Komplexitätsklasse ergibt sich (Daumenregel) aus dem am schnellsten wachsenden Term der Funktion.

% \textit{Code-Beispiel:} Aufsummieren aller Zahlen bis $n$ 
% \lstset{numbers=none, language=ada, tabsize=2}
% \begin{center}
% \begin{tabular}{lcl}
% \multicolumn{1}{c}{\textbf{Schleife}} & & \multicolumn{1}{c}{\textbf{Direkt}} \\ 
% \begin{minipage}{0.4\linewidth}
% \begin{lstlisting}
% sum(n): 
% 	sum = 0; // a
% 	loop n do
% 		sum = sum + n // b
% 	end
% \end{lstlisting}	
% \end{minipage}
% &
% ~
% &
% \begin{minipage}{0.4\linewidth}
% \begin{lstlisting}
% sum(n): 
% 	sum = n * (n + 1)/2 // c
% \end{lstlisting}			
% \end{minipage}
% \\
% $g(n) = a + n \cdot b$
% &&
% $g(n) = c$\\
% $\Rightarrow g(n) \in \mathcal{O}(n)$ & & $\Rightarrow g(n) \in \mathcal{O}(1)$\\
% \end{tabular}
% \end{center}
\subsection{Problemklassen}
\begin{description}\itemsep0em
	\item [Einfaches Problem] 
	in polynominaler Zeit lösbar
	
	\item [Schwieriges Problem] 
	nur in mindestens exponentieller Zeit (d.\,h. nur für \enquote{kleine} Eingabewerte) lösbar
	
	Es gibt bei den schwierigen Problemen eine Unterklasse von Problemen, die mit einer nicht-deterministischen TM in polynominaler Zeit gelöst werden können (d.\,h. es ist schwer eine Lösung zu finden, aber einfach diese zu verifizieren)

	\item [P]
	Eine Sprache $L$ ist in der Klasse P enthalten, wenn $L = L(\mathcal{M})$ für eine deterministische TM $\mathcal{M}$ mit einer polynominalen Zeitkomplexität $T(n)$ gilt. D.\,h. $\mathcal{M}$ kann für $w \in \Sigma^*$ mit polynominalen Zeitaufwand prüfen, ob $w \in L$.

	\item [NP]
	Eine Sprache $L$ ist in der Klasse NP enthalten, wenn $L = L(\mathcal{M})$ für eine nichtdeterministische TM mit einer polynominalen Zeitkomplexität $T(n)$ gilt und für jede Eingabe der Länge $n$ es keine Folge mit mehr als $T(n)$ Bewegungen in $\mathcal{M}$ gibt. D.\,h. $\mathcal{M}$ \enquote{rät} die richtige Lösung und prüft diese mit polynominalen Zeitaufwand.

	Offensichtlich gilt P $\subset$ NP. Die offene Frage ist: 
	\begin{equation*}
	\mbox{P} \overset{?}{=} \mbox{NP}
	\end{equation*}
\end{description}
~\\\vfill
\begin{description}
	\item [NP-schwierig] 
	Eine Sprache $L$ heisst NP-schwierig, wenn für jede Sprache $L' \in$ NP eine Polynominalzeitreduktion von $L'$ auf $L$ existiert. D.\,h. $L$ ist mindestens so schwierig, wie alle Probleme in NP (die übrigens alle gleich schwierig sind)

	\item[NP-vollständig (NPC)]
	Eine Sprache $L$ heisst NP-vollständig (NP-complete) wenn $L \in $ NP und für jede Sprache $L' \in $ NP eine Polynominalzeitreduktion von $L'$ auf $L$ existiert. D.\,h. im Gegensatz zu NP-schwierig existiert immerhin eine Lösung in NP.

	\item[Polynominalzeitreduktion]
	Ein Algorithmus der ein Problem $P_1$ auf ein Problem $P_2$ reduziert darf höchstens polynominale Zeit verwenden. D.\,h. Der Algorithmus zum lösen von $P_2$ unterscheidet sich nur polynominal von dem von $P_1$. Darus folgt:
	\begin{equation*}
		P_1 \in \mbox{NP} \Rightarrow P_2 \in \mbox{NP}
	\end{equation*} 
\end{description}

\subsection{Beweistechnik $L \in$ NP}
Via (transitiver) Polynominalzeitreduktion reduziert man ein Problem soweit, bis man bei einem bekannten NP-vollständigen Problem angelangt ist.

\subsection{Bekannte NP-vollständige-Probleme}
\begin{description}\itemsep0em
	\item [SAT]
	Das \enquote{satisfiable} Problem (Erfüllbarkeitsproblem): Gibt es eine Belegung der Variablen eines aussagelogischen Ausdrucks, so dass der Ausdruck wahr wird. (Satz von Cook, 1971)
	\item [TST]
	Das Travelling Salesman Problem (kürzest mögliche Verbindung von verschiedenen Punkten)
	\item [\enquote{Färbungsproblem}] 
	Wieviele Farben sind mindestens notwendig, um eine Landkarte so einzufärben, dass benachbarte Länder immer unterschiedliche Farben haben?
	\item [\enquote{Rucksack}-Problem] Wie fülle ich einen Rucksack so, des Inhalts maximal, sein Gewicht aber minimal wird?
\end{description}

Lösungsalgorithmen sind für diese Probleme bekannt, doch sie erfordern exponentiellen Zeitaufwand. Es gelten: 
\begin{enumerate}\itemsep0em
	\item jedes NP-vollständige Problem ist höchstens so schwierig wie das SAT-Problem
	\item alle NP-vollständigen Probleme sind mindestens so schwierig wie das SAT-Problem
\end{enumerate}
%. Und ebenfalls, . Somit sind alle NP-vollständigen Probleme gleich schwierig. Kann also für nur ein einziges NP-vollständiges Problem gezeigt werden, dass es in polynominaler Zeit lösbar ist, ist %die Frage P $\overset{?}{=}$ NP mit \enquote{ja} zu beantworten.

\subsection{Lösungsstrategien für NP-vollständige Probleme}
\begin{itemize}\itemsep0em
	\item Näherungsverfahren/Heuristiken, so dass 
	\begin{itemize}\itemsep0em
		\item fast alle Lösungen optimal sind
		\item alle Lösungen fast optimal sind
	\end{itemize}
	\item Generierung von zufälligen Lösungen, bis man keine bessere mehr findet
	\item Optimierung des Basiswerts $x$, so dass $1 \leq x \leq 2$
	\item Exakte Algorithmen für kleine $n$
	\item \enquote{Lokale Suche} (neu, noch ohne Relevanz)
\end{itemize}

