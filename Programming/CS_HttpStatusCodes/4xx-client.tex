\section{4xx - Client Errors}
\begin{description}\itemsep0em
	\item [400 - Bad Request] The request cannot be fulfilled due to bad syntax.
	
	\item [401 - Unauthorized] Similar to 403 Forbidden, but specifically for use when authentication is required and has failed or has not yet been provided.[2] The response must include a WWW-Authenticate header field containing a challenge applicable to the requested resource. See Basic access authentication and Digest access authentication.

	\item [402 - Payment required] Reserved for future use. The original intention was that this code might be used as part of some form of digital cash or micropayment scheme, but that has not happened, and this code is not usually used. As an example of its use, however, Apple's MobileMe service generates a 402 error if the MobileMe account is delinquent.[citation needed] In addition, YouTube uses this status if a particular IP address has made excessive requests, and requires the person to enter a CAPTCHA.

	\item [403 - Forbidden] The request was a valid request, but the server is refusing to respond to it. Unlike a 401 Unauthorized response, authenticating will make no difference. On servers where authentication is required, this commonly means that the provided credentials were successfully authenticated but that the credentials still do not grant the client permission to access the resource (e.g. a recognized user attempting to access restricted content).

	\item [404 - Not found] The requested resource could not be found but may be available again in the future. Subsequent requests by the client are permissible.

	\item [405 - Method Not Allowed] A request was made of a resource using a request method not supported by that resource; for example, using GET on a form which requires data to be presented via POST, or using PUT on a read-only resource.

	\item [406 - Not Acceptable] The requested resource is only capable of generating content not acceptable according to the Accept headers sent in the request.

	\item [407 - Proxy Authentication Required] The client must first authenticate itself with the proxy.

	\item [408 - Request Timeout] The server timed out waiting for the request.[2] According to W3 HTTP specifications: \enquote{The client did not produce a request within the time that the server was prepared to wait. The client MAY repeat the request without modifications at any later time.}

	\item [409 - Conflict] Indicates that the request could not be processed because of conflict in the request, such as an edit conflict.

	\item [410 - Gone] Indicates that the resource requested is no longer available and will not be available again.[2] This should be used when a resource has been intentionally removed and the resource should be purged. Upon receiving a 410 status code, the client should not request the resource again in the future. Clients such as search engines should remove the resource from their indices. Most use cases do not require clients and search engines to purge the resource, and a "404 Not Found" may be used instead.

	\item [411 - Length Required] The request did not specify the length of its content, which is required by the requested resource.

	\item [412 - Precondition failed] The server does not meet one of the preconditions that the requester put on the request.

	\item [413 - Request Entity Too Large] The request is larger than the server is willing or able to process.

	\item [414 - Request-URI Too Long] The URI provided was too long for the server to process.

	\item [415 - Unsupported Media Type] The request entity has a media type which the server or resource does not support. For example, the client uploads an image as image/svg+xml, but the server requires that images use a different format.

	\item [416 - Request Range Not Satisfiable] The client has asked for a portion of the file, but the server cannot supply that portion.[2] For example, if the client asked for a part of the file that lies beyond the end of the file.

	\item [417 - Expectation Failed] The server cannot meet the requirements of the Expect request-header field.

	\item [422 - Unprocessable Entity (WebDAV)] The request was well-formed but was unable to be followed due to semantic errors.

	\item [423 - Locked (WebDAV)] The resource that is being accessed is locked.

	\item [424 - Failed Dependency (WebDAV)] The request failed due to failure of a previous request (e.g. a PROPPATCH).

	\item [426 - Upgrade Required] The client should switch to a different protocol such as TLS/1.0.

	\item [428 - Precondition Required] The origin server requires the request to be conditional. Intended to prevent \enquote{the \textit{lost update} problem, where a client GETs a resource's state, modifies it, and PUTs it back to the server, when meanwhile a third party has modified the state on the server, leading to a conflict.}
	
	\item [429 - Too Many Requests] The user has sent too many requests in a given amount of time. Intended for use with rate limiting schemes.

	\item [431 - Request Header Fields Too Large] The server is unwilling to process the request because either an individual header field, or all the header fields collectively, are too large.
\end{description}