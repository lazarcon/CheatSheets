\section{Konfigurationsmanagement}
URL: edu.panter.ch/Username: student; Passwort: hsz-t
\subsection{Build Circle}
\begin{description}
 \item [clean] räumt Dateien auf (löscht target-folder)
 \item [compile] Kompiliert das Projekt
 \item [test] Führt Unit-Tests aus
 \item [package] Erzeugt Artefakt (jar/war/ear)
 \item [deploy] Verschiebt Artefakt auf einen anderen Server
\end{description}

\subsection{Typen von Artefakten}
\begin{description}
 \item [jar] Java Archive (enthält META-INF Ordner mit Manifest-Datei)
 \item [war] Web Archive (kann mehrere jars enthalten, läuft in Servlet-Containern)
 \item [ear] Enterprise Archive (kann mehrere war-files enthalten)
\end{description}

\subsection{Alternativen IDEs}
\begin{itemize}
 \item Eclipse, IntelliJ (IDEs)
 \item Gradle (wie Maven, in Groovy)
 \item Ant
 \item make (C-Programme)
 \item rake (...)
\end{itemize}

