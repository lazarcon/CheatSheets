\section{GUI}
Es gibt verschiedene Gui-Implementationen (AWT, Swing, SWT).
%Die hier aufgeführten beziehen sich auf \enquote{Swing}.

Für die GUI-Komponeneten gilt allgemein: Komponenten werden der Elternkomponente mit \java{parent.getContentPane().add(newComponent)} hinzugefügt.

\subsection{Beispiel Fenster}
\begin{lstlisting}[language=Java]
public class Buttons {
	private JFrame frame;
	public void show() {
	frame = new JFrame("WindowTitle");
	frame.setDefaultCloseOperation(JFrame.EXIT_ON_CLOSE);
	final JButton button1 = new JButton("Button 1");
	final JButton button2 = new JButton("Button 2");
	final JLabel label = new JLabel();
	label.setSize(300, 50);
	button1.addActionListener(new ActionListener() {
		@Override
		public void actionPerformed(ActionEvent e) {
			label.setText("Button 1 wurde angeklickt");
		}
	});
	button2.addActionListener(new ActionListener() {
		@Override
		public void actionPerformed(ActionEvent e) {
			label.setText("Button 2 wurde angeklickt");
		}
	});
	Container pane = frame.getContentPane();
	pane.setLayout(new BorderLayout());
	pane.add(label, BorderLayout.CENTER);
	pane.add(button1, BorderLayout.NORTH);
	pane.add(button2, BorderLayout.SOUTH);
	frame.setSize(300, 100);
	frame.setVisible(true)
}
\end{lstlisting}

\subsection{Beispiel Menu}
\begin{lstlisting}[language=Java]
public void createMenuBar(final Frame frame) {

	// Menüzeile (JMenuBar) erzeugen und in das Fenster (JFrame) einfügen
	final JMenuBar bar = new JMenuBar();
	frame.setJMenuBar(bar);

	// Menü (JMenu) erzeugen und in die Menüzeile (JMenuBar) einfügen
	final JMenu dateiMenu = new JMenu("Datei");
	bar.add(dateiMenu);

	// Menüeinträge (JMenuItem) erzeugen und dem Menü (JMenu) "Datei" hinzufügen
	final JMenuItem oeffnenItem = new JMenuItem("Öffnen");
	oeffnenItem.addActionListener(new OpenActionListener());
	dateiMenu.add(oeffnenItem);

	final JMenuItem beendenItem = new JMenuItem("Beenden");
	beendenItem.addActionListener(new ExitActionListener());
	dateiMenu.add(beendenItem);
}
\end{lstlisting}
% \subsection{Fenster und Dialoge}
% Haben alle einen Titel (\java{String}) und einen Inhalt (\java{contentPane}).
% \settowidth{\MyLenA}{\java{JColorChooser}~}
% \begin{tabular}{@{}p{\the\MyLenA}%
% 				@{}p{(\linewidth - \the\MyLenA)}}
% 	\java{JFrame} & Hauptfenster, kann eine Menüleiste haben.\\
% 	\java{JDialog} & Hängt von einem anderen Fenster ab.\\
% 	\java{JColorChooser} & Dialog um Farben auszuwählen. Kann Teil von \java{JFrame} oder \java{JDialog} sein.\\
% 	\java{JFileChoose} & Dialog um Dateien auszuwählen.\\
% 	\java{JOptionPane} & Zuständig für Fehlermeldungen und Rückfragen.\\
% \end{tabular}
% \java{JOptionPane} bietet vier Darstellungen:
% \begin{itemize}\itemsep0em
% 	\item \java{showMessageDialog()}
% 	\item \java{showOptionDialog()}
% 	\item \java{showConfirmDialog()}
% 	\item \java{showInputDialog()}
% \end{itemize}
% 
% 
% \subsection{Menüs}
% 
% \settowidth{\MyLenA}{\java{JRadioButtonMenuItem}~}
% \begin{tabular}{@{}p{\the\MyLenA}%
% 				@{}p{(\linewidth - \the\MyLenA)}}
% 	\java{JMenuBar} & Menüleiste eine Fenster. Kann beliebig viele \java{JMenu} Objecte enthalten\\
% 	\java{JMenu}	& Container für \java{JMenuItem} oder weiter \java{JMenu} (Submenu)\\
% 	\java{JMenuItem} & Allgemeiner Menü-Eintrag. Kann \java{JMenu} enthalten\\
% 	\java{JCheckBoxMenuItem} & Menü-Eintrag mit einer Checkbox\\
% 	\java{JRadioButtonMenuItem} & Menü-Eintrag mit Radio-Buttons. Radio-Buttons werden durch eine \java{ButtonGroup} zusammengehalten.\\
% 	\java{JSeparator} & Trennstrich zwischen \java{JMenuItem}\\
% 	\java{JPopupMenu} & Kontext-Menü, aktiviert durch rechte Maustaste.\\
% \end{tabular}
% 
% \subsection{Container}
% Container strukturieren die eigentliche Bedienelemente.
% 
% \settowidth{\MyLenA}{\java{JInternalFrame}~}
% \begin{tabular}{@{}p{\the\MyLenA}%
% 				@{}p{(\linewidth - \the\MyLenA)}}
% 	\java{JPanel} & Standard-Container\\
% 	\java{JTabbedPane} & verwaltet Container über Registerkarten\\
% 	\java{JSplitPane} & Zweigeteilter, grössenvariabler Container\\
% 	\java{JScrollPane} & Ermöglich scrollen innerhalb eines Containers\\
% 	\java{JToolBar} & Werkzeugleiste\\
% 	\java{JDesktopPane} & Kann innere Fenster (\java{JInternalFrame}) enthalten\\
% 	\java{JInternalFrame} & Inneres Fenster (ähnlich \java{JFrame})\\
% 	\java{JLayeredPane} & Wie \java{JPanel}, nur mit unterschiedlichen Ebenen\\
% \end{tabular}
% 
\subsubsection{LayoutManager}
LayoutManager ermöglichen das Platzieren der Inhaltselemente:
\settowidth{\MyLenA}{\java{BorderLayout}~}
\begin{tabular}{@{}p{\the\MyLenA}%
				@{}p{(\linewidth - \the\MyLenA)}}
	\java{BorderLayout} & Besteht aus fünf Bereichen: North, South (ganze Breite), East, West und Center, die einzeln gefüllt werden können.(Default für \java{JFrame})\\
%	\java{BoxLayout} & Stappelt die Elemente, oder platziert sie in einer Reihe.\\
	\java{FlowLayout} & Platziert alle Elemente nebeneinander und bricht um, wenn der
	Platz nicht ausreicht (Default für \java{JPanel})\\
%	\java{BoxLayout} & Unterteilt das Fenster in eine Tabelle, deren Zellen mit
%	Elementen gefüllt werden können\\
\end{tabular}
% 
% \subsection{Inhaltselemente}
% Können ineinander verschachtelt werden und erben von \java{JComponent}.
% \settowidth{\MyLenA}{\java{JFormattedTextField}~}
% \begin{tabular}{@{}p{\the\MyLenA}%
% 				@{}p{(\linewidth - \the\MyLenA)}}
% \java{JLabel} & 	 statischer Text, nicht editierbar\\
% \java{JButton} &	 Schaltfläche (Button)\\
% \java{JToggleButton} &	Button mit zwei Zuständen (an/aus)\\
% \java{JCheckBox} &	 Auswahlkästchen\\
% \java{JRadioButton} & Auswahl einer von mehreren Optionen. Sind via \java{ButtonGroup} verknüpft\\
% \java{ButtonGroup} & Gruppierung von \java{JRadioButton}\\
% \java{JComboBox} &	Dropdown-Liste (auch als Auswahlliste oder Listbox bezeichnet)\\
% \java{JList}	 & Liste, die mehrere Elemente enthalten kann. Mehrfachauswahl möglich\\
% \java{JTextField} &	Einzeilige Texteingabe\\
% \java{JTextArea}  & Mehrzeilige Texteingabe\\
% \java{JScrollBar}	&  Schieberregler zum Scrollen\\
% \java{JSlider}	 	& Skalierbarer Schieberregler\\
% \java{JProgressBar} & Fortschrittsbalken\\
% \java{JFormattedTextField} & Textfeld, für das eine Formatierung (z.\,B. Beträge) festlegbar ist\\
% \java{JPasswordField} & Textfeld zur Passworteingabe.\\
% \java{JSpinner}		&	Ähnlich der JComboBox, enthält Liste, die mit Pfeiltasten gesteuert wird.\\
% \java{JSeperator} &	Trennlinie\\
% \java{JEditorPane}	& mehrzeiliges Textfeld zur Bearbeitung von formatiertem Text, untersützt HTML- und RTF\\
% \java{JTextPane} &	Zur Bearbeitung und Anzeige grafisch aufbereiteter Texte.\\
% \java{JTree}	& Baumstruktur.\\
% \java{JTable}	& Tabelle, kann auch Texteingaben entgegen nehmen.\\
% \end{tabular}


% \subsection{Events}
% Durch Dritte (Benutzer, System) ausgelöste Ereignisse
% \settowidth{\MyLenA}{\java{WindowEvent} $\rightarrow$ \java{WindowListener}~}
% \begin{tabular}{@{}p{\the\MyLenA}%
% 				@{}p{(\linewidth - \the\MyLenA)}}
% 	\java{ActionEvent} $\rightarrow$ \java{ActionListener} & Ausgelöst durch GUI-Komponenten\\
% 	\java{FocusEvent} $\rightarrow$ & wenn eine Komponente den Fokus bekommt oder verliert\\
% 	\java{MouseEvent} $\rightarrow$ \java{MouseListener} & wenn die Maus ein Komponente berührt, bewegt oder gedrückt wird.\\
% 	\java{WindowEvent} $\rightarrow$ \java{WindowListener} & wenn sich der Zustand eines Fensters ändert\\
% 	\java{TextEvent} & wenn sich der Text einer Komponente ändert\\
% \end{tabular}

\subsubsection{Implentieren von Listenern}
\begin{itemize}\itemsep0em
	\item Das entsprechende Element (z.\,B. \java{JPanel}) den entsprechenden Listener implementieren lassen.
	\item Durch eine anonyme innere Klasse
\end{itemize}
