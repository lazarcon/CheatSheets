\section{Datentypen}
\begin{itemize}\itemsep0em
	\item einfacher, elementarer, primitiver Datentypen
	\begin{itemize}
		\item boolean: Boolean (1 Bit, [\texttt{true, false}])
		\item numerisch:
		\begin{itemize}
			\item ganzzahlige
			\begin{itemize}
				\item byte: Byte (8 Bit $[-2^7, 2^7-1]$)
				\item char: Character (16 Bit, $65\,536$ Unicode-Zeichen)
				\item short: Short (16 Bit $[-2^{15}, 2^{15}-1]$)
				\item int: Integer (32 Bit $[-2^{31}, 2^{31}-1]$)
				\item long: Long (64 Bit $[-2^{63}]m 2^{63}-1]$)
			\end{itemize}
			\item kommastellige
			\begin{itemize}
				\item float: Float (32 Bit $[\pm\ 3.4 \cdot 10^{28}]$)
				\item double: Double (64 Bit $[\pm\ 1.7 \cdot 10^{308}]$)
			\end{itemize}
		\end{itemize}
	\end{itemize}
	\item Referenztypen
	\begin{itemize}
		\item Arrays: geordnete Gruppe gleicher Datentypen
		\item Klassen: Datentyp, der sich aus anderen Datentypen und Methoden zusammensetzt
		\item Interfaces: Datentyp, der Anforderungen definiert
	\end{itemize}
\end{itemize}
Primitive Zahl-Datentypen können \texttt{unsigned} sein. Dann nehmen sie nur positive Werte an. Die obere Grenze ist dann doppelt so hoch.\\
\texttt{void} ist kein Datentyp, sondern ein Kennzeichen für Methoden ohne Rückgabewert.

\section{Operatoren}
\begin{description}\itemsep0em
	\item [Unäre, monadische] haben einen Operanden: \texttt{index++}
	\item [Binäre, dyadische] haben zwei Operanden: \texttt{row * col}
	\item [Ternäre, tryadische] haben drei Operanden: \texttt{finished ? 42 : 71}
\end{description}

\subsection{Auswertungsreihenfolge}
\begin{enumerate}
	\item Klammerausdrücke
	\item Unäre Operatoren von links nach rechts
	\item Binäre Operatoren von links nach rechts
\end{enumerate}

\subsection{Inkrement/Dekrement Operatoren ($++/--$)}
\begin{description}\itemsep0em
	\item [links] erhöhen, dann rechnen (\texttt{(a = 1: b = ++a //b = 2, a = 2})
	\item [rechts] rechnen, dann erhöhen (\texttt{(a = 1: b = a++ //b = 1, a = 2)})
\end{description}

\section{Arithmetische Operatoren}
\settowidth{\MyLenA}{[\texttt{>>>}]~}
\begin{tabular}{@{}p{\the\MyLenA}%
				@{}p{(\linewidth - \the\MyLenA)}}
$+$ & Addition: $1 + 2 = 3$ \\
$-$ & Subtraktion: $3 - 2 = 1$\\
$*$ & Multiplikation: $2 * 3 = 6$\\
$/$ & Division: $6 / 3 = 2$\\
$\%$ & Modulo: $3 \% 2 = 1$\\
\end{tabular}

\subsection{Bit-Operationen}
\settowidth{\MyLenA}{[\texttt{>>>}]~~~}
\begin{tabular}{@{}p{\the\MyLenA}%
				@{}p{(\linewidth - \the\MyLenA)}}
$\sim$ & Negation: $\sim0111 \to 1000$\\
$\&$ & Bitweises \enquote{und}: $0101 \& 0011 = 0001$\\
$|$ & Bitweises \enquote{oder}: $0101 | 0011 = 0111$\\
$\hat\ $ & Bitweises \enquote{entweder/oder}, XOR: $0010 \hat\ 1010 = 1000$\\
$<<$ & Bitweise Linksverschiebung (*2): $01001111 <<  1 = 10011110$\\
$>>$ & Bitweise Rechtsverschiebung (/2): $01001111 >>  1 = 00100111$\\
$>>>$ & Vorzeichenlose Rechtsvers: $11110000 >>> 2 = 00111100$\\
\end{tabular}

\subsection{Weitere Operatoren}
\settowidth{\MyLenA}{[Bedingungsoperator]~~}
\begin{tabular}{@{}p{\the\MyLenA}%
				@{}p{(\linewidth - \the\MyLenA)}}
Bedingungsoperator & Ist $A$ wahr, dann $B$ sonst $C$: \texttt{A ? B : C}\\
\texttt{[x]} & Zugriff auf das $x$-te Element eines Arrays\\
\texttt{new} & Erzeugt ein neues Objekt\\
casting & \texttt{(MyClass) object} wandelt \texttt{object} in den Typ \texttt{MyClass} um. Kann eine \texttt{ClassCastException} werfen\\
. & Zugriff auf eine Methode: \texttt{myObject.doSomething()}\\
\texttt{instanceof} & Prüft, ob ein Objekt zu einem Datentyp gehört: \texttt{myObj instanceof String}\\
\end{tabular}

\section{Kontrollstrukturen}
\subsection{Verzweigungen}
\begin{description}\itemsep0em
	\item [if \dots else] Wenn \dots, dann \dots Anweisung
	\begin{lstlisting}[language=Java]
if (value == expected) {
	// do this
} else if (value > expected) {
	// do that
} else {
	// do something completly different
} 
	\end{lstlisting}
	\texttt{else} und \texttt{else if} sind optional
	\item [switch] Fall Unterscheidung, funktioniert für Integer (=Enums) und seit Java 7 auch für Strings:
	\begin{lstlisting}
switch (value) {
	case "a":
		// do this
		break;
	case "b":
	default:
		// same as for b;
}
	\end{lstlisting}
Trifft ein Fall zu, werden alle weitere Anweisungen ohne weitere \texttt{case}-Prüfungen durchgeführt, bis ein \texttt{break} oder \texttt{return} zum Verlassen der Struktur auffordert.
\end{description}

\subsection{Schleifen}
\begin{description}

\item [while] Wird solange wiederholt, wie die Bedingung wahr ist:
\begin{lstlisting}[language=Java]
while (condition) {
	// do something
}
\end{lstlisting}
Wird eventuell gar nicht ausgeführt.


\item [do \dots while] Wird solange wiederholt, wie die Bedingung wahr ist:
\begin{lstlisting}[language=Java]
do {
	// do something
} while (condition)
\end{lstlisting}
wird mindestens ein mal ausgeführt

\item [for (\dots; \dots; \dots)] Vorbestimmte Anzahl Wiederholungen:
\begin{lstlisting}[language=Java]
for(intital = x; initial < end; ++initial) {
	// do something
}
\end{lstlisting}

\item [foreach] Wird mit jedem Element einer Menge durchgeführt:
\begin{lstlisting}[language=Java]
for(MyItem item : myCollection) {
	// do something
}
\end{lstlisting}
\end{description}

\subsubsection{break/continue}
Schleifendurchläufe können mit \texttt{break} abgebrochen werden.
Mit Hilfe von \texttt{continue} \textit{myLabel} kann zu einer Stelle im Code gesprungen werden.
Ist kein \textit{myLabel} definiert, wird zum Anfang der Schleife gesprungen.
\begin{lstlisting}[language=Java]
int index = 0;
label:
while (index == 0) {
	for(index = 1; index < 10; ++index) {
		// do something
		continue label;
	}
}
\end{lstlisting}

\section{Arrays}
\begin{lstlisting}[language=Java]
int[] wert1; // 1-dimensionales Array von Integern
int wert2[]; // 1-dimensionales Array von Integern
wert1 = new int[10]; // Erzeugt ein mit 0 gefülltes Array aus 10 Integern
wert2 = {1, 2, 3}; // Füllt die Werte 1, 2 und 3 in das Array
int[][] feld = new int[8][8]; // Erzeugt 2-dimensionales Array
int feld[][] = {{1, 2, 3}, {4, 5, 6}}; // Erzeugt ein 2x3-Array 
\end{lstlisting}

\section{Klasse}
\begin{description}\itemsep0em
	\item[Klassendeclaration] \texttt{(public|protected|private|) (abstract) class} \textit{MyClass} \texttt{(<T (extends R)>) (extends} \textit{Parent}\texttt{(<T (extends R)>)) (implements} \textit{Interface1, \dots} \texttt{)}
	\item[Datenfelder, Attribute] \texttt{(public|protected|private|)(final)}\\\texttt{(static|transient|synchronized) AClass }\texttt{aClass}\texttt{;}
	\item[Methoden] \texttt{(public|protected|private|)(final)(<T>)} \textit{ReturnType doSomething}\texttt{((final)}\textit{ ParameterType parameter, \dots}\texttt{)}\texttt{ (throws }\textit{Exception1, \dots}\texttt{)}
	\item[Innere Klassen] wie Klasse an sich
\end{description}
ReturnType \texttt{void} steht für keine Rückgabe.

\subsection{Modifikatoren}
\subsubsection{Sichtbarkeit/Zugriff}
\begin{description}\itemsep0em
	\item [public] für alle sichtbar
	\item [protected] für die Klasse und ihre Kinder sichtbar
	\item [private] nur für die Klasse sichtbar
	\item [ohne Bezeichner] Nur innerhalb des Java-Paketes (\texttt{ch.zhaw.example}) sichtbar
\end{description}

\subsubsection{Besonderes}
\begin{description}\itemsep0em
	\item[final] 
	bedeutet bei primitiven Datentypen, dass sie so bleiben (Konstante), bei Nicht-Primitiven, dass die Referenz erhalten bleibt, der Inhalt sich aber ändern kann.\\
	\item[static] 
	Ist nicht von der Klasse abhängig und für alle abgeleiteten Klassen gleich.\\

	\item [transient] 
	wird bei der Serialisierung nicht gespeichert
	
	\item [synchronized] 
	kann nur von einem Thread manipuliert werden
\end{description}

\subsection{Methoden}
\begin{itemize}\itemsep0em
	\item primitiven Datentypen werden als Wert übergeben (Call-by-value)
	\item nicht-primitive Datentypen werden als Referenz übergeben (Call-by-reference)
\end{itemize}

\subsubsection{Überladen}
Methoden müssen sich lediglich in ihrer Signatur, d.\,h. der kompletten Deklaration unterscheiden. Erlaubt ist also das Überladen:
\begin{lstlisting}[language=Java]
public int quadrat(final int x) {
	return x * x;
}
public int quadrat(final float x) {
	return x * x;
}
\end{lstlisting}
(Unterscheidung bei den Parametern). Aber für generische Datentypen (\texttt{List<T>}) funktioniert das nicht:
\begin{lstlisting}[language=Java]
public void manipulate(final List<String> strings) {
	// do something with strings
}
public void manipulate(final List<Integer> integers) {
	// do something with integers
}
\end{lstlisting}
funktioniert nicht. (Der Datentyp ist in beiden Fällen \texttt{List})

\subsubsection{Polymorphie}
Erbende Klassen können Methoden ihrer Elternklasse überschreiben.

\section{Strings (Zeichenketten)}
\begin{itemize}\itemsep0em
	\item Sind eine Folge von Einzelzeichen: \texttt{"Hallo" = 'h', 'a', 'l', 'l', 'o'}.
	\item Explizites Erzeugen: \texttt{String string = new String("Hallo");}
	\item Implizites Erzeugen: \texttt{String string = "Hallo";}
	\item Strings sind Konstant und können im Grunde genommen nicht verändert werden.
	\item Wird eine existierende String-Variable manipuliert, wird dabei ein neuer String erzeugt.
	\item Vergleiche über \texttt{myString.equals(anotherString)}
\end{itemize}

\subsection{String.format}
Mit Hilfe von \texttt{String.format(String format, Object\dots args} können komplizierte Strings zusammengesetzt werden.
\begin{description}
	\item [\%n] ergibt das/die Zeichen für den Zeilenvorschub (Betriebssystem-abhängig)
	\item [\%\%] ergibt ein Prozentzeichen (\textbackslash\textbackslash) ergibt einen \textbackslash
	\item [\%s] ergibt einen String, \texttt{null}$\mapsto$\texttt{null}
	\item [\%b (\%B)] ergibt einen Boolean mit \texttt{true|false}, \%B ergibt \texttt{TRUE|FALSE}. \texttt{NULL} ergibt immer \texttt{false}, alles anderen nicht-Booleans ergeben \texttt{true}
	\item [\%c] erbibt ein Zeichen (aus Character, Byte oder Short), \%C schreibt das Zeichen in Grossbuchstaben
	\item [\%d, \%x (\%X), \%o] ergibt einen Dezimalwert (\%d), einen Hexadezimalwert (\%x, \%X) und einen Oktalwert (\%o) für die Typen Byte, Short, Integer, Long und BigInteger
	\item [\%f, \%e (\%E), \%g (\%G), \%a (\%A)] wie vorher, erlaubt aber zusätzlich die Typen Float und Double
	\item [\%t, \%T] formatiert ein Datum der Typen Long, Calendar und Date. \%t benötigt ein Suffix.
	\item [\%h, \%H] gibt den Hash-Code aus. \texttt{null} $\mapsto$ \texttt{null}
\end{description}
Beispiele:
\begin{lstlisting}[language=Java]
int i = 123; 
String.format("|%d|   |%d|%n", i, -i); // |123|   |-123| 
String.format("|%5d| |%5d|%n" , i, -i); // |  123| | -123| 
String.format("|%-5d| |%-5d|%n" , i, -i); // |123  | |-123 | 
String.format("|%+-5d| |%+-5d|%n" , i, -i); // |+123 | |-123 | 
String.format("|%05d| |%05d|%n%n", i, -i); // |00123| |-0123| 

String.format("|%X| |%x|%n", 0xabc, 0xabc ); // |ABC| |abc| 
String.format("|%08x| |%#x|%n%n", 0xabc, 0xabc ); // |00000abc| |0xabc| 

double d = 12345.678; 
String.format("|%f| |%f|%n", d, -d); // |12345,678000| |-12345,678000| 
String.format("|%.2f| |%.2f|%n", d, -d); // |12345,68| |-12345,68| 
String.format("|%,10f| |%,10f|%n", d, -d ); // |1234,567800| |-1234,567800| 
String.format("|%10.2f| |%10.2f|%n", d, -d); // |  12345,68| | –12345,68| 
String.format("|%010.2f| |%010.2f|%n", d, -d); // |0012345,68| |-012345,68| 

String s = "Monsterbacke"; 
String.format("%n|%s|%n", s); // |Monsterbacke| 
String.format("|%20s|%n", s); // |        Monsterbacke| 
String.format("|%-20s|%n", s); // |Monsterbacke        | 
String.format("|%7s|%n", s); // |Monsterbacke| 
String.format("|%.7s|%n", s ); // |Monster| 
String.format("|%20.7s|%n", s); // |             Monster| 

Date t = new Date(); 
String.format("%tT%n", t); // 11:01:39 
String.format("%tD%n", t); // 05/23/06 
String.format("%1$te. %1$tb%n", t); // 23. Mai 
\end{lstlisting}

\subsection{StringBuffer}
Dient dem effizienten Zusammensetzen von Strings.
Intitialisierung:
\begin{itemize}\itemsep0em
	\item \texttt{StringBuffer buffer = new StringBuffer();}: Leerer StringBuffer
	\item \texttt{StringBuffer buffer = new StringBuffer(10);}: Leerer StringBuffer der Länge 10
	\item \texttt{StringBuffer buffer = new StringBuffer("Hallo");}: StringBuffer mit dem Startwort \enquote{Hallo}
\end{itemize}

Verwendungsbeispiel:
\begin{lstlisting}[language=Java]
public String toString(final List<Item> list) {
	final StringBuffer buffer = new StringBuffer();
	for(final Item item : list) {
		buffer.append(item.toString());
		buffer.append(',');
	}
	// Delete last comma
	buffer.deleteCharAt(buffer.lastIndexOf(','));
	return buffer.toString();
}
\end{lstlisting}

\section{Nachrichten}
In der OOP bezeichnet \enquote{Nachricht senden} den Aufruf einer Methode eines Objektes durch ein anderes Objekt.

\begin{lstlisting}[language=Java]
myObject.doSomething(); // Aktuelle Klasse sendet myObject die Nachricht doSomething
\end{lstlisting}

\begin{itemize}\itemsep0em
	\item Alle Objekte Nachrichten zusenden, egal von welcher Klasse
	\item Objekte können sich selbst Nachrichten zusenden (\texttt{this.})
	\item Der Sender wartet bis der Empfänger die Nachricht verarbeitet hat.
\end{itemize}

\section{Vererbung}
\begin{itemize}\itemsep0em
	\item eine Klasse übernimmt durch Vererbung (alle) Eigenschaften und das Verhalten einer Klasse
	\item die vererbende Klasse heisst Basisklasse (Superklasse)
	\item die verbende Klasse ist der allgemeinere Fall (generalisiert)
	\item die erbende Klasse heisst Subklasse (Ableitung)
	\item die erbende Klasse ist spezialisiert
	\item eine erbende Klasse ist auch ein Objekt vom Typ Superklasse
	\item eine mit \texttt{final} deklarierte Klasse kann keine Kinder haben (Bsp. \texttt{java.lang.String})
	\item Falls K nicht eine Art von E ist, soll K auch nicht von E erben! (Grady Booch)
	\item Jede Klasse erbt mindestens von der Klasse \texttt{Object} das folgende:
	\begin{itemize}
		\item \texttt{clone()} gibt eine Referenz auf eine Kopie von sich selbst zurück
		\item \texttt{equals(Object object)} Prüft zwei Objekte auf Gleichheit (i.\,d.\,R. auf gleiche Speicherplatz-Referenz)
		\item \texttt{hashCode()} HashCode des Objekts
		\item \texttt{finalize()} wird aufgerufen wenn das Objekt aufgelöst wird (Garbage Collector)
		\item \texttt{getClass()} Meta-Informationen zur Klasse des Objekts
		\item \texttt{toString()} Beschreibung des Objektes als String (i.\,d.\,R. der HashCode)
	\end{itemize}
\end{itemize}

\section{Konstruktoren}
\begin{itemize}\itemsep0em
	\item führen die zum Erzeugen des Objektes notwendigen Arbeiten durch
	\item heissen gleich wie die Klasse
	\item haben keinen Rückgabewert
	\item werden nicht vererbt
	\item können weitere eigene (\texttt{this()}) oder Eltern-Konstruktoren (\texttt{super()})aufrufen. Ein solcher Aufruf muss aber die erste Anweisung im Konstruktor sein
	\item können benutzt werden um die Instantiierung eines Objektes zu verhindern.
	Dazu wird der Konstruktor als \texttt{private} deklariert
	\item hat eine Kind-Klasse keinen Konstruktor wird automatisch der Default-Konstruktor der Elternklasse ausgeführt
\end{itemize}

\subsection{Instantiierung}
\begin{enumerate}\itemsep0em
	\item einer Referenzvariablen wird der Wert \texttt{null} zugewiesen
	\item mit \texttt{new} wird eine Objekt der Klasse im Heap-Speicher angelegt
	\item der Konstruktor wird aufgerufen
	\item der \texttt{new} Operator gibt eine Referenz auf das erzeugt Objekt zurück
\end{enumerate}
