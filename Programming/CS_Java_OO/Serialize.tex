\subsection{Serialisierung/Deserialisierung}
Dient dem lesen und schreiben von Objekten.
Um ein Objekt in eine Datei zu muss es das Interface \java{Serializable} implementieren.\\
Beispiel Code Serializerung/Deserialisierung:
\begin{lstlisting}[language=Java]
public void serialize(final Serializable object, final String filename) {
	FileOutputStream fs = null;
	ObjectOutputStream os = null;
	try {
		fs = new FileOutputStream(new File(filename));
		os = new ObjectOutputStream(fs);
		os.writeObject(object);
		os.close();
		fs.close();
	} catch (final IOException e) {
		e.printStackTrace();
	} 
}

public Serializable load(final String filename) {
	FileInputStream fi = null;
	ObjectInputStream oi = null;
	Serializable object = null;
	try {
		fi = new FileInputStream(filename);
		oi = new ObjectInputStream(fi);
		object = oi.readObject();
		oi.close();
		fi.close();
	} catch (final IOException e) {
		e.printStackTrace();
	}
	return (object == null) ? new Object() : object;
}
\end{lstlisting}
Das serialisierte Objekt ist in den entsprechenden Typ zu casten.
Grundsätzlich wird der gesamte Objektgraph gespeichert. Instanzvariablen werden
dabei transient gespeichert. Falls die Superklasse nicht serialisierbar ist,
wird einfach deren Default-Konstruktor ausgeführt.

\subsubsection{Datei-Operationen}
Datein lasse sich mit Hilfe von \java{File} und \java{FileWriter} schreiben.
\begin{lstlisting}[language=Java]
public void write(final String content, final String path) {
	BufferedWriter buffer = null;
	try {
		buffer = new BufferedWriter(new FileWriter(new File(path)));
		buffer.write(content);
	} catch (final IOException e) {
		e.printStackTrace();
	} finally {
		// Close the buffer
		try {
			if (buffer != null) {
				buffer.close();
			}
		} catch (final IOException e) {
			e.printStackTrace();
		}
	}
}
\end{lstlisting}

Das Lesen funktioniert über \java{File} und \java{FileReader}:
\begin{lstlisting}
public String read(final String path) {
	BufferedReader reader = null;
	try {
		final File file = new File(path);
		reader = new BufferedReader(new FileReader(file));
		final StringBuffer stringBuffer = new StringBuffer();
		String line = reader.readLine();
		while (line != null) {
			stringBuffer.append(line);
			line = reader.readLine();
		}
	} catch (final FileNotFoundException e) {
		e.printStackTrace();
	} catch (final IOException e) {
		e.printStackTrace();
	} finally {
		// Close the buffer
		try {
			if (reader != null) {
				reader.close();
			}
		} catch (final IOException e) {
			e.printStackTrace();
		}
	}
	return buffer.toString();
}	
\end{lstlisting}

\java{File} repräsentiert die Datei auf einem Datenträger. Nicht den Inhalt der Datei.
Bietet Methoden wie \java{mkdir, isDirectory, delete,} \dots.


