\section{Collections}
\settowidth{\MyLenA}{Interface~}
\settowidth{\MyLenB}{IdentityHashMap,~}
\begin{tabular}{@{}p{\the\MyLenA}%
				@{}p{\the\MyLenA}
				@{}p{\linewidth - \the\MyLenA - \the\MyLenB}
}
\multicolumn{1}{c}{\texttt{Interface}} & \multicolumn{1}{c}{\texttt{Interface}} & \multicolumn{1}{c}{\texttt{Beschreibung}}\\ 
List & ArrayList, LinkedList, Vector, Stack & Sammlung gleichartiger Objekte\\
Set & HashSet, LinkedHashSet & Menge von Objekten (d.\,h. keine Doppelten)\\
Map & Attributes, HashMap, Hashtable, IdentityHashMap, RenderingHints, WeakHashMap & Key-Value-Speicher\\
SortedSet & TreeSet & Geordnete Menge von Objekten\\
SortedMap & TreeMap & Geordnete Key-Value-Paare\\ 
\end{tabular}
\begin{itemize}\itemsep0em
	\item Seit Java 6 sind die Collections typsicher und werden als \java{List<T>} deklariert
	\item Vector ist im Gegensatz zu ArrayList synchronisiert
\end{itemize}

\section{Generics}
Verwendungsbeispiele
\begin{lstlisting}[language=Java]
public class TierWaage {
	public static <T extends Tier> void printGewicht(T t) {
		System.out.println(t.getGewicht());
	}
}
public class TierComparator implements Comparator<Tier> {
	@Override
	public int compare(Tier o1, Tier o2) {
		return o1.getGewicht() - o2.getGewicht();
	}
}
public class Regal<T extends Ware> {
	final List<T> waren = new ArrayList<T>();
	
	public void add(T ware) {
		waren.add(ware);
	}

	public List<T> getWaren() {
		return waren;
	}
}
\end{lstlisting}