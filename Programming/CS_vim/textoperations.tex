\section{Text Operations}
\subsection{Inserting}
\settowidth{\MyLen}{\texttt{q!}~~}
\begin{tabular}{@{}p{\the\MyLen}%
		  @{}p{\linewidth-\the\MyLen}}
\verb!i! 	& Insert before cursor\\
\verb!I!	& Insert to the start of the current line\\
\verb!a!	& Append after cursor\\
\verb!A!	& Append to the end of the current line\\
\verb!o!	& Open a new line below and insert\\
\verb!O!	& Open a new line above and insert\\
\verb!C!	& Change the rest of the current line\\
\verb!r!	& Overwrite one character. After overwriting the single character, go back to command mode\\
\verb!R!	& Enter insert mode but replace characters rather than inserting\\
\verb!The ESC key! &	Exit insert/overwrite mode and go back to command mode\\
\end{tabular}

\subsection{Deleting}
\settowidth{\MyLen}{\texttt{dd}~~}
\begin{tabular}{@{}p{\the\MyLen}%
		  @{}p{\linewidth-\the\MyLen}}
\verb!x! & Delete characters \textit{under} the cursor\\
\verb!X! & Delete characters \textit{before} the cursor\\
\verb!dd! & Delete the current line
\end{tabular}

\subsection{Highlighting}
\settowidth{\MyLen}{\texttt{d}~~}
\begin{tabular}{@{}p{\the\MyLen}%
		  @{}p{\linewidth-\the\MyLen}}
\verb!v!	& Start highlighting characters. Use movement keys and commands to select text for highlighting\\
\verb!V!	& Start highlighting lines\\
\end{tabular}

\subsection{Editing blocks of text}
Note: the Vim commands marked with (V) work in visual mode, when you've selected some text. The other commands work in the command mode, when you haven't selected any text.
(Visual Mode)
\begin{tabular}{@{}p{\the\MyLen}%
		  @{}p{\linewidth-\the\MyLen}}
\verb!>! & Shift right (indent)\\
\verb!<! & Shift left (de-indent)\\
\verb!c! & Change the highlighted text\\
\verb!y! & (Yank) Copy the selected text to clipboard\\
\verb!d! & Cut the selected text to clipboard
\end{tabular}

(Non-visual Mode)
\settowidth{\MyLen}{\texttt{yy or Y}~~}
\begin{tabular}{@{}p{\the\MyLen}%
		  @{}p{\linewidth-\the\MyLen}}
\verb!~! & Change the case of characters. This works both in visual and command mode.\\
\verb!yy! or \verb!Y! & Yank the current line\\
\verb!dd! & Delete the current line\\
\verb!p!  & Put the text you yanked or deleted.\\
\verb!P!  & Put characters before the cursor. Put lines above the current line.\\
\end{tabular}