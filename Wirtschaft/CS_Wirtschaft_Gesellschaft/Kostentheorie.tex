\section{Kosten- und Gewinntheorie}
\begin{description}\itemsep0em
	\item [Fixkosten] Unabhängig von der produzierten Gütermenge
	\item [Variable Kosten] Direkt von der produzierten Gütermenge abhängig
\end{description}

\begin{itemize}\itemsep0em
	\item Im Bereich steigender Grenzerträge wird die Totalkostenkurve flacher $\rightarrow$ Stückkosten sinken
	\item Am Wendepunkt der Totalkostenkurve steigen die Grenzkosten
	\item Werden die Grenzkosten grösser als die Durchschnittskosten, steigen die Durchschnittskosten
	\item Liegt der Preis beim Minimum der Durchschnittskosten entsteht weder Gewinn noch Verlust (Gewinnschwelle)
	\item Betriebsminimum bezeichnet den Preis, der dem Minimum der variablen Kosten entspricht 
\end{itemize}

