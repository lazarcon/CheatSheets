\subsection{Angebot}
Je höher der Preis, desto grösser das Angebot
\begin{center}
\begin{tikzpicture}
    %\draw[very thin,color=gray] (-0.1,-1.1) grid (3.9,3.9);
    \draw[->] (0,0) -- (5,0) node[below] {Menge};
    \draw[->] (0,0) -- (0,4.2) node[left] {Preis};
    \draw[domain=0.1:1.6] plot[id=xl] function{1 - 1/(0.4 * x - 1)};
    \draw [thick, domain=0.1:2.97] plot[id=x] function{0.5 - 1/(0.4 * x - 1.5)}  node[above] {Angebot};
    \draw [domain=0.1:4.33] plot[id=xr] function{- 1/(0.4 * x - 2)};
	% lines
	\draw [dashed] (0,2.5)  node [left] {$P$} -- (4, 2.5);
	\draw [dashed](0.83,0) node [below] {$M_l$} -- (0.83,2.5);
	\draw [dashed](2.5,0) node [below] {$M$} -- (2.5, 2.5);
	\draw [dashed](4, 0) node [below] {$M_r$}-- (4, 2.5);
\end{tikzpicture}
\end{center}

 \begin{tabular}{@{}p{\linewidth/2}%
				@{}p{\linewidth/2}}
	\multicolumn{1}{c}{\underline{Linksverschiebung}} & \multicolumn{1}{c}{\underline{Rechtsverschiebung}}\\
	Steigende Kosten & Sinkende Kosten \\
	Negative externe Grössen & Positive externe Grössen \\
	Erwartete Preiserhöhungen & Erwartete Preissenkung \\
\end{tabular}

\subsection{Ertragsgesetz}
\textbf{Grenzertrag} Zuwachs des Ertrags (oder des Nutzens), der durch den Einsatz 
einer jeweils weiteren Einheit eines Produktionsfaktors erzielt wird.

Wird der Einsatz eines Produktionsfaktors bei Konstanz der Menge der übrigen Faktoren erhöht,
so nimmt der Output (Ertrag) zunächst mit steigenden, dann mit fallenden Grenzerträgen zu, bis
schliesslich der Output sinkt, der Grenzertrag also negativ wird.




