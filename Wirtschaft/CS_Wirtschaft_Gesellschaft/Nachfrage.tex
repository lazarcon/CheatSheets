\section{Angebot und Nachfrage}
Angebot und Nachfrage lassen sich in Kurven ausdrücken. Dabei gilt:
\begin{itemize}\itemsep0em
	\item Preisänderung: Auf der Kurve
	\item Andere Änderungen: Verschiebung der Kurve (links oder rechts)
\end{itemize}

\subsection{Nachfrage}

% Einflussfaktoren:
% \begin{itemize}\itemsep0em
% 	\item Preis
% 	\item Nutzenerwartung
% 	\item Preis anderer Güter
% 	\item Einkommen
% 	\item Zukunftserwartungen
% \end{itemize}
Je höher der Preis, desto weniger Nachfrage.

\begin{center}
\begin{tikzpicture}
    %\draw[very thin,color=gray] (-0.1,-1.1) grid (3.9,3.9);
    \draw[->] (0,0) -- (5,0) node[below] {Menge};
    \draw[->] (0,0) -- (0,4.2) node[left] {Preis};
    \draw[domain=0.25:4.5] plot[id=xl] function{1/x} node[right] {};
    \draw [thick, domain=1.34:4.5] plot[id=x] function{1/(x - 1) + 1}  node[right] {Nachfrage};
    \draw [domain=2.5:4.5] plot[id=xr] function{1/(x - 2) + 2}  node[right] {};
	% lines
	\draw [dashed] (0,2.5)  node [left] {$P$} -- (4, 2.5);
	\draw [dashed](4,0) node [below] {$M_r$} -- (4,2.5);
	\draw [dashed](1.67,0) node [below] {$M$} -- (1.67, 2.5);
	\draw [dashed](0.4, 0) node [below] {$M_l$}-- (0.4, 2.5);
\end{tikzpicture}
\end{center}

%\settowidth{\MyLenA}{}
 \begin{tabular}{@{}p{\linewidth/2}%
				@{}p{\linewidth/2}}
	\multicolumn{1}{c}{\underline{Linksverschiebung}} & \multicolumn{1}{c}{\underline{Rechtsverschiebung}}\\
	Tiefere Nutzenerwartung & Höhere Nutzenerwartung \\
	Substitutionsgüter $\rightarrow$ billiger & Substitutionsgüter $\rightarrow$ teurer \\
	Komplementärgüter $\rightarrow$ teurer & Komplementärgüter $\rightarrow$ billiger \\
	Tieferes Einkommen & Höheres Einkommen \\
	Erwartete Preissenkungen & Erwartete Preissteigerungen \\
	Bevölkerung nimmt ab & Bevölkerung nimmt zu \\
\end{tabular}

\begin{description}\itemsep0em
	\item [Substitutionsgut] Ein Gut, das ein anderes ersetzen kann
	\item [Komplemtärgut] Ein Gut, das ein anderes ergänzt
\end{description}

\subsubsection{Erstes Gossensches Gesetz}
\textbf{Grenznutzen} bezeichnet den zusätzlicher Nutzen pro zusätzlicher Einheit.

Die Grösse eines und desselben Genusses nimmt, wenn wir mit Bereitung des Genusses ununterbrochen fortfahren, 
fortwährend ab, bis zuletzt Sättigung eintritt.

$\Rightarrow$ Der Grenznutzen ist abnehmend; je mehr man von etwas hat, desto weniger will man für zusätzliche Einheiten
bezahlen. Die Nachfragekurve ist eine Grenznutzenkurve!

\subsubsection{Zweites Gossensches Gesetz}
Wird ein Gut teurer, werden alle anderen Güter relativ billiger. 

Der Grenznutzen ist ausgleichend; haben zwei Güter den gleichen Grenznutzen zu unterschiedlichen Preisen,
nehmen wir das billigere Gut. Wir sind bestrebt den Grenznutzen pro Geldeinheit in allen Verwendungsrichtungen
gleich gross zu halten.

Ein Haushalt befindet sich demnach in einem Haushaltsoptimum, wenn seine Grenznutzen für alle Güter, 
jeweils geteilt durch den Preis des Gutes, übereinstimmen. 
Andernfalls könnte er seinen Nutzen steigern, da sich eine Umstrukturierung des Konsums so vornehmen liesse, 
dass eine Ausgabenreduzierung bei einem Gut weniger Nutzeneinbuße als eine entsprechende Ausgabenerhöhung 
bei einem anderen Gut Nutzenzuwachs bedeutet.
