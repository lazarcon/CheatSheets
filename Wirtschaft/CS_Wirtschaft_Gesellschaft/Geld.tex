%\vfill
\section{Geld}

Die Funktionen des Geldes sind:
\begin{enumerate}\itemsep0em
	\item Zahlungsmittel
	\item Rechnungseinheit (Umrechnung von einem Gut in ein anderes $\Rightarrow$ Kostenersparnis, 
	weil weniger Aufwand)
	\item Wertaufbewahrungsmittel (Geld verdirbt nicht, nur den Charakter)
\end{enumerate}
Geld senkt die Transaktionskosten (all jene Kosten, die entstehen, um ein Tauschgeschäft abzuwickeln)

Geld im engeren Sinn ist alles, womit jederzeit bezahlt werden kann:
\begin{description}\itemsep0em
	\item [Bargeld] Münzen und Noten
	\item [Buchgeld] Guthaben auf Banken oder Post:
		\begin{description}\itemsep0em
		\item [Sicht-/Giroguthaben, Kontokorrentkonten] Jederzeit verfügbare Guthaben
		\item [Transaktionskonti] Andere Konten, die ebenfalls dem Zahlungsverkehr dienen können (Spar- und Einlagekonten)
		\end{description}
	\item [Spar- und Termineinlagen] sind nicht jederzeit verfügbar.
\end{description}

Geldbestände der Banken gehören nicht zur Geldmenge weil:
\begin{enumerate}\itemsep0em
	\item Sie keine Güternachfrage erzeugen
	\item Doppelzählung vermeiden; Einbezahlt = Guthaben; Geld im Banktresor ist bereits im Guthaben gezählt
\end{enumerate}
Die Geldmenge $G$ ist:
\begin{equation*}
	G = G_{\mbox{Haushalte}} \cup G_{\mbox{Unternehmen}} \cup G_{\mbox{Staat}}
\end{equation*}
Man unterscheidet 3 Teilmengen:
\begin{align*}
	G_{M1} & = \sum \mbox{Bargeld} + \sum \mbox{Giroguthaben} + \sum \mbox{Transaktionkonti}\\
	G_{M2} & = G_{M1} + \sum \mbox{Spareinlagen} - \sum \mbox{Vorsorgegelder (BVG, 3te Säule)}\\
	G_{M3} & = G_{M2} + \sum \mbox{Termineinlagen}
\end{align*}
Das Verhältnis von Bar- zu Buchgeld beträgt $\approx 1 : 9$ 

Die \textbf{Notenbankgeldmenge}, also die, die unter der Kontrolle der Nationalbank steht, ist:
\begin{equation*}
	G_{N} = \sum \mbox{Notenumlauf} + \sum \mbox{Giroguthaben Geschäftsbanken bei SNB}
\end{equation*}

\subsection{Geldentsehung und -vernichtung}
Geld entsteht durch ein Tauschgeschäft mit einer Bank (Nichtgeld gegen Geld).
\begin{itemize}\itemsep0em
	\item Kauf von Devisen durch SNB: Notenbankgeldmenge $\uparrow$, M-Geldmenge $\rightarrow$
	\item Transaktionen zwischen SNB und Geschäftsbank: Notenbankgeldmenge $\uparrow$, M-Geldmenge $\rightarrow$
	\item Geschäftsbank gewährt Kredit: Notenbankgeldmenge $\rightarrow$, M-Geldmenge $\uparrow$
\end{itemize}
Die Vernichtung ist die Umkehr der Entstehung.

Weil Banken Kundenguthaben bis auf einen Reservesatz weiter verleihen können, entsteht ein Multiplikator$ = \frac{1}{\mbox{Reservesatz}}$
auf die M-Geldmenge. So erklärt sich das Verhältnis von Bargeld zu Buchgeld.

\subsection{Probleme beim Geld}
\begin{enumerate}\itemsep0em
	\item Geldkreislauf $\neq$ Güterkreislauf
	\item Zinsen von Krediten; das Zinsgeld existiert gar nicht!
\end{enumerate}
\subsubsection{Zinseszinsformel}
Kapital $K_n$ nach $n$ Jahren bei einem Einsatz von $K_0$ und dem Zinssatz $i$:
\begin{align*}
	K_n& = K_0 \cdot (1 + i)^n = K_0 \cdot q^n\\
	n& = \frac{\ln{\frac{K_n}{K_0}}}{\ln{(1 + i)}}
\end{align*}
5 Rappen zu 2\% sind nach 2\,000 Jahren $3.9 \cdot 10^{14}$ CHF!

\subsection{Kritische Fragen}
\begin{enumerate}\itemsep0em
	\item Warum leihen sich Regierungen Geld, wenn sie es auch zinsfrei selber machen könnten? (Inflation)
	\item Warum wird Geld über Schulden geschaffen?
	\item Wie kann ein Geldsystem auf exponentiellem Wachstum basieren?
	\item Was muss gemacht werden, um ein nachhaltige Wirtschaft zu erschaffen?
	\item Warum kommen Produktionsfortschritte heute grösstenteils den Managern und Aktionären und nicht auch den
	Arbeitnehmern zu gute?
\end{enumerate}

\begin{description}
	\item [Bail-out] Umwandlung privater Schulden in staatliche
\end{description}

\subsection{Schweizerische Nationalbank (SNB)}
\begin{description}\itemsep0em
	\item [Rechtsform] Gemischtwirtschaftliche AG (51\% Kantone, 49\% Privat)
	\item [Hauptaufgabe] Für Stabilität in der Geld- und Währungspolitik sorgen.
	\item [Instrumente] ~\\
	\begin{itemize}\itemsep0em
	\item Repo-Geschäft: SNB kauft Wertpapiere von einer Geschäftsbank und vereinbart, dass die Geschäftsbank nach einer festgelegten Frist für weniger wieder zurückkauft. Die Differenz ist der Repo-Satz. 
	Je tiefer, desto weniger derartige Geschäft, desto weniger Geld im Umlauf $\rightarrow$ bekämpft Inflation
	\item Devisengeschäft: SNB kauft ausländische Währungen $\rightarrow$ mehr CH-Geld im Umlauf, Kurs fällt
	\item SNB Bills: Schuldverschreibungen der SNB $\rightarrow$ entzieht dem Markt Geld.
 	\item 3-Monate-Libor: Zinssatz für 3-monatige Anlagen $\rightarrow$ Je höher, desto weniger Geld im Umlauf
	\end{itemize}
\end{description}

\subsubsection{Geldpolitik}
Ziel: Innenwert des Geldes, den Geldwert im Inland stabil halten. \\
Instrumente: Repo-Geschäft, 3-Monate-Libor, SNB-Bills\\

\begin{center}
	\begin{tabular}{ll}
		\multicolumn{1}{c}{\textbf{Geldwert steigt}} & \multicolumn{1}{c}{\textbf{Geldwert sinkt}}\\
		Deflation (Preise fallen)	& Inflation (Preise steigen)\\
		Immoblienpreise fallen		& Immobilienpreise steigen\\
		Schulden steigen			& Schulden sinken\\
	\end{tabular}
\end{center}


\subsubsection{Währungspolitik}
Ziel: Aussenwert des Geldes, den Geldwert im Ausland stabil halten.\\
Instrumente: Devisengeschäft, SNB Bills\\

\begin{center}
	\begin{tabular}{ll}
		\multicolumn{1}{c}{\textbf{Franken steigt}} & \multicolumn{1}{c}{\textbf{Franken sinkt}}\\
		Import steigt				& Export steigt\\
		Touristen gehen				& Touristen kommen\\
	\end{tabular}
\end{center}

\subsubsection{Euro-Mindeskurs}
\begin{itemize}\itemsep0em
	\item[$\Rightarrow$] Produziert die SNB zu viele CHF um EUR zu kaufen, ist mehr Geld im Umlauf und die Preise steigen
	\item[$\Rightarrow$] Bis jetzt keine Inflation, weil das Geld gar nicht im Umlauf ist (Geldmenge blieb konstant)
	\item[$\Rightarrow$] Probleme entstehen erst, wenn Geld ausgegeben wird, also in den Umlauf kommt, weil beispielsweise die Wirtschaft anzieht.
\end{itemize}

\subsubsection{Quantitätsgleichung}
Es gilt die Quantitätsgleichung (auch Transaktionsgleichung, Verkehrsgleichung oder Tauschgleichung), die Anhaltspunkte liefert 
über die Beziehung zwischen Geld und Gütertransaktionen innerhalb einer Volkswirtschaft.
\begin{equation*}
	M \cdot V = P \cdot T
\end{equation*}
\begin{description}
	\item [$M$] Geldmenge (durchschnittlich umlaufende Menge an Geld innerhalb einer Periode)
	\item [$V$] Umlaufgeschwindigkeit (gibt an, wie häufig eine Geldeinheit in einer Betrachtungsperiode durchschnittlich verwendet wurde)
	\item [$P$] Preisniveau (stellt den Durchschnittspreis der Güter und Dienstleistungen dar)
	\item [$T$] Transaktionen (gibt die durchschnittliche Anzahl, der in einer Periode stattfindenden Transaktionen, an) 
\end{description}
