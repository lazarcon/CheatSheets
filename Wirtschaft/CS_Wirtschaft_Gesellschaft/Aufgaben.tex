\section{Aufgaben der Wirtschaftswissenschaften}
\begin{description}\itemsep0em
	\item [Beschreiben] von wirtschaftlichen Vorgängen
	\item [Erklären] der wirtschaftlichen Vorgängen
	\item [Prognostizieren] des zukünftigen Ablaufs
	\item [Beeinflussen] der wirtschaftlichen Entwicklung
\end{description}

\subsection{Ziele der (Schweizerischen) Wirtschaftspolitik}
Die Ziele der Wirtschaftspolitik leiten sich ab aus Art. 2 und 94 BV ab und lassen sich im \enquote{magischen Sechseck}
darstellen:
\begin{center}
\tikzstyle{block} = [draw, rectangle, minimum width=2cm, minimum height=1.25cm, text width=2cm, text centered]
\begin{tikzpicture}[auto, node distance=2.5cm,>=latex']
	\node [block] (vb) {Voll-\\beschäftigung};
 	\node [block, right of=vb] (ww) {Wirtschafts-\\wachstum};
 	\node [block, below left of=vb] (sa) {Sozialer Ausgleich};
 	\node [block, below right of=ww] (uq) {Umwelt-\\qualität};
 	\node [block, below right of=sa] (ps) {Preis-\\stabilität};
 	\node [block, below left of=uq] (ag) {Aussenwirt-\\schafliches Gleichgewicht};
	\draw [-] (vb) -- (ww);
	\draw [-] (ww) -- (uq);
	\draw [-] (uq) -- (ag);
	\draw [-] (ag) -- (ps);
	\draw [-] (ps) -- (sa);
	\draw [-] (sa) -- (vb);
\end{tikzpicture}
\end{center}
Dabei stehen die sechs Ziele in unterschiedlichen Zielbeziehungen zu einander:
\begin{description}\itemsep0em
	\item [Zielharmonie] Das eine Ziel fördert das Erreichen eines anderen 
	(z.\,B. Wirtschaftswachstum und Vollbeschäftigung)
	\item [Zielneutralität] Ein Ziel hat (zumindest zeitweise) keinen Einfluss auf ein anderes
	(z.\,B. Preisstabilität und Umweltqualität)
	\item [Zielkonkurrenz] Ein Ziel behindert (zumindest kurzfristig) ein anderes
	(z.\,B. Preisstabilität und Vollbeschäftigung)
\end{description}

\subsubsection{Ziele 2012/13}
\begin{itemize}\itemsep0em
	\item Wettbewerbsfähigkeit der Wirtschaft stärken
	\item Standortattraktivität gegenüber dem Ausland steigern
\end{itemize}

\subsubsection{Werkzeugkasten}
Mehrwertsteuer, Handelsabkommen, Zinspolitik, Preisüberwachung, Investitionsprogramme,
Subventionen, Stipendien, Bankgeheimnis, Arbeitslosenversicherung.

Aber: Der Schuss kann auch nach hinten los gehen: Steuererhöhung kann zu weniger Steuereinnahmen führen
(Wohnortswechsel, Steuerhinterziehung und Schwarzarbeit). Politische Steuerungsversuche sind immer von
 ungewissen Nebenfolgen begleitet!

