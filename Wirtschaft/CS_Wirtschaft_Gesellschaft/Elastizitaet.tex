\section{Elastizität}
Die Preiselastizität ist definiert als relative Mengenänderung (der am Markt zu diesem Preis angebotenen Güter) dividiert durch relative Preisänderung.
\begin{equation*}
	\mbox{Elastizität:} E = \frac{\mbox{relative Veränderung der abhängigen Variable}}{\mbox{relative Veränderung der unabhängigen Variable}}
\end{equation*}

\begin{description}\itemsep0em
	\item [vollkommen elastisch ($E = -\infty$)] Geringe Änderungen haben riesige Wirkung, z.\,B. Preis von Banknoten
	\item [sehr elastisch ($E < -1$)] Änderung hat grosse Wirkung (überproportional) z.\,B. Nägel
	\item [proportional elastisch ($E = -1$)] Änderung gleich grosse
	\item [unelastisch ($-1 < E < 0$)] Änderung hat kleine Wirkung (unterproportional) z.\,B. Nahrungsmittel
	\item [vollkommen unelastisch ($E = 0$)] Änderung hat überhaupt keine Wirkung  z.\,B. lebenswichtige Medikamente
	\item [anomal elastisch ($E > 0$)] höherer Preis = höhere Nachfrage z.\,B. Snobeffekt, Hamsterkäufe
\end{description}

% \subsection{Rechenbeispiel}
% Gegeben sei die Funktion $m(p) = 24 - 6 \cdot p$, die die umgesetzte Menge $m$ in Abhängigkeit vom Preis $p$ beschreibt. 
% Wie ist die Elastizität bei $p = 3$?
% \begin{align*}
% 	m(p)& = 24 - 6 \cdot p\\
% 	m'(p)& = - 6\\
% 	m(3)& = 24 - 6 \cdot 3 = 24 - 18 = 6\\ 
% 	E& = \frac{\mathrm d m}{\mathrm d p} \cdot \frac{p}{m} = m'(p) \cdot \frac{p}{m}\\
% 	& = -6 \cdot \frac{3}{6} = -6 \cdot \frac{1}{2} = - \frac{6}{2} = - 3 \mbox{ (sehr elastisch)}
% \end{align*}

%\subsection{Elastizität der Nachfrage}
% \begin{equation*}
% 	E_{\mbox{Nachfrage}} = \frac{\mbox{relative Veränderung der nachgefragten Menge}}{\mbox{relative Veränderung des Preises}}
% \end{equation*}

\textbf{Einflussfaktoren auf die Nachfrageelastizität}
\begin{itemize}\itemsep0em
	\item (Lebens-)Wichtigkeit des Produkts: je wichtiger, desto unelastisch
	\item Anteil am Budget: je geringer, desto unelastisch
	\item Substitutionsmöglichkeiten: je substitutierbar, desto elastischer
	\item Zeit: je länger die Periode, desto elastisch
\end{itemize}

%\subsection{Elastizität des Angebots}
% \begin{equation*}
% 	E_{\mbox{Angebot}} = \frac{\mbox{relative Veränderung der angebotenen Menge}}{\mbox{relative Veränderung des Preises}}
% \end{equation*}

\textbf{Einflussfaktoren auf die Angebotselastizität}
\begin{itemize}\itemsep0em
	\item Haltbarkeit und Lagerfähigkeit: je lagerbar, desto elastisch
	\item Herstellbarkeit: je herstellbar, desto elastisch
	\item Zeit: je länger die Periode, desto elastischer
\end{itemize}

\textbf{Einkommenselastitzität}
\begin{equation*}
	E_{\mbox{Einkommen}} = \frac{\mbox{relative Veränderung der nachgefragten Menge}}{\mbox{relative Veränderung des Einkommens}}
\end{equation*}

\begin{description}\itemsep0em
	\item [vollkommen unelastisch ($E = 0$)] Z.\,B. Toilettenpapier, Salz
	\item [unelastisch ($0 < E < 1$)] Normale Güter; Kleidung, Nahrungsmittel
	\item [elastisch ($E > 1$)] Luxusgüter; Reisen, Schmuck
	\item [anomal elastisch ($E < 0$)] Inferiore Güter; Bohnen oder Kartoffeln
\end{description}

