\section{Menschenbild}
\subsection{Wirtschaftliche Sicht}
Aus wirtschaftlicher Sicht geht man vom \enquote{Homo oeconomicus} aus. Das heisst: Nutzenoptimierung unter Nebenbedingungen
\begin{itemize}\itemsep0em
	\item Handlungseinheit ist das Individuum (Methodologischer Individualismus)
	\item Anreize und Belohnungen steuern das menschliche Verhalten
	\item Anreize sind durch Präferenzen (Vorlieben) und Einschränkungen (Einkommen) bestimmt
	\item Individuen sind auf ihren eigenen Vorteil bedacht und verhalten sich eigennützig
\end{itemize}

\subsection{Psychologische Sicht}
\begin{itemize}\itemsep0em
	\item Handlungseinheit sind Gruppen
	\item Jeder hat ein bestimmtes Rollenverhalten
	\item Jeder folgt einer Tradition
	\item Sitte und Moral begrenzen die Möglichkeiten
\end{itemize}

\subsection{Verdrängungseffekt}
Um menschliches Verhalten zu steuern kann man auf \textit{intrinsische Motivation} oder \textit{externe Anreize} setzen.
Die beiden Möglichkeiten verdrängen sich gegenseitig.

\subsection{Teilungsexperiment}
Zwei Teilnehmer, einer hat (beispielsweise) 10 CHF. Er teilt die 10 CHF in zwei Haufen, einen für sich, einen für den anderen.
Akzeptiert der andere seinen Haufen, können beide das Geld behalten. Lehnt er ab, bekommen beide nichts.

Rational (\enquote{Homo oeconomicus}) für den zweiten wäre, jeden Haufen zu akzeptieren. Tatsächlich handelt die Mehrheit 
irrational und lehnt Teilungen von unter 4:6 ab (Fairnessnorm).

\subsection{Verteilungsexperiment}
Jedes Mitglied einer Gruppe von Teilnehmern bekommt den gleichen Betrag. Jeder setzt einen bestimmten Teilbetrag, so dass
die anderen nicht wissen, wie viel er setzt. Alle Einsätze werde verdoppelt und gleichmässig an alle Teilnehmer verteilt.

Rational wäre den kleinst möglichen Betrag (0) zu setzen (Trittbrettfahrer, Nash-Gleichgewicht). Tatsächlich werden aber
höhere Beträge gesetzt. Wird allerdings öffentlich gemacht, wer wie viel setzt, werden die eingesetzten Beträge grösser.

\subsection{Broken Window}
Ist eine Scheibe eines Gebäudes kaputt gehen bald auch die anderen Scheiben kaputt. Gleiches gilt für Plätze usw.

\subsection{Merkwürdiges}
\begin{itemize}\itemsep0em
	\item Strafen könne die Kooperation erhöhen
	\item Kooperation ist nicht stabil (obwohl sie zu Wohlstand führen würde), weil Trittbrettfahren (das zur Armut führt) profitabler ist.
	\item Teamarbeit kann die Produktivität um bis zu 20\% steigern
	\item Der Wunsch zu helfen ist hedonisch (d.\,h. es bereitet Freude, Vergnügen, Lust oder Genuss) verankert
	\item Religionen predigen Nächstenliebe
\end{itemize}
 
Das einfach Rezept zum Glück ist Hoffnung, Grosszügigkeit und Vergebung.
