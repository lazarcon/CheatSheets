\section{Einordnung}
% \settowidth{\MyLenA}{Word~~}
% \begin{tabular}{@{}p{\the\MyLenA}%
% 				@{}p{\linewidth-\the\MyLenA}}
% 	Bit & [0,1], Masseinheit für übertragene Datenmengen und Informationsgehalt\\
% 	Byte & 8-Tupel von Bit, Masseinheit für Speicherung\\
% 	Word & Grundverarbeitungsdatengrösse in einem Rechner. Immer eine 2er Potenz\\
% \end{tabular}

Wissenschaft lässt sich in zwei Gruppen unterteilen:
\begin{description}\itemsep0em
	\item [Realwissenschaften] haben reale Sachverhalte zum Forschungsgegenstand. Beispiele sind die Naturwissenschaften und die Kulturwissenschaften (insbesondere Wirtschaft, Psychologie).
	\item [Formalwissenschaften] sind Wissenschaften, die sich der Analyse von Formalen Systemen widmen. Beispiele sind Mathematik, Logik, allgemeine Linguistik und theoretische Informatik.
\end{description}

\textit{ceteris paribus (cet. par.)} (lat. alles andere bleibt gleich).\\