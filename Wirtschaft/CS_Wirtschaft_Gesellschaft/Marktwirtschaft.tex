\section{Marktwirtschaft}
Der Mensch hat unbeschränkte Bedürfnisse, die Erde als geschlossenes System hat beschränkte Ressourcen (Knappheit)\\ 
$\Rightarrow$ Je mehr Menschen, desto grösser der Kampf um Ressourcen.

\subsection{Zentrale Fragen}
\begin{itemize}\itemsep0em
	\item Was soll produziert werden?
	\item Wie soll produziert werden?
	\item Für wen soll produziert werden?
\end{itemize}

\subsection{Wirtschaftssysteme}
Die Wirtschaftssysteme versuchen diese Fragen zu beantworten.

\subsubsection{Zentrale Planwirtschaft}
Politisch \enquote{links} angesiedeltes System, wie etwa in Kuba oder Nordkorea (typischerweise Diktaturen)
\begin{itemize}\itemsep0em
	\item Die staatliche Kontrolle strebt gegen 100\%
	\item Produktionsmittel sind in staatlicher Hand
	\item Zentrale Planstellen entscheiden über Produktionsmengen und Preise
	\item [$\Rightarrow$] kleine Planungsfehler bezüglich Preis oder Menge haben riesige Folgen für Preis und Menge.
\end{itemize}

\subsubsection{Freie Marktwirtschaft}
Politisch \enquote{rechts} angesiedeltes System, wie etwa in den USA oder GB (typischerweise Demokratien)
\begin{itemize}\itemsep0em
	\item Die staatliche Kontrolle strebt gegen 0\%
	\item Produktionsmittel sind in privater Hand
	\item Die Märkte entscheiden über Produktionsmengen und Preise
	\item [$\Rightarrow$] versagen die Märkte ist das Leid gross (Immobilenblase)
\end{itemize}

\subsubsection{Sozial Marktwirtschaft}
Politisch in der \enquote{mitte} angesiedeltes System, wie etwa CH, D oder auch F (typischerweise Demokratien)
\begin{itemize}\itemsep0em
	\item Der Staat kontrolliert, was notwendig ist
	\item Die Produktionsmittel sind in privater Hand
	\item Die Märkte entscheiden über Produktionsmengen und Preise
	\item [$\Rightarrow$] der Kompromiss aus zentraler Planwirtschaft und freier Marktwirtschaft
\end{itemize}


\subsection{Markt- und Preisfunktionen}
Der Markt ist ein Verfahren, bei dem durch Zusammenwirken von Anbietern und Nachfragern Entscheidungen über Preis und Menge von Gütern und Produktionsfaktoren getroffen werden.

\begin{description}\itemsep0em
	\item [Informationsträger] Preise zeigen an, \textbf{was} produziert werden soll
	\item [Steuerung und Allokation] (Allokation = Zuweisung verfügbarer Mittel an die Herstellung bestimmter Güter) Preise zeigen an, \textbf{wie}, d.\,h. mit
	welchen Mitteln, produziert werden soll (Hohe Preise $\Rightarrow$ Produktionsfaktoren für diese Güter verwenden)
	\item [Koordination] Der Markt-Preis-Mechanismus steuert auf den Märkten für Produktionsfaktoren ebenfalls \textbf{für wen} produziert wird
\end{description}

Der Markt sorgt für die fortwährende Suche nach neuen Produkten, besseren Technologien um Ressourcen bestmöglich zu nutzen. Das Preissystem
signalisiert Knappheit und Überschüsse. Gemäss \textit{Adam Smith} (1723-1790) lenkt der Markt wie ein \enquote{unsichtbare Hand} die Handlungen aller Individuen, so dass der Nutzen aller
maximiert wird.

%Damit die Wirtschaft blüht, braucht es risikofreudige, innovative Unternehmer, die mit neuen Produkten und Technologien die Märkte für sich erobern.  

\subsection{Marktversagen}
Die Bedingungen für den idealen Markt fehlen:
\begin{enumerate}\itemsep0em
	\item Es gibt keine homogenen Güter
	\item Es gibt nur wenige Anbietern
	\item Märkte sind aufgrund von Eintrittsbarrieren selten frei zugänglich
	\item Beide Seiten sind nicht vollständig informiert
\end{enumerate}

\subsubsection{Ursachen}
\begin{description}\itemsep0em
	\item [Wettbewerbsbeschränkungen] Monopole (Autoimport), Normen, Preis- und Mengenabsprachen (Kartelle, OPEC) führen zu künstlicher Knappheit
	und damit zu höheren Preisen (\enquote{rent seeking} -- Einkommen ohne rechtfertigende Leistung)
	\item [Öffentliche Güter] gelten beide der folgenden Bedingungen:
	\begin{enumerate}\itemsep0em
		\item Das Ausschlussprinzip (man kann vom Konsum ausgeschlossen werden) funktioniert nicht
		\item Es herrscht Nicht-Rivalität im Konsum, d.\,h. Konsum durch den Einen schränkt den Konsum durch einen Anderen nicht ein
	\end{enumerate}
	ist \enquote{freeriding} (Trittbrettfahrern) die Konsum-Option, die alle wählen $\Rightarrow$ kein Angebot (z.\,B. saubere Umwelt)
	\item [Externe Effekte] treten auf, wenn unbeteiligte Dritte die Folgen eines Konsums spüren. Dabei ist zu unterscheiden:
	\begin{description}\itemsep0em
		\item [Externe Kosten] sind Kosten die nicht vom Verursacher getragen werden: Preise zu tief $\Rightarrow$ Konsum zu hoch.
		\item [Externer Nutzen] sind Vorteile ohne eigenes zu tun: Kein Preis $\Rightarrow$ keine Produktion
	\end{description}
	\item[Asymmetrische Information] Weiss eine Vertragspartei mehr als die andere kommt es zur \enquote{adverse selection}, der falschen Auslese (Nicht optimaler Preis, nicht optimale Menge).
	Hat ein Tauschpartner Möglichkeit und Anreiz Kosten auf den Tauschpartner zu überwälzen, liegt ein \enquote{moral hazard} ein moralisches Risiko vor. (z.\,B. Versicherungen, Ärzte)
	\item[Soziale Fragen] Der Markt ist gar nicht für soziale Fragen gemacht; Alte, Kranke, Schwache nehmen nicht teil. 
\end{description}

\subsection{Staatliche Eingriffe}
Damit der Markt funktioniert muss der Staat sicherstellen:
\begin{enumerate}\itemsep0em
	\item Privateigentum der Güter
	\item Vertrags- und Rechtssicherheit
	\item Schliessen von Zutrittsbarrieren zum Markt $\rightarrow$ offene Märkte
	\item Sicherung des Wettbewerbs (Verhindern von \enquote{rent-seeking} durch Preisüberwacher, WEKO)
	\item Bereitstellen öffentlicher Gütern, Förderung deren Produktion
	\item Externe Kosten verhindern (z.\,B. Lärm- und Abgasvorschriften -- Internalisierung externer Kosten) 
	\item Asymmetrische Information minimieren (z.\,B. Standesrichtlinien, Konsumentenschutz, staatliche Information)
	\item Umverteilung der Einkommen zugunsten von Invaliden, Alten und Schwachen (z.\,B. Sozialtransfers, Versicherungen)
\end{enumerate}

\subsubsection{Staatsversagen}
Lösen staatliche Eingriffe ein Problem nicht oder schaffen gar neue kommt es zum Staatsversagen (statt zu fördern behindert der Staat). Ursachen:
\begin{itemize}\itemsep0em
	\item Politisch motivierte Entscheide (Lobbyismus $\rightarrow$ \enquote{rent-seeking}
	\item Regulierungskosten (Überregulierung verursacht marktverzerrende Kosten, Unterregulierung gewährt zu viele Freiräume)
	\item Verzerrung der Allokationseffizienz (Durch Regulierung, etwa Importbeschränkungen, profitiert eine Branche ohne den Preis dafür zu zahlen $\rightarrow$ Konsumenten zahlen zu hohe
	Preise, aber eigentlich müsste die Branche billiger produziern)
\end{itemize}

\subsection{Ursachen Finanzkrise}
\begin{enumerate}\itemsep0em
	\item Asymmetrische Information: Die Risiken von Wertpapieren waren weitgehende unbekannt
	\item Externe Kosten: Der Untergang einer Bank zog andere Banken mit sich
	\item Zu tiefe Zinsen: Die amerikanische Zentralbank verlangte zu tiefe Zinsen für Kredite
\end{enumerate}

% 
% \subsection{Eigentum}
% 
% \subsection{Die Wirtschaftsordnung der Schweiz}
% 
\subsection{Exkurs: Warum Gratis Exporte schlecht sind}
Gratis-Exporte unserer Überproduktion nach Afrika führt zu zwei Problemen:
\begin{enumerate}
	\item Unser Markt wird kaputt gemacht -- Die Preise sind höher als sie sein müssten
	\item Der Markt in Afrika wird kaputt gemacht -- Die Preise dort sind tiefer als sie sein müssten
\end{enumerate}
% 
% 
% \subsection{Exkurs: Spieltheorie}
