\section{Rechnerarithmetik}
\subsection{Maschinenzahlen}
Eine gegebene Zahl $x \in \mathbb{R}$ lässt sich darstellen als: $x = m \cdot B^e$.
\begin{description}
 \item [Basis $B$] ist eine Zahl aus $\mathbb{N}$
 \item [Mantisse $m$] ist eine Zahl in $\mathbb{R}$. $m_1$ ist die erste Ziffer von $m$.
 \item [Exponent $e$] ist eine Zahl in $\mathbb{Z}$. $e_1$ ist die erste Ziffer von $e$.
 \item [Maschinendarstellbare Zahlen] Es gilt:
 \begin{equation*}
  M:= \{x \in \mathbb{R} | x = \pm 0.m_1m_2m_{\dots}m_n \cdot B^{e_1e_2e_{\dots}e_j}\}
 \end{equation*}
 Zahlen werden normiert, so dass die Ziffer vom dem Komma immer eine $0$ ist. (spart ein Bit)
\item [Wert im Dezimal] Die Basis ist somit 10.
\begin{equation*}
 \mbox{Wert} = \sum_{i=1}^n m_i \cdot B^{e-i}
\end{equation*}
Bsp:
  \begin{align*}
   x& = 0.101 \cdot 2^3 & n = 3\\
    & = 0 \cdot 2^3 + 1 \cdot 2^2 + 0 \cdot 2^1 + 1 \cdot 2^0 = 0 + 4 + 0 + 1 = 5 
  \end{align*}
\end{description}

\subsection{Musteraufgaben}
\subsubsection{Wie viele Stellen benötigt folgende Zahl als $n$-stellige Gleitpunktzahl ($g$)?}
\begin{align*}
  x&= 1230001\\
  g(x)& = 0.1230001 \cdot 10^{7} \Rightarrow n = 7
\end{align*}
\subsubsection{Bestimmen Sie alle dualen positiven 3-stelligen Gleitpunktzahlen mit einstelligem binären Exponenten, sowie deren Dezimalwert}
\begin{align*}
  0.100 \cdot 2^0& = 0.5	& 0.100 \cdot 2^1& = 1\\
  0.101 \cdot 2^0& = 0.625	& 0.101 \cdot 2^1& = 1.25\\
  0.110 \cdot 2^0& = 0.75	& 0.110 \cdot 2^1& = 1.5\\
  0.111 \cdot 2^0& = 0.875	& 0.111 \cdot 2^1& = 1.75\\
\end{align*}
Hinzu kommt noch die immer vorhandene 0 ($0.000 \cdot 2^x$)
\subsubsection{Wie viele verschiedene Maschinenzahlen gibt es auf einem Rechner, der 20-stellige Gleitpunktzahlen mit 4-stelligem binären Exponenten
sowie zugehörige Vorzeichen im Dualsystem verwendet? Was ist die kleinste und die grösste Maschinenzahl?}
\begin{enumerate}
 \item Die erste Stelle ist per Definition eine 1 $\rightarrow$ 19 freie Stellen, die die Werte 0 oder 1 annehmen können. $\rightarrow 2^{19}$ Möglichkeiten.
 \item 4 Stellen für den Exponenten, die jeweils die Werte 0 und 1 annehmen können $\rightarrow 2^4$ Möglichkeiten.
 \item 1 Vorzeichen für das Vorzeichen der Mantisse $\rightarrow$ 2 Möglichkeiten.
 \item 1 Vorzeichen für den Exponent $\rightarrow$ 2 Möglichkeiten.
\end{enumerate}
Verschiedene Zahlen: $2^{19} \cdot 2^4 \cdot 2 \cdot 2 = 2^{25}$ Möglichkeiten. Hinzu kommt noch die 0. Also $2^{25} + 1$ Möglichkeiten.
Minimum $= - 0.1000_2\dots \cdot 2^{1111_2}$, Maximum $= 0.111_2\dots \cdot 2^{1111_2}$ 
