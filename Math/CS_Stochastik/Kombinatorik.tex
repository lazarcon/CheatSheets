\section{Kombinatorik}
Voraussetzung: Alle betrachteten Fälle sind gleich wahrscheinlich.
\begin{description}
	\item [Urnenmodell] In einer Gefäss befinden sich $n$ unterscheidbare Kugeln.
Nach und nach werden $k$ davon zufällig ausgewählt
	\item [Permutationen] Es gibt $n!$ Möglichkeiten, um $n$ Elemente einer Menge anzuordnen
	\item [Binominalkoeffizient] $\binom nk = \frac{n!}{k! \cdot (n-k)!}$
\end{description}

\begin{center}
\begin{tabular}{|c|c|p{2.5cm}|p{2.5cm}|}
\cline{3-4}
\multicolumn{2}{c|}{} & \multicolumn{2}{c|}{\textbf{Zurücklegen}} \\ \cline{3-4}
\multicolumn{2}{c|}{} & \multicolumn{1}{c|}{\textbf{mit}} & \multicolumn{1}{c|}{\textbf{ohne}} \\ \cline{1-4}
\multirow{2}{*}{\rotatebox{90}{\textbf{Reihenfolge}~~~~~~~~~~~~~~~~~~~~~~~~~~~~}} &
\textbf{mit} &
%% mit zurücklegen, mit Reihenfolge
\begin{equation*}
	|\Omega| = n^k
\end{equation*}
z.\,B. Zahlenschloss mit $k=3$ Ziffern von 0-9 ($n=10$) $\rightarrow |\Omega| = 10^3$ 

& 
% ohne zurücklegen, mit Reihenfolge
\begin{multline*}
	|\Omega| = \frac{n!}{(n - k)!}\\ = \mbox{npr}(n, k)
\end{multline*}
z.\,B. $k=5$ Plätze bei einem Rennen mit $n=12$ Läufern $\rightarrow |\Omega| = \frac{12!}{7!}$

\\\cline{2-4}
	&
\textbf{ohne} & 
% mit zurücklegen, ohne Reihenfolge
\begin{multline*}
	|\Omega| = \binom{n + k - 1}{k}\\ = \mbox{ncr}(n, k)
\end{multline*}
z.\,B. $k=2$ Würfel mit $n=6$ Seiten $\rightarrow |\Omega| = \binom{6 + 2 - 1}{2}$

 & 
% ohne zurücklegen, ohne Reihenfolge
\begin{equation*}
	|\Omega| = \binom{n}{k}
\end{equation*}
z.\,B. Zahlenlotto: 6 ($k$) aus 49 ($n$) Kugeln: $|\Omega| = \binom{49}{6}$

\\ \cline{1-4}
\end{tabular}
\end{center}

\subsection{Verteilungen}

\subsection{Wahrscheinlichkeitsfunktion}
Beschreibt die Wahrscheinlichkeit dafür, dass eine Zufallsvariable $X$ einen bestimmten Wert $x$ annimmt.
\begin{itemize}\itemsep0em
	\item Diskrete Verteilungen: $p(x) = P(X = x)$
	\item Stetigen Verteilungen: 
	\begin{equation*}
	P(a \le X \le b) \, = \, \int_a^b f(x)\,\mathrm dx \mbox{ bzw. } \mu([a,b]) \, = \, \int_a^b f(x)\,\mathrm dx	
	\end{equation*}
\end{itemize}


\subsubsection{Binominalverteilung}
Beschreibt die Anzahl der Erfolge in einer Serie von gleichartigen und unabhängigen Versuchen, die jeweils genau zwei mögliche Ergebnisse haben (\enquote{Erfolg} oder \enquote{Misserfolg}) (Bernoulli-Prozesse). Für $n$ Versuche mit $k$ Erfolgen ($E$) gilt ($p = P(E)$):
\begin{equation*}
	p_k = \binom {n}{k}\, p^k\, (1-p)^{n-k}
\end{equation*}
Es werden bei $n$ Versuchen genau $k$ Erfolge erzielt. Für $p \neq 1/2$ ist die Binominalverteilung unsymmetrisch.

\subsubsection{Poisson-Verteilung}
Spezialfall der Binominalverteilung, der die Verteilung von seltenen Ereignissen beschreibt. $n$ wird weggelassen (d.\,h. schlägt ein Blitz ein? ist entscheidbar, wie oft er einschlägt kann hingegen nicht berechnet werden)
\begin{equation*}
	P_\lambda (k) = \frac{\lambda^k}{k!}\, \mathrm{e}^{-\lambda}
\end{equation*}
Mit $\lambda = g \cdot w$ und $g = $Anzahl Ereignisse pro Einheitsintervall während des Intervalls $w$. Beispiel: Blitzhäufigkeit = 10 Einschläge pro km$^2$. Wie gross ist die Wahrscheinlichkeit für Blitzeinschläge pro ha ($10{,}000$ km$^2$)?
$\lambda = 10/\mbox{km}^2 \cdot 0.01 \mbox{km}^2 = 0.1$

\subsubsection{Geometrische Verteilung}
Beschreibt, wie oft ein Versuch wiederholt werden muss, bis $E$ Eintritt.
\begin{equation*}
	p_k = (1 - p)^{k-1} p
\end{equation*}

\subsubsection{Normal- oder Gauss-Verteilung}
Dichtefunktion:
\begin{equation*}
	f_X(x) = \frac{1}{\sqrt{2\pi\sigma^2}}\operatorname{exp}\left(-\frac{\left(x-\mu\right)^2}{2\sigma^2}\right)
\end{equation*}
\begin{description}
	\item [Erwartungswert] $E(x) = \mu$ entspricht dem Mittelwert
	
	\item [Varianz] $\operatorname{Var}(x) = \sigma^2$ (Standardabweichung$^2$)
	
	\item [Dichtefunktion der Standartnormalverteilung]
	\begin{equation*}
		f_x(X) = \varphi(x)=\frac{1}{\sqrt{2\pi}}  e^{-\frac{1}{2} x^2}\,.
	\end{equation*}
	
	\item [Verteilungsfunktion]
	\begin{equation*}
		F(x) = \frac{1}{\sigma\sqrt{2\pi}} \int_{-\infty}^x e^{-\frac{1}{2} \left(\frac{t-\mu}{\sigma}\right)^2} \mathrm dt
	\end{equation*}

	\item [Maximum]
	\begin{align*}
		x_{\mbox{max}}& = \mu & f(x_{\mbox{max}})& = \frac{1}{\sigma\,\sqrt{2\pi}}
	\end{align*}

	\item [Wendepunkte]
	\begin{align*}
		x_{\mbox{wende}}& = \mu \pm \sigma & f(x_{\mbox{wende}}) = \frac{1}{\sigma\,\sqrt{2\pi \mathrm e}}
	\end{align*}
\end{description}
%Standardnormalverteilung:
%Für $X \approx N(\mu, \sigma^2)$ gilt: $X^* = \frac{X - \mu}{\sigma} \approx N(0, 1)$
%d.\,h. 

\textbf{Transformation}\\
Eine Normalverteilung mit beliebigen $\mu$ und $\sigma$ und der Verteilungsfunktion $F$ hat folgende Beziehung zur Standardnormalverteilung ($\mathcal{N}(0,1)$):
\begin{equation*}
	F_X(x) = \Phi \left(\frac{x-\mu}{\sigma} \right)
\end{equation*}
Für $X \sim \mathcal{N}(\mu, \sigma^2)$ gilt die Transformation $Z = \frac{X - \mu}{\sigma}$.
$Z$ ist dann standardnormalverteilt ($\mathcal{N}(\mu, \sigma^2) \rightarrow \mathcal{N}(0, 1)$)

Die Wahrscheinlichkeit, dass ein Ereignis $X$, bei denen die Wahrscheinlichkeiten normalverteilt sind, im Intervall $[a, b]$ liegt lässt sich wie folgt berechnen:
\begin{equation*}
	P(a \leq X \leq b) = \Phi\left(\frac{b - \mu}{\sigma}\right) - \Phi\left(\frac{a - \mu}{\sigma}\right)
\end{equation*}
Die Werte für $\Phi(x)$ lassen sich einer Tabelle entnehmen
