\section{Zufallsvariablen}
\begin{tabular}{@{}p{\linewidth/2-1em}%
				@{}|p{\linewidth/2-1em}}
	\multicolumn{1}{c}{\textbf{Diskret}} & \multicolumn{1}{|c}{\textbf{Stetig}}\\\hline
	Nur abzählbare Werte	& Nur Werte in einem Interval\\
	Verteilungsfunktion: $X \leq x_i$
	\begin{multline*}
		F_X(x) = P(X \leq x)\\ = \sum_{x_i \leq x} f_X(x_i)
	\end{multline*}
	Die Grenzwerte streben gegen 1 bzw. gegen 0 für $x \rightarrow \pm \infty$
	&
	Verteilungsfunktion
	\begin{multline*}
		F_X(x) = P(X \leq x)\\ = \int_{-\infty}^{x} f(u)\,\mathrm du
	\end{multline*}
	\\
	Wahrscheinlichkeitsfunktion (Zähldichte) 
	\begin{multline*}
		P(X = x_i) = f_X(x_i)\\
		f_X(x_i) \geq 0 \mbox{ und } \sum f_X(x_i) = 1\\
		P(a \leq X \leq b)  = \sum_{a \leq x_i \leq b} f_X(x_i)
	\end{multline*}
	&
	Dichtefunktion (1te Ableitung der Verteilungsfkt.)
	\begin{equation*}
		f_X(x) = \frac{\mathrm dF_x(x)}{\mathrm dx}
	\end{equation*}
	\\
	Erwartungswert (artithm. Mw.)
	\begin{equation*}
		\mu \equiv E(x) =\sum_i f_X(x_i) \cdot x_i
	\end{equation*}
	&
	Erwartungswert
	\begin{equation*}
		E(X) = \int_{-\infty}^{\infty} x \cdot f_X(x) \, \mathrm dx
	\end{equation*}
	\\
	Varianz
	\begin{multline*}
		\sigma^2 = E(X - E(X))^2 \\=  \sum_i (x_i - E(X))^2 f_X(x_i)\\ = E(X^2) - (E(X)^2)
	\end{multline*}
	&
	Varianz
	\begin{multline*}
		\sigma^2 = \int_{-\infty}^{\infty} (x - E(X)^2) \cdot f_X(x) \, \mathrm dx\\ = \int_{-\infty}^{\infty} x^2 \cdot f_X(x)\, \mathrm dx - (E(x))^2
	\end{multline*}
\end{tabular}

\section{Beschreibende Statistik}
\begin{description}
	\item [Histogramm] Säulendiagramm. Höhe der Säulen widerspiegelt die relative oder absolute Häufigkeit, die $x$-Achse zeigt die einzelnen Ereignisse
	\item [Streuungsdiagrammm] Jedem Datenpaar ($x_i, y_i$) wird ein Punkt der $x, y$-Ebene zugeordnet
	\item [Lineare Regression] geht von einem linearen Zusammenhang zwischen den Merkmalen $X$ und $Y$ aus: $Y = \alpha + \beta\cdot X + \varepsilon$. Wobei $\varepsilon$ ein Störterm ist. $X$ heisst Regressor, $Y$ heisst Regressand (die zu erklärende Variable)
\end{description}

