\section{Natürliche Zahlen $\N$}
\subsection{Peano-Axiome}
\begin{enumerate}
\item Die 0 ist eine natürliche Zahl: $0 \in \N$
\item Der Nachfolger einer natürliche Zahl $n$ sei $\sigma(n)$: $n \in \N \Rightarrow \sigma(n) \in \mathbb{N}$
\item Die 0 ist kein Nachfolger: $\forall n \in \N \colon \sigma(n) \neq 0$
\item Zwei verschiedene Zahlen haben verschiedene Nachfolger: $\forall n, m \in \N: (n \neq m) \Rightarrow (\sigma(n) \neq \sigma(m))$
\item Für jede Teilmenge $M$ von $\mathbb{N}$ gilt, wenn $M$ die folgenden Eigenschaften erfüllt:
	\begin{enumerate}
		\item 0 ist in $M$
		\item Für jedes $n$ in $M$ ist auch $\sigma(n)$ in $M$
	\end{enumerate}
	dass $M = \N$. Formal: $0\in M \wedge \forall n \in \N \colon (n \in M \Rightarrow \sigma(n)\in M) \Rightarrow M \subseteq \N$\\
	(Schliesst parallele Strukturen aus.) Induktionsprinzip.
\end{enumerate}

\subsection{Addition}
Die Addition ist rekursiv definiert:
\begin{enumerate}
	\item Eine Zahl + 0 ist wieder die Zahl selbst. $n + 0 = n$
	\item Wird zu einer Zahl, der Nachfolger einer anderen Zahl addiert, ist das so, wie wenn
	erst die beiden Zahlen addiert werden und dann der Nachfolger gebildet wird: $m + \sigma(n) = \sigma(m + n)$
\end{enumerate}

\subsection{Vollständige Induktion}
\begin{equation*}
(E(0)
\wedge \forall n \in \N (E(n) \Rightarrow E(n + 1)))
\Leftrightarrow \forall n \in \N (E(n))
\end{equation*}

Man zeigt etwas für die 0, anschliessend nimmt man an, dass wenn es für eine
natürliche Zahl gilt, dann auch für deren Nachfolger. Gilt es für 0 und alle Nachfolger, gilt es für alle.

Zu zeigen: $\forall n \in \N \colon n + 0 = n$\\
Problem: bisherige Theoreme lassen sich nicht anwenden.\\
\begin{proof}[Beweis: Vollständige Induktion über $n$]
	\begin{align*}
		n	& = 0	& \mbox{(Induktionsanfang)}\\
		0 + 0& = 0	& \mbox{(Addition 1: Für die 0 bewiesen)}\\
	\end{align*}
	Induktionschritt: Wir schliessen von $n = k$ auf $n = \sigma(k)$
	\begin{align*}
		k + 0 	& = k	& \mbox{Induktionsannahme; für $k$ gilt es bereits}\\
		\sigma(k) + 0	& = \sigma(k)		& \mbox{(Neu zu zeigen: gilt für Nachfolger)}\\
		\sigma(k) + 0	& = \sigma(k + 0)	& \mbox{(Addition 2)}\\
						& = \sigma(k)		& \mbox{(Induktionsannahme: für $\sigma(k)$ bewiesen)}
	\end{align*}
\end{proof}
Der Trick ist, die Induktionsannahme im Beweis zu verwenden. Es wird nur angenommen, dass die Formel für eine Zahl gilt.
Bewiesen wird dann, wenn die Formel für eine Zahl gilt, gilt sie auch für alle anderen.

\subsubsection{Direkter Beweis}
Zu zeigen: $\forall n \in \N \colon n + 1 = \sigma(n)$
\begin{proof}
	Sei $n$ fest, aber beliebig
	\begin{align*}
	n + 1	& = n + \sigma(0) &\mbox{Definition der 1}\\
			& = \sigma(n + 0)	& \mbox{Addition 2}\\
			& = \sigma(n)		& \mbox{Addition 1}
	\end{align*}
\end{proof}

% Rechenbeispiel: $1 + 2 = ?$
% \begin{align*}
% 	1 + 2	& = \sigma(0) + \sigma(\sigma(0))			& \mbox{Einsetzen der Definitionen}\\
% 			& = \sigma(\sigma(0) + \sigma(0))			& \mbox{Addition 2 auf das Ganze}\\
% 			& = \sigma(\sigma(\sigma(0) + 0))			& \mbox{Addition 2 in der Klammer}\\
% 			& = \sigma(\sigma(\sigma(0)))				& \mbox{Addition 1 auf die innere Klammer}\\
% 			& = 3										& \mbox{Einsetzen der Definition}
% \end{align*}
% Durch mehrfaches Anwenden von Addition 2 wird nach und nach ein $\sigma$ nach vorne geholt. Dieser Teil ist rekursiv.


\subsubsection{Summen}
\begin{align*}
	\sum_{i=1}^0 a_i& = 0\\
	\sum_{i=1}^{n+1} a_i& = \left( \sum_{i=1}^n a_i\right ) + a_{n+1}\\
\end{align*}

\subsubsection{Die Gaussche Summenformel}
\begin{equation*}
	1 + 2 + \dots + n = \Sigma_{i = 1}^{n} i = \frac{n \cdot (n+1)}{2}
\end{equation*}
Zu zeigen: $\forall n \in \N\textbackslash\{0\} \colon \Sigma_{i = 1}^n i = \frac{n \cdot (n + 1)}{2}$
\begin{proof}
	Induktion über $n$
	\begin{align*}
		n& = 1 &\mbox{(Induktionsanfang)}\\
		\Sigma_{i=1}^1 i & = 1 = \frac{2}{2} = \frac{1 \cdot 2}{2} = \frac{1 \cdot(1 + 1)}{2} &\mbox{Beweis für 1}
	\end{align*}
	Induktionsschritt: Wir schliessen von $n = k$ auf $n = k + 1$
	\begin{align*}
		\Sigma_{i=1}^k i& = \frac{k \cdot (k + 1)}{2} &\mbox{Induktionsannahme}\\
		\Sigma_{i=1}^{k+1} i& = \frac{(k + 1) \cdot ((k + 1) + 1)}{2} &\mbox{Neu zu zeigen}\\
		\Sigma_{i=1}^{k+1} i& = \Sigma_{i=1}^{k} i + (k + 1)&\mbox{Summe um einen Term verkürzt}\\
		& = \frac{k \cdot (k + 1)}{2} + (k + 1)& \mbox{Induktionsannahme}\\
		& = \frac{k \cdot (k + 1)}{2} + \frac{2 \cdot (k + 1)}{2} & \mbox{Rechts um zwei erweitert}\\
		& = \frac{k \cdot (k + 1) + 2 \cdot (k + 1)}{2}& \mbox{Zusammenzug}\\
		& = \frac{(k + 1) \cdot (k + 2)}{2} & k + 1 \mbox{ausklammern}\\
		& = \frac{(k + 1) \cdot ((k + 1) + 1)}{2} & \mbox{Terme vertauscht}
	\end{align*}
\end{proof}

\subsubsection{Rechenregeln}
\settowidth{\MyLenA}{Neutrales Element~~}
\begin{tabular}{@{}p{\the\MyLenA}%
				@{}p{(\linewidth - \the\MyLenA)/2}}
	Neutrales Element 		& $0 + n = n$\\
	Kommutativität			& $n + m = m + n$\\
	Assoziativität			& $(n + m) + k = n + (m + k)$\\
	Kürzbarkeit				& $(n + k = m + k) \Rightarrow n = m$\\
\end{tabular}

\subsection{Multiplikation}
Die Multiplikation ist rekursiv definiert:
\begin{enumerate}
	\item Eine Zahl mal 0 ist 0: $n \cdot 0 = 0$\\
	\item Wird eine Zahl mit dem Nachfolger einer anderen Zahl multipliziert, ist das so, wie wenn
	erst die beiden Zahlen multipliziert und anschliessend die ursprüngliche Zahl addiert wird: $m \cdot \sigma(n) = \sigma(m \cdot n) + m$
\end{enumerate}

\subsubsection{Potenzen}
\begin{enumerate}
	\item Eine Zahl hoch 0 ist 1: $n^0 = 1$
	\item Eine Zahl hoch dem Nachfolger einer anderen Zahl ist die eigentliche Zahl multipliziert mit der eigentlichen Zahl hoch der anderen Zahl: $m^{\sigma(n)} = m \cdot (m^n)$
\end{enumerate}

\subsubsection{Fakultät}
\begin{enumerate}
	\item $0! = 1$
	\item $\sigma(m)! = \sigma(m) \cdot m!$
\end{enumerate}

Beispiel: $n! > 2^n$ für $n \geq 4$
\begin{proof}
	Anfang: $4! = 24 > 16 = 2^4$\\
	Annahme: $k! > 2^k$\\
	Zu zeigen: $(k + 1)! > 2^{k + 1}$
	\begin{equation*}
		(k + 1)! = (k + 1) \cdot k! > (k + 1) \cdot 2^k > 2 \cdot 2^k = 2^{k+1}
	\end{equation*}
\end{proof}
Beweisstrategie: Von Links und rechts der Mitte hin annähern.


\subsubsection{Rechenregeln}
\settowidth{\MyLenA}{Neutrales Element~~}
\begin{tabular}{@{}p{\the\MyLenA}%
				@{}p{(\linewidth - \the\MyLenA)/2}}
	Absorbtion				& $0 \cdot n = 0$\\
	Neutrales Element 		& $1 \cdot n = n$\\
	Kommutativität			& $n \cdot m = m \cdot n$\\
	Assoziativität			& $(n \cdot m) \cdot k = n \cdot (m \cdot k)$\\
	Distributivität			& $n \cdot (m + k) = n \cdot m + n \cdot k$\\
\end{tabular}
Partialsummen
\begin{equation*}
	\sum_{i=1}^n (c \cdot(a_1 + b_1)) = c \left (\sum_{i=1}^n a_i + \sum_{i=1}^n b_i \right )
\end{equation*}

\subsubsection{Produkte}
\begin{align*}
	\prod_{i=1}^0 a_i& = 1\\
	\prod_{i=1}^{n + 1} a_i& = a_{n+1} \cdot \prod_{i=1}^n a_i\\
\end{align*}

\subsection{Die Ordnung der natürlichen Zahlen}
Es sei $A$ eine Menge und $\preceq$ eine Ordnung auf $A$.
\begin{enumerate}
	\item Sei $X \subset A$, dann ist $m \in X$ dass minimale Element, wenn gilt: $\forall x \in X (m \preceq x)$.
	\item Das Paar $(A, \preceq)$ heisst Wohlordnung, wenn jede nichtleere Teilmenge $X \subset A$ ein minimales Element besitzt.
\end{enumerate}
Die 0 ist die kleinste natürliche Zahl.

\subsubsection{Relationen}
\settowidth{\MyLenA}{$a \leq b$ ~~}
\begin{tabular}{@{}p{\the\MyLenA}%
				@{}p{(\linewidth - \the\MyLenA)/2}}
	$a \leq b$				& $\Leftrightarrow \exists k \in \N (m = k + n)$\\
	$a < b$ 				& $\Leftrightarrow (n \leq m \wedge n \neq m)$\\
\end{tabular}

Die Kleiner-als-Relation wird rekursiv definiert:
\begin{enumerate} 
	\item Keine Zahl ist kleiner als null: $\forall n \in \N \colon \neg n < 0$
	\item Eine Zahl ist kleiner als der Nachfolger einer anderen Zahl genau dann, wenn beide Zahlen gleich gross oder die erste Zahl kleiner als die zweite ist:
		$\forall m, n \in \N \colon m < \sigma(n) \Leftrightarrow (m = n) \vee (m < n)$
\end{enumerate}
Zu zeigen: $3 < 5$
\begin{proof}
	\begin{align*}
		(3 < 5) & = (3 < \sigma(4))				& \mbox{(Definition der 5)}\\
		& \Leftrightarrow (3 = 4) \vee (3 < 4)	& \mbox{(Kleiner 2: links falsch)}\\
		(3 < 4) & = (3 < \sigma(3))				& \mbox{(Definition der 4)}\\
		& \Leftrightarrow (3 = 3) \vee (3 < 3)	& \mbox{(Kleiner 2: (3 = 3) ist wahr)}
	\end{align*}
	\qedhere
\end{proof}

Das Paar $(\N, \leq)$ ist eine Wohlordnung.

\subsubsection{Rechenregeln}
\begin{align*}
	n < m& \Leftrightarrow (n + 1) \leq m\\
	n < m& \Leftrightarrow (n + k) < (m + k)\\
	n \leq m& \Leftrightarrow (n + k) \leq (m + k)\\
	(n \leq n') \wedge (m \leq m')& \Leftrightarrow (n + m) \leq (n' + m')\\
\end{align*}
