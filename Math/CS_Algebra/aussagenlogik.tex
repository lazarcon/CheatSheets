\section{Logik}
\subsection{Aussagenlogik}
\textbf{Aussage} Eine Aussage ist eine sprachliche Äusserung, 
die wahr ($w$) oder falsch($f$) sein kann.\\
Bsp: \enquote{Es gibt unendlich viele natürliche Zahlen.}

\subsubsection{Junktoren (Verknüpfungsoperatoren)}
\begin{tabular}{cl}
	$\neg$				& Negation, \enquote{nicht \dots}\\
	$\wedge$ 			& Konjunktion, \enquote{\dots\ und \dots}\\
	$\vee$ 				& Disjunktion, \enquote{\dots\ oder \dots}\\
	$\Rightarrow$ 		& Implikation, \enquote{wenn \dots, dann \dots}\\
	$\Leftrightarrow$	& Äquivalenz, \enquote{\dots\ genau dann, wenn \dots} \\
	$\bigoplus$			& Antivalenz, \enquote{entweder \dots oder  \dots}\\
\end{tabular}\\

Für eine einzelne Aussage $A$ gilt:\\
\begin{center}
\begin{tabular}{c||c}
	$A$ & $\neg A$ \\\hline
	$f$ & $w$ \\
	$w$ & $f$ \\
\end{tabular}
\end{center}

Für zwei Ausagen $A$ und $B$ gilt:
\begin{center}
\begin{tabular}{c|c||c|c|c|c|c}
	$A$ & $B$ 	& $A \wedge B$ 	& $A \vee B$ 	& $A \Rightarrow B$ 	& $A \Leftrightarrow B$ 	& $A \bigoplus B$ \\\hline
	$f$ & $f$ 	& $f$				& $f$			& $w$				& $w$					 	& $f$\\
	$f$ & $w$ 	& $f$				& $w$			& $w$				& $f$					 	& $w$\\	
	$w$ & $f$ 	& $f$				& $w$			& $f$				& $f$						& $w$\\
	$w$ & $w$ 	& $w$				& $w$			& $w$				& $w$						& $f$\\
\end{tabular}\\
\end{center}
Bindungsstärke: $\neg$ vor $\wedge$ vor $\vee$ vor $\Rightarrow$ vor $\Leftrightarrow$.\\

\subsubsection{Rechenregeln}
\settowidth{\MyLenA}{Doppelte Negation~~}
\begin{equation*}
\begin{array}{lrl}
	\mbox{Duplizität}			& A \wedge A				& \Leftrightarrow A\\
								& A \vee A					& \Leftrightarrow A\\
	\mbox{Doppelte Negation}	& \neg\neg A 				& \Leftrightarrow A\\
	\mbox{Kommutativität}		& A \wedge B 				& \Leftrightarrow B \wedge A\\
								& A \vee B					& \Leftrightarrow B \vee A\\
	\mbox{Assoziativität} 		& (A \wedge B) \wedge C 	& \Leftrightarrow A \wedge (B \wedge C)\\
								& (A \vee B) \vee C 		& \Leftrightarrow A \vee (B \vee C)\\
	\mbox{Distributiviät}		& A \wedge (B \vee C) 		& \Leftrightarrow (A \wedge B) \vee (A \wedge C)\\
								& A \vee (B \wedge C) 		& \Leftrightarrow (A \vee B) \wedge (A \vee C)\\
	\mbox{De Morgan Regeln}		& \neg (A \wedge B) 		& \Leftrightarrow \neg A \vee \neg B\\
								& \neg (A \vee B) 			& \Leftrightarrow \neg A \wedge \neg B\\
	\mbox{Implikation}			& (A \Rightarrow B) 		& \Leftrightarrow (\neg A \vee B)\\
	\mbox{Kontraposition}		& (A \Rightarrow B) 		& \Leftrightarrow (\neg B \Rightarrow \neg A)\\
	\mbox{Äquivalenz}			& (A \Leftrightarrow B)		& \Leftrightarrow (A \Rightarrow B) \wedge (B \Rightarrow A)\\
	\mbox{Absorbtion}			& A \wedge (A \vee B)		& \Leftrightarrow A\\
								& A \vee (A \wedge B)		& \Leftrightarrow A\\
\end{array}
\end{equation*}

\subsection{Prädikatenlogik}
\textbf{Aussageform} Eine Aussageform ist eine sprachliche Äusserung, in 
der Variablen vorkommen und die in Abhängigkeit der Variablenwerte wahr ($w$) oder falsch ($f$) 
sein kann -- Aussageformen sind manchmal wahr, manchmal falsch.\\
Bsp: \enquote{Die Zahl $x$ ist eine gerade Zahl.}

Es wird unterschieden zwischen Objekt und Prädikat (Eigenschaft): für oberes Beispiel ist \enquote{ist gerade}
das Prädikat, während $x$ das Objekt ist.

\subsubsection{Quantoren (Variablenbinder)}
$A(x)$ sei eine Aussageform, $M$ eine Menge von Objekten.
\settowidth{\MyLenA}{$\forall x \in M A(x))$~~}
\begin{tabular}{@{}p{\the\MyLenA}%
				@{}p{\linewidth-\the\MyLenA}}
	$\forall x \in M A(x)$ & Für alle $x$ der Menge $M$ gilt $A(x)$\\
	$\exists x \in M A(x)$ & Für mindestens ein $x$ der Menge $M$ gilt $A(x)$\\
\end{tabular}

\subsubsection{Besondere Ausdrücke}
\begin{tabular}{ll}
	Für mindestens zwei gilt~~ & $\exists x\, \exists y ((A(x) \wedge A(y)) \Rightarrow (x \neq y))$\\
	Es gibt höchstens ein & $\forall x\, \forall y\,((A(x) \wedge A(y)) \Rightarrow (x = y))$\\
\end{tabular}
Beispiel: Prädikat: $L$ = $x$ liebt $y$
\begin{itemize}\itemsep0em
	\item Jeder wird von jemanden geliebt: $\forall x\,\exists y\,Lyx$
	\item Jeder liebt jemanden: $\forall x\, \exists y\, Lxy$
	\item Jemand liebt alle: $\exists x\, \forall y\, Lxy$
	\item Jemand wird von allen geliebt: $\exists x\, \forall y\, Lyx$
	\item Jemand liebt sich selbst: $\exists x\, Lxx$
	\item Alle lieben sich selbst: $\forall x\, Lxx$
	\item Einer liebt einen: $\exists x\, \exists y\, Lxy$
	\item Einer wird geliebt: $\exists x\, \exists y\, Lyx$
	\item Jeder liebt jeden: $\forall x\, \forall y\, Lxy$
	\item Jeder wird von jedem geliebt: $\forall x\, \forall y\, Lyx$
\end{itemize}

\subsubsection{Rechenregeln}
\begin{equation*}
	\forall x A(x) \Leftrightarrow \neg \exists x \neg A(x)
\end{equation*}
\enquote{Für alle $x$ gilt $A(x)$} ist äquivalent zu: \enquote{Es existiert kein $x$ für das $A(x)$ nicht gilt.}
\begin{equation*}
		\neg \forall x A(x) \Leftrightarrow \exists x \neg A(x)
\end{equation*}
\enquote{Nicht für alle $x$ gilt $A(x)$.} ist äquivalent zu: \enquote{Es existiert ein $x$ für das $A(x)$ nicht gilt.}
\begin{equation*}
			\neg \exists x A(x) \Leftrightarrow \forall x. \neg A(x)
\end{equation*}
\enquote{Es existiert kein $x$ für das $A(x)$ gilt.} ist äquivalent zu: \enquote{Für alle $x$ gilt $A(x)$ nicht.}\\
Rechenregeln mit beschränkten Quantoren:
\begin{equation*}
	\begin{array}{rl}
		\forall x \in M A(x) & \Leftrightarrow \neg \exists x \in M \neg A(x)\\
		\forall x \in M A(x) & \Leftrightarrow \forall x(x \in M \Rightarrow A(x))\\
		\exists x \in M A(x) & \Leftrightarrow \exists x (x \in M \wedge A(x))\\
	\end{array}
\end{equation*}

Die Rechenregel der Aussagenlogik werden mit Hilfe von Wahrheitstafeln bewiesen.

