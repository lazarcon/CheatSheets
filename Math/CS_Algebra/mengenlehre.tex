\section{Mengenlehre}
Eine Menge ist jede Zusammenfassung von bestimmten, wohlunterscheidbarer Objekte
unserer Anschauung oder unseres Denkens zu einem Ganzen.\\
Kein Objekt kann in einer Menge doppelt vorkommen.

\subsection{Symbole}
\settowidth{\MyLenA}{$\mathbb{Q} = \{x\mid x = \frac{a}{b} \wedge a \in \mathbb{Z} \wedge b \in \mathbb{N}$*$\}$~~}
\begin{tabular}{@{}p{\the\MyLenA}%
				@{}p{\linewidth-\the\MyLenA}}
	$x \in M (x \notin M)$ & $x$ ist (kein) Element der Menge $M$ \\
	$x \mid A(x)$ & für $x$ gilt die Aussageform $A(x)$\\
	$\emptyset = \{\}$ & Leere Menge; enthält nichts\\
	$\mathbb{N} = \{0, 1, 2, \dots\}$	& Natürliche Zahlen\\
	$\mathbb{N}^* = \{1, 2, 3, \dots\}$	& Positive ganze Zahlen\\
	$\mathbb{Z} = \{0, \pm1, \pm2, \dots\}$	& Ganze Zahlen\\
	$\mathbb{Q} = \{x\mid x = \frac{a}{b} \wedge a \in \mathbb{Z} \wedge b \in \mathbb{N}$*$\}$ & Rationale Zahlen (=Brüche)\\
	$\mathbb{R}$ & Reelle Zahlen (= alle Zahlen)\\
	$\mathbb{C}$ & Komplexe Zahlen\\
\end{tabular}

\subsection{Intervalle}
Fortlaufende Mengen lassen sich als Intervalle schreiben: $(a,b) = \{x \mid a \leq x \leq b\}$.
Runde Klammern bedeuten exklusive, eckige Klammern inklusive. $\infty$ hat immer eine Runde Klammer zur Folge.

\subsubsection{Beispiele für Mengendefinitionen}
\begin{tabular}{ll}
	Aufzählende Notation & $M = \{x_1, x_2, \dots, x_n\}$\\
	Intensionale Notation & $M = \{x \in X \mid A(x)\}$ 
\end{tabular}\\
Beispiel geraden Zahlen: $G = \{n \in \mathbb{N} \mid \exists m \in \mathbb{N}~2 \cdot m = n\}$ \\

\subsection{Mächtigkeit}
Die Mächtigkeit $|M|$ entspricht im Prinzip der Anzahl Elemente einer Menge. Allerdings nur im Prinzip,
weil: $|\mathbb{N}^*| = |\mathbb{N}| = |\mathbb{Z}| = |\mathbb{Q}| < |\mathbb{R}|$

\subsection{Mengenoperationen}
\settowidth{\MyLenA}{$C = A \cup B$~~}
\begin{tabular}{@{}p{\the\MyLenA}%
				@{}p{\linewidth-\the\MyLenA}}
	$A \cup B$	& Vereinigungsmenge: $\forall x \in (A \cup B) \Leftrightarrow (x \in A \vee x \in B)$ \\
	$A \cap B$	& Schnittmenge: $\forall x \in (A \cap C) \Leftrightarrow (x \in A \wedge x \in B)$\\
	$A \backslash B$ & Differenzmenge: $\forall x \in (A \backslash B) \Leftrightarrow ((x \in A) \wedge (x \notin B))$\\
\end{tabular}
Disjunkt (elementfremd) sind zwei Mengen wenn, $A \cap B = \emptyset$.

\subsection{Mengenrelationen}
\settowidth{\MyLenA}{$C = A \cup B$~~}
\begin{tabular}{@{}p{\the\MyLenA}%
				@{}p{\linewidth-\the\MyLenA}}
	$A = B$	& Gleichheit: $(A = B) \Leftrightarrow (\forall x (x \in A) \Leftrightarrow (x \in B))$\\
	$A \subseteq B$	& Teilmenge: $(A \subseteq B) \Leftrightarrow (\forall x (x \in A) \Rightarrow (x \in B))$\\
	$A \subsetneq B$ & Echte Teilmenge: $(A \subsetneq B) \Leftrightarrow ((A \subseteq B) \wedge (A \neq B))$
\end{tabular}

\subsection{Rechenregeln}
\settowidth{\MyLenA}{Doppelte Negation~~}
\begin{equation*}
\begin{array}{lrl}
	\mbox{Kommutativität}		& A \cup B 									& = B \cup A\\
								& A \cap B									& = B \cap A\\
	\mbox{Assoziativität} 		& (A \cup B) \cup C 						& = A \cup (B \cup C)\\
								& (A \cap B) \cap C 						& = A \cap (B \cap C)\\
	\mbox{Distributiviät}		& A \cup (B \cap C) 						& = (A \cup B) \cap (A \cup C)\\
								& A \cap (B \cup C) 						& = (A \cap B) \cup (A \cap C)\\
	\mbox{De Morgan Regeln}		& (C \backslash A) \cup (C \backslash B) 	& = C \backslash (A \cap B)\\
								& (C \backslash A) \cap (C \backslash B) 	& = C \backslash (A \cup B)\\
	\mbox{Idempotenzgesetz}		& A \cup A 									& = A\\
								& A \cap A									& = A
\end{array}
\end{equation*}

Beweise in der Mengenlehre werden durch Umwandlung zu logischen Aussagen geführt.
Bsp: Zu zeigen: $\forall x \in (A \cap B) \cap C \Leftrightarrow x \in A \cap (B \cap C)$
\begin{proof}
Sei $x$ fest, aber beliebig.
\begin{align*}
	x \in (A \cap B) \cap C	& \Leftrightarrow x \in (A \cap B) \wedge x \in C\\
							& \Leftrightarrow (x \in A \wedge x \in B) \wedge x \in C\\
							& \Leftrightarrow x \in A \wedge (x \in B \wedge x \in C)\\
							& \Leftrightarrow x \in A \wedge x \in (B \cap C)\\
							& \Leftrightarrow x \in A \cap (B \cap C)\qedhere
\end{align*}
\end{proof}

\subsection{Mengenbildung}
Sei $A$ ein Menge von Mengen $A = \{X, Y\}$ mit $X = \{x_1, x_2, \dots, x_n\}, Y = \{y_1, y_2, \dots, y_n\}$\\

\subsubsection{Vereinigung}
Die Vereinigung $\bigcup A$ ist die Menge aller Elemente aller in $A$ enthaltener Mengen: 
$\bigcup_{i \in I} Y_i := \bigcup A$, wobei $I$ eine Indexmenge und $A = \{Y_i \mid i \in I\}$
\begin{equation*}
	x \in \bigcup A \Leftrightarrow \exists Y \in A (x \in Y)
\end{equation*}

\subsubsection{Schnittmenge}
Die Schnittmenge $\bigcap A$ ist die Menge aller Elemente die in jeder in $A$ enthaltener Menge vorkommt:
$\bigcap_{i \in I} Y_i := \bigcap A$, wobei $I$ eine Indexmenge und $A = \{Y_i \mid i \in I\}$
\begin{equation*}
	x \in \bigcap A \Leftrightarrow \forall Y \in A (x \in Y)
\end{equation*}

\subsubsection{Potenzmenge}
Die Potenzmenge enthält alle Teilmengen von $A$ als Elemente.
\begin{equation*}
	\mathcal{P}(A) := \{x \mid x \subseteq A\}
\end{equation*}
Es gelten:
\begin{align*}
	|P(A)|& = 2^{|A|}\\
	A& = \bigcup \mathcal{P}(A)\\
	\cap \mathcal{P}(A)& = \emptyset\\
	\mathcal{P}(A \cap B)& = \mathcal{P}(A) \cap \mathcal{P}(B) \\
	\mathcal{P}(A \cup B)& \neq \mathcal{P}(A) \cup \mathcal{P}(B)
\end{align*}

\subsubsection{Kartesisches Produkt (Kreuzprodukt)}
Das Kreuzprodukt zweier Mengen ist die Menge aller möglichen Kombination von Elementen aus der ersten
Menge mit Elementen aus der zweiten Mengen. Die Ergebnismenge besteht aus Tupeln, wobei jedes Tupel
eine mögliche Kombination darstellt. Diese 2-Tupel heissen geordnete Paare.
\begin{equation*}
		A \times B = \{(a, b) \mid a \in A \wedge b \in B\} \neq B \times A
\end{equation*}
Die Reihenfolge ist dabei relevant. Beispiel: $A = \{0, 1, 2\}, B = \{s, t\}$\\
$ A \times B = \{(0, s), (0, t), (1, s), (1, t), (2, s), (2, t)\}$
Die Anzahl Elemente des Kreuzproduktes errechnet sich nach $|K| = |A| \cdot |B|$.

\subsubsection{$n$-Tupel}
Ein $n$-Tupel ist ein Term der Form $(a_1, a_2, \dots, a_n)$.
Zwei Tupel sind gleich, wenn: $(a_1, \dots, a_n) = (b_1, \dots, b_n) \Leftrightarrow (a_1 = b_1) \wedge \dots \wedge (a_n = b_n)$