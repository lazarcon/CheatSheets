\section{Lösen von Gleichungen}
Es gibt keinen allgemeinen Ansatz. Lässt sich eine Gleichung auch als Funktion ausdrücken, kommen die Regeln der Nullstellen-Bestimmung des Funktionstyps zu Anwendung.

\subsection{Gleichungssysteme}
In Gleichungssystemen liegen $m$ Gleichungen mit $n$ Variablen vor. Sie besitzen keine, eine oder unendlich viele Lösungen.

\subsubsection{Lineare Gleichungssysteme}
Allgemeine Form:
\begin{equation*}
\begin{matrix}
a_{11} x_1 +  a_{12} x_2 \, + & \cdots & +\, a_{1n} x_n & = & b_1\\
a_{21} x_1 +  a_{22} x_2 \, + & \cdots & +\, a_{2n} x_n & = & b_2\\
&&&\vdots&\\
a_{m1} x_1 +  a_{m2} x_2 \, + & \cdots & +\, a_{mn} x_n &a = & b_m\\
\end{matrix}
\end{equation*}

Lösungsansatz:\\
1. Durch Umformen Unbekannte eliminieren.\\
2. Einsetzen der Umformung.\\

Eliminierungsregeln:\\
1. Gleichungen dürfen miteinander vertauscht werden.\\
2. Gleichungen dürfen mit $q \in \mathbb{Q} \backslash \{0\}$ multipliziert werden.\\
3. Zu jeder Gleichung darf darf eine andere Gleichung addiert oder subtrahiert werden.\\

\subsection{Nicht-lineare Gleichungssysteme}
Nicht lineare Gleichungssystem enthalten mindestens einen nicht-linearen Term. Sie lassen sich nur numerisch lösen.
