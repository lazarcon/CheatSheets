\section{Arithmetik}
\subsection{Regeln}
\settowidth{\MyLenA}{Kommutativität~~}
\begin{tabular}{@{}p{\the\MyLenA}%
				@{}p{\linewidth-\the\MyLenA}}
	Kommutativität	& $a + b = b + a$\\
					& $ab = ba$\\
	Assoziativität 	& $a + (b + c) = (a + b) + c$\\
					& $a(bc) = (ab)c$\\
	Distributiviät	& $a(b + c) = ab + ac$\\
\end{tabular}

\subsection{Brüche}
Terminologie: $\frac{\mbox{Divident(Zähler)}}{\mbox{Divisor(Nenner)}} = \mbox{Quotient(Bruch)}$\\
Division ist nur erlaubt für $(\mbox{Divisor} \neq 0)$!
\settowidth{\MyLenA}{Addition/Subtraktion~~}
\begin{tabular}{@{}p{\the\MyLenA}%
				@{}p{\linewidth-\the\MyLenA}}
	Erweitern				& $\frac{a}{b} = \frac{a \cdot u}{b \cdot u}$\\
	Kürzen					& $\frac{a \cdot u}{b \cdot u} = \frac{a \cdot \msout{u}}{b \cdot \msout{u}} = \frac{a}{b}$\\
	Addition/Subtraktion	& $\frac{a}{b} \pm \frac{c}{d} = \frac{ad \pm c \cdot b}{d \cdot b}$\\
	Multiplikation 			& $\frac{a}{b} \cdot \frac{c}{d} = \frac{a \cdot c}{b \cdot d}$\\
	Division				& $\frac{a}{b} \div {\frac{c}{d}} = \frac{a \cdot d}{b \cdot c}$\\
\end{tabular}

\subsection{Beträge}
$\left|a \pm b\right| \leq \left|a\right| + \left|b\right|$ (Dreiecksungleichung)\\
$\left|a\right| - \left|b\right| \leq \left|a\right| + \left|b\right|$\\
$\left|a_{1} + a_{2} + \dots + a_{n}\right| \leq \left|a_{1}\right| + \left|a_{2}\right| + \dots + \left|a_{n}\right|$\\
$\left|ab\right| = \left|a\right| \cdot \left|b\right|$\\
$\left|\frac{a}{b}\right| = \frac{\left|a\right|}{\left|b\right|}$
$\left|x\right| = a \Leftrightarrow x_{1,2} = \pm a\,\,\,(a > 0)$\\

\subsection{Binomischer Lehrsatz}
Ein Binom ist die Summe zweier Glieder der allgemeinen Form $(a + b)$.\\
Es gilt: $(a + b)^{n} = \sum_{k=0}^n{\binom{n}{k} a^{n-k} \cdot b^{k}}$\\
Wobei: $\binom{n}{k} = \frac{n(n - 1)(n - 2)\cdots[n - (k - 1)]}{k!}\,(0 < k \leq n)$
Die Ergebnisse des Binominalkoeffizienten lassen sich aus dem Pascalschen Dreieck ablesen, wobei der Wert in Zeile $(n + 1)$ und Spalte $(k + 1)$ steht.
