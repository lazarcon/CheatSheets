\section{Ungleichungen}
\subsection{Ungleichungsregeln}
Es muss stehts auf beiden Seiten das gleiche gemacht werden.
\settowidth{\MyLenA}{$+/-$~~}
\begin{tabular}{@{}p{\the\MyLenA}%
				@{}p{\linewidth-\the\MyLenA}}
	$+/-$	& beliebiger Term $T(x)$\\
	$*/:$ 	& beliebiger Term $x (x > 0)$\\
	$*/:$ 	& beliebigen Term $x (x < 0)$ mit Relationszeichenumkehrung.\\ 
\end{tabular}
Ist bei Multiplikation oder Division der Wert des Terms $T(x)$ unbekannt, muss eine Fallunterscheidung vorgenommen und beide Fälle betrachtet werden.
(Fall I: $T(x) > 0$ \dots/Fall II: $T(x) < 0$ \dots). Die anschliessende Lösung darf der Annahme nicht widersprechen, sonst handelt es sich um eine Scheinlösung,
die nicht in die Lösungsmenge gehört. Lösung der Ungleichung und Annahme gehören in die Lösungsmenge.