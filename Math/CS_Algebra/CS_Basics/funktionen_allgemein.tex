\subsection{Allgemein}
Eine Funktion ist eine Vorschrift, die jedem Element $x$ aus eine Menge $D$ genau ein Element $y$ aus einer Menge $W$ zuordnet.
\begin{equation*}
	\begin{matrix}
	f: x \rightarrow y = f(x) & \mbox{mit} & x \in D\\	
	\end{matrix}	
\end{equation*}
Darstellungen:\\
1. Analytisch: $y = f(x)$ (explizit), $F(x;y) = 0$ (implizit)\\
2. Als Wertetabelle\\
3. Graphisch\\
4. Parametrisch (wenn $x = x(t)$ und $y = y(t)$). Die Graphik erhält man aus einer Wertetabelle beginnend mit $t$.

\subsection{Funktionseigenschaften}
Gegeben sei eine Funktion $y = f(x)$.
\settowidth{\MyLenA}{Streng monoton wachsend~~}
\begin{tabular}{@{}p{\the\MyLenA}%
				@{}p{\linewidth - \the\MyLenA}}
Nullstellen & $f(x_0) = 0$\\
Symmetrie (gerade)	& $f(-x) = f(x)$ \\
Symmetrie (ungerade) & $f(-x) = - f(x)$\\
Monoton wachsend & $f(x_1) \leq f(x_2)$ ($x_1 < x_2$) \\
Streng monoton wachsend & $f(x_1) < f(x_2)$ ($x_1 < x_2$)\\
Monoton fallend & $f(x_1) \geq f(x_2)$ ($x_1 < x_2$) \\
Streng monoton fallend & $f(x_1) > f(x_2)$ ($x_1 < x_2$)\\
Periodisch mit Periode $p$ & $f(x \pm p) = f(x)$\\
Umkehrbar & $x_1 \neq x_2 \Rightarrow f(x_1) \neq f(x_2)$\\
\end{tabular}

Funktionen können zu einem Teil wachsend und zu einem anderen Teil fallend sein.\\
Bestimmen der Umkehrfunktion:\\
1. $y = f(x)$ nach $x$ auflösen. Ergebnis: $x = f^{-1}(y)$.\\
2. Vertauschen von $x$ und $y$ im Ergebnis: $y = f^{-1}(x)$.\\
Bei der Umkehrfunktion sind Definitions- und Wertebereich der ursprünglichen Funktion vertauscht.\\
Nur streng monotone Funktionen sind umkehrbar.\\
Die Umkehrfunktion entspricht graphisch der Spiegelung an $y=x$.
\begin{equation*}
	x\mathop{\rightleftarrows}^{f}_{f^{-1}}f(x)
\end{equation*}

\subsection{Grenzwert}
Reelle Zahlenfolge: $\langle{a}\rangle = f(n)$ mit $n \in \mathbb{N}$*. $f(n)$ heisst Bildungsgesetz, die Zahlen: $a_1, a_2, \dots a_n$ heissen Glieder der Folge.\\
Eine reelle Zahl $g$ heisst Grenzwert der Zahlenfolge $\langle{A}\rangle$, wenn es zu jedem $\epsilon > 0$ eine
natürliche Zahl $n_0 > 0$ gibt, so dass für all $n \geq n_0$ stets gilt: $\left|a_n - g\right| < \epsilon$.
Eine Folge heisst \textit{konvergent} wenn sie eine Grenzwert besitzt. Andernfalls heisst sie \textit{divergent}.

Die Funktion $f(x)$ hat an der Stelle $x_0$ einen Grenzwert $g$, wenn gilt
\begin{equation*}
	\lim_{x\to x_0\atop x < x_0}f(x) = \lim_{x\to x_0\atop x > x_0}f(x) = \lim_{x\to x_0}f(x) = g 
\end{equation*}

Lösungsschema zur Bestimmung des Grenzwerts an nicht definierten Stellen:\\
1. Grundsätzlich $x_0$ in $f(x)$ einsetzen. Wenn keine Division durch Null entsteht ist $g = x_0$.\\
2. Bei einer Division durch Null, den Divisior eliminieren.\\
3. Annähern von links und rechts.\\

Polstelle: Der Grenzwert ist $+\infty$ oder $-\infty$.

\subsubsection{Rechenregeln}
\begin{align*}
\lim_{x\to x_0}{(k \cdot f(x))}& = k (\lim_{x\to x_0}{f(x)})\\
\lim_{x\to x_0}{(f(x) \pm g(x))}& = (\lim_{x\to x_0}{f(x)}) \pm (\lim_{x\to x_0}{g(x)})\\
\lim_{x\to x_0}{(f(x) \cdot g(x))}& = (\lim_{x\to x_0}{f(x)}) \cdot (\lim_{x\to x_0}{g(x)})\\
\lim_{x\to x_0}{\frac{(f(x)}{g(x)})}& = \frac{\lim_{x\to x_0}f(x)}{\lim_{x\to x_0}g(x)}\\
\end{align*}
\subsection{Stetigkeit}
Eine Funktion $f(x)$ heisst an der Stelle $x_0$ stetig, wenn der Grenzwert vorhanden ist und mit dem Funktionswert übereinstimmt:
\begin{equation*}
	\lim_{x\to x_0}{f(x)} = f(x_0)
\end{equation*}

Eine Funktion ist an der Stelle $x_0$ unstetig, wenn:\\
1. $f(x)$ an der Stelle $x_0$ nicht definiert ist (Definitionslücke).\\
2. An der Stelle $x_0$ kein Grenzwert vorhanden ist.\\
3. Funktions- und Grenzwert zwar vorhanden, aber verschieden sind.\\

