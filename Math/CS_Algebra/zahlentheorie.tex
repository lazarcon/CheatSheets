\section{Zahlentheorie}
% \subsection{Satz vom kleinsten Teiler}
% Der kleinste Teiler $d > 1$ einer natürlichen Zahl $n \geq 2$ ist eine Primzahl.
% 
% \begin{proof}
% 	Sei $T(n)$ die Menge aller Teiler von $n$.\\
% 	$T(n)\textbackslash\{1\}$ ist nicht leer, weil $n \in T(n)$\\
% 	dann gibt es nach dem Wohlordnungsprinzip 
% 	eine kleinste Zahl $d$ in $T(n)\textbackslash\{1\}$.\\
% 	Zu zeigen: $d$ ist eine Primzahl\\
% 	Annahme: $d$ ist keine Primzahl\\
% 	Dann: $\exists a, b \in \N \setminus\{1, d\} \colon a \cdot b = d \wedge a, b < d$\\
% 	$\Rightarrow a \mid d$\\
% 	Da $a$ kleiner als $d$ sein muss, kann $d$ nicht das kleinste Element in $T(n)\textbackslash{1}$ sein.\\
% 	Das ist ein Widerspruch zur Annahme. Die Annahme muss falsch sein, dann muss aber $d$ eine Primzahl sein.
% 	\qedhere
% \end{proof}

% \subsection{Hauptsatz}
% Jede natürliche Zahl $n \geq 2$ besitzt eine eindeutige Primfaktorzerlegung.
% Die Reihenfolge der Primfaktoren kann variieren.
% 
% \begin{proof}
% 	Sei $n \in \N \setminus\{1\}$\\
% 	Fall I: $n$ ist eine Primzahl, ein eindeutiger Primfaktor gefunden ($n$).\\
% 	Fall II: $n$ ist keine Primzahl\\
% 	\begin{equation*}
% 		\Rightarrow \exists d_1, q_1 \in \N \setminus\{1\} \colon d_1 \cdot q_1 = n
% 	\end{equation*}
% 	und $d_1$ ist das kleinste Element in $T(n)\setminus\{1\}$, also $d_1$ ist Primzahl\\
% 	Betrachte $q_1$:
% 	Fall I: $q_1$ ist eine Primzahl, eindeutige Primfaktorzerlegung gefunden.\\
% 	Fall II: $q_1$ ist keine Primzahl, dann wiederhole diesen Prozess mit $q_1$\\ 
% 	\qedhere
% \end{proof}
% Es gibt also eine Primfaktorzerlegung.
% Bleibt zu beweisen, dass es nur eine Primfaktorzerlegung gibt.
% \begin{proof}
% 	Sei $n$ die kleinste natürliche Zahl mit mehr als einer Primfaktorzerlegung.\\
% 	\begin{align*}
% 	n& = p_1 \cdot p_2 \cdot \dots \cdot p_n = q_1 \cdot q_2 \cdot \dots \cdot q_m & p, q \in \mathbb{P}\\
% 	\mbox{$p_i$ und $q_j$ sind paarweise verschieden}	
% 	\end{align*}
% 
% \end{proof}

\subsection{Primzahlen}
Primzahlen sind folgendermassen definiert:
\begin{align*}
 \forall n, m \in \N& (P|n \cdot m \Rightarrow p |n \wedge p | m) \mbox{ und } p \neq 1\\
  T(P) & = \{1, p\} \mbox{ und } p \neq 1\\
  |T(P)|& = 2\\
\end{align*}
\begin{itemize}\itemsep0em
\item Es gibt unendlich viele Primzahlen!
\item Jede natürliche Zahl $n \in \N$ ist das Produkt endlich vieler Primzahlen: $k = \prod_{i=1}^n p_i = p_1 \cdot p_2 \cdot \dots$
\item Der kleinste Teiler $d > 1$ einer natürlichen Zahl $n \geq 2$ ist eine Primzahl (Satz vom kleinsten Teiler) 
\end{itemize}

\subsection{Modulare Arithmetik}
Sei $n \in \N$ beliebig. Wir definieren die Modulo-Relation $\equiv_n$ auf $\Z$ wie folgt:
\begin{equation*}
 r \equiv_n s :\Leftrightarrow n | (r - s)
\end{equation*}
Synonyme schreibweisen:
\begin{align*}
 r &\equiv_n x \\
 r \mod n& = x
\end{align*}

\subsubsection{Chinesischer Restsatz}
\begin{align*}
  x& \equiv_{m_i} a_i\\
  M& := \prod m_1\\
  M_i & := M / m_i
\end{align*}
Finde $r_i, s_i$ sodass $r_i \cdot m_i + s_i \cdot M_i = 1$ (Euklid). Setze $e_i = s_i \cdot M_i$. Die Lösung ist dann $x = \sum a_i \cdot e_i \equiv y \mod M$

Beispiel:
\begin{align*}
 x &\equiv_3 2& \to a_1& = 2\\
 x &\equiv_4 3& \to a_2& = 3\\
 x &\equiv_5 2& \to a_3& = 5\\
 M &= m_1 \cdot m_2 \cdot m_3 = 3 \cdot 4 \cdot 5 = 60\\
 M_1 & = M / m_1 = 20, & (M_2& = 15, M_3 = 12)\\
\end{align*}
Als nächstes $s_1$ mit Hilfe des erweiterten Euklid bestimmen:
\begin{align*}
 1 &= r_1 \cdot m_1 + s_1 \cdot M_1  \Rightarrow r_1 = 7, s_1 = -1 \Rightarrow e_1 = -20\\
 1 &= r_2 \cdot m_2 + s_2 \cdot M_2  \Rightarrow r_2 = 4, s_2 = -1 \Rightarrow e_1 = -15\\
 1 &= r_3 \cdot m_3 + s_3 \cdot M_3  \Rightarrow r_3 = 5, s_1 = -2 \Rightarrow e_1 = -24\\
 x &= 2 \cdot (-20) + 3 \cdot (-15) + 5 \cdot (-24) = - 133 = 47 \mod 60\\
\end{align*}
