\section{Matrizen}
Eine Matrix ist eine rechteckige Anordnung von Elementen. Matrizen können beliebige Dimensionalität besitzen. Die Elemente der Matrix über $K$ heissen Komponenten. Sie entstammen einer Menge $K$ (Körper oder Ring). Formal handelt es sich um eine Funktion:
\begin{equation*}
	A: \{1,\ldots,m\}\times\{1,\dots,n\} \to K,\quad (i,j) \mapsto a_{ij}
\end{equation*}
Sie ordnet jedem Indexpaar $(i, j)$ einen Funktionswert $a_{ij}$ zu.

Eine Matrix mit $m$-Zeilen und $n$-Spalten wie folgt darstellen:
\begin{equation*}
A=\mathbf{A}=\underline{A}=\begin{pmatrix}
a_{11} & a_{12} & \cdots & a_{1n}\\
a_{21} & a_{22} & \cdots & a_{2n}\\
\vdots & \vdots & \ddots & \vdots\\
a_{m1} & a_{m2} & \cdots & a_{mn}\\
\end{pmatrix} = (a_{ij})
\end{equation*}
\subsection{Addition}
\begin{equation*}
	A + B := (a_{ij} + b_{ij})_{i=1\dots\ m; j=1\dots\ n}
\end{equation*}
\begin{equation*}
  \begin{pmatrix}
    1 & -3 \\
    1 & 2
  \end{pmatrix}
  +
  \begin{pmatrix}
    0 & 3 \\
    2 & 1
  \end{pmatrix}
  =
  \begin{pmatrix}
    1+0 & (-3)+3 \\
    1+2 & 2+1 
  \end{pmatrix}
  =
  \begin{pmatrix}
    1 & 0\\
    3 & 3
  \end{pmatrix}
\end{equation*}
Die Addition ist assoziativ, kommutativ und besitzt mit der Nullmatrix ein neutrales Element.

\subsection{Skalarmultiplikation}
\begin{equation*}
	\lambda\cdot A := (\lambda\cdot a_{ij})_{i=1, \ldots , m; \ j=1, \ldots , n}
\end{equation*}
\begin{equation*}
5 \cdot
  \begin{pmatrix}
    1 & -3 & 2 \\
    1 &  2 & 7
  \end{pmatrix}
  =
  \begin{pmatrix}
   5 \cdot 1 & 5 \cdot (-3) & 5 \cdot 2 \\
   5 \cdot 1 & 5 \cdot   2  & 5 \cdot 7
  \end{pmatrix}
  =
  \begin{pmatrix}
    5 & -15 & 10 \\
    5 & 10  & 35
  \end{pmatrix}
\end{equation*}

$\lambda$ und die Elemente entstammen demselben Ring $(K, +, \cdot, 0)$.

\subsection{Multiplikation}
Zwei Matrizen können multipliziert werden, wenn die Spaltenanzahl der linken ($A = (a_{ij})$) mit der Zeilenanzahl der rechten Matrix ($B = (b_{ij})$)übereinstimmt.
\begin{equation*}
	c_ {ij} = \sum_{k=1}^m a_{ik} \cdot b_{kj}
\end{equation*}
\newcommand{\myunit}{0.75 cm}
\tikzset{
    node style sp/.style={draw,circle,minimum size=\myunit},
    node style ge/.style={circle,minimum size=\myunit},
    arrow style mul/.style={draw,sloped,midway,fill=white},
    arrow style plus/.style={midway,sloped,fill=white},
}

\begin{tikzpicture}[>=latex]
% les matrices
\matrix (A) [matrix of math nodes,%
             nodes = {node style ge},%
             left delimiter  = (,%
             right delimiter = )] at (0,0)
{%
  a_{11} & a_{12} & \ldots & a_{1p}  \\
  \node[node style sp] {a_{21}};%
         & \node[node style sp] {a_{22}};%
                  & \ldots%
                           & \node[node style sp] {a_{2p}}; \\
  \vdots & \vdots & \ddots & \vdots  \\
  a_{n1} & a_{n2} & \ldots & a_{np}  \\
};
\node [draw,below=10pt] at (A.south) 
    { $A$ : \textcolor{red}{$n$ Zeilen} $p$ Spalten};

\matrix (B) [matrix of math nodes,%
             nodes = {node style ge},%
             left delimiter  = (,%
             right delimiter =)] at (6*\myunit,6*\myunit)
{%
  b_{11} & \node[node style sp] {b_{12}};%
                  & \ldots & b_{1q}  \\
  b_{21} & \node[node style sp] {b_{22}};%
                  & \ldots & b_{2q}  \\
  \vdots & \vdots & \ddots & \vdots  \\
  b_{p1} & \node[node style sp] {b_{p2}};%
                  & \ldots & b_{pq}  \\
};
\node [draw,above=10pt] at (B.north) 
    { $B$ : $p$ Zeilen \textcolor{red}{$q$ Spalten}};
% matrice résultat
\matrix (C) [matrix of math nodes,%
             nodes = {node style ge},%
             left delimiter  = (,%
             right delimiter = )] at (6*\myunit,0)
{%
  c_{11} & c_{12} & \ldots & c_{1q} \\
  c_{21} & \node[node style sp,red] {c_{22}};%
                  & \ldots & c_{2q} \\
  \vdots & \vdots & \ddots & \vdots \\
  c_{n1} & c_{n2} & \ldots & c_{nq} \\
};
% les fleches
\draw[blue] (A-2-1.north) -- (C-2-2.north);
\draw[blue] (A-2-1.south) -- (C-2-2.south);
\draw[blue] (B-1-2.west)  -- (C-2-2.west);
\draw[blue] (B-1-2.east)  -- (C-2-2.east);
\draw[<->,red](A-2-1) to[in=180,out=90]
	node[arrow style mul] (x) {$a_{21}\times b_{12}$} (B-1-2);
\draw[<->,red](A-2-2) to[in=180,out=90]
	node[arrow style mul] (y) {$a_{22}\times b_{22}$} (B-2-2);
\draw[<->,red](A-2-4) to[in=180,out=90]
	node[arrow style mul] (z) {$a_{2p}\times b_{p2}$} (B-4-2);
\draw[red,->] (x) to node[arrow style plus] {$+$} (y)%
                  to node[arrow style plus] {$+\raisebox{.5ex}{\ldots}+$} (z)%
                  to (C-2-2.north west);


\node [draw,below=10pt] at (C.south) 
    {$ C=A\times B$ : \textcolor{red}{$n$ Zeilen}  \textcolor{red}{$q$ Spalten}};

\end{tikzpicture}
Die Multiplikation ist nicht kommutativ!
\subsection{Einheitsmatrix}
Die $(n \times n)$-Einheitsmatrix ist definiert als:
\begin{equation*}
E_n=\begin{pmatrix}
1 & 0 & \cdots & 0\\
0 & 1 & \cdots & 0\\
\vdots & \vdots & \ddots & \vdots\\
0 & 0 & \cdots & 1\\
\end{pmatrix}
\end{equation*}
\subsection{Darstellende Matrix}
Es sei $v = \lambda_1, \dots, \lambda_n) \in K^n$, für $i=1,\dots,n$. Mit $v[i] := \lambda_i$ ist die $i$-te Komponente von $v$ bezeichnet.
\begin{itemize}\itemsep0em
	\item [$\Theta$:] Hom$_K(K^n, K^m) \to K^{m,n}$
	\begin{equation*}
		\Theta (f) =   
		\begin{pmatrix}
   f(e_1)[1] & \cdots & f(e_i)[1] & \cdots & f(e_n)[1] \\
	\vdots & & \vdots & & \vdots\\
	f(e_1)[m] & \cdots & f(e_i)[m] & \cdots & f(e_n)[m]
		\end{pmatrix}
	\end{equation*}
	Beispiel: $f: \mathbb{R}^3 \to \mathbb{R^2}, f(x, y, z) = (2x - 3y, x - 2y + z)$\\
	Im Urbild ($\mathbb{R}^3$) und im Zielraum ($\mathbb{R}^2$) die Standardbasis wählen:
	\begin{equation*}
	A=\left(
		\begin{pmatrix} 1\\ 0\\ 0\end{pmatrix}, \begin{pmatrix} 0\\ 1\\ 0\end{pmatrix}, \begin{pmatrix} 0\\ 0\\ 1\end{pmatrix} \right) 
		\, , \quad
	B=\left(
		\begin{pmatrix} 1\\ 0\end{pmatrix}, \begin{pmatrix} 0\\ 1\end{pmatrix}
		\right) 
	\end{equation*}
	Es gilt:
	\begin{equation*}
		f\begin{pmatrix} 1\\0\\0\end{pmatrix} = 
		\begin{pmatrix}
			2 \cdot 1 - 3 \cdot 0\\
			1 - 2 \cdot 0 + 0
		\end{pmatrix}
		= \begin{pmatrix}2\\1\end{pmatrix}
	\end{equation*}
	\begin{equation*}
		\mbox{entsprechend: }f\begin{pmatrix}0\\1\\0\end{pmatrix} = \begin{pmatrix}-3\\-2\end{pmatrix}
		\, ,
		f\begin{pmatrix}0\\0\\1\end{pmatrix} = \begin{pmatrix}0\\1\end{pmatrix}
	\end{equation*}
	Somit wird $ \Theta_B^A (f)=\begin{pmatrix} 2 & -3 & 0\\ 1 & -2 & 1\end{pmatrix} $

	\item [$\Psi$:] $K^{m,n} \to $ Hom$_K(K^n, K^m)$
	\begin{equation*}
		\Psi (A)(v) = A \circ
		\begin{pmatrix}
			v[1]\\
			\vdots\\
			v[n]
		\end{pmatrix}
	\end{equation*}
	Beispiel: $\Psi(\Theta(f))(2, 3, 1)$:
	\begin{equation*}
		\Psi(\Theta(f))(2, 3, 1) =
		\begin{pmatrix}
			2 & -3 & 0\\
			1 & -2 & 1
		\end{pmatrix}
		\circ
		\begin{pmatrix}
			2\\3\\1
		\end{pmatrix}
		=
		\begin{pmatrix}
			2 \cdot 2 - 3 \cdot 3\\
			2 - 2 \cdot 3 + 1\\
		\end{pmatrix}
	\end{equation*}
	$ = (-5, -3) = f(2, 3, 1)$
\end{itemize}
% \subsubsection{Berechnung}
% Sei $f: V \to \mathbb{R}^m$ eine lineare Abbildung und $A = (v_1, v_2, \dots, v_n)$ eine geordnete Basis von $V$.
% \begin{enumerate}
% 	\item Basis $B$ für $\mathbb{R}^m$ wählen: $B = (e_1, e_2, \dots, e_m)$
% 	\item Die Abbildungsmatrix ergibt sich, indem man die Bilder der Basisvektoren von $V$ als Spalten einer Matrix auffasst:
% 	\begin{equation*}
% 		\Theta_B^A(f) = 
% 		\begin{pmatrix}
% 			\vdots 		& \vdots 		&  			& \vdots \\
% 			f(\vec v_1) & f(\vec v_2) 	& \cdots 	& f(\vec v_n)\\
% 			\vdots 		& \vdots 		&  			& \vdots
% 		\end{pmatrix}
% 	\end{equation*}
% \end{enumerate}





%Ring, darstellende Matrix, Theta (Homomorphismus ...), Phi (bis S. 83)
