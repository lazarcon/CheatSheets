\section{Zahlentheorie}
\subsection{Satz vom kleinsten Teiler}
Der kleinste Teiler $d > 1$ einer natürlichen Zahl $n \geq 2$ ist eine Primzahl.

\begin{proof}
	Sei $T(n)$ die Menge aller Teiler von $n$.\\
	$T(n)\textbackslash\{1\}$ ist nicht leer, weil $n \in T(n)$\\
	dann gibt es nach dem Wohlordnungsprinzip 
	eine kleinste Zahl $d$ in $T(n)\textbackslash\{1\}$.\\
	Zu zeigen: $d$ ist eine Primzahl\\
	Annahme: $d$ ist keine Primzahl\\
	Dann: $\exists a, b \in \N \setminus\{1, d\} \colon a \cdot b = d \wedge a, b < d$\\
	$\Rightarrow a \mid d$\\
	Da $a$ kleiner als $d$ sein muss, kann $d$ nicht das kleinste Element in $T(n)\textbackslash{1}$ sein.\\
	Das ist ein Widerspruch zur Annahme. Die Annahme muss falsch sein, dann muss aber $d$ eine Primzahl sein.
	\qedhere
\end{proof}

\subsection{Hauptsatz}
Jede natürliche Zahl $n \geq 2$ besitzt eine eindeutige Primfaktorzerlegung.
Die Reihenfolge der Primfaktoren kann variieren.

\begin{proof}
	Sei $n \in \N \setminus\{1\}$\\
	Fall I: $n$ ist eine Primzahl, ein eindeutiger Primfaktor gefunden ($n$).\\
	Fall II: $n$ ist keine Primzahl\\
	\begin{equation*}
		\Rightarrow \exists d_1, q_1 \in \N \setminus\{1\} \colon d_1 \cdot q_1 = n
	\end{equation*}
	und $d_1$ ist das kleinste Element in $T(n)\setminus\{1\}$, also $d_1$ ist Primzahl\\
	Betrachte $q_1$:
	Fall I: $q_1$ ist eine Primzahl, eindeutige Primfaktorzerlegung gefunden.\\
	Fall II: $q_1$ ist keine Primzahl, dann wiederhole diesen Prozess mit $q_1$\\ 
	\qedhere
\end{proof}
% Es gibt also eine Primfaktorzerlegung.
% Bleibt zu beweisen, dass es nur eine Primfaktorzerlegung gibt.
% \begin{proof}
% 	Sei $n$ die kleinste natürliche Zahl mit mehr als einer Primfaktorzerlegung.\\
% 	\begin{align*}
% 	n& = p_1 \cdot p_2 \cdot \dots \cdot p_n = q_1 \cdot q_2 \cdot \dots \cdot q_m & p, q \in \mathbb{P}\\
% 	\mbox{$p_i$ und $q_j$ sind paarweise verschieden}	
% 	\end{align*}
% 
% \end{proof}
