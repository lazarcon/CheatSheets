\documentclass[10pt,landscape]{scrartcl}
\usepackage{multicol}
\usepackage{calc}
\usepackage{ifthen}
\usepackage[landscape]{geometry}
% Umlaute ermöglichen
\usepackage[T1]{fontenc}
% Format dieses Dokuments
\usepackage[utf8]{inputenc}
% Deutsche Trennungsregeln
\usepackage[ngerman]{babel}
\usepackage[babel,german=quotes]{csquotes}

%\usepackage{color}
%\definecolor{lightgrey}{gray}{0.75}
%\definecolor{darkgrey}{gray}{0.25}

\usepackage{amssymb,amsmath,amsthm,mathtools}
\newtheorem*{satz}{Satz}
\usepackage{wasysym}

\usepackage{tikz}

\usepackage{ulem}

\usepackage{hyperref}
%\pdfoutput=1
\hypersetup{
	pdfauthor   = {Lazari, Constantin},
	pdftitle    = {Cheat Sheet Algebra},
	pdfsubject  = {Mathematik},
	pdfkeywords = {},
	pdfcreator  = {Kile},		% Texnic Center oder Kile z.B.
	pdfproducer = {pdflatex},
	colorlinks  = false		% Links nicht farbig hervorheben (sieht Scheisse aus).
} 

% To make this come out properly in landscape mode, do one of the following
% 1.
%  pdflatex latexsheet.tex
%
% 2.
%  latex latexsheet.tex
%  dvips -P pdf  -t landscape latexsheet.dvi
%  ps2pdf latexsheet.ps

% This sets page margins to .5 inch if using letter paper, and to 1cm
% if using A4 paper. (This probably isn't strictly necessary.)
% If using another size paper, use default 1cm margins.
\ifthenelse{\lengthtest { \paperwidth = 11in}}
	{ \geometry{top=.5in,left=.5in,right=.5in,bottom=.5in} }
	{\ifthenelse{ \lengthtest{ \paperwidth = 297mm}}
		{\geometry{top=1cm,left=1cm,right=1cm,bottom=1cm} }
		{\geometry{top=1cm,left=1cm,right=1cm,bottom=1cm} }
	}

% Turn off header and footer
\pagestyle{empty}
 
% Redefine section commands to use less space
\makeatletter
\renewcommand{\section}{\@startsection{section}{1}{0mm}%
                                {-1ex plus -.5ex minus -.2ex}%
                                {0.5ex plus .2ex}%x
                                {\normalfont\large\bfseries}}
\renewcommand{\subsection}{\@startsection{subsection}{2}{0mm}%
                                {-1explus -.5ex minus -.2ex}%
                                {0.5ex plus .2ex}%
                                {\normalfont\normalsize\bfseries}}
\renewcommand{\subsubsection}{\@startsection{subsubsection}{3}{0mm}%
                                {-1ex plus -.5ex minus -.2ex}%
                                {1ex plus .2ex}%
                                {\normalfont\small\bfseries}}
\newcommand{\msout}[1]{\text{\sout{\ensuremath{#1}}}}

\newcommand{\N}{\mathbb{N}} % Natürliche Zahlen
\newcommand{\Z}{\mathbb{Z}}	 % Ganze Zahlen

\makeatother

% Define BibTeX command
\def\BibTeX{{\rm B\kern-.05em{\sc i\kern-.025em b}\kern-.08em
    T\kern-.1667em\lower.7ex\hbox{E}\kern-.125emX}}

% Don't print section numbers
\setcounter{secnumdepth}{0}
\setlength{\parindent}{0pt}

\setlength{\parskip}{0pt plus 0.5ex}


% -----------------------------------------------------------------------

\begin{document}

	\raggedright
	\footnotesize
	\begin{multicols}{3}


	% multicol parameters
	% These lengths are set only within the two main columns
	%\setlength{\columnseprule}{0.25pt}
	\setlength{\premulticols}{1pt}
	\setlength{\postmulticols}{1pt}
	\setlength{\multicolsep}{1pt}
	\setlength{\columnsep}{2pt}
	\newlength{\MyLenA}
	\newlength{\MyLenB}

	\begin{center}
	\Large{\textbf{Cheat Sheet Algebra}} \\
	\end{center}

	%\section{Mathematische Symbole und Begriffe}
\settowidth{\MyLenA}{Primzahl~~}
\begin{tabular}{@{}p{\the\MyLenA}%
				@{}p{\linewidth-\the\MyLenA}}
	$w$, $f$ & wahr, falsch \\
	$\in$, $\notin$  & Element von, kein Element von \dots\\
	$\forall$ & Allquantor; für alle \dots\\
	$\exists$ & Existenzquantor; es existiert mindestens ein \dots \\
	$\exists !$ & Sonderfall: es existiert genau ein \dots \\
	$\mid$ & für die gilt; Bsp: $\{x\mid x < 0\}$ (alle x für die gilt x < 0)\\
	$:=$ & Definiert als\\
	$\equiv$ & Kongruent bzw. identisch, identisch gleich\\
\end{tabular}


	\section{Logik}
\subsection{Aussagenlogik}
\textbf{Aussage} Eine Aussage ist eine sprachliche Äusserung, 
die wahr ($w$) oder falsch($f$) sein kann.\\
Bsp: \enquote{Es gibt unendlich viele natürliche Zahlen.}

\subsubsection{Junktoren (Verknüpfungsoperatoren)}
\begin{tabular}{cl}
	$\neg$				& Negation, \enquote{nicht \dots}\\
	$\wedge$ 			& Konjunktion, \enquote{\dots\ und \dots}\\
	$\vee$ 				& Disjunktion, \enquote{\dots\ oder \dots}\\
	$\Rightarrow$ 		& Implikation, \enquote{wenn \dots, dann \dots}\\
	$\Leftrightarrow$	& Äquivalenz, \enquote{\dots\ genau dann, wenn \dots} \\
	$\bigoplus$			& Antivalenz, \enquote{entweder \dots oder  \dots}\\
\end{tabular}\\

Für eine einzelne Aussage $A$ gilt:\\
\begin{tabular}{c||c}
	$A$ & $\neg A$ \\\hline
	$f$ & $w$ \\
	$w$ & $f$ \\
\end{tabular}

Für zwei Aussagen $A$ und $B$ gilt:\\
\begin{tabular}{c|c||c|c|c|c|c}
	$A$ & $B$ 	& $A \wedge B$ 	& $A \vee B$ 	& $A \Rightarrow B$ 	& $A \Leftrightarrow B$ 	& $A \bigoplus B$ \\\hline
	$f$ & $f$ 	& $f$				& $f$			& $w$				& $w$					 	& $f$\\
	$f$ & $w$ 	& $f$				& $w$			& $w$				& $f$					 	& $w$\\	
	$w$ & $f$ 	& $f$				& $w$			& $f$				& $f$						& $w$\\
	$w$ & $w$ 	& $w$				& $w$			& $w$				& $w$						& $f$\\
\end{tabular}\\
Bindungsstärke: $\neg$ vor $\wedge$ vor $\vee$ vor $\Rightarrow$ vor $\Leftrightarrow$.\\

\subsubsection{Rechenregeln}
\settowidth{\MyLenA}{Doppelte Negation~~}
\begin{equation*}
\begin{array}{lrl}
	\mbox{Duplizität}			& A \wedge A				& \Leftrightarrow A\\
								& A \vee A					& \Leftrightarrow A\\
	\mbox{Doppelte Negation}	& \neg\neg A 				& \Leftrightarrow A\\
	\mbox{Kommutativität}		& A \wedge B 				& \Leftrightarrow B \wedge A\\
								& A \vee B					& \Leftrightarrow B \vee A\\
	\mbox{Assoziativität} 		& (A \wedge B) \wedge C 	& \Leftrightarrow A \wedge (B \wedge C)\\
								& (A \vee B) \vee C 		& \Leftrightarrow A \vee (B \vee C)\\
	\mbox{Distributiviät}		& A \wedge (B \vee C) 		& \Leftrightarrow (A \wedge B) \vee (A \wedge C)\\
								& A \vee (B \wedge C) 		& \Leftrightarrow (A \vee B) \wedge (A \vee C)\\
	\mbox{De Morgan Regeln}		& \neg (A \wedge B) 		& \Leftrightarrow \neg A \vee \neg B\\
								& \neg (A \vee B) 			& \Leftrightarrow \neg A \wedge \neg B\\
	\mbox{Implikation}			& (A \Rightarrow B) 		& \Leftrightarrow (\neg A \vee B)\\
	\mbox{Kontraposition}		& (A \Rightarrow B) 		& \Leftrightarrow (\neg B \Rightarrow \neg A)\\
	\mbox{Äquivalenz}			& (A \Leftrightarrow B)		& \Leftrightarrow (A \Rightarrow B) \wedge (B \Rightarrow A)\\
	\mbox{Absorbtion}			& A \wedge (A \vee B)		& \Leftrightarrow A\\
								& A \vee (A \wedge B)		& \Leftrightarrow A\\
\end{array}
\end{equation*}
Die Rechenregel der Aussagenlogik werden mit Hilfe von Wahrheitstafeln bewiesen.


\settowidth{\MyLenA}{Kanonisch konjunktive Normalform (KKNF)~~}
\begin{description}
	\item [Erfüllbarkeit] Es gibt eine Interpretation der Variablen, so dass die Formel wahr wird.
	\item [Allgemeingültigkeit] Ein Formel wird mit jeder Interpretation der Variablen wahr.
	\item [Widerlegbarkeit] Es gibt mindestens eine Interpretation der Variablen, so dass die Formel falsch wird.
	\item [Entscheidbarkeit] Man kann wissen, bei welcher Variablen-Kombination eine Formel wahr wird.
	\item [Konjunktive Normalform (KNF)] Konjunktion von Disjunktionstermen $(a \wedge b) \vee (c \wedge b)$
	\item [Kanonisch konjunktive Normalform (KKNF)] (Minterm) Jede Variable tritt pro KNF-Term genau einmal auf.
	\item [Disjunktive Normalform (DNF)] Disjunktion von Konjunktionstermen $(a \vee b) \wedge (c \vee d)$
	\item [Kanonisch disjunktive Normalform (KDNF)] (Maxterm) Jede Variable tritt pro DNF-Term genau einmal auf.\\	
\end{description}

Normalformen werden über Wahrheitstafeln oder durch Term-Umformung konstruiert.

Für die KDNF alle Belegungen einer Wahrheitstafel, die $w$ ergeben konjunktiv verknüpfen. Einzelterme disjunktiv verknüpfen.
Für die KKNF alle Belegungen einer Wahrheitstafel, die $f$ ergeben disjunktiv verknüpfen. Einzelterme konjunktiv verknüpfen.

\subsubsection{Beispiel Formelumwandlung zu DNF/KDNF}
Gegeben $F = ((a_1 \Rightarrow a_2) \Rightarrow \neg a_3) \vee \neg a_2$\\

1. Umrechung in DNF:\\
	\begin{align*}
		F& = ((a_1 \Rightarrow a_2) \Rightarrow \neg a_3) \vee \neg a_2\\
		(a_1 \Rightarrow a_2) & \Leftrightarrow \neg a_1 \vee a_2	& \mbox{Impl.}\\
		((\neg a_1 \vee a_2) \Rightarrow \neg a_3) & \Leftrightarrow (\neg (\neg a_1 \vee a_2)) \vee \neg a_3) & \mbox{Impl.}\\
		& \Leftrightarrow \neg(a_2 \vee \neg a_1) \vee \neg a_3 & \mbox{Assoz.}\\
		\neg(a_2 \vee \neg a_1) & \Leftrightarrow \neg a_2 \wedge \neg \neg a_1 & \mbox{De Morgan}\\
		& \Leftrightarrow \neg a_2 \wedge a_1 & \mbox {D. Vernein.}\\
		((\neg a_2 \wedge a_1) \vee \neg a_3) \vee \neg a_2 & \Leftrightarrow (\neg a_2 \wedge a_1) \vee \neg a_3 \vee \neg a_2 & \mbox{Absorbtion}\\
		& \Leftrightarrow \neg a_2 \wedge \neg a_3 
	\end{align*}

2. KDNF erstellen:\\
\begin{center}
	\begin{tabular}{c|c|c||c||c}
		$a_1$ 	&$a_2$ 	& $a_3$	 	& $F_2$ & Ausdruck\\\hline
		$w$ 	& $w$	& $w$		& $f$	& \\ 
		$w$ 	& $w$	& $f$		& $w$	& $a_1 \wedge a_2 \wedge \neg a_3$\\ 
		$w$ 	& $f$	& $w$		& $w$	& $a_1 \wedge \neg a_2 \wedge a_3$\\ 
		$w$ 	& $f$	& $f$		& $w$	& $a_1 \wedge \neg a_2 \wedge \neg a_3$\\ 
		$f$ 	& $w$	& $w$		& $f$	& \\
		$f$ 	& $w$	& $f$		& $w$	& $\neg a_1 \wedge a_2 \wedge \neg a_3$\\ 
		$f$ 	& $f$	& $w$		& $w$	& $\neg a_1 \wedge \neg a_2 \wedge a_3$\\ 
		$f$ 	& $f$	& $f$		& $w$	& $\neg a_1 \wedge \neg a_2 \wedge \neg a_3$\\ 
	\end{tabular}
\end{center}

3. Ergebnis zusammensetzen:
\begin{align*} F_2:& (a_1 \wedge a_2 \wedge \neg a_3) \vee (a_1 \wedge \neg a_2 \wedge a_3) \vee (a_1 \wedge \neg a_2 \wedge \neg a_3)\\ 
& \vee (\neg a_1 \wedge a_2 \wedge \neg a_3) \vee (\neg a_1 \wedge \neg a_2 \wedge a_3) \vee (\neg a_1 \wedge \neg a_2 \wedge \neg a_3)
\end{align*}

\subsection{Prädikatenlogik}
\textbf{Aussageform} Eine Aussageform ist eine sprachliche Äusserung, in 
der Variablen vorkommen und die in Abhängigkeit der Variablenwerte wahr ($w$) oder falsch ($f$) 
sein kann -- Aussageformen sind manchmal wahr, manchmal falsch.\\
Bsp: \enquote{Die Zahl $x$ ist eine gerade Zahl.}

Es wird unterschieden zwischen Objekt und Prädikat (Eigenschaft): für oberes Beispiel ist \enquote{ist gerade}
das Prädikat, während $x$ das Objekt ist.

\subsubsection{Quantoren (Variablenbinder)}
$A(x)$ sei eine Aussageform, $M$ eine Menge von Objekten.
\settowidth{\MyLenA}{$\forall x \in M A(x))$~~}
\begin{tabular}{@{}p{\the\MyLenA}%
				@{}p{\linewidth-\the\MyLenA}}
	$\forall x \in M A(x)$ & Für alle $x$ der Menge $M$ gilt $A(x)$\\
	$\exists x \in M A(x)$ & Für mindestens ein $x$ der Menge $M$ gilt $A(x)$\\
\end{tabular}

\subsubsection{Besondere Ausdrücke}
\begin{tabular}{ll}
	Für mindestens zwei gilt~~ & $\exists x\, \exists y ((A(x) \wedge A(y)) \Rightarrow (x \neq y))$\\
	Es gibt höchstens ein & $\forall x\, \forall y\,((A(x) \wedge A(y)) \Rightarrow (x = y))$\\
\end{tabular}



	\section{Mengenlehre}
Eine Menge ist jede Zusammenfassung von bestimmten, wohlunterscheidbarer Objekte
unserer Anschauung oder unseres Denkens zu einem Ganzen.\\
Kein Objekt kann in einer Menge doppelt vorkommen.

\subsection{Symbole}
\settowidth{\MyLenA}{$\mathbb{Q} = \{x\mid x = \frac{a}{b} \wedge a \in \mathbb{Z} \wedge b \in \mathbb{N}$*$\}$~~}
\begin{tabular}{@{}p{\the\MyLenA}%
				@{}p{\linewidth-\the\MyLenA}}
	$x \in M (x \notin M)$ & $x$ ist (kein) Element der Menge $M$ \\
	$x \mid A(x)$ & für $x$ gilt die Aussageform $A(x)$\\
	$\emptyset = \{\}$ & Leere Menge; enthält nichts\\
	$\mathbb{N} = \{0, 1, 2, \dots\}$	& Natürliche Zahlen\\
	$\mathbb{N}^* = \{1, 2, 3, \dots\}$	& Positive ganze Zahlen\\
	$\mathbb{Z} = \{0, \pm1, \pm2, \dots\}$	& Ganze Zahlen\\
	$\mathbb{Q} = \{x\mid x = \frac{a}{b} \wedge a \in \mathbb{Z} \wedge b \in \mathbb{N}$*$\}$ & Rationale Zahlen (=Brüche)\\
	$\mathbb{R}$ & Reelle Zahlen (= alle Zahlen)\\
	$\mathbb{C}$ & Komplexe Zahlen\\
\end{tabular}

\subsection{Intervalle}
Fortlaufende Mengen lassen sich als Intervalle schreiben: $(a,b) = \{x \mid a \leq x \leq b\}$.
Runde Klammern bedeuten exklusive, eckige Klammern inklusive. $\infty$ hat immer eine Runde Klammer zur Folge.

\subsubsection{Beispiele für Mengendefinitionen}
\begin{tabular}{ll}
	Aufzählende Notation & $M = \{x_1, x_2, \dots, x_n\}$\\
	Intensionale Notation & $M = \{x \in X \mid A(x)\}$ 
\end{tabular}\\
Beispiel geraden Zahlen: $G = \{n \in \mathbb{N} \mid \exists m \in \mathbb{N}~2 \cdot m = n\}$ \\

\subsection{Mächtigkeit}
Die Mächtigkeit $|M|$ entspricht im Prinzip der Anzahl Elemente einer Menge. Allerdings nur im Prinzip,
weil: $|\mathbb{N}^*| = |\mathbb{N}| = |\mathbb{Z}| = |\mathbb{Q}| < |\mathbb{R}|$

\subsection{Mengenoperationen}
\settowidth{\MyLenA}{$C = A \cup B$~~}
\begin{tabular}{@{}p{\the\MyLenA}%
				@{}p{\linewidth-\the\MyLenA}}
	$A \cup B$	& Vereinigungsmenge: $\forall x \in (A \cup B) \Leftrightarrow (x \in A \vee x \in B)$ \\
	$A \cap B$	& Schnittmenge: $\forall x \in (A \cap C) \Leftrightarrow (x \in A \wedge x \in B)$\\
	$A \backslash B$ & Differenzmenge: $\forall x \in (A \backslash B) \Leftrightarrow ((x \in A) \wedge (x \notin B))$\\
\end{tabular}
Disjunkt (elementfremd) sind zwei Mengen wenn, $A \cap B = \emptyset$.

\subsection{Mengenrelationen}
\settowidth{\MyLenA}{$C = A \cup B$~~}
\begin{tabular}{@{}p{\the\MyLenA}%
				@{}p{\linewidth-\the\MyLenA}}
	$A = B$	& Gleichheit: $(A = B) \Leftrightarrow (\forall x (x \in A) \Leftrightarrow (x \in B))$\\
	$A \subseteq B$	& Teilmenge: $(A \subseteq B) \Leftrightarrow (\forall x (x \in A) \Rightarrow (x \in B))$\\
	$A \subsetneq B$ & Echte Teilmenge: $(A \subsetneq B) \Leftrightarrow ((A \subseteq B) \wedge (A \neq B))$
\end{tabular}

\subsection{Rechenregeln}
\settowidth{\MyLenA}{Doppelte Negation~~}
\begin{equation*}
\begin{array}{lrl}
	\mbox{Kommutativität}		& A \cup B 									& = B \cup A\\
								& A \cap B									& = B \cap A\\
	\mbox{Assoziativität} 		& (A \cup B) \cup C 						& = A \cup (B \cup C)\\
								& (A \cap B) \cap C 						& = A \cap (B \cap C)\\
	\mbox{Distributiviät}		& A \cup (B \cap C) 						& = (A \cup B) \cap (A \cup C)\\
								& A \cap (B \cup C) 						& = (A \cap B) \cup (A \cap C)\\
	\mbox{De Morgan Regeln}		& (C \backslash A) \cup (C \backslash B) 	& = C \backslash (A \cap B)\\
								& (C \backslash A) \cap (C \backslash B) 	& = C \backslash (A \cup B)\\
	\mbox{Idempotenzgesetz}		& A \cup A 									& = A\\
								& A \cap A									& = A
\end{array}
\end{equation*}

Beweise in der Mengenlehre werden durch Umwandlung zu logischen Aussagen geführt.
Bsp: Zu zeigen: $\forall x \in (A \cap B) \cap C \Leftrightarrow x \in A \cap (B \cap C)$
\begin{proof}
Sei $x$ fest, aber beliebig.
\begin{align*}
	x \in (A \cap B) \cap C	& \Leftrightarrow x \in (A \cap B) \wedge x \in C\\
							& \Leftrightarrow (x \in A \wedge x \in B) \wedge x \in C\\
							& \Leftrightarrow x \in A \wedge (x \in B \wedge x \in C)\\
							& \Leftrightarrow x \in A \wedge x \in (B \cap C)\\
							& \Leftrightarrow x \in A \cap (B \cap C)\qedhere
\end{align*}
\end{proof}

\subsection{Mengenbildung}
Sei $A$ ein Menge von Mengen $A = \{X, Y\}$ mit $X = \{x_1, x_2, \dots, x_n\}, Y = \{y_1, y_2, \dots, y_n\}$\\

\subsubsection{Vereinigung}
Die Vereinigung $\bigcup A$ ist die Menge aller Elemente aller in $A$ enthaltener Mengen: 
$\bigcup_{i \in I} Y_i := \bigcup A$, wobei $I$ eine Indexmenge und $A = \{Y_i \mid i \in I\}$
\begin{equation*}
	x \in \bigcup A \Leftrightarrow \exists Y \in A (x \in Y)
\end{equation*}

\subsubsection{Schnittmenge}
Die Schnittmenge $\bigcap A$ ist die Menge aller Elemente die in jeder in $A$ enthaltener Menge vorkommt:
$\bigcap_{i \in I} Y_i := \bigcap A$, wobei $I$ eine Indexmenge und $A = \{Y_i \mid i \in I\}$
\begin{equation*}
	x \in \bigcap A \Leftrightarrow \forall Y \in A (x \in Y)
\end{equation*}

\subsubsection{Potenzmenge}
Die Potenzmenge enthält alle Teilmengen von $A$ als Elemente.
\begin{equation*}
	\mathcal{P}(A) := \{x \mid x \subseteq A\}
\end{equation*}
Es gelten:
\begin{align*}
	|P(A)|& = 2^{|A|}\\
	A& = \bigcup \mathcal{P}(A)\\
	\cap \mathcal{P}(A)& = \emptyset\\
	\mathcal{P}(A \cap B)& = \mathcal{P}(A) \cap \mathcal{P}(B) \\
	\mathcal{P}(A \cup B)& \neq \mathcal{P}(A) \cup \mathcal{P}(B)
\end{align*}

\subsubsection{Kartesisches Produkt (Kreuzprodukt)}
Das Kreuzprodukt zweier Mengen ist die Menge aller möglichen Kombination von Elementen aus der ersten
Menge mit Elementen aus der zweiten Mengen. Die Ergebnismenge besteht aus Tupeln, wobei jedes Tupel
eine mögliche Kombination darstellt. Diese 2-Tupel heissen geordnete Paare.
\begin{equation*}
		A \times B = \{(a, b) \mid a \in A \wedge b \in B\} \neq B \times A
\end{equation*}
Die Reihenfolge ist dabei relevant. Beispiel: $A = \{0, 1, 2\}, B = \{s, t\}$\\
$ A \times B = \{(0, s), (0, t), (1, s), (1, t), (2, s), (2, t)\}$
Die Anzahl Elemente des Kreuzproduktes errechnet sich nach $|K| = |A| \cdot |B|$.

\subsubsection{$n$-Tupel}
Ein $n$-Tupel ist ein Term der Form $(a_1, a_2, \dots, a_n)$.
Zwei Tupel sind gleich, wenn: $(a_1, \dots, a_n) = (b_1, \dots, b_n) \Leftrightarrow (a_1 = b_1) \wedge \dots \wedge (a_n = b_n)$	

	\section{Relationen}
Eine Relation ist eine Teilmenge des kartesischen Produktes zweier Mengen.
\begin{equation*}
	R \subseteq A \times B
\end{equation*}
Eine Relation $R = A \times A = A^2$ heisst Relation auf $A$.

Eine Relation lässt sich umkehren: $R^{-1} = \{(a,b) \in B \times A\}$

Wenn $a \in A$ und $b \in B$ und $(a, b) \in R$, dann steht $a$ in Relation zu $b$ ($aRb$ oder $a \sim_R\ b$).

\subsection{Eigenschaften von Relationen}
\begin{tabular}{ll}
	reflexiv		& $\forall a \in A :\ (a,a) \in R $\\
					& $\Delta\subseteq R$\\
	irreflexiv		& $\forall a \in A :\ \neg(a, a) \in R$\\ 
					& $\Delta\cap R= \emptyset$\\
	symmetrisch 	& $\forall a, b \in A :\ (a, b) \in R \Rightarrow (b, a) \in R$ \\
					& $R\subseteq R^{-1}$\\
	asymmetrisch 	& $\forall a, b \in A :\ (a, b) \in R \Rightarrow (b, a) \notin R$\\ 
					& $R\cap R^{-1}=\emptyset$\\
	antisymmetrisch & $\forall a, b \in A :\ (a, b) \in R \wedge (b, a) \in R \Rightarrow a = b$\\
					& $R\cap R^{-1} \subseteq \Delta$\\
	linear			& $\forall a, b \in A :\ (a, b) \in R \vee (b, a) \in R $\\
					& $R\cup R^{-1}= A\times A$\\
	transitiv 		& $\forall a, b, c \in A :\ (a, b) \in R \wedge (b, c) \in R \Rightarrow (a, c) \in R$\\
					& $R\circ R\subseteq R$\\
\end{tabular}
Achtung: reflexiv $\neq \neg$irreflexiv, symmetrisch $\neq \neg$ asymmetrisch $\neq \neg$ antisymmetrisch.

\subsection{Besondere Relationen}
\begin{description}\itemsep0em
	\item [Äquivalenzrelation]
	Eine reflexive, symmetrische und transitive Relation heisst Äquivalenzrelation.

	Beispiel: $R = $\enquote{$a$ sitzt in der selben Reihe wie $b$.} Die Relation ist:
	\begin{itemize}
		\item reflexiv (jeder sitzt in der selben Reihe wie er selbst,
		\item symmetrisch (wenn $a$ in der selben Reihe sitzt wie $b$, dann sitzt auch $b$ in der selben Reihe wie $a$)
		\item transitiv (wenn $a$ in der selben Reihe sitzt wie $b$ und $b$ in der selben Reihe sitzt wie $c$, dann sitzt auch $a$ in der selben Reihe wie $c$)
	\end{itemize}
	Bei Anwendung einer Äquivalenzrelation auf eine Menge, zerfällt die Menge in disjunkte Teilmengen (Äquivalenzklassen), deren Elemente in Relation zueinander stehen.

	\item[Halbordnung]
	Eine reflexive, antisymmetrische und transitive Relation heisst Halbordnung.
	Beispiel: $R = $\enquote{$x \leq y$}
	\begin{itemize}
		\item reflexiv ($x \leq x$)
		\item antisymmetrisch ($x \leq y \wedge y \leq x \Rightarrow x = y$) 
		\item transitiv ($x \leq y \wedge y \leq z \Rightarrow x \leq z$)
	\end{itemize}

	\item[Ordnung]
	Eine reflexive, antisymmetrische, transitive und lineare Relation heisst Ordnung.

	\item[Wohlordnung]
	Jede nicht-leere Teilmenge hat ein kleinstes Element.
	
\end{description}

\subsubsection{Funktionen}
Eine Funktion ist eine Relation, die jedem Element einer Menge, genau ein Element einer anderen Menge zuordnet.
\begin{equation*}
	F: X \rightarrow Y
\end{equation*}
$f(x)$ bezeichnet dann das Element $y$ von $Y$ für das gilt: $(x, y) \in F$.
Es gelten folgende Begriffe:
\settowidth{\MyLenA}{Definitionsmenge (Domäne)~~}
\begin{tabular}{@{}p{\the\MyLenA}%
				@{}p{\linewidth-\the\MyLenA}}
	Definitionsmenge (Domäne) 	& $\mbox{dom}(F) := X = \{x \mid \exists y ((x, y) \in F)\}$\\
	Bild von $F$			  	& $m(F) := \{y \in Y \mid \exists x (F(x) = y)\}$\\
	Erweiterte Funktion			& $G \circ F: X \rightarrow Z$ oder $G \circ F(x) := G(F(x))$\\
	Assoziativität				& $(G \circ F) \circ H = G \circ (F \circ H)$\\
\end{tabular}

Einschränkung: Sind $F: A \rightarrow B$ und $X \subseteq B$ gegeben, dann ist $F \upharpoonright X := F \cap (X \times B)$
auch eine Funktion und es gilt: $F \upharpoonright X : X \rightarrow B$. $F \upharpoonright X$ heisst Einschränkung
von $F$ nach $X$.

	\section{Natürliche Zahlen $\N$}
\subsection{Peano-Axiome}
\begin{enumerate}
\item Die 0 ist eine natürliche Zahl: $0 \in \N$
\item Der Nachfolger einer natürliche Zahl $n$ sei $\sigma(n)$: $n \in \N \Rightarrow \sigma(n) \in \mathbb{N}$
\item Die 0 ist kein Nachfolger: $\forall n \in \N \colon \sigma(n) \neq 0$
\item Zwei verschiedene Zahlen haben verschiedene Nachfolger: $\forall n, m \in \N: (n \neq m) \Rightarrow (\sigma(n) \neq \sigma(m))$
\item Für jede Teilmenge $M$ von $\mathbb{N}$ gilt, wenn $M$ die folgenden Eigenschaften erfüllt:
	\begin{enumerate}
		\item 0 ist in $M$
		\item Für jedes $n$ in $M$ ist auch $\sigma(n)$ in $M$
	\end{enumerate}
	dass $M = \N$. Formal: $0\in M \wedge \forall n \in \N \colon (n \in M \Rightarrow \sigma(n)\in M) \Rightarrow M \subseteq \N$\\
	(Schliesst parallele Strukturen aus.) Induktionsprinzip.
\end{enumerate}

\subsection{Rekursion}
Eine Menge $X \subset \N$ heisst \textit{initiales Segment} von $\N$, wenn mit jeder Nachfolgerzahl $\sigma(n) = n + 1$ auch deren Vorgänger $n$ ein Element von $X$ ist.
\begin{equation*}
 \forall n \in \N (n + 1 \in X \Rightarrow n \in X)
\end{equation*}
Allgemeine rekursive Definition: Sei $M$ eine beliebige Menge und $g: M \to M$ sowie $c \in M$, dann gibt es eine eindeutig bestimmte Funktion $f: \N \to M$ welche die Rekursionsgleichung erfüllt:
\begin{equation*}
 f(n) = \begin{cases}
         f(0) = c \mbox{ für n = 0}\\
	 f(n + 1) = g(f(n)) \mbox{ sonst}
        \end{cases}
\end{equation*}

\subsection{Verknüpfung}
Ist $A$ eine Menge, dann nennen wir die Abbildung $\circ: A \times A \to A$ eine Verknüpfung auf $A$ (Bsp. Addition, Multiplikation \dots).
Seien $g: \N \to \N$ und $c: \N \to \N$, so gelten:
\begin{align*}
 n \circ 0& = c(n)\\
 n \circ (k + 1)& = g(n \circ k)
\end{align*}


\subsection{Addition}
Die Addition (Verknüpfung $+$)ist rekursiv definiert:
\begin{enumerate}
	\item Eine Zahl + 0 ist wieder die Zahl selbst. $n + 0 = n$
	\item Wird zu einer Zahl, der Nachfolger einer anderen Zahl addiert, ist das so, wie wenn
	erst die beiden Zahlen addiert werden und dann der Nachfolger gebildet wird: $m + \sigma(n) = \sigma(m + n)$
\end{enumerate}

Zu zeigen: $\forall n \in \N \colon 0 + n = n$
\begin{proof}[Beweis: Vollständige Induktion über $n$]
	\begin{align*}
		n	& = 0	& \mbox{(Induktionsanfang)}\\
		0 + 0& = 0	& \mbox{(Addition 1: Für die 0 bewiesen)}\\
	\end{align*}
	Induktionschritt: Wir schliessen von $n = k$ auf $n = \sigma(k)$
	\begin{align*}
		0 + k 		& = k	& \mbox{Induktionsannahme; für $k$ gilt es bereits}\\
		0 + \sigma(k) 	& = \sigma(k)		& \mbox{(Neu zu zeigen: gilt für Nachfolger)}\\
		0 + \sigma(k)	& = \sigma(0 + k)	& \mbox{(Addition 2)}\\
				& = \sigma(k)		& \mbox{(Induktionsannahme: für $\sigma(k)$ bewiesen)}
	\end{align*}
\end{proof}
Der Trick ist, die Induktionsannahme im Beweis zu verwenden. Es wird nur angenommen, dass die Formel für eine Zahl gilt.
Bewiesen wird dann, wenn die Formel für eine Zahl gilt, gilt sie auch für alle anderen.

Für alle $n, k \in \N$ gilt: $(n + 1) + k = n + (k + 1)$

% \subsubsection{Direkter Beweis}
% Zu zeigen: $\forall n \in \N \colon n + 1 = \sigma(n)$
% \begin{proof}
% 	Sei $n$ fest, aber beliebig
% 	\begin{align*}
% 	n + 1	& = n + \sigma(0) 	& \mbox{Definition der 1}\\
% 			& = \sigma(n + 0)	& \mbox{Addition 2}\\
% 			& = \sigma(n)		& \mbox{Addition 1~~~~}\qedhere
% 	\end{align*}
% \end{proof}
% 
% Rechenbeispiel: $1 + 2 = ?$
% \begin{align*}
% 	1 + 2	& = \sigma(0) + \sigma(\sigma(0))			& \mbox{Einsetzen der Definitionen}\\
% 			& = \sigma(\sigma(0) + \sigma(0))			& \mbox{Addition 2 auf das Ganze}\\
% 			& = \sigma(\sigma(\sigma(0) + 0))			& \mbox{Addition 2 in der Klammer}\\
% 			& = \sigma(\sigma(\sigma(0)))				& \mbox{Addition 1 auf die innere Klammer}\\
% 			& = 3										& \mbox{Einsetzen der Definition}
% \end{align*}
% Durch mehrfaches Anwenden von Addition 2 wird nach und nach ein $\sigma$ nach vorne geholt. Dieser Teil ist rekursiv.

\subsubsection{Summen}
\begin{align*}
	\sum_{i=1}^0 a_i& = 0&
	\sum_{i=1}^{n+1} a_i& = \left( \sum_{i=1}^n a_i\right ) + a_{n+1}\\
\end{align*}

\subsubsection{Die Gaussche Summenformel}
\begin{equation*}
	1 + 2 + \dots + n = \Sigma_{i = 1}^{n} i = \frac{n \cdot (n+1)}{2}
\end{equation*}
Zu zeigen: $\forall n \in \N\textbackslash\{0\} \colon \Sigma_{i = 1}^n i = \frac{n \cdot (n + 1)}{2}$
\begin{proof}
	Induktion über $n$
	\begin{align*}
		n& = 1 &\mbox{(Induktionsanfang)}\\
		\Sigma_{i=1}^1 i & = 1 = \frac{2}{2} = \frac{1 \cdot 2}{2} = \frac{1 \cdot(1 + 1)}{2} &\mbox{Beweis für 1}
	\end{align*}
	Induktionsschritt: Wir schliessen von $n = k$ auf $n = k + 1$
	\begin{align*}
		\Sigma_{i=1}^k i& = \frac{k \cdot (k + 1)}{2} &\mbox{Induktionsannahme}\\
		\Sigma_{i=1}^{k+1} i& = \frac{(k + 1) \cdot ((k + 1) + 1)}{2} &\mbox{Neu zu zeigen}\\
		\Sigma_{i=1}^{k+1} i& = \Sigma_{i=1}^{k} i + (k + 1)&\mbox{Summe um einen Term verkürzt}\\
		& = \frac{k \cdot (k + 1)}{2} + (k + 1)& \mbox{Induktionsannahme}\\
		& = \frac{k \cdot (k + 1)}{2} + \frac{2 \cdot (k + 1)}{2} & \mbox{Rechts um zwei erweitert}\\
		& = \frac{k \cdot (k + 1) + 2 \cdot (k + 1)}{2}& \mbox{Zusammenzug}\\
		& = \frac{(k + 1) \cdot (k + 2)}{2} & k + 1 \mbox{ausklammern}\\
		& = \frac{(k + 1) \cdot ((k + 1) + 1)}{2} & \mbox{Terme vertauscht~~~~}\qedhere
	\end{align*}
\end{proof}

\subsubsection{Rechenregeln der Addition}
\settowidth{\MyLenA}{Neutrales Element~~}
\begin{tabular}{@{}p{\the\MyLenA}%
				@{}p{(\linewidth - \the\MyLenA)/2}}
	Neutrales Element 		& $0 + n = n$\\
	Kommutativität			& $n + m = m + n$\\
	Assoziativität			& $(n + m) + k = n + (m + k)$\\
	Kürzbarkeit				& $(n + k = m + k) \Rightarrow n = m$\\
\end{tabular}

\subsection{Multiplikation}
Die Multiplikation ist rekursiv definiert:
\begin{enumerate}
	\item Eine Zahl mal 0 ist 0: $n \cdot 0 = 0$\\
	\item Wird eine Zahl mit dem Nachfolger einer anderen Zahl multipliziert, ist das so, wie wenn
	erst die beiden Zahlen multipliziert und anschliessend die ursprüngliche Zahl addiert wird: $m \cdot \sigma(n) = \sigma(m \cdot n) + m$
\end{enumerate}

\subsubsection{Potenzen}
\begin{enumerate}
	\item Eine Zahl hoch 0 ist 1: $n^0 = 1$
	\item Eine Zahl hoch dem Nachfolger einer anderen Zahl ist die eigentliche Zahl multipliziert mit der eigentlichen Zahl hoch der anderen Zahl: $m^{\sigma(n)} = m \cdot (m^n)$
\end{enumerate}

\subsubsection{Fakultät}
\begin{enumerate}
	\item $0! = 1$
	\item $\sigma(m)! = \sigma(m) \cdot m!$
\end{enumerate}

Beispiel: $n! > 2^n$ für $n \geq 4$
\begin{proof}
	Anfang: $4! = 24 > 16 = 2^4$\\
	Annahme: $k! > 2^k$\\
	Zu zeigen: $(k + 1)! > 2^{k + 1}$
	\begin{equation*}
		(k + 1)! = (k + 1) \cdot k! > (k + 1) \cdot 2^k > 2 \cdot 2^k = 2^{k+1}\qedhere
	\end{equation*}
\end{proof}
Beweisstrategie: Von links und rechts der Mitte hin annähern.


\subsubsection{Rechenregeln der Multiplikation}
\settowidth{\MyLenA}{Neutrales Element~~}
\begin{tabular}{@{}p{\the\MyLenA}%
				@{}p{(\linewidth - \the\MyLenA)/2}}
	Absorbtion				& $0 \cdot n = 0$\\
	Neutrales Element 		& $1 \cdot n = n$\\
	Kommutativität			& $n \cdot m = m \cdot n$\\
	Assoziativität			& $(n \cdot m) \cdot k = n \cdot (m \cdot k)$\\
	Distributivität			& $n \cdot (m + k) = n \cdot m + n \cdot k$\\
\end{tabular}\\
Partialsummen
\begin{equation*}
	\sum_{i=1}^n (c \cdot(a_1 + b_1)) = c \left (\sum_{i=1}^n a_i + \sum_{i=1}^n b_i \right )
\end{equation*}

\subsubsection{Produkte}
\begin{align*}
	\prod_{i=1}^0 a_i& = 1&
	\prod_{i=1}^{n + 1} a_i& = a_{n+1} \cdot \prod_{i=1}^n a_i\\
\end{align*}

\subsection{Die Ordnung der natürlichen Zahlen}
Es sei $A$ eine Menge und $\preceq$ eine Ordnung auf $A$.
\begin{enumerate}
	\item Sei $X \subset A$, dann ist $m \in X$ dass minimale Element, wenn gilt: $\forall x \in X (m \preceq x)$.
	\item Das Paar $(A, \preceq)$ heisst Wohlordnung, wenn jede nichtleere Teilmenge $X \subset A$ ein minimales Element besitzt.
\end{enumerate}
Die 0 ist die kleinste natürliche Zahl.

\subsubsection{Relationen}
\settowidth{\MyLenA}{$a \leq b$ ~~}
\begin{align*}
	n \leq m & :\Leftrightarrow \exists k \in \N (m = k + n)\\
	n < m & :\Leftrightarrow (n \leq m \wedge n \neq m)\\
\end{align*}
\begin{satz}
Das Paar $(\N, \leq)$ ist eine Wohlordnung. 
\end{satz}

\subsubsection{Rechenregeln}
\begin{align*}
  0 & \leq n\\
  n < m& \Leftrightarrow (n + 1) \leq m\\
  n < m& \Leftrightarrow (n + k) < (m + k)\\
  n \leq m& \Leftrightarrow (n + k) \leq (m + k)\\
  (n \leq n') \wedge (m \leq m')& \Leftrightarrow (n + m) \leq (n' + m')\\
  \mbox{Für }c \in \N (c \neq 0)&: c\cdot n < c \cdot m \Leftrightarrow n < m\\
  \mbox{Für }c \in \N (c \neq 0)&: c\cdot n \leq c \cdot m \Leftrightarrow n \leq m\\
  n < n' \wedge m < m' &\Rightarrow n + m < n' + m' \wedge n \cdot m < n' \cdot m'\\
  n \leq n' \wedge m \leq m' &\Rightarrow n + m \leq n' + m' \wedge n \cdot m \leq n' \cdot m'
\end{align*}
  

\begin{proof}
Zu zeigen: $3 < 5$
	\begin{align*}
		(3 < 5) & = (3 < \sigma(4))				& \mbox{(Definition der 5)}\\
		& \Leftrightarrow (3 = 4) \vee (3 < 4)	& \mbox{(Kleiner 2: links falsch)}\\
		(3 < 4) & = (3 < \sigma(3))				& \mbox{(Definition der 4)}\\
		& \Leftrightarrow (3 = 3) \vee (3 < 3)	& \mbox{(Kleiner 2: (3 = 3) ist wahr)}
	\end{align*}
	\qedhere
\end{proof}


	\section{Ganze Zahlen}
$\Z = \{\cdots -2, -1, 0, 1, 2, \dots\}$\\
Es gilt: $x < y \Leftrightarrow \exists n \in \N > 0 \mid x + n = y$\\
\subsection{Rechenregeln}
\settowidth{\MyLenA}{Neutrales Element bzgl. $+$~~}
\begin{tabular}{@{}p{\the\MyLenA}%
		@{}p{(\linewidth - \the\MyLenA)/2}}
					& $-1 \dot z = -z$\\
					& $-(-z) = z$\\
	Inverses Element bzgl. $+$ & $-z + z = 0$\\
	Absorbtion	& $0 \cdot z = 0$\\
	Neutrales Element bzgl. $+$ & $0 + z = z$\\
	Neutrales Element bzgl. $\cdot$ & $1 cdot z = z$\\
	Assoziativität bzgl. $+$ & $a + (b + c) = (a + b) + c$\\
	Assoziativität bzgl. $\cdot$ & $a \cdot (b \cdot c) = (a \cdot b) \cdot c$\\
	Kommutativität bzgl. $+$ & $a + b = b + a$\\
	Kommutativität bzgl. $\cdot$ & $a \cdot b = b \cdot a$\\
	Distributiviät & $a  \cdot (b + c) = a \cdot b + a \cdot c$\\
\end{tabular}

\subsection{Definitionen}
\subsubsection{Subtraktion (-:$\Z \to \Z$)}
\begin{equation*}
	a - b = a + (- b)
\end{equation*}

\subsubsection{Betrag ($|.|: \Z \to \Z$)}
\begin{equation*}
	|z| = \begin{cases}
	      	z & \mbox{falls } z \in \N\\
			-1 \cdot z & \mbox{sonst}
	      \end{cases}
\end{equation*}

\subsubsection{Teilbarkeit}
\begin{equation*}
	x \mid y \Leftrightarrow \exists q \in \Z (y = x \cdot q)
\end{equation*}
Teilbarkeit ist transitiv: $x \mid y \wedge y \mid z \Rightarrow x \mid z$.

Beispiel: $\forall a, b, c, d \in \Z\ (a \mid b) \wedge (a \mid d) \Rightarrow (a \cdot c) \mid (b \cdot d)$
\begin{proof}
	Es seien $a, b, c, d$ fest, aber beliebig.\\
	Es gelte $a \mid b$ und $c \mid d$\\
	Zeige: $a \cdot c \mid b \cdot d$\\
	\begin{align*}
		a \mid b & \Leftrightarrow \exists q_1 \in \Z\ b = a \cdot q_1\\
		c \mid d & \Leftrightarrow \exists q_2 \in \Z\ d = c \cdot q_2\\
		(a \cdot q_1) \cdot (c \cdot q_2) = b \cdot d\\
	\end{align*}
Mit $q_3 = q_1 \cdot q_2$ gilt: $a \cdot c \cdot q_3 = b \cdot d \Rightarrow (a \cdot c) \mid (b \cdot d)$\qedhere
\end{proof}

Beispiel: $\forall a, b, c \in \Z\ a \mid b \wedge a \nmid c \Rightarrow a \nmid (b + c)$
\begin{proof}
	Es seien $a, b, c$ fest, aber beliebig.\\
	Es gelte $a \mid b$ und $a \nmid b$.\\
	Zeige: $a \nmid (b + c)$.
	Anmerkung: Beweis durch Widerspruch: $\neg (a \mid b \wedge a \nmid c \Rightarrow a \nmid (b + c)) = a \mid b \wedge a \nmid c \wedge a \mid (b + c)$\\
	Annahme: $a \mid (b + c)$\\
	\begin{align*}
		a \mid b& \Leftrightarrow a \cdot q_1 = b\\
		a \mid (b + c) & \Leftrightarrow a \cdot q_2 = b + c\\
		a \cdot q_2& = a \cdot q_1 + c\\
		a (q_2 - q_1)& = c \Leftrightarrow a \mid c. \mbox{Widerspruch zum geltenden}\qedhere
	\end{align*}
\end{proof}

\subsubsection{Hilfssätze}
\begin{align*}
  s \mid t \wedge s \mid u & \Rightarrow s \mid (t + u)\\
  s \mid t \wedge s \mid u & \Rightarrow s \mid (t - u)
\end{align*}

\subsubsection{Teilen mit Rest}
Sind $x, y \in \N \setminus \{0\}$, dann gibt es eindeutig bestimmte Zahlen, so dass:
\begin{enumerate}
	\item $y = q \cdot x + r$
	\item $r < x$
\end{enumerate}

\subsubsection{Kleinstes gemeinsames Vielfaches (KGV)}
(engl. \textit{least common multiple}, zum Erweitern von Brüchen)\\
\begin{equation*}
	\mbox{kgV}(x, y) = \min\{q \in \N \mid x \mid q \wedge y \mid q\}
\end{equation*}


\subsubsection{Grösster gemeinsamer Teiler (GGT)}
(engl. \textit{greatest common divisor}, zum Kürzen von Brüchen)\\
\begin{align*}
	\mbox{ggT}(x, y) &= \max\{q \in \N \mid q \mid x \wedge q \mid y\}\\
	\mbox{kgV}(x, y) \cdot \mbox{ggT}(x, y) &= x \cdot y
\end{align*}


\subsubsection{Euklidischer Algorithmus}
\begin{equation*}
	\mbox{ggt}(n, m) = \mbox{ggt}(n, m - n) = \mbox{ggt}(m, m - n)
\end{equation*}

\subsubsection{Erweiterter euklidischer Algorithmus}
Erklärt sich am besten an einem Beispiel:
Gesucht: ggt$(64, 14) = s \cdot 64 + t \cdot 14$
\begin{center}

\begin{tabular}
{
c%
@{}|p{0.18\linewidth}%
@{}|p{0.18\linewidth}%
@{}|p{0.24\linewidth}%
@{}|p{0.23\linewidth}%
}
 \multicolumn{1}{c}{\textbf{n}} & \multicolumn{1}{|c}{\textbf{a}} & \multicolumn{1}{|c}{\textbf{q}} & \multicolumn{1}{|c}{\textbf{s}} & \multicolumn{1}{|c}{\textbf{t}} \\\hline
 1 & \multicolumn{1}{c|}{64} & & \multicolumn{1}{|c}{1} & \multicolumn{1}{|c}{0} \\\hline
 2 & \multicolumn{1}{c|}{14} & & \multicolumn{1}{|c}{0} & \multicolumn{1}{|c}{1} \\\hline
 3 & $a_1 \% a_2 = 8$ & ${a_1}/{a_2} = 4$ & $s_1 - q_3 \cdot s_2 = 1$ 	& $t_1 - q_3 \cdot t_2 = -4$ \\\hline
 4 & $a_2 \% a_3 = 6$ & ${a_2}/{a_3} = 1$ & $s_2 - q_4 \cdot s_3 = -1$ 	& $t_2 - q_4 \cdot t_3 = 5$ \\\hline
 \textbf{5} & $a_3 \% a_4 = 2$ & ${a_3}/{a_4} = 1$ & $s_3 - q_5 \cdot s_4 = 2$ 	& $t_3 - q_5 \cdot t_4 = -9$ \\\hline
 6 & $a_4 \% a_5 = 0$ & ${a_4}/{a_5} = 3$ & $s_4 - q_6 \cdot s_5 = -7$ 	& $t_4 - q_6 \cdot t_5 = 32$ \\\hline
\end{tabular}
\end{center}
Die Lösung steht in Zeile 5 (mit $s = 2$ und $t = -9$):\\ ggt$(64, 14) = 2 = 2 \cdot 64 + (-9) \cdot 14 = 128 - 126 = 2$ 

\subsubsection{Teilerfremdheit}
Zahlen sind teilerfremd, wenn ggt$(m, n) = 1 \Leftrightarrow k \cdot x + k' \cdot y$.


	\section{Zahlentheorie}
% \subsection{Satz vom kleinsten Teiler}
% Der kleinste Teiler $d > 1$ einer natürlichen Zahl $n \geq 2$ ist eine Primzahl.
% 
% \begin{proof}
% 	Sei $T(n)$ die Menge aller Teiler von $n$.\\
% 	$T(n)\textbackslash\{1\}$ ist nicht leer, weil $n \in T(n)$\\
% 	dann gibt es nach dem Wohlordnungsprinzip 
% 	eine kleinste Zahl $d$ in $T(n)\textbackslash\{1\}$.\\
% 	Zu zeigen: $d$ ist eine Primzahl\\
% 	Annahme: $d$ ist keine Primzahl\\
% 	Dann: $\exists a, b \in \N \setminus\{1, d\} \colon a \cdot b = d \wedge a, b < d$\\
% 	$\Rightarrow a \mid d$\\
% 	Da $a$ kleiner als $d$ sein muss, kann $d$ nicht das kleinste Element in $T(n)\textbackslash{1}$ sein.\\
% 	Das ist ein Widerspruch zur Annahme. Die Annahme muss falsch sein, dann muss aber $d$ eine Primzahl sein.
% 	\qedhere
% \end{proof}

% \subsection{Hauptsatz}
% Jede natürliche Zahl $n \geq 2$ besitzt eine eindeutige Primfaktorzerlegung.
% Die Reihenfolge der Primfaktoren kann variieren.
% 
% \begin{proof}
% 	Sei $n \in \N \setminus\{1\}$\\
% 	Fall I: $n$ ist eine Primzahl, ein eindeutiger Primfaktor gefunden ($n$).\\
% 	Fall II: $n$ ist keine Primzahl\\
% 	\begin{equation*}
% 		\Rightarrow \exists d_1, q_1 \in \N \setminus\{1\} \colon d_1 \cdot q_1 = n
% 	\end{equation*}
% 	und $d_1$ ist das kleinste Element in $T(n)\setminus\{1\}$, also $d_1$ ist Primzahl\\
% 	Betrachte $q_1$:
% 	Fall I: $q_1$ ist eine Primzahl, eindeutige Primfaktorzerlegung gefunden.\\
% 	Fall II: $q_1$ ist keine Primzahl, dann wiederhole diesen Prozess mit $q_1$\\ 
% 	\qedhere
% \end{proof}
% Es gibt also eine Primfaktorzerlegung.
% Bleibt zu beweisen, dass es nur eine Primfaktorzerlegung gibt.
% \begin{proof}
% 	Sei $n$ die kleinste natürliche Zahl mit mehr als einer Primfaktorzerlegung.\\
% 	\begin{align*}
% 	n& = p_1 \cdot p_2 \cdot \dots \cdot p_n = q_1 \cdot q_2 \cdot \dots \cdot q_m & p, q \in \mathbb{P}\\
% 	\mbox{$p_i$ und $q_j$ sind paarweise verschieden}	
% 	\end{align*}
% 
% \end{proof}

\subsection{Primzahlen}
Primzahlen sind folgendermassen definiert:
\begin{align*}
 \forall n, m \in \N& (P|n \cdot m \Rightarrow p |n \wedge p | m) \mbox{ und } p \neq 1\\
  T(P) & = \{1, p\} \mbox{ und } p \neq 1\\
  |T(P)|& = 2\\
\end{align*}
\begin{itemize}\itemsep0em
\item Es gibt unendlich viele Primzahlen!
\item Jede natürliche Zahl $n \in \N$ ist das Produkt endlich vieler Primzahlen: $k = \prod_{i=1}^n p_i = p_1 \cdot p_2 \cdot \dots$
\item Der kleinste Teiler $d > 1$ einer natürlichen Zahl $n \geq 2$ ist eine Primzahl (Satz vom kleinsten Teiler) 
\end{itemize}

\subsection{Modulare Arithmetik}
Sei $n \in \N$ beliebig. Wir definieren die Modulo-Relation $\equiv_n$ auf $\Z$ wie folgt:
\begin{equation*}
 r \equiv_n s :\Leftrightarrow n | (r - s)
\end{equation*}
Synonyme schreibweisen:
\begin{align*}
 r &\equiv_n x \\
 r \mod n& = x
\end{align*}

\subsubsection{Chinesischer Restsatz}
\begin{align*}
  x& \equiv_{m_i} a_i\\
  M& := \prod m_1\\
  M_i & := M / m_i
\end{align*}
Finde $r_i, s_i$ sodass $r_i \cdot m_i + s_i \cdot M_i = 1$ (Euklid). Setze $e_i = s_i \cdot M_i$. Die Lösung ist dann $x = \sum a_i \cdot e_i \equiv y \mod M$

Beispiel:
\begin{align*}
 x &\equiv_3 2& \to a_1& = 2\\
 x &\equiv_4 3& \to a_2& = 3\\
 x &\equiv_5 2& \to a_3& = 5\\
 M &= m_1 \cdot m_2 \cdot m_3 = 3 \cdot 4 \cdot 5 = 60\\
 M_1 & = M / m_1 = 20, & (M_2& = 15, M_3 = 12)\\
\end{align*}
Als nächstes $s_1$ mit Hilfe des erweiterten Euklid bestimmen:
\begin{align*}
 1 &= r_1 \cdot m_1 + s_1 \cdot M_1  \Rightarrow r_1 = 7, s_1 = -1 \Rightarrow e_1 = -20\\
 1 &= r_2 \cdot m_2 + s_2 \cdot M_2  \Rightarrow r_2 = 4, s_2 = -1 \Rightarrow e_1 = -15\\
 1 &= r_3 \cdot m_3 + s_3 \cdot M_3  \Rightarrow r_3 = 5, s_1 = -2 \Rightarrow e_1 = -24\\
 x &= 2 \cdot (-20) + 3 \cdot (-15) + 5 \cdot (-24) = - 133 = 47 \mod 60\\
\end{align*}


	\section{Grundstrukturen}
\begin{description}
	\item [n-stellige Verknüpfung]
	Sind $A_1, \dots, A_n, B$ Mengen, dann nennt man eine Abbildung $\circ : A_1 \times \dots \times A_n \rightarrow B$
	eine $n$-stellige Verknüpfung auf B. $\circ A^n \rightarrow A$ nennt man eine $n$-stellige Verknüpfung auf A. 
	
	\item[Einfache algebraische Strukur] bezeichnet ein Paar $S = (A, (f_i)_{i \in I})$. Dabei heisst die Menge $A$ Grundmenge von $S$. $(f_i)_{i \in I}$ ist eine endliche Familie von Verknüpfungen auf diese Grundmenge.

	\item [Zusammengesetze algebraische Struktur] bezeichnete die verallgemeinerte einfache algebraische Struktur. Sie ist ein Tupel $S = (A_1, \dots, A_n, (f_i)_{i \in I})$. Sie besteht aus endlich vielen Grundmengen ($A_1, \dots, A_n$) und einer endlichen Familie von von Vernupfungen, so dass es für alle $i \in I$ natürliche Zahlen $p, m$ und Grundmengen $A_r, A_s, A_k$ gibt mit:
		\begin{equation*}
			f_i: A^p_r  \times A^m_s \rightarrow A_k
		\end{equation*}

	\item [Signatur von $S$] $(f_i)_{i \in I}$ heisst Signatur von $S$. 
\end{description}

Für zweiwertige (binäre) Verknüpfunge $\circ$ werden folgende Begriffe verwendet:
\settowidth{\MyLenA}{Kommutativität~~}
\begin{tabular}{@{}p{\the\MyLenA}%
				@{}p{(\linewidth - \the\MyLenA)}}
	Assoziativitat: & wenn $\forall a, b, c \in A (A \circ (b \circ c) = (a \circ b) \circ c)$\\
	Kommutativität: & wenn $\forall a, b \in A (A \circ b = b \circ a)$\\
\end{tabular}

\subsection{Neutralität}
Ein Element $e_i \in A$ ist:
\settowidth{\MyLenA}{linksneutral bezüglich $\circ$ falls~}
\begin{tabular}{@{}p{\the\MyLenA}%
				@{}p{(\linewidth - \the\MyLenA)}}
	linksneutral bezüglich $\circ$ falls & $\forall a \in A (e_i \circ a = a)$\\
	linksneutral bezüglich $\circ$ falls & $\forall a \in A (a \circ e_1 = a)$\\
	neutral bezüglich $\circ$ falls & $\forall a \in A (e_i \circ a = a \circ e_i = a)$\\
\end{tabular}
\\
Wenn es ein neutrales Element gibt, kann es kein zweites neutrales Element geben.


\subsection{Halbgruppen, Gruppen und Monoide}
Eine Struktur $(G, \circ)$ bestehend aus einer Menge $G$\\
 und einer Verknüpfung $\circ: G \times G \rightarrow G$ heisst:
\settowidth{\MyLenA}{Kommutative Gruppe ~~}
\begin{tabular}{@{}p{\the\MyLenA}%
				@{}p{(\linewidth - \the\MyLenA)}}
	Halbgruppe & wenn die Verknüpfung assoziativ ist\\
	Monoid & wenn zusätzlich ein neutrales Element $e \in G$ existiert\\
	Gruppe & wenn zusätzlich ein für jedes $g \in G$ ein inverses Element $g^{-1}$existiert\\
	Kommutative Gruppe & wenn die Verknüpfung zusätzlich kommutativ ist.\\
\end{tabular}
Für inverse Elemente gilt: $(a^{-1})^{-1} = a$. (Das inverse vom inversen ist das element selbst)

\begin{itemize}
	\item In Gruppen kann gekürzt werden ($a \cdot x = b \cdot x \Rightarrow a = b$)
\end{itemize}

\subsubsection{Beispiele für Halbgruppen, Gruppen und kommutative Gruppen}
\settowidth{\MyLenA}{Kommutative Gruppe ~~}
\begin{tabular}{@{}p{\the\MyLenA}%
				@{}p{(\linewidth - \the\MyLenA)/2}}
	Halbgruppen & $(\mathbb{N}, +), (\mathbb{Z, -})$\\
	Monoid & $(\mathbb{N} \cup {0}, +)$\\
	Gruppe & $(\mathbb{Q}, *)$\\
	Kommutative Gruppe & $(\mathbb{Z}, +)$\\
\end{tabular}

\subsection{Unterstrukturen}
Sei $(A, \circ)$ eine Struktur und $U \subset A$. $U$ heisst abgeschlossen falls gilt:
	\begin{equation*}
		\forall a, b \in U (a \circ b \in U)
	\end{equation*}
Je nach übergeordneter Struktur handelt es sich um Unterhalbgruppen, Untermonoide oder Untergruppen.

\subsubsection{Regeln}
\begin{itemize}
	\item Ist $(G, \circ)$ eine Halbgruppe und seien $(U_i)_{i \in I}$ Unter\dots, dann ist $\bigcap_{i \in I} U_i$ ebenfalls eine Unter\dots.
\end{itemize}

Jede (Halb-) Gruppe besitzt eine kleinste Unter(halb)gruppe und jeder Monoid
besitzt einen kleinsten Untermonoid, die eine gegebene Teilmenge der (Halb-) Gruppe bzw.
des Monoids enthalten.


	\section{Morphismen}
\subsection{Homomorphismus}
Ein (Halb-) Gruppenhomomorphismus ist die Abbildung $f: G \rightarrow G'$ einer Struktur $(G, \circ)$ in eine andere Struktur $(G', \sim)$, so dass für alle $a, b \in G$ gilt:
\begin{equation*}
	f(a \circ b) = f(a) \sim f(b)
\end{equation*}
Beim Monoidhomomorphismus wird zusätzliche das neutrale Element von $(G, \circ)$ auf das neutrale Element von $(G', ~)$ abgebildet.

\begin{description}
	\item [Monomorphismus] bezeichnet injektive (d.\,h. jedes $a \in G$ wird auf ein anderes $b \in G'$ abgebildet) Homomorphismen.
	\item [Epimorphismus] bezeichnet surjektive (d.\,h. durch die Abbildung wird jedes $b \in G'$ erreicht) Homomorphismen
	\item [Isomorphismus] bezeichnet Homomorphismen die sowohl injektiv als auch bijektiv sind.
\end{description}

Nicht jeder Homomorphismus zwischen zwei Monoiden ist zwingend ein Monoidhomomorphismus.
Beispiel: 
\begin{align*}
f:& (\mathbb{N}, +) \rightarrow (\mathbb{N}, \cdot)	\\
f(0)& = 0\\
f(0 + 0)& = f(0) \cdot f(0) = 0
\end{align*}
Aber das neutrale Element der Addition $(1)$ wird nicht auf das neutralen Element der Multiplikation abgebildet.


\subsection{Regeln}

\begin{enumerate}
	\item Sind $f: (G, \cdot) \rightarrow (G', \circ)$ und $h: (G', \sim) \rightarrow (G'', \bullet)$ Homomorphismen, dann ist auch $h \circ f: (G, \cdot) \rightarrow (G'', \bullet)$ ein entsprechender Homomorphismus.
	\item Ist $f: (G, \sim) \rightarrow (G', \circ)$ ein Homomorphismus, dann ist das Bild $Im(f) \subset G'$ eine entsprechende Unterstruktur von $(G', \circ).$
	\item Es sei $f: G \rightarrow G'$ ein Gruppenhomomorphismus zwischen den Gruppen $(G, \sim)$ und $(G', \circ)$ mit den neutralen Elementen $e$ und $e'$, dann gelten:
	\begin{itemize}
		\item $f(e) = e'$
		\item $\forall a \in G(f(a^{-1}) = f(a)^{-1})$
	\end{itemize}
	\item Ist $f: (G, \sim) \rightarrow (G', \circ)$ ein Gruppenhomomorphismus, dann der Kern $ker(f) = \{a \in G | f(a) = e'\}$
	\item Ist $f: (G, \circ) \rightarrow (G', \sim)$ ein Gruppenhomomorphismus mit $ker(f) = \{e\}$,
	dann ist $f$ injektiv.
	\item Ist $f: (G, \sim) \rightarrow (G', \circ)$ ein Isomorphismus, dann ist auch $f^{-1}: (G', \circ) \rightarrow (G, \sim)$ ein Isomorphismus.
\end{enumerate}

\begin{description}
	\item [Bild (Im)] Menge die durch eine Funktion erzeugt wird.
	\item [Kern] Alle $g_n \in G$ die auf $e \in G'$ abgebildet werden. Wobei $e$ das neutrale Element von $G'$ ist.
\end{description}


	\section{Ringe und Körper}
Eine Struktur $(G, +, \cdot)$  heisst Ring, wenn folgende Bedingungen erfüllt sind:
\begin{enumerate}
	\item $(G, +)$ ist eine kommutative Gruppe
	\item $(G, \cdot)$ ist eine Halbgruppe
	\item Es gilt das Distributivgesetz, d.\,h. für alle Elemente $a, b, c$ des Ringes gelten:
	\begin{itemize}
		\item $a \cdot (b + c) = (a \cdot b) + (a \cdot c)$
		\item $(a + b) \cdot c = (a \cdot c) + (b \cdot c)$
	\end{itemize}
\end{enumerate}

\subsection{Konventionen}
\begin{itemize}
	\item Wenn $(R, +, \cdot)$ ein Ring ist, dann bezeichnen wir das neutrale Element von $(G, +)$ mit 0
	\item Falls vorhanden bezeichnen wir das neutrale Element von $(G, \cdot)$ mit 1.
	\item Das inverse Element von $g \in G$ bezüglich $\sim$ bezeichnen wir mit $-g$.
	\item Das inverse Element von $g \in G$ bezüglich $\circ$ bezeichnen wir mit $g^{-1}$.
\end{itemize}

\subsection{Typische Ringe}
$(\mathbb{Z}, +, \cdot), (\mathbb{Q}, +, \cdot)$ und $(\mathbb{Z}, +, \cdot)$. Sowie der Nullring $(\{0\}, +, \cdot)$

\subsection{Potenz}
Sei $(G, +, \cdot)$ ein Ring mit 1, dann ist die $n$-te Potenz von $g \in G$ definiert als:
\begin{align*}
	r^0& := 1\\
	r^{n + 1}& := r \cdot r^n
\end{align*}

\subsection{Rechenregeln in Ringen}
Sei $G, +, \cdot)$ ein Ring. Für alle Elemente $a, b \in R$ und alle Zahlen $n, k \in \mathbb{N}$ gelten folgende Identitäten:
\begin{enumerate}
	\item $0 \cdot a = a \cdot 0 = 0$
	\item $(- a) = (-1) \cdot a$
	\item $- (a \cdot b) = (- a) \cdot b = a \cdot (-b)$
	\item $(- a) \cdot (- b) = a \cdot b$
	\item $0 = 1 \Rightarrow G = \{0\}$
	\item $a^n \cdot a^k = a^{n + k}$
	\item $a^{n \cdot k} = (r^n)^k$
\end{enumerate}

\subsection{Begriffe}
\settowidth{\MyLenA}{rechter Nullteiler~~}
\begin{tabular}{@{}p{\the\MyLenA}%
				@{}p{(\linewidth - \the\MyLenA)}}
	rechter Nullteiler & $b \in G$ heisst rechter Nullteiler, falls ein $a \in G\setminus\{0\}$ existiert, so dass $a \cdot b = 0$\\
	linker Nullteiler & $b \in G$ heisst linker Nullteiler, falls ein $a \in G\setminus\{0\}$ existiert, so dass $b \cdot a = 0$\\
	Nullteiler & ist sowohl rechter, wie auch linker Nullteiler\\
	Integritätsring & Die Verknüpfung $\cdot$ ist kommutativ und $0 \in G$ ist der einzige Nullteiler.\\
	Körper & Integritätsring mit $(G\setminus \{0\}, \cdot)$ ist eine kommutative Gruppe.
\end{tabular}

In einem Integritätsring gilt stets: $1 \neq 0$.\\

Ein kommuativer Ring $(G, +, \cdot)$ mit $G \neq \{0\}$, ist genau dann ein Integritätsring, wenn für jedes $g \in G\setminus\{0\}$ die Abbildung $f_g: (G, +) \rightarrow (G, +)$ mit $f_g(x) := g \cdot x$ ein injektiver Gruppenhomomorphismus ist.

Ein Integritätsring $(R, +, \cdot)$ ist genau dann ein Körper, wenn alle Funktionen $f_g: G \rightarrow G$ mit $f_g(x) = r \cdot x$ mit $r \in G\setminus\{0\}$ surjektiv sind.

\subsection{Folgerungen}
\begin{enumerate}
	\item Jeder endliche Integritätsring ist ein Körper.
	\item Für $p \in \mathbb{N} gilt: (\mathbb{Z}_{/p}, +, \cdot)$ ist ein Körper $\Leftrightarrow p$ ist eine Primzahl.
\end{enumerate}

\subsection{Ringhomomorphismus}
Es seien die Ringe $(R, +, \cdot)$ und $(R', +', \cdot')$ gegeben. Ein Ringhomomorphismus $f: (R, +, \cdot) \rightarrow (R', +', \cdot')$ ist eine Abbildung $f : R \rightarrow R'$, die:
\begin{enumerate}
	\item Ein Gruppenhomomorphismus $f: (R, +) \rightarrow (R', +')$ und
	\item ein Halbgruppenhomomorphismus $f: (R, \cdot) \rightarrow (R', \cdot')$ ist.
	\item Sind $(R, \cdot)$ und $(R', \cdot')$ Monoide, muss $f$ ein Monoidhomomorphismus sein.
\end{enumerate}



	\section{Vektorräume}
Es sei $K$ ein Körper, seine Elemente heissen \textit{Skalare}. Sie sind mit $k$ bezeichnet.

\begin{description}
	\item [$K$-Vektorraum] ($K$-VR) ist ein Tripel $(V, +, \cdot)$ mit:
	\begin{enumerate}\itemsep0em
		\item $(V, +)$ ist eine kommutative Gruppe (s.\,o.)
		\item Es ist $\cdot : K \times V \rightarrow V$ und für alle Elemente $k_1, k_2 \in K$ und $v_1, v_2 \in V$ gelten:
		\begin{enumerate}\itemsep0em
			\item $k_1 \cdot (k_2 \cdot v_1) = k_1\cdot k_2 \cdot v_1$
			\item $k_1 \cdot (v_1 + v_2) = k_1 \cdot v_1 + k_1 \cdot v_2$
			\item $(k_1 + k_2) \cdot v_1 = k_1 \cdot v_1 + k_2 \cdot v_1$
			\item Für die $1$ von $K$ gilt: $1 \cdot v_1 = v_1$
		\end{enumerate}
	\end{enumerate}
	Elemente von $V$ werden mit $v$ bezeichnet.
\end{description}
\begin{itemize}\itemsep0em
	\item [$\Rightarrow$] $K$ selbst mit seiner Addition und Multiplikation ist ein $K$-VR
	\item [$\Rightarrow$] Der Körper $\mathbb{C}$ ist ein 2-dimensionaler VR über $\mathbb{R}$
	\item [$\Rightarrow$] Der Körper $\mathbb{R}$ ist ein $\infty$-dimensionaler VR über $\mathbb{Q}$
	\item [$\Rightarrow$] Die Menge $\mathbb{R} \times \mathbb{R}$ ist ein 2-dimensionaler $\mathbb{R}$-VR.
\end{itemize}

\subsection{Rechenregeln}
\begin{itemize}\itemsep0em
	\item[] $0_K \cdot v = 0_V = k \cdot 0_V$
	\item[] $-k \cdot v = -(k \cdot v) = k \cdot (-v)$
	\item[] $k \cdot v \Rightarrow (k = 0_K) \vee (v = 0_v)$
\end{itemize}

\subsection{Untervektorraum}
Ist $V$ ein $K$-VR und $U \subset V$ ($U \neq \emptyset$) abgeschlossen unter den Verknüpfungen $\cdot, +$, dann ist $U$ ein Untervektorraum von $V$ und somit auch ein $K$-VR.

Jede Gerade in $\mathbb{R}^2$ durch den Nullpunkt ist ein solcher 1-dimensionaler Untervektorraum.

\subsubsection{Erzeugender Untervektorraum}
\begin{equation*}
	\langle U \rangle := \left\lbrace \sum_{i=1}^n k_i \cdot v_i | (n \in \mathbb{N}) \wedge (k_1, \dots, k_n \in K) \wedge (v_1, \dots v_n \in V) \right\rbrace
\end{equation*}
Beispiele: 
\begin{itemize}\itemsep0em
	\item $\langle \emptyset \rangle = \{(0, 0, 0)\}$
	\item $\langle\{(1, 0), (0, 1)\}\rangle = \mathbb{R}^2$
\end{itemize}

\begin{description}
	\item [Erzeugendensystem] 
	bezeichnet eine Menge $U$, wenn gilt $\langle U \rangle = V$ mit $U \subset V$

	\item [Lineare unabhängig] 
	(frei) ist eine Menge $U$ wenn für alle paarweise verschiedenen Vektoren $v_1, \dots, v_n \in U$ und für alle Skalare $k_1, \dots k_n \in K$ stet gilt:
	\begin{equation*}
		\sum_{i=0}^n k_i \cdot v_i \neq 0 \mbox{ oder } r_1, \dots, r_n = 0  
	\end{equation*}
	
	\item[Basis]
	bezeichnet ein linear unabhängiges (freies) Erzeugendensystem (geschrieben als $B$).
\end{description}
Lässt sich ein Vektor eines Untervektorraums aus anderen Vektoren desselben Untervektorraums erzeugen, dann ist der Untervektorraum nicht frei.
\begin{equation*}
	(v = \sum_{i=1}^n k_i \cdot v_i) \wedge (v, v_1, \dots, v_n \in U) \wedge (k_1, \dots, k_n \in K) \Leftrightarrow U \mbox{ ist nicht frei} 
\end{equation*}
Jeder Vektor $v$ aus $V$ lässt sich aus jeder beliebigen Basis erzeugen:
\begin{equation*}
	v = \sum_{i=1}^n k_i \cdot b_i \Leftrightarrow B = \{b_1, \dots, b_n\} \mbox{ist eine Basis von }V
\end{equation*}

\subsection{Sätze, Axiome, Theoreme}
\begin{itemize}\itemsep0em
	\item Ist $A$ eine Menge und \enquote{$\leq$} eine Halbordnung auf $A$, so dass für jede total geordnete Teilmenge eine obere Schranke bezüglich $\leq$ existiert, dann besitzt $A$ maximale Elemente.

	\item Ist $\mathcal{F}$ eine Familie von Mengen mit der Eigenschaft, dass mit jeder Kette $U \subset \mathcal{F}$ die Beziehung $\cup U \in \mathcal{F}$ gilt, dann hat das Paar $(\mathcal{F}, \subset)$ maximale Elemente.

	\item Ist $V$ ein $K$-VR und ist $E \subset V$ ein Erzeugendensystem und $F \subset V$ eine freie Teilmenge von $V$, dann gibt es eine Menge $U \subset V$ mit $X \cap F \neq \emptyset$, so dass $F \cup U$ eine Basis von $V$ ist.

	\item Jeder Vektorraum hat eine Basis. Hat ein Vektorraum eine endliche Basis, dann ist jede weitere Basis dieses Vektorraums ebenfalls endlich und besitzt gleich viele Elemente.
\end{itemize}

\subsection{Dimension}
Die Dimension eines Vektorraums $V$ über $K$ ist dim$_K(V) = |B|$. 
Beispiele:
\begin{itemize}\itemsep0em
	\item dim$_{\mathbb{Q}}(\mathbb{R}) = \infty$
	\item dim$_{\mathbb{Q}}(\mathbb{Q}) = 1$ (weil $\{1\} \subset \mathbb{Q}$ eine Basis von $\mathbb{Q}$ ist) 
\end{itemize}

\subsection{Lineare Abbildungen und Matrizen}
Sind $W$ und $V$ beides $K$-VR. Eine Abbildung $f: V \rightarrow W$ heisst $K$-linear oder $K$-VR Homomorphismus falls für alle Element $\lambda \in K$ und alle Vektoren $v, w \in V$ die Gleichungen:
\begin{align*}
	f(v + w)& = f(v) + f(w) & f(\lambda v)& = \lambda f(v)
\end{align*}
erfüllt werden. Die Menge aller derartiger Abbildungen wird als Hom$_K (V, W)$ bezeichnet.\\
Für $K$-lineare Abbildungen gilt:
\begin{equation*}
	f(\sum_{i=1}^n \lambda_i \cdot v_i) = \sum_{i=1}^n \lambda_i \cdot f(v_i)
\end{equation*}
Beispiele: 
\begin{itemize}\itemsep0em
	\item $f: \mathbb{R}^3 \rightarrow \mathbb{R}^2$ mit $f((x, y, z)) := (y, z)$
	\item $f: \mathbb{R}^2 \rightarrow \mathbb{R}^3$ mit $f((x, y)) := (0, y, z)$
\end{itemize}

\begin{description}
	\item [Kern] Für $f \in $ Hom$_K(V, W)$ ist der Kern definiert als:
	\begin{equation*}
		\mbox{ker}(f) := \{v \in V | f(v) = 0\}
	\end{equation*}
\end{description}
\begin{itemize}\itemsep0em
	\item Sind $V$ und $W$ zwei $K$-VR und $f, g \in $ Hom$_K(V, W)$, so dass $f$ und $g$ auf einer Basis von $V$ dieselben Werte annehmen, dann gilt: $f = g$
	\item Sind $V$ und $W$ zwei $K$-VR und ist $B = \{b_1, \dots, b_n\}$ eine Basis von $V$ sowie $f: B \rightarrow W$ eine beliebige Funktion, dann lässt sich $f$ eindeutig zu einer $K$-linearen Abbildung $f: V \rightarrow W$ fortsetzen.
	\item Zwei $K$-VR gleicher, endlicher Dimension sind stets isomorph zueinander. $\Rightarrow$ Ist $V$ ein endlich dimensionaler $K$-VR, dann gibt es eine Zahl $n \in \mathbb{N}$, so dass $V$ isomorph zu $K^n$ ist.
	\item Sind $V$ und $W$ zwei $K$-VR und ist $f \in$ Hom$_K(V, W)$, dann ist ker$(f)$ ein Untervektorraum von $V$ und im$(f)$ ein Untervektorraum von $W$.
	\item Sind $V$ und $W$ zwei $K$-VR endlicher Dimension und ist $f \in$ Hom$_K(V, W)$, dann gilt:
	\begin{equation*}
		\mbox{dim}_K(\mbox{im}(f)) + \mbox{dim}_K(\mbox{ker}(f)) = \mbox{dim}_K(V)
	\end{equation*}
	\item Sind $V$ und $W$ zwei $K$-VR endlicher und gleicher Dimension, dann sind für $f \in$ Hom$_K(V, W)$ folgende Aussage äquivalent:
	\begin{itemize}
		\item $f$ ist ein Isomorphismus
		\item $f$ ist ein Epimorphismus
		\item $f$ ist ein Monomorphismus
	\end{itemize}
\end{itemize}





	\section{Beweistechniken}
\subsection{Direkter Beweis}
Strategie: Zwingende Argumente für die Richtigkeit von $A$ finden.\\
Bsp: \enquote{Beweisen, dass jede durch 4 teilbare natürliche Zahl gerade ist.}
\begin{proof}
Direkt
\begin{enumerate}\itemsep0em
	\item Sei $n$ eine beliebige durch 4 teilbare Zahl
	\item $n$ muss also von der Form $n = 4 \cdot m (m \in \N)$ sein
	\item Sei $k = 2 \cdot m \Rightarrow n = 2 \cdot k$ sein
	\item Deswegen muss $n$ gerade sein \qedhere
\end{enumerate}
\end{proof}


\subsection{Beweis durch Widerspruch}
Strategie: Annehmen, das $A$ falsch sei. Unter dieser Annahme eine Folgerung
herleiten, von der entweder bekannt ist, dass sie falsch ist, oder die im
Widerspruch zu Annahme steht.\\
Bsp: \enquote{Beweisen, dass es keine grösste natürliche Zahl gibt.}

\begin{proof}
\renewcommand{\qedsymbol}{\lightning}
durch Widerspruch
	\begin{enumerate}\itemsep0em
		\item Sei $m$ die grösste natürliche Zahl
		\item Es gilt für jede natürliche Zahl $n$: $n + 1$ ist ebenfalls eine natürliche Zahl $\wedge n < n + 1$
		\item Also muss $m + 1$ eine natürliche Zahl sein $> m$ sein \qedhere
	\end{enumerate}
\end{proof}

\subsection{Erweiterte Techniken}
%Problem: Beweisen, dass die Verknüpfung von zwei Aussagen $A$ und $B$ wahr ist.

\subsection{Beweis durch Implikation}
Problem: Beweisen, dass $A \Rightarrow B$ wahr ist.\\
Strategie: Unter der Annahme, dass $A$ wahr ist, folgern, dass dann $B$ wahr sein muss.\\
Bsp: \enquote{Für jede natürliche Zahl $n$ gilt: $(n^2 + 1 = 1) \Rightarrow (n = 0)$}\\
\begin{proof}
durch Implikation
\begin{enumerate}\itemsep0em
	\item Angenommen, $n^2 + 1 = 1$ sei wahr
	\item Dann ist $n^2 = 0$ bzw. $n = \sqrt{0} = 0$
	\item Also: $(n^2 + 1 = 1) \Rightarrow (n = 0)$\qedhere
\end{enumerate}
\end{proof}

\subsubsection{Beweis durch Kontraposition}
Problem: Beweisen, dass $A \Rightarrow B$ wahr ist.\\
Strategie: Die Kontraposition $(\neg B \Rightarrow \neg A)$ beweisen.\\
Bsp: \enquote{Für jede natürliche Zahl $n$ gilt: $(n^2 + 1 = 1) \Rightarrow (n = 0)$}\\
\begin{proof}
durch Kontraposition
\begin{enumerate}\itemsep0em
	\item Es muss gelten: $n \neq 0 \Rightarrow (n^2 + 1 \neq 1)$
	\item Ist $n \neq 0 \Rightarrow n^2 \neq 0$
	\item daraus folgt, dass $n^2 + m \neq m$ für jedes $n \neq 0$
	\item Also muss $n^2 + 1 \neq 1 (n \neq 0)$\qedhere
\end{enumerate}
\end{proof}

\subsubsection{Beweis durch Äquivalenz}
Problem: Beweisen, dass $A \Leftrightarrow B$ wahr ist.
Strategie: Beweisen, dass $A \Rightarrow B \wedge B \Rightarrow A$\\
Als erstes also beweisen, dass $A \Rightarrow B$ und als zweites
beweisen, dass $B \Rightarrow A$
Bsp: \enquote{Für jede natürliche Zahl $n$ gilt: $(n^2 + 1 = 1) \Leftrightarrow (n = 0)$}\\
\begin{proof}
	durch Äquivalenz
\begin{enumerate}\itemsep0em
	\item Für den Beweis $(n^2 + 1 = 1) \Rightarrow (n = 0)$ siehe z.B. Implikation
	\item Bleibt zu beweisen, dass $n = 0 \Rightarrow n^2 + 1 = 1$
	\begin{enumerate}\itemsep0em
		\item Einsetzen von $n = 0: 0^2 + 1 = 1$, d.\,h. $1 = 1$, was wahr ist
		\item Folglich gilt: $n = 0 \Rightarrow n^2 + 1 = 1$ 
	\end{enumerate}
	\item Beide Teilaussagen sind wahr, also ist die ganze Aussage wahr\qedhere
\end{enumerate}
\end{proof}

\subsubsection{Beweistechnik durch vollständige Induktion}
\begin{equation*}
(E(0)
\wedge \forall n \in \N (E(n) \Rightarrow E(n + 1)))
\Leftrightarrow \forall n \in \N (E(n))
\end{equation*}

Man zeigt etwas für die 0, anschliessend nimmt man an, dass wenn es für eine
natürliche Zahl gilt, dann auch für deren Nachfolger. Gilt es für 0 und alle Nachfolger, gilt es für alle.
Beispiele bei der Addition der natürlichen Zahlen.

	\rule{0.3\linewidth}{0.25pt}\\
	\scriptsize
	Copyright \copyright\ 2013 Constantin Lazari\\
	% Should change this to be date of file, not current date.
	Revision: 1.0, Datum: \today\\
	\end{multicols}
\end{document}