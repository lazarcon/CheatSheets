\section{Ringe und Körper}
Eine Struktur $(G, +, \cdot)$  heisst Ring, wenn folgende Bedingungen erfüllt sind:
\begin{enumerate}
	\item $(G, +)$ ist eine kommutative Gruppe
	\item $(G, \cdot)$ ist eine Halbgruppe
	\item Es gilt das Distributivgesetz, d.\,h. für alle Elemente $a, b, c$ des Ringes gelten:
	\begin{itemize}
		\item $a \cdot (b + c) = (a \cdot b) + (a \cdot c)$
		\item $(a + b) \cdot c = (a \cdot c) + (b \cdot c)$
	\end{itemize}
\end{enumerate}

\subsection{Konventionen}
\begin{itemize}
	\item Wenn $(R, +, \cdot)$ ein Ring ist, dann bezeichnen wir das neutrale Element von $(G, +)$ mit 0
	\item Falls vorhanden bezeichnen wir das neutrale Element von $(G, \cdot)$ mit 1.
	\item Das inverse Element von $g \in G$ bezüglich $\sim$ bezeichnen wir mit $-g$.
	\item Das inverse Element von $g \in G$ bezüglich $\circ$ bezeichnen wir mit $g^{-1}$.
\end{itemize}

\subsection{Typische Ringe}
$(\mathbb{Z}, +, \cdot), (\mathbb{Q}, +, \cdot)$ und $(\mathbb{Z}, +, \cdot)$. Sowie der Nullring $(\{0\}, +, \cdot)$

\subsection{Potenz}
Sei $(G, +, \cdot)$ ein Ring mit 1, dann ist die $n$-te Potenz von $g \in G$ definiert als:
\begin{align*}
	r^0& := 1\\
	r^{n + 1}& := r \cdot r^n
\end{align*}

\subsection{Rechenregeln in Ringen}
Sei $G, +, \cdot)$ ein Ring. Für alle Elemente $a, b \in R$ und alle Zahlen $n, k \in \mathbb{N}$ gelten folgende Identitäten:
\begin{enumerate}
	\item $0 \cdot a = a \cdot 0 = 0$
	\item $(- a) = (-1) \cdot a$
	\item $- (a \cdot b) = (- a) \cdot b = a \cdot (-b)$
	\item $(- a) \cdot (- b) = a \cdot b$
	\item $0 = 1 \Rightarrow G = \{0\}$
	\item $a^n \cdot a^k = a^{n + k}$
	\item $a^{n \cdot k} = (r^n)^k$
\end{enumerate}

\subsection{Begriffe}
\settowidth{\MyLenA}{rechter Nullteiler~~}
\begin{tabular}{@{}p{\the\MyLenA}%
				@{}p{(\linewidth - \the\MyLenA)}}
	rechter Nullteiler & $b \in G$ heisst rechter Nullteiler, falls ein $a \in G\setminus\{0\}$ existiert, so dass $a \cdot b = 0$\\
	linker Nullteiler & $b \in G$ heisst linker Nullteiler, falls ein $a \in G\setminus\{0\}$ existiert, so dass $b \cdot a = 0$\\
	Nullteiler & ist sowohl rechter, wie auch linker Nullteiler\\
	Integritätsring & Die Verknüpfung $\cdot$ ist kommutativ und $0 \in G$ ist der einzige Nullteiler.\\
	Körper & Integritätsring mit $(G\setminus \{0\}, \cdot)$ ist eine kommutative Gruppe.
\end{tabular}

In einem Integritätsring gilt stets: $1 \neq 0$.\\

Ein kommuativer Ring $(G, +, \cdot)$ mit $G \neq \{0\}$, ist genau dann ein Integritätsring, wenn für jedes $g \in G\setminus\{0\}$ die Abbildung $f_g: (G, +) \rightarrow (G, +)$ mit $f_g(x) := g \cdot x$ ein injektiver Gruppenhomomorphismus ist.

Ein Integritätsring $(R, +, \cdot)$ ist genau dann ein Körper, wenn alle Funktionen $f_g: G \rightarrow G$ mit $f_g(x) = r \cdot x$ mit $r \in G\setminus\{0\}$ surjektiv sind.

\subsection{Folgerungen}
\begin{enumerate}
	\item Jeder endliche Integritätsring ist ein Körper.
	\item Für $p \in \mathbb{N} gilt: (\mathbb{Z}_{/p}, +, \cdot)$ ist ein Körper $\Leftrightarrow p$ ist eine Primzahl.
\end{enumerate}

\subsection{Ringhomomorphismus}
Es seien die Ringe $(R, +, \cdot)$ und $(R', +', \cdot')$ gegeben. Ein Ringhomomorphismus $f: (R, +, \cdot) \rightarrow (R', +', \cdot')$ ist eine Abbildung $f : R \rightarrow R'$, die:
\begin{enumerate}
	\item Ein Gruppenhomomorphismus $f: (R, +) \rightarrow (R', +')$ und
	\item ein Halbgruppenhomomorphismus $f: (R, \cdot) \rightarrow (R', \cdot')$ ist.
	\item Sind $(R, \cdot)$ und $(R', \cdot')$ Monoide, muss $f$ ein Monoidhomomorphismus sein.
\end{enumerate}

