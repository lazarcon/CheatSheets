\section{Ganze Zahlen}
$\Z = \{\cdots -2, -1, 0, 1, 2, \dots\}$\\
Es gilt: $x < y \Leftrightarrow \exists n \in \N > 0 \mid x + n = y$\\
\subsection{Rechenregeln}
\settowidth{\MyLenA}{Neutrales Element bzgl. $+$~~}
\begin{tabular}{@{}p{\the\MyLenA}%
		@{}p{(\linewidth - \the\MyLenA)/2}}
					& $-1 \dot z = -z$\\
					& $-(-z) = z$\\
	Inverses Element bzgl. $+$ & $-z + z = 0$\\
	Absorbtion	& $0 \cdot z = 0$\\
	Neutrales Element bzgl. $+$ & $0 + z = z$\\
	Neutrales Element bzgl. $\cdot$ & $1 cdot z = z$\\
	Assoziativität bzgl. $+$ & $a + (b + c) = (a + b) + c$\\
	Assoziativität bzgl. $\cdot$ & $a \cdot (b \cdot c) = (a \cdot b) \cdot c$\\
	Kommutativität bzgl. $+$ & $a + b = b + a$\\
	Kommutativität bzgl. $\cdot$ & $a \cdot b = b \cdot a$\\
	Distributiviät & $a  \cdot (b + c) = a \cdot b + a \cdot c$\\
\end{tabular}

\subsection{Definitionen}
\subsubsection{Subtraktion (-:$\Z \to \Z$)}
\begin{equation*}
	a - b = a + (- b)
\end{equation*}

\subsubsection{Betrag ($|.|: \Z \to \Z$)}
\begin{equation*}
	|z| = \begin{cases}
	      	z & \mbox{falls } z \in \N\\
			-1 \cdot z & \mbox{sonst}
	      \end{cases}
\end{equation*}

\subsubsection{Teilbarkeit}
\begin{equation*}
	x \mid y \Leftrightarrow \exists q \in \Z (y = x \cdot q)
\end{equation*}
Teilbarkeit ist transitiv: $x \mid y \wedge y \mid z \Rightarrow x \mid z$.

Beispiel: $\forall a, b, c, d \in \Z\ (a \mid b) \wedge (a \mid d) \Rightarrow (a \cdot c) \mid (b \cdot d)$
\begin{proof}
	Es seien $a, b, c, d$ fest, aber beliebig.\\
	Es gelte $a \mid b$ und $c \mid d$\\
	Zeige: $a \cdot c \mid b \cdot d$\\
	\begin{align*}
		a \mid b & \Leftrightarrow \exists q_1 \in \Z\ b = a \cdot q_1\\
		c \mid d & \Leftrightarrow \exists q_2 \in \Z\ d = c \cdot q_2\\
		(a \cdot q_1) \cdot (c \cdot q_2) = b \cdot d\\
	\end{align*}
Mit $q_3 = q_1 \cdot q_2$ gilt: $a \cdot c \cdot q_3 = b \cdot d \Rightarrow (a \cdot c) \mid (b \cdot d)$\qedhere
\end{proof}

Beispiel: $\forall a, b, c \in \Z\ a \mid b \wedge a \nmid c \Rightarrow a \nmid (b + c)$
\begin{proof}
	Es seien $a, b, c$ fest, aber beliebig.\\
	Es gelte $a \mid b$ und $a \nmid b$.\\
	Zeige: $a \nmid (b + c)$.
	Anmerkung: Beweis durch Widerspruch: $\neg (a \mid b \wedge a \nmid c \Rightarrow a \nmid (b + c)) = a \mid b \wedge a \nmid c \wedge a \mid (b + c)$\\
	Annahme: $a \mid (b + c)$\\
	\begin{align*}
		a \mid b& \Leftrightarrow a \cdot q_1 = b\\
		a \mid (b + c) & \Leftrightarrow a \cdot q_2 = b + c\\
		a \cdot q_2& = a \cdot q_1 + c\\
		a (q_2 - q_1)& = c \Leftrightarrow a \mid c. \mbox{Widerspruch zum geltenden}\qedhere
	\end{align*}
\end{proof}

\subsubsection{Hilfssätze}
\begin{align*}
  s \mid t \wedge s \mid u & \Rightarrow s \mid (t + u)\\
  s \mid t \wedge s \mid u & \Rightarrow s \mid (t - u)
\end{align*}

\subsubsection{Teilen mit Rest}
Sind $x, y \in \N \setminus \{0\}$, dann gibt es eindeutig bestimmte Zahlen, so dass:
\begin{enumerate}
	\item $y = q \cdot x + r$
	\item $r < x$
\end{enumerate}

\subsubsection{Kleinstes gemeinsames Vielfaches (KGV)}
(engl. \textit{least common multiple}, zum Erweitern von Brüchen)\\
\begin{equation*}
	\mbox{kgV}(x, y) = \min\{q \in \N \mid x \mid q \wedge y \mid q\}
\end{equation*}


\subsubsection{Grösster gemeinsamer Teiler (GGT)}
(engl. \textit{greatest common divisor}, zum Kürzen von Brüchen)\\
\begin{align*}
	\mbox{ggT}(x, y) &= \max\{q \in \N \mid q \mid x \wedge q \mid y\}\\
	\mbox{kgV}(x, y) \cdot \mbox{ggT}(x, y) &= x \cdot y
\end{align*}


\subsubsection{Euklidischer Algorithmus}
\begin{equation*}
	\mbox{ggt}(n, m) = \mbox{ggt}(n, m - n) = \mbox{ggt}(m, m - n)
\end{equation*}

\subsubsection{Erweiterter euklidischer Algorithmus}
Erklärt sich am besten an einem Beispiel:
Gesucht: ggt$(64, 14) = s \cdot 64 + t \cdot 14$
\begin{center}

\begin{tabular}
{
c%
@{}|p{0.18\linewidth}%
@{}|p{0.18\linewidth}%
@{}|p{0.24\linewidth}%
@{}|p{0.23\linewidth}%
}
 \multicolumn{1}{c}{\textbf{n}} & \multicolumn{1}{|c}{\textbf{a}} & \multicolumn{1}{|c}{\textbf{q}} & \multicolumn{1}{|c}{\textbf{s}} & \multicolumn{1}{|c}{\textbf{t}} \\\hline
 1 & \multicolumn{1}{c|}{64} & & \multicolumn{1}{|c}{1} & \multicolumn{1}{|c}{0} \\\hline
 2 & \multicolumn{1}{c|}{14} & & \multicolumn{1}{|c}{0} & \multicolumn{1}{|c}{1} \\\hline
 3 & $a_1 \% a_2 = 8$ & ${a_1}/{a_2} = 4$ & $s_1 - q_3 \cdot s_2 = 1$ 	& $t_1 - q_3 \cdot t_2 = -4$ \\\hline
 4 & $a_2 \% a_3 = 6$ & ${a_2}/{a_3} = 1$ & $s_2 - q_4 \cdot s_3 = -1$ 	& $t_2 - q_4 \cdot t_3 = 5$ \\\hline
 \textbf{5} & $a_3 \% a_4 = 2$ & ${a_3}/{a_4} = 1$ & $s_3 - q_5 \cdot s_4 = 2$ 	& $t_3 - q_5 \cdot t_4 = -9$ \\\hline
 6 & $a_4 \% a_5 = 0$ & ${a_4}/{a_5} = 3$ & $s_4 - q_6 \cdot s_5 = -7$ 	& $t_4 - q_6 \cdot t_5 = 32$ \\\hline
\end{tabular}
\end{center}
Die Lösung steht in Zeile 5 (mit $s = 2$ und $t = -9$):\\ ggt$(64, 14) = 2 = 2 \cdot 64 + (-9) \cdot 14 = 128 - 126 = 2$ 

\subsubsection{Teilerfremdheit}
Zahlen sind teilerfremd, wenn ggt$(m, n) = 1 \Leftrightarrow k \cdot x + k' \cdot y$.
