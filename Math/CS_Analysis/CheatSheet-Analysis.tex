\documentclass[10pt,landscape]{scrartcl}
\usepackage{multicol}
\usepackage{calc}
\usepackage{ifthen}
\usepackage[landscape]{geometry}
% Umlaute ermöglichen
\usepackage[T1]{fontenc}
% Format dieses Dokuments
\usepackage[utf8]{inputenc}
% Deutsche Trennungsregeln
\usepackage[ngerman]{babel}
\usepackage[babel,german=quotes]{csquotes}

%\usepackage{color}
%\definecolor{lightgrey}{gray}{0.75}
%\definecolor{darkgrey}{gray}{0.25}

\usepackage{amssymb,amsmath}
\usepackage{amsthm, mathtools}

\usepackage{tikz}
\usetikzlibrary{arrows}

\usepackage{ulem}

\usepackage{hyperref}
%\pdfoutput=1
\hypersetup{
	pdfauthor   = {Lazari, Constantin},
	pdftitle    = {Cheat Sheet - Analysis},
	pdfsubject  = {Mathematik},
	pdfkeywords = {},
	pdfcreator  = {Kile},		% Texnic Center oder Kile z.B.
	pdfproducer = {pdflatex},
	colorlinks  = false		% Links nicht farbig hervorheben (sieht Scheisse aus).
} 

% To make this come out properly in landscape mode, do one of the following
% 1.
%  pdflatex latexsheet.tex
%
% 2.
%  latex latexsheet.tex
%  dvips -P pdf  -t landscape latexsheet.dvi
%  ps2pdf latexsheet.ps

% This sets page margins to .5 inch if using letter paper, and to 1cm
% if using A4 paper. (This probably isn't strictly necessary.)
% If using another size paper, use default 1cm margins.
\ifthenelse{\lengthtest { \paperwidth = 11in}}
	{ \geometry{top=.5in,left=.5in,right=.5in,bottom=.5in} }
	{\ifthenelse{ \lengthtest{ \paperwidth = 297mm}}
		{\geometry{top=1cm,left=1cm,right=1cm,bottom=1cm} }
		{\geometry{top=1cm,left=1cm,right=1cm,bottom=1cm} }
	}

% Turn off header and footer
\pagestyle{empty}
 
% Redefine section commands to use less space
\makeatletter
\renewcommand{\section}{\@startsection{section}{1}{0mm}%
                                {-1ex plus -.5ex minus -.2ex}%
                                {0.5ex plus .2ex}%x
                                {\normalfont\large\bfseries}}
\renewcommand{\subsection}{\@startsection{subsection}{2}{0mm}%
                                {-1explus -.5ex minus -.2ex}%
                                {0.5ex plus .2ex}%
                                {\normalfont\normalsize\bfseries}}
\renewcommand{\subsubsection}{\@startsection{subsubsection}{3}{0mm}%
                                {-1ex plus -.5ex minus -.2ex}%
                                {1ex plus .2ex}%
                                {\normalfont\small\bfseries}}
\newcommand{\msout}[1]{\text{\sout{\ensuremath{#1}}}}
\makeatother

\newcommand{\dop}[1]{\,\mathrm d#1}
%\newcommand{\lint}[2]{\int\limits_{#1}^{#2}}
\newcommand{\lint}[2]{\int_{#1}^{#2}}

\DeclareMathOperator{\arccot}{arccot}

% Define BibTeX command
\def\BibTeX{{\rm B\kern-.05em{\sc i\kern-.025em b}\kern-.08em
    T\kern-.1667em\lower.7ex\hbox{E}\kern-.125emX}}

% Don't print section numbers
\setcounter{secnumdepth}{0}
\setlength{\parindent}{0pt}

\setlength{\parskip}{0pt plus 0.5ex}

\newcommand{\N}{\mathbb{N}}

% -----------------------------------------------------------------------

\begin{document}

	\raggedright
	\footnotesize
	\begin{multicols}{3}


	% multicol parameters
	% These lengths are set only within the two main columns
	%\setlength{\columnseprule}{0.25pt}
	\setlength{\premulticols}{1pt}
	\setlength{\postmulticols}{1pt}
	\setlength{\multicolsep}{1pt}
	\setlength{\columnsep}{2pt}
	\newlength{\MyLenA}
	\newlength{\MyLenB}

	\begin{center}
	\Large{\textbf{Cheat Sheet Analysis}} \\
	\end{center}

	%% Teil 1
	\section{Funktionen}
Eine Funktion ist eine Vorschrift, die jedem Element $x$ aus eine Menge $D$ genau ein Element $y$ aus einer Menge $W$ zuordnet.
\begin{equation*}
	f\colon\, D\to W,\; x\mapsto y.	
\end{equation*}
Darstellungen:\\
1. Analytisch ($y = f(x)$ (explizit), $F(x;y) = 0$ (implizit)),\\ 2. Wertetabelle, 3. Graphisch, 
4. Parametrisch ($x = x(t)$, $y = y(t)$, Wertetabelle beginnt mit $t$)

\subsection{Funktionseigenschaften}
\subsubsection{Symmetrie}
\begin{tabular}{lcclc}
gerade:	& $f(-x) = f(x)$ & ~~~ & ungerade: & $f(-x) = - f(x)$\\
\end{tabular}

\subsubsection{Monotonie}
\settowidth{\MyLenA}{Streng monoton wachsend~~}
\begin{tabular}{@{}p{\the\MyLenA}%
				@{}p{\linewidth - \the\MyLenA}}
Monoton wachsend & $f(x_1) \leq f(x_2)$ ($x_1 < x_2$) \\
Streng monoton wachsend & $f(x_1) < f(x_2)$ ($x_1 < x_2$)\\
Monoton fallend & $f(x_1) \geq f(x_2)$ ($x_1 < x_2$) \\
Streng monoton fallend & $f(x_1) > f(x_2)$ ($x_1 < x_2$)\\
\end{tabular}

\subsubsection{Umkehrbarkeit}
Umkehrbar: $x_1 \neq x_2 \Rightarrow f(x_1) \neq f(x_2)$ (streng monton)\\
Bestimmen der Umkehrfunktion (Spiegelung an $y=x$):\\
1. $y = f(x)$ nach $x$ auflösen. Ergebnis: $x = f^{-1}(y)$.\\
2. Vertauschen von $x$ und $y$ im Ergebnis: $y = f^{-1}(x)$.\\
Definitions- und Wertebereich sind vertauscht.
\begin{equation*}
	x\mathop{\rightleftarrows}^{f}_{f^{-1}}f(x)
\end{equation*}

\subsubsection{Periodizität}
Periodisch mit Periode: $p: f(x \pm p) = f(x)$

\subsubsection{Stetigkeit}
Eine Funktion $f(x)$ heisst an der Stelle $x_0$ stetig, wenn der Grenzwert vorhanden ist und mit dem Funktionswert übereinstimmt:
\begin{equation*}
	\lim_{x\to x_0}{f(x)} = f(x_0)
\end{equation*}

Eine Funktion ist an der Stelle $x_0$ unstetig, wenn:\\
1. $f(x)$ an der Stelle $x_0$ nicht definiert ist (Definitionslücke).\\
2. An der Stelle $x_0$ kein Grenzwert vorhanden ist.\\
3. Funktions- und Grenzwert zwar vorhanden, aber verschieden sind.\\

\subsection{Grenzwert}
% Reelle Zahlenfolge: $\langle{a}\rangle = f(n)$ mit $n \in \mathbb{N}$*. $f(n)$ heisst Bildungsgesetz, die Zahlen: $a_1, a_2, \dots a_n$ heissen Glieder der Folge.\\
% Eine reelle Zahl $g$ heisst Grenzwert der Zahlenfolge $\langle{A}\rangle$, wenn es zu jedem $\epsilon > 0$ eine
% natürliche Zahl $n_0 > 0$ gibt, so dass für all $n \geq n_0$ stets gilt: $\left|a_n - g\right| < \epsilon$.

Die Funktion $f(x)$ hat an der Stelle $x_0$ einen Grenzwert $g$, wenn gilt
\begin{equation*}
	\lim_{x\to x_0\atop x < x_0}f(x) = \lim_{x\to x_0\atop x > x_0}f(x) = \lim_{x\to x_0}f(x) = g 
\end{equation*}
\textit{konvergent} = hat Grenzwert, \textit{divergent} hat keinen Grenzwert.

Lösungsschema zur Bestimmung des Grenzwerts $g = \lim_{x\to x_0} f(x)$:\\
1. Grundsätzlich $x_0$ in $f(x)$ einsetzen. Wenn $f(x_0)$ definiert ist: $g = \lim_{x\to x_0} f(x) = f(x_0)$.\\
2. Falls $f(x_0)$ nicht definiert ist, $f(x)$ vereinfachen.\\
3. Falls das nicht geht, den links und rechtsseitigen Grenzwert durch annähern von links und rechts ermitteln.\\
Polstelle: Der Grenzwert ist $+\infty$ oder $-\infty$.\\

\subsubsection{Rechenregeln}
\begin{align*}
\lim_{x\to x_0}{(k \cdot f(x))}& = k (\lim_{x\to x_0}{f(x)})\\
\lim_{x\to x_0}{(f(x) \pm g(x))}& = (\lim_{x\to x_0}{f(x)}) \pm (\lim_{x\to x_0}{g(x)})\\
\lim_{x\to x_0}{(f(x) \cdot g(x))}& = (\lim_{x\to x_0}{f(x)}) \cdot (\lim_{x\to x_0}{g(x)})\\
\lim_{x\to x_0}{(\frac{f(x)}{g(x)})}& = \frac{\lim_{x\to x_0}f(x)}{\lim_{x\to x_0}g(x)}\\
\end{align*}



	\section{Polynomfunktionen}
Allgemein: $f(x) = a_n \cdot x^n + a_{n-1} \cdot x^{n-1} + \dots + a_1 \cdot x^1 + a_0$\\
Der Grad des Polynoms ist $n$. Es gibt $n$ Nullstellen.

\subsection{Nullstellen-Formeln}
\settowidth{\MyLenA}{Biquadratisch~~}
\settowidth{\MyLenB}{$ax^3 + bx^2 + cx = 0$~~}
\begin{tabular}{@{}p{\the\MyLenA}%
				@{}p{\the\MyLenB}
				@{}p{\linewidth - \the\MyLenA - \the\MyLenB}}
Linear & $ax + b = 0$ & $x = - \frac{b}{a}$\\
Quadratisch & $ax^2 + bx + c = 0$ & $x_{1,2} = \frac{-b \pm \sqrt{b^{2} - 4ac}}{2a}$\\
Kubisch & $ax^3 + bx^2 + cx = 0$ & $x(ax^2 + bx + c) = 0$ mit $x_1 = 0$\\ 
Biquadratisch & $ax^4 + bx^2 + c = 0$ & $y = x^2 \Rightarrow ay^2 + by + c = 0$\\ 
\end{tabular}

\subsection{Geraden (erster Grad)}
Es sei $m$ die Steigung, $a$ der x- und $b$ der y-Achsenabschnitt. 
\settowidth{\MyLenA}{Achsenabschnittsform~~}
\settowidth{\MyLenB}{$\frac{y - y_1}{x - x_1} = \frac{y_2 - y_1}{x_2 - x_1}$~~}
\begin{tabular}{@{}p{\the\MyLenA}%
				@{}p{\the\MyLenB}
				@{}p{\linewidth - \the\MyLenA - \the\MyLenB}}
y-Achse, Steigung & $y = mx + b$ & \\
Achsenabschnittsform & $\frac{x}{a} + \frac{y}{b} = 1$ & \\
Punkt-Steigung & $\frac{y - y_1}{x - x_1} = m$ & Durch $P(x_1;y_1)$\\
Zwei-Punkte-Form & $\frac{y - y_1}{x - x_1} = \frac{y_2 - y_1}{x_2 - x_1}$ & Durch $P_1(x_1;y_1)$, $P_2(x_2;y_2)$\\
\end{tabular}

\subsection{Parabeln (zweiter Grad)}
Es sei $S$ der Scheitelpunkt.

\settowidth{\MyLenA}{Scheitelpunktsform~~}
\settowidth{\MyLenB}{$y = a(x - x_1)(x - x_2)$~~}
\begin{tabular}{@{}p{\the\MyLenA}%
				@{}p{\the\MyLenB}
				@{}p{\linewidth - \the\MyLenA - \the\MyLenB}}
Hauptform & $y = ax^2 + bx + c$ & $S = (-\frac{b}{2a}; \frac{4ac - b^2}{4a})$\\
Produktform & $y = a(x - x_1)(x - x_2)$ & $x_1$, $x_2$ sind Nullstellen\\
Scheitelpunktsform & $y - y_0 = a(x - x_0)^2$ & $S = (x_0;y_0)$\\
\end{tabular}

\subsection{Höhere Grade}
Besitzt eine Polynomfunktion $f(x)$ vom Grad $n$ an der Stelle $x_n$ eine Nullstelle,
so lässt sie sich schreiben als: $f(x) = (x - x_n) \cdot f_1(x)$.\\
$(x - x_n)$ heisst Linearfaktor, $f_1(x)$ heisst reduziertes Polynom vom Grad $n - 1$.\\
Besitzt eine Polynom vom Grad $n$ genau $n$ Nullstellen, so lässt es sich schreiben als:
\begin{equation*}
	f(x) = a_n(x - x_1)(x - x_2) \dots (x - x_n)
\end{equation*}

1. Das reduzierte Polynom erhält man durch das Horner-Schema.\\
2. Polynome solange reduzieren (raten weiterer Nullstellen) bis man auf eine Polynomfunktion zweiten Grades stösst, 
deren Nullstellen sich durch lösen der quadratischen Gleichung ergeben.

\subsubsection{Horner-Schema}
% 1. Koeffizienten in Reihenfolge notieren ($a_n, a_{n-1}, \dots, a_0$).\\
% 2. Denn ersten Koeffizienten ($a_n$) in die zweite Spalte der ersten Zeile schreiben.\\
% 3. Denn Wert der dritten Zeile mit der Nullstelle multiplizieren und in die nächste Spalte der zweiten Zeile schreiben.\\
% 4. Die Summe der ersten beiden Zeilen der nächsten Spalten bilden und in der dritten Zeile notieren.\\
% 5. Schritte 3 und 4 wiederholen, bis am Ende der Tabelle angelangt. \\

Gegeben: $y = 3 x^3 + 18 x^2 + 9 x - 30 = 3 (x^3 + 6 x^2 + 3 x - 10)$\\
Durch raten findet man eine Nullstelle bei $x = 1$ ($1 + 6 + 3 - 10 = 0$). 
\begin{tabular}{c|c|c|c|c}
			& $a_3 = 1$ 	& $a_2 = 6$ 			& $a_1 = 3$ 			& $a_0 = - 10$ 			\\ \hline
$x_0 = 1$ 	& 				& $a_3 \cdot x_0 = 1$	& $7 \cdot x_0 = 7$ 	& $10 \cdot x_0 = 15$	\\ \hline
			& $a_3 = 1$		& $6 + 1 = 7$			& $3 + 7 = 10$			& $-10 + 10 = 0$		\\ \hline
\end{tabular}

Umgeformt: $y = 3 (x - 1)(x^2 + 7 x + 10) \Rightarrow y = 3 (x - 1)(x + 2)(x + 5)$.\\

\subsection{Gebrochenrationale Funktionen}
Funktionen, die sich als Quotient zweier Polynomfunktionen $g(x)$ und $h(x)$ darstellen lassen heissen gebrochenrationale Funktionen: $f(x) = \frac{g(x)}{h(x)}$
Diese Funktionen sind echt gebrochen, wenn der Grad von $g(x)$ kleiner ist als der Grad von $h(x)$.
Sie werden mit Hilfe der Polynom-Division gelöst.

Nullstellen: $x_0: g(x_0) = 0$ und $h(x_0) \neq 0$.\\
Definitionslücke: Alle Stellen wo $h(x_0) = 0$.\\
Bestimmen der Null- und Polstellen:\\
1. Zähler- und Nennerpolynom in Linearfaktoren zerlegen.\\
2. die Zähler Linearfaktoren sind die Nullstellen,\\ 
3. die Nenner Linearfaktoren sind die Polstellen.\\

\subsection{Kreis und Ellipse}
Kreisgleichung (Mittelpunkt $M = (x_0; y_0)$, Radius $r$):\\
$(x - x_0)^2 + (y - y_0)^2 = r^2 $ oder $y = y_0 \pm \sqrt{r^2 - (x - x_0)^2}$\\

Ellipsengleichung (Mittelunkt $M = (x_0; y_0)$, x-Halbachse $a$, y-Halbachse $b$):\\ 
%$M$: Mittelpunkt; $F_1, F_2$: Brennpunkte, $a$: grosse, $b$: kleine Halbachse, $e$: Brennweite\\
$\frac{(x - x_0)^2}{a^2} + \frac{(y - y_0)^2}{b^2} = 1$ oder $y = y_0 \pm \frac{b}{a} \sqrt{a^2 - (x - x_0)^2}$


	\subsection{Potenzen-, Wurzel- und Logarithmusfunktionen}
Terminologie: $\mbox{Basis}^{\mbox{Exponent}}\,\,\mathbb{D} := (a, b, u, v \in \mathbb{R})$\\

\settowidth{\MyLenA}{$a^{u} \cdot a^{v} = a^{u + v}$~~}
\settowidth{\MyLenB}{$\sqrt[u]{a} \cdot \sqrt[v]{a} = \sqrt[uv]{a^{u + v}}$~~}
\begin{tabular}
	{
		@{}p{\the\MyLenA}%
		@{}p{\the\MyLenB}%
		@{}p{\linewidth - \the\MyLenA - \the\MyLenB}%
	}
	Potentenzen & Wurzeln & Logarithmen \\\hline
	$a^{0} = 1\,\,(a \neq 0)$ & $\sqrt[u]{0} = 0$ & $\log_0{a}; \log_a{0}$ sind undefiniert.\\
	$a^{-u} = \frac{1}{a^{u}}$ & $\sqrt[u]{a} = \frac{1}{a^{u}}$ & $\log_a{a} = 1$\\
	$a^{u} \cdot a^{v} = a^{u + v}$ & $\sqrt[u]{a} \cdot \sqrt[v]{a} = \sqrt[uv]{a^{u + v}}$	 & $\log_a{(u\cdot v)}=\log_a{u}+\log_a{v}$\\
	$\frac{a^{u}}{a^{v}} = a^{u - v}$ & $\frac{\sqrt[u]{a}}{\sqrt[v]{a}} = \sqrt[uv]{a^{u - v}}$ & $\log_a{\frac{u}{v}} = \log_a{u} - \log_a{v}$\\
	$(a^{u})^v = a^{uv}$ & $\sqrt[u]{\sqrt[v]{a}} = \sqrt[u \cdot v]{a}$ & $\log_a{u^v} = v \cdot log_a{u}$\\
	$a^u \cdot b^u = (a \cdot b)^u$ & $\sqrt[u]{a} \cdot \sqrt[u]{b} = \sqrt[u]{a \cdot b}$ & $\log_a{u} \cdot \log_b{u} = \frac{(\log_a{u})^2}{\log_a{b}}$\\
	$\frac{a^u}{b^u} = (\frac{a}{b})^u$ & $\frac{\sqrt[u]{a}}{\sqrt[u]{b}} = \sqrt[u]{\frac{a}{b}}$ & $\frac{\log_a{u}}{\log_b{u}} = \log_a{b}$\\
\end{tabular}
Es gibt keine Logarithmen von negativen Zahlen. Generell löst der Logarithmus folgendes Problem: $a^x = b \rightarrow x = \log_b a$\\

Basiswechsel: $\log_b{x} = \frac{\log_a{x}}{\log_a{b}}$, es gilt auch:  $a^{b} = \mathrm{e}^{b \cdot \ln{a}}\,\, (a > 0)$

\subsubsection{Übersicht Eigenschaften}
Angaben für $D$ und $W$ gelten allgemein. Im Einzelfall genauer prüfen.
\settowidth{\MyLenA}{Monotonie~~}
\begin{tabular}{@{}p{\the\MyLenA}%
				@{}p{(\linewidth - \the\MyLenA)/2}%
				@{}p{(\linewidth - \the\MyLenA)/2}}
	\textbf{$f(x)$} 		& \textbf{$x^n$}  					& \textbf{$a^x$} \\\hline
	$D$ 	 				& $\mathbb{R}$						& $\mathbb{R}$	\\
	$W$						& $\mathbb{R}$						& $(0, \infty)$ \\
	Monotonie				& wachsend							& $a < 1~\searrow, a > 1, \nearrow$\\
	$f'(x)$					& $nx^{n-1}$						& $(\ln a) \cdot a^x$\\
	$f^{-1}(x)$				& $\sqrt[n]{x}$						& $\log_a x$ \\
	$f^{-1'}(x)$			& $\frac{1}{n}\sqrt[n]{x^{1-n}}$ 	& $\frac{1}{(\ln a) \cdot x}$\\ 
\end{tabular}
Spezialfälle:
\settowidth{\MyLenA}{Logarithmusfunktion~~}
\begin{tabular}{@{}p{\the\MyLenA}%
				@{}p{(\linewidth - \the\MyLenA)/3}%
				@{}p{(\linewidth - \the\MyLenA)/3}}
Exponentialfunktion	& $f(x) = \mathrm e^x$	 	& $f'(x) = \mathrm e^x$\\
Logarithmusfunktion	& $f(x) = \ln x$			& $f'(x) = \frac{1}{x}$\\
\end{tabular}




	\section{Trigonometrie}
Winkel in griechischen Buchstaben ($\alpha$, $\beta$ \dots) werden in $^\circ$ Grad, Winkel mit lateinischen Buchstaben ($x$, $y$, \dots) in Radian ausgedrückt.
Für Radian (= Bogenmass) gilt: der Winkel $x$ ist die Länge des Bogens $b$ im Verhältnis zum Radius $r$. Die Beziehung zwischen Grad und Radian ist:
\begin{equation*}
	\frac{\alpha}{360^{\circ}} = \frac{x}{2\pi}
\end{equation*}

In einem rechtwinkligem Dreieck mit der Hypotenuse $c$, der Gegenkathete $a$ und der Ankathete $b$ gilt:
\begin{tabular}{@{}p{\linewidth/3}%
				@{}p{\linewidth/3}%
				@{}p{\linewidth/3}}
	$\sin{\alpha} = \frac{a}{c}$ & $\cos{\alpha} = \frac{b}{c}$ & $\tan{\alpha} = \frac{a}{b} = \frac{\sin{\alpha}}{\cos{\alpha}}$\\
	%$\csc{\alpha} = \frac{c}{a}$ & $\sec{\alpha} = \frac{c}{b}$ & $\cot{\alpha} = \frac{b}{a} = \frac{1}{\tan{\alpha}}$\\
\end{tabular}
Die weiteren trigonometrischen Funktionen ($\csc\alpha = \frac{1}{\sin\alpha}$, $\sec\alpha = \frac{1}{\cos\alpha}$ und $\cot\alpha = \frac{1}{\tan\alpha}$) werden hier
nicht weiter betrachtet.

\subsection{Einheitskreis}
\begin{tikzpicture}[scale=2,cap=round]
  % Local definitions
  \def\costhirty{0.8660256}

  % Colors
  \colorlet{lightgrey}{white!80!black}
  %\colorlet{anglecolor}{green!50!black}
  %\colorlet{sincolor}{red}
  %\colorlet{tancolor}{orange!80!black}
  %\colorlet{coscolor}{blue}

  % Styles
  \tikzstyle{axes}=[]
  \tikzstyle{important line}=[very thick]
  \tikzstyle{information text}=[rounded corners,inner sep=0.5ex] %,fill=black!10

  % The graphic
  \draw[style=help lines,step=0.5cm] (-1.4,-1.4) grid (1.4,1.4);

  \draw (0,0) circle (1cm);

  \begin{scope}[style=axes]
    \draw[->] (-1.5,0) -- (1.5,0) node[right] {$x$};
    \draw[->] (0,-1.5) -- (0,1.5) node[above] {$y$};

    \foreach \x/\xtext in {-1, -.5/-\frac{1}{2}, 1}
      \draw[xshift=\x cm] (0pt,1pt) -- (0pt,-1pt) node[below,fill=white]
            {$\xtext$};

    \foreach \y/\ytext in {-1, -.5/-\frac{1}{2}, .5/\frac{1}{2}, 1}
      \draw[yshift=\y cm] (1pt,0pt) -- (-1pt,0pt) node[left,fill=white]
            {$\ytext$};
  \end{scope}

  \filldraw[fill=lightgrey] (0,0) -- (3mm,0pt) arc(0:30:3mm);
  \draw (15:2mm) node {$\alpha$};

  \draw[style=important line]
    (30:1cm) -- node[left=1pt,fill=white] {$\sin \alpha$} +(0,-.5);

  \draw[style=important line]
    (0,0) -- node[below=2pt,fill=white] {$\cos \alpha$} (\costhirty,0);

  \draw[style=important line] (1,0) --
    node [right=1pt,fill=white]
    {
      $\displaystyle \tan \alpha$
    } (intersection of 0,0--30:1cm and 1,0--1,1) coordinate (t);

  \draw (0,0) -- (t);

	\draw[xshift=1.6cm] node [right,text width=2.8cm,style=information text]
    {
    Der Winkel $\alpha$ ist im Beispiel $30^\circ$: 
      \[
      \sin \alpha = 1/2.
      \]
      Gemäss Pythagoras:
	  \[
	  	cos^2\alpha + \sin^2\alpha = 1
	  \]
	  Also:
      \[
      \cos\alpha = \sqrt{1 - \frac{1}{4}} = \textstyle \frac{1}{2} \sqrt 3. %
      \]%
      Und:
      \[
      \tan\alpha = \frac{\sin
          \alpha}{\cos \alpha} = \textstyle \frac{1}{\sqrt 3}.
      \]%
    };
\end{tikzpicture}

\subsubsection{Rechenregeln}
\begin{align*}
	\sin{x}& = \cos{x} + \frac{\pi}{2}\\
	\cos{x}& = \sin{x} - \frac{\pi}{2}\\
\sin(\alpha \pm \beta)& = \sin{\alpha} \cdot \cos{\beta} \pm \cos{\alpha} \cdot \sin{\beta} \\ 
\cos(\alpha \pm \beta)& = \cos{\alpha} \cdot \cos{\beta} \mp \sin{\alpha} \cdot \sin{\beta} \\
\tan(\alpha \pm \beta)& = \frac{\tan{\alpha} \pm \tan{\beta}}{1 \mp \tan{\alpha} \cdot \tan{\beta}}
\end{align*}

\subsubsection{Übersicht Eigenschaften}
\settowidth{\MyLenA}{$f^{-1'}(x)$~~}
\begin{tabular}{@{}p{\the\MyLenA}%
				@{}p{(\linewidth - \the\MyLenA)/4}%
				@{}p{(\linewidth - \the\MyLenA)/4}%
				@{}p{(\linewidth - \the\MyLenA)/4}}
	\textbf{$f(x)$}	& \textbf{$\sin x$} 					& \textbf{$\cos x$} 					& \textbf{$\tan x$}\\\hline
	$\mathbb{D}$ 	& $\mathbb{R}$ 							& $\mathbb{R}$ 							& $\mathbb{R} \backslash \{\frac{\pi}{2} + k\pi\}$\\
	$\mathbb{W}$	& $[-1, +1]$							& $[-1, +1]$							& $(-\infty, +\infty)$\\
	Peri			& $2\pi$								& $2\pi$								& $\pi$\\	
	Symm.			& ungerade								& gerade								& ungerade \\
	Null			& $x_k = k \cdot \pi$ 					& $x_k = \frac{\pi}{2} + k \cdot \pi$	& $x_k = k \cdot \pi$\\
	$f'(x)$			& $\cos x$								& $-\sin x$								& $\frac{1}{\cos^2 x}$\\
	$f^{-1}(x)$		& $\arcsin x$							& $\arccos x$							& $\arctan x$\\
	$f^{-1'}(x)$	& $\frac{1}{\sqrt{1-x^2}}$				& $-\frac{1}{\sqrt{1-x^2}}$				& $\frac{1}{1+x^2}$
\end{tabular}





	\section{Differentialrechnung}
Berechnet die Steigung der Kurventangente an der Stelle $x_0$.\\
Voraussetzungen:
\begin{equation*}
	\lim_{\Delta x \to 0} \frac{\Delta y}{\Delta x} = \lim_{\Delta x \to 0} \frac{f(x_0 + \Delta x) - f(x_0)}{\Delta x}
\end{equation*}
und linksseitiger Grenzwert = rechtsseitiger Grenzwert.
Dann:
\begin{align*}
	m& = \tan \alpha = \frac{\Delta y}{\Delta x} = \frac{y_2 - y_1}{x_2 - x_1}\\
	\alpha& = \arctan m = \arctan \frac{\Delta y}{\Delta x}
\end{align*}
Eine Funktion ist differenzierbar wenn:
Stetigkeit $\nRightarrow$ diff.-bar, diff.-bar $\Rightarrow$ Stetigkeit, unstetig $\Rightarrow$ undiff.-bar

\subsection{Ableitungsregeln}
Ableitungen zusammengesetzter Funktionen, z.B. $y = \sin(2x)$ oder $ y = x^2 \cdot \mathrm e^{-x^2} $ auf elementare  
Ableitungen zurückführen.

Seien $f(x), g(x)$  und $h(x)$ (im Definitionsbereich) differenzierbare, reelle Funktionen,  und $a, b$ reelle Zahlen, dann gelten:
 
\settowidth{\MyLenA}{Konstante Funktion~~}
\begin{tabular}
	{
		@{}p{\the\MyLenA}%
		@{}p{\linewidth - \the\MyLenA}%
	}
	Konstante Funktion 	& $(a)' = 0$\\
	Faktorregel		& $(a \cdot f(x))' = a \cdot f'(x)$\\
	Summenregel		& $(f(x) \pm g(x))' = f'(x) \pm g'(x)$\\
	Produktregel		& $(f(x) \cdot g(x))' = f'(x) \cdot g(x) + f(x) \cdot g'(x)$\\
	Quotientenregel		& $(\frac{f(x)}{g(x)})' = \frac{f'(x)\cdot g(x) - f(x) \cdot g'(x)}{(g(x))^2}$\\
	Potenzregel		& $(x^n)' = nx^{n-1}$\\
	Kettenregel		& $(f(g(x)))' = (f \circ g)'(x) = f'(g(x)) \cdot g'(x)$\\
	Logarithmisch		& $f'(x) = (g(x)^{h(x)}) = f(x) \cdot (h'(x) \cdot \ln (g(x)) + h(x) \cdot \frac{g'(x)}{g(x)})$\\
\end{tabular}
Die Kettenregel ist im wesentlichen äussere Ableitung mal innere Ableitung. Beispiel:
\begin{align*}
	f:x \rightarrow f(x)& = (x^2 + 4)^3\\
	u:x \rightarrow u(x)& = x^2 + 4 \rightarrow u'(x) = 2x\\
	v:u \rightarrow v(u)& = u^3 \rightarrow v'(u) = 3u^2\\
	f(x)& = (v \circ u)(x) = v(u(x)) \rightarrow f'(x) = 3(x^2 + 4)^2 \cdot 2x
\end{align*}

\subsubsection{Ableitung Umkehrfunktion}
1. Umkehrfunktion bestimmen: $y = f(x) \Rightarrow x = g(y)$\\
2. $g'(y) = \frac{1}{f'(x)}$\\
3. Mit Hilfe von $y = f(x)$ $g'(y)$ als Funktion von $y$ schreiben\\
4. $x$ und $y$ in $g'(y)$ vertauschhen\\

\subsubsection{Ableitung in Parameterform}
$(x = x(t), y = y(t))' \Rightarrow y' = \frac{y'(t)}{x'(t)} = \frac{\dot{y}}{\dot{x}}$

\subsection{Differential}
$\mathrm dy = \mathrm df = f'(x_0) \cdot \mathrm dx$: Zuwachs der Ordinate an der Stelle $x_0$ bei Änderung von $x$ um $\mathrm dx$.

\subsection{Tangente und Normale}
\begin{align*}
	y_T& = f'(x_0)(x - x_0) + y_0 & \mbox{Tangente}\\
	y_N& = \frac{1}{f'(x)} \cdot (x - x_0) + y_0 & \mbox{Normale}
\end{align*}

\subsection{Linearisierung}
In der Umgebung von $P(x_0, y_0)$ gilt $\Delta y = f'(x_0) \Delta x$.

\subsection{Monotonie}
$y' = f'(x) > 0 \Rightarrow \mbox{streng monoton wachsend}$\\
$y' = f'(x) < 0 \Rightarrow \mbox{streng monoton fallend}$\\

\subsubsection{Krümmung}
Linkskrümmung: $y'' = f''(x_0) > 0$\\
Rechtskrümmung: $y'' = f''(x_0) < 0$\\
 

		
	\section{Kurvendiskussion}
% \begin{tikzpicture}[domain=-0.2:3.5, info/.style={very thin, color=gray}]
% 	\coordinate (x0w) at (2,0.666);
% 	\coordinate (x0wx) at (2,0);
% 	\coordinate (x0wy) at (0,0.666);
% 	\coordinate (x0max) at (1,1.333);
% 	\coordinate (x0maxx) at (1,0);
% 	\coordinate (x0maxy) at (0,1.333);
%     \coordinate (x0min) at (3,0);
% 	\coordinate (zero) at (0,0);
% 
% 	%\draw[very thin,color=gray] (-1.1,-1.1) grid (3.9,3.9);
%     \draw[->] (-0.7,0) -- (3.7,0) node[right] {$x$};
%     \draw[->] (0,-0.7) -- (0,3.7) node[above] {$f(x)$};
%     \draw[thick] plot[id=x] function{(x*x*x)/3 - 2*x*x + 3*x} 
%         node[right] {$f(x) = \frac{1}{3}x^3 - 2x^2 + 3x$};
%     \draw plot[id=x1] function{x*x - 4 * x + 3} 
%         node[right] {$f'(x) = x^2 - 4 x + 3$};
%     \draw[dashed] plot[id=x2,domain=1.5:3.5] function{2*x - 4} 
%         node[right] {$f''(x) = 2x - 4$};
% 	\draw[info] (x0maxx) -- (x0max) node[above,color=black] {Maximum};
% 	\draw[info] (x0maxy) -- (x0max);
% 	\draw[info] (x0wx) -- (x0w) node[above,color=black] {Wendepunkt}; 
% 	\draw[info] (x0wy) -- (x0w);
% 	\node[below right] at (x0min) {Minimum};
% 	\node[below left] at (zero) {0};
% \end{tikzpicture}

\subsection{Definitionsbereich und Definitionslücken}
Definitionslücken liegen vor bei nicht-definierten Werten:\\
Division durch 0, negative Wurzeln, Logarithmus von 0.\\

\subsection{Symmetrie}
\begin{align*}
	f(x)& = f(-x) \Rightarrow \mbox{gerade, gespiegelt y-Achse}\\
	f(-x)& = -f(x) \Rightarrow \mbox{ungerade, gespiegelt 0-Punkt}
\end{align*}

\subsection{Nullstellen}
\begin{equation*}
	f(x) = 0	
\end{equation*}

\subsection{Pole}
$x_0$ sei eine Definitionslücke, dann Pol, wenn $\lim_{x_0 \rightarrow 0}f(x_0) = \pm \infty$

\subsection{Ableitungen}
$f'(x), f''(x), f'''(x)$ berechnen

\subsection{Extremwerte}
Extremwerte: $f'(x) = 0, f''(x) < 0 \Rightarrow \mbox{max.}, f''(x) > 0 \Rightarrow \mbox{min.}$
$f^{(n)}(x_0) \neq 0 \Rightarrow (n = \mbox{gerade} \Rightarrow \mbox{Extremwert}) \wedge (n = \mbox{ungerade} \Rightarrow \mbox{Sattelpunkt})$

\subsection{Wende- und Sattelpunkte}
Wendepunkt: $f''(x) = 0, f'''(x) \neq 0$\\
Sattelpunkt: $f'(x) = 0, f''(x) = 0, f'''(x) \neq 0$

\subsection{Asymptoten}
$\lim_{x \rightarrow \infty} f(x), \lim_{x \rightarrow -\infty} f(x)$

\subsection{Wertebereich}
Entweder aus der Zeichnung oder aus Definitionlücken der Umkehrfunktion.


	%% Teil 2
	\section{Allgemein}
Die Integration ist die Umkehrung der Ableitung.
\begin{equation*}
	y = f(x) \xrightarrow{\mbox{Differentiation}} y' = f'(x) \xrightarrow{\mbox{Integration}} y = f(x) 
\end{equation*}

\subsection{Stammfunktion}
Es sei $f(x)$ eine auf dem Intervall $[a, b]$ definierte Funktion. Eine Funktion $F(x)$ heisst Stammfunktion
von $f(x)$ falls für alle $x \in [a, b]$ gilt: $F'(x) = f(x)$.\\
Eigenschaften:
\begin{enumerate}\itemsep0em
	\item Hat eine stetige Funktion $f(x)$ mindestens eine Stammfunktion, so hat sie unendliche viele Stammfunktionen.
	\item Zwei beliebige Stammfunktione $F_1(x) und F_2(x)$ unterscheiden sich durch eine additive Konstante $C$. ($F_1(x) - F_2(x) = $ konstant.
	\item Ist $F_1(x)$ eine beliebige Stammfunktion von $f(x)$, so ist auch $F_1(x) + C$ eine Stammfunktion von $f(x)$.
	Die allgemeine Stammfunktion ist: $F(x) = F_1(x) + C$, wobei $C$ eine beliebige reelle Konstante ist.
\end{enumerate}

\subsection{Flächeninhalt (bestimmtes Integral)}
Um die Fläche $A$ unterhalb einer Funktion $f(x)$ zu berechnen gilt folgendes Vorgehen:
\begin{enumerate}\itemsep0em
	\item Fläche in $n$ Streifen teilen
	\item Alle Streifenflächen berechnen
	\item Flächen aufsummieren
\end{enumerate}

In der Theorie wird eine Fläche in Rechtecke zerlegt, Untersumme ($U_n$) und Obersumme ($O_n$) berechnet. Die Fläche liegt zwischen diesen beiden Werten.
\begin{align*}
	U_n& = \sum_{k=1}^n f(x_{k-1}) \cdot \Delta x_k &   
	O_n& = \sum_{k=1}^n f(x_{k}) \cdot \Delta x_k\\
\end{align*}
\begin{equation*}
	A = \lim_{n \rightarrow \infty} U_n = \lim_{n \rightarrow \infty} O_n = \lim_{n \rightarrow \infty} \sum_{k=1}^n f(x_k) \cdot \Delta x_k = \int\limits_a^b f(x)\dop{x} = \int\limits_{x=a}^{x=b} \dop{A}
\end{equation*}

Das bestimmte Integral ist eine Zahl, die der Fläche entspricht.

\subsection{Flächenfunktion (unbestimmtes Integral)}
\begin{equation*}
	I(x) = \int\limits_a^x f(t)\dop{t}
\end{equation*}
Die obere Intervall Grenze wird offengelassen. Das unbestimmte Integral ist eine Funktion. Eigenschaften:
\begin{enumerate}\itemsep0em
	\item Das unbestimmte Integral $I(x) = \int_a^x f(t)\dop{t}$ repräsentiert den Flächeninhalt zwischen $y = f(t)$ und der $t$-Achse im Intervall $a \leq t \leq x$ in Abhängigkeit
	von der oberen Grenze $x$.
	\item Zu jeder stetigen Funktion $f(t)$ gibt es unendliche viele unbestimmte Integrale, die sich in ihrer unteren Grenze voneinander unterscheiden.
	\item Die Differenz zweier unbestimmter Integrale $I_1(x)$ und $I_2(x)$ von $f(t)$ ist eine Konstante.
\end{enumerate}


	\subsection{Fundamentalsatz}
Jedes unbestimmte Integral $I(x) = \int\limits_{a}^{x} f(x) \dop{x}$ der stetigen Funktion $f(x)$ ist eine Stammfunktion von $f(x)$:
\begin{equation*}
	I(x) = \int\limits_{a}^{x} f(x) \dop{x} \Rightarrow I'(x) = f(x)
\end{equation*}
heisst: Die Ableitung jedes unbestimmten Integrals ergibt die Integrandfunktion. Jeds unbestimmte Integral einer Funktion ist die Menge
aller Stammfunktionen.

\begin{itemize}\itemsep0em
	\item $I(x)$ ist eine stetig differenzierbare Funktion.
	\item Jedes unbestimmte Integral lässt sich schreiben als:
	\begin{equation*}
		I(x) = \lint{a}{x} f(x) \dop{x} = F(x) + C
	\end{equation*}
	\item Die Funktionenschar aller unbestimmter Integrale eine Funktion $f(x)$ schreibt man als
	\begin{equation*}
		\int f(x) \dop{x} = F(x) + C
	\end{equation*}
	wobei $F(x)$ eine beliebige Stammfunktion ist.
	\item Für stetige Funktionen sind die Begriffe \enquote{unbestimmtes Integral} und \enquote{Stammfunktion} synonym.
\end{itemize}





	\subsection{Grund- oder Stammintegrale}
\begin{align*}
	\int 0 \dop{x}& = C \\ 
	\int 1 \dop{x}& = x + C \\
	\int x^n \dop{x}& = \frac{x^{n + 1}}{n + 1} + C \\
	\int \frac{1}{x} \dop{x}& = \ln |x| + C\\
	\int \mathrm e^x \dop{x}& = \mathrm e^x + C \\
	\int a^x \dop{x}& = \frac{a^x}{\ln a} + C\\
	\int \sin x \dop{x}& = - \cos x + C \\
	\int \cos x \dop{x}& = \sin x + C\\
	\int \frac{1}{\sin^2 x} \dop{x}& = - \cot x + C \\
	\int \frac{1}{\cos^2 x} \dop{x}& = \tan x + C \\
 	\int \frac{1}{\sqrt{1 - x^2}} \dop{x}& = \arcsin x + C_1 = -\arccos x + C_2 \\
 	\int \frac{1}{1 + x^2} \dop{x}& = \arctan x + C_1 = -\arccot x + C_2\\
\end{align*}

\subsubsection{Beweistechniken}
Verifizierung: Ableiten der Stammfunktion ($I(x)$ muss den Integrand ($f(x)$) ergeben.
\textbf{Beispiel}
\begin{proof}[Verifizierung]
	\begin{align*}
		\int \ln x \dop{x}& = x \cdot \ln x - x + C & (C \in \mathbb{R})\\
		\frac{\mathrm d}{\dop x} (x \cdot \ln x - x + C) & = 1 \cdot \ln x + x \cdot \frac{1}{x} - 1\\
		& = \ln x + 1 - 1\\
		& = \ln x\qedhere
	\end{align*}
\end{proof}

	\subsection{Berechnen des bestimmten Integrals}
\begin{enumerate}\itemsep0em
	\item Zunächst eine beliebige Stammfunktion bestimmen
	\item Mit der Stammfunktion $F(b)$ und $F(a)$ berechnen:
	\begin{equation*}
		\lint{a}{b} f(x) \dop{x} = \left[ F(x) \right]_a^b = F(b) - F(a)
	\end{equation*}
\end{enumerate}



	\section{Integrationsregeln}
\subsection{Faktorregel}
Ein konstanter Faktor darf vor das Integral gezogen werden.
\begin{equation*}
	\lint{a}{b} C \cdot f(x) \dop{x} = C \cdot \lint{a}{b} f(x) \dop{x}
\end{equation*}

\subsection{Summenregel}
Eine endliche Summe von Funktionen darf gliedweise integriert werden.
\begin{equation*}
	\lint{a}{b} (f_1(x) + \dots + f_n(x)) \dop{x} = \lint{a}{b} f_1(x) \dop{x} + \dots + \lint{a}{b} f_n(x) \dop{x}
\end{equation*}

\subsection{Vertauschungsregel}
Vertauschen der Integrationsgrenzen bewirkt einen Vorzeichenwechsel.
\begin{equation*}
	\lint{a}{b} f(x) \dop{x} = - \lint{b}{a} f(x) \dop{x}
\end{equation*}

\subsection{Gleiche Intervallgrenzen}
Fallen die Integrationsgrenzen zusammen ($a = b$), so ist der Integralwert gleich Null.
\begin{equation*}
	\lint{a}{a} f(x) \dop{x} = 0
\end{equation*}

\subsection{Zerlegen des Integrationsintervalls}
Für jede Stelle $c$ aus dem Integrationsinterval $a \leq c \leq b$ gilt:
\begin{equation*}
	\lint{a}{b} f(x) \dop{x} = \lint{a}{c} f(x) \dop{x} + \lint{c}{b} f(x) \dop{x}
\end{equation*}


	\section{Integrationsmethoden}
\subsection{Substitution}
\begin{equation*}
	\int f(x) \dop{x} = ?
\end{equation*}

\begin{enumerate}\itemsep0em
	\item Aufstellen der Substitutionsgleichungen:
	\begin{equation*}
		u = g(x) \rightarrow \frac{\dop{u}}{\dop{x}} = g'(x) \rightarrow \dop{x} = \frac{\dop{u}}{g'(x)}
	\end{equation*}
	\item Durchführen der Integralsubstitution durch Einsetzen der Substitutionsgleichungen in das vorgegebene Integral:
	\begin{equation*}
		\int f(x) \dop{x} = \int \varphi(u) \dop{u}
	\end{equation*}
	Das neue Integral enthält nur noch die Hilfsvariable $u$ und deren Differential $\dop{u}$. Der Integrand
	ist nur noch eine von $u$ abhängige Funktion $\varphi(u)$
	\item Integration (Berechnung des neuen Integrals)
	\begin{equation*}
		\int \varphi(u) \dop{u} = \Phi(u)
	\end{equation*}
	\item Rücksubsitution (mittels $u = g(x)$)
	\begin{equation*}
		\int f(x) \dop{x} = \Phi(u) = \Phi(g(x)) = F(x)
	\end{equation*}
\end{enumerate}
\begin{itemize}
	\item Die Funktion muss stetig differenzier- und umkehrbar sein.
	\item Die Substitution muss zu einer Vereinfachung führen
	\item Nach einsetzen der Substitutionsgleichung darf $x$ im Integral nicht mehr vorkommen
	\item Bei Wurzelausdrücken ist eine Substitutionsgleichung vom Typ $x = h(u)$ günstiger
	\item Bei bestimmten Integralen kann auf die Rücksubsitution verzichtet werden. 
	Dafür sind die Integrationsgrenzen mit $u = g(x)$ bzw. $x = h(u)$ zu berechnen.
\end{itemize}

\subsubsection{Beispiel mit $u = g(x)$}
\begin{align*}
	\lint{0}{1} x \cdot \sqrt{1 + x^2} \dop{x}& = ?\\
	u &= 1 + x^2 \rightarrow \frac{\dop{u}}{\dop{x}} = 2\cdot x \rightarrow \dop{x} = \frac{\dop{u}}{2 \cdot x}\\
	\mbox{Untergrenze: }& x = 0 \Rightarrow u = 1 + (0)^2 = 1\\
	\mbox{Obergrenze: }& x = 1 \Rightarrow u = 1 + (1)^2 = 2\\
	\lint{0}{1} x \cdot \sqrt{1 + x^2} \dop{x}& = \lint{u=1}{u=2} x \sqrt{u} \frac{\dop{u}}{2 \cdot x}\\
	& = \frac{1}{2} \cdot \lint{1}{2} \sqrt{u} \dop{u} = \frac{1}{2} \lint{1}{2} u^{\frac{1}{2}} \dop{u}\\
	& = \frac{1}{2} \left[ \frac{u^{\frac{3}{2}}}{\frac{3}{2}} \right]_1^2 = \frac{1}{3} \left[ \sqrt{u^3} \right]_1^2\\
	& = \frac{1}{3} (\sqrt{8} - \sqrt{1}) \approx 0{,}6095
\end{align*}

\subsection{Integralsubstitutionen}
\subsubsection{Typ A}
\begin{equation*}
	\int f(a \cdot x + b) \dop{x} = \frac{1}{a} \int f(u) \dop{u}
\end{equation*}
Subsitution: $u = a \dot x + b \rightarrow \dop{x} = \frac{\dop{u}}{a}$\\
Beispiel: $\int \sqrt{4x + 5} \dop{x}; u = 4x + 5$

\subsubsection{Typ B}
\begin{equation*}
	\int f(x) \cdot f'(x) \dop{x} = \frac{1}{2} (f(x))^2 + C
\end{equation*}
Substitution: $u = f(x) \rightarrow \dop{x} = \frac{\dop{u}}{f'(x)}$\\
Beispiel: $\int \sin x \cdot \cos x \dop{x}; u = \sin x$

\subsubsection{Typ C}
\begin{equation*}
	\int (f(x))^n \cdot f'(x) \dop{x} = \frac{1}{n + 1} (f(x))^{n + 1} + C
\end{equation*}
Substitution: $u = f(x) \rightarrow \dop{x} = \frac{\dop{u}}{f'(x)}$\\
Beispiel: $\int (\ln x)^2 \cdot \frac{1}{x} \dop{x}; u = \ln x$

\subsubsection{Typ D}
\begin{equation*}
	\int f(g(x)) \cdot g'(x) \dop{x} = \int f(u) \dop{u}
\end{equation*}
Substitution: $u = g(x) \rightarrow \dop{x} = \frac{\dop{u}}{g'(x)}$\\
Beispiel: $\int x \cdot \mathrm e^{x^2} \dop{x}; u = x^2$

\subsubsection{Typ E}
\begin{equation*}
	\int \frac{f'(x)}{f(x)} \dop{x} = \ln |f(x)| + C
\end{equation*}
Substitution: $u = f(x) \rightarrow \dop{x} = \frac{\dop{u}}{f'(x)}$\\
Beispiel: $\int \frac{2x - 3}{x^2 - 3x + 1} \dop{x}; u = x^2 - 3x + 1$

\subsection{Partielle (Produkt-)Integration}
\begin{equation*}
	\int f(x) \dop{x} = \int u \cdot v' \dop{x} = u \cdot v - \int u' \cdot v \dop{x}
\end{equation*}

Eine Funktion muss geschickt nach $u \cdot v'$ zerlegt werden. Die Stammfunktion von $v'$ muss sich ohne Schwierigkeiten ergeben.
Häufig muss erneut integriert oder substituiert werden.

\subsubsection{Beispiel}
\begin{align*}
	\int \overset{\downarrow}{x} \cdot \overset{\uparrow}{\mathrm e^x} \dop{x}& = ?\\
	u& = x \rightarrow u' = 1\\
	v'& = e^x \rightarrow v = e^x\\
	\int x \cdot \mathrm e^x \dop{x}& = x \cdot e^x - \int 1 \cdot e^x \dop x\\
	& = x  \cdot e^x - e^x + C = (x - 1) \cdot e^x + C
\end{align*}


	\section{Flächeninhalt}
Die Fläche ist immer ein positiver Wert $\rightarrow$ mit Beträgen arbeiten.

\subsection{Allgemeiner Fall}
Flächen, die teils oberhalb, teils unterhalb der $x$-Achse verlaufen, müssen in Teilflächen zerlegt werden,
die entweder oberhalb oder unterhalb der $x$-Achse verlaufen:
\begin{enumerate}\itemsep0em
	\item Nullstellen im Interval $a \leq x \leq b$ bestimmen
	\item Teilflächen aufsummieren (ggf. Skizze erstellen)
\end{enumerate}

\subsection{Fläche zw. zwei Kurven (ohne Schnittpunkte)}
Gegeben seien zwei Kurven $f_1(x)$ und $f_2(x)$
\begin{equation*}
	A = \left|\lint{a}{b} (f_1(x) - f_2(x)) \dop{x}\right|
\end{equation*}

\subsection{Fläche zw. zwei Kurven (mit Schnittpunkten)}
Erst die Schnittpunkte berechnen,
dann wie ohne Schnittpunkt bis zum Schnittpunkt berechnen
und aufsummieren.

	\section{Rotationskörper}
\subsection{x-Achse}
\begin{equation*}
	V_x = \pi \cdot \lint{a}{b} y^2 \dop{x} = \pi \cdot \lint{a}{b} (f(x))^2 \dop{x}
\end{equation*}

\subsection{y-Achse}
$y = f(x)$ in $x = g(y)$ umrechnen und Intervallgrenzen berechnen ($c = f(a), d = f(b)$)
\begin{equation*}
	V_y = \pi \cdot \lint{c}{d} x^2 \dop{y} = \pi \cdot \lint{c}{d} (g(x))^2 \dop{y} 
\end{equation*}


	\section{Anwendungen}
\begin{equation*}
	\mbox{Ort: } s(t) = \int v(t) \dop{t} = \int \int a(t) \dop{t}
\end{equation*}
\begin{equation*}
	\mbox{Geschwindigkeit: } v(t) = \frac{\mathrm d}{\dop{t}} s(t) = \dot s = \int a(t) \dop{t}
\end{equation*}
\begin{equation*}
	\mbox{Beschleunigung: } a(t) = \frac{\mathrm d}{\dop{t}} v(t) = \dot v = \ddot s
\end{equation*}



	% Teil 3
	\section{Folgen}
\begin{description}\itemsep0em
	\item [Folge] Sei $\N$ die Menge der natürlichen Zahlen und $A$ eine nicht leere Menge.
	Ein Folge entsteht, indem man jedem Element $n \in \N$ ein Element $a$ von $A$ zuordnet; man schreibt
	dann für diese Zuordnung:
	\begin{equation*}
		n \mapsto a_n
	\end{equation*}
	Die entstande Folge wird selbst mit
	\begin{equation*}
		\{a_n\}_{n \in \N} \mbox{ oder einfach mit } \{a_n\} \mbox {bezeichnet}
	\end{equation*}

	\item[Obere Schranke] Gibt es eine reele Zahl $K_O$ so, dass
	\begin{equation*}
		a_n \leq K_O \mbox{ für alle } n \in \N
	\end{equation*}
	gilt, so ist die Folge $\{a_n\}$ nach oben beschränkt. Man nennt $K_O$ die obere Schranke der Folge.

	\item[Untere Schranke] Gibt es eine reele Zahl $K_U$ so, dass
	\begin{equation*}
		a_n \geq K_U \mbox{ für alle } n \in \N
	\end{equation*}
	gilt, so ist die Folge $\{a_n\}$ nach unten beschränkt. Man nennt $K_U$ die untere Schranke der Folge.

	\item[Beschränkt] falls eine Folge sowohl nach oben, wie auch nach unten beschränkt ist.
\end{description}

\subsection{Monotonie}
\settowidth{\MyLenA}{Streng monoton wachsend~~}
\begin{tabular}{@{}p{\the\MyLenA}%
				@{}p{\linewidth - \the\MyLenA}}
Monoton steigend & $a_n \leq a_{n + 1}$ für alle $n \in \N$ \\
Streng monoton steigend & $a_n < a_{n + 1}$ für alle $n \in \N$\\
Monoton fallend & $a_n \geq a_{n + 1}$ für alle $n \in \N$ \\
Streng monoton fallend & $a_n > a_{n + 1}$ für alle $n \in \N$\\
\end{tabular}

Eine monoton steigende Folge mit der Indexmenge $\N$ ist immer nach unten beschränkt.
Die untere Schranke ist $a_1$.\\
Eine monoton fallende Folge mit der Indexmenge $\N$ ist immer nach oben beschränkt.
Die obere Schranke ist $a_1$

\subsection{Konvergenz}
Es sei $\{a_n\}_{n \in \N}$ eine reelle Folge und $a$ eine reelle Zahl. Man sagt,
die Folge konvergiert gegen den Grenzwert $a$, wenn für jede beliebige reelle Zahl
$\epsilon > 0$ ein Index $n_0$ existiert, so dass gilt:
\begin{equation*}
	|a_n - a| < \epsilon \mbox{ für alle } n \geq n_0
\end{equation*}
Man schreibt dann
\begin{equation*}
	a = \lim_{n \rightarrow \infty} a_n
\end{equation*}
oder auch
\begin{equation*}
	a_n \rightarrow a \mbox{ für } n \rightarrow \infty
\end{equation*}

\subsection{Rechenregeln}
Es seien $\{a_n\}$ eine konvergierende Folge mit dem Grenzwert $a$ und 
$\{b_n\}$ eine konvergierende Folge mit dem Grenzwert $b$. Dann gilt:

\settowidth{\MyLenA}{Multiplikation~~}
\begin{tabular}{@{}p{\the\MyLenA}%
				@{}p{\linewidth - \the\MyLenA}}
Addition & Die Folge $\{a_n + b_n\}$ konvergiert gegen $a + b$ \\
Subtraktion & Die Folge $\{a_n - b_n\}$ konvergiert gegen $a - b$ \\
Multiplikation & Die Folge $\{a_n \cdot b_n\}$ konvergiert gegen $a \cdot b$ \\
Division & Die Folge $\{\frac{a_n}{b_n}\}$ konvergiert gegen $\frac{a}{b}$ \\
\end{tabular}

Nach oben beschränkte, monoton steigende Folgen konvergieren.\\
Nach unten beschränkte, monoton fallende Folgen konvergieren.\\
Jede konvergente Folge ist beschränkt.\\
Der Grenzwert einer konvergenten Folge ist eindeutig bestimmt:
Jede Folge hat höchstens einen Grenzwert.

	\vfill
\section{Reihen}
Informell: Eine Reihe ist eine Folge, die dadurch entsteht, dass man die Glieder einer anderen Folge aufsummiert und die entstanden Partialsummen als neue Folge interpretiert.

Sei $\{a_i\}$ ein Folge von Zahlen und $p$ eine natürliche Zahl. Dann betrachtet man die Summe $\sum_{i=1}^{p} a_i$
der ersten $p$ Zahlen einer Folge. Gibt es eine Zahl $S$, so dass
\begin{equation*}
	\lim_{p \rightarrow \infty} \sum_{i = 1}^{p} a_i = S
\end{equation*}
ist, konvergiert also die bis ins unendliche fortgesetzte Summation der Folgeglieder $a_i$ gegen einen festen Wert, so sagt man, die
Reihe konvergiert gegen $S$ und schreibt in symbolischer Notation
\begin{equation*}
	\sum_{i = 1}^{\infty} a_i = S
\end{equation*}
Die Zahl $S$ bezeichnet den Summenwert der Reihe (oder auch den Reihenwert). Liegt keine konvergenz vor, so sagt man, die Reihe divergiert.

\subsection{Konvergenzkriterien}
Damit eine Reihe $\sum_{i = 1}^{\infty} a_i$ konvergieren kann ist es notwendig, dass
\begin{equation*}
	\lim_{i \rightarrow \infty} a_i = 0
\end{equation*}


\subsubsection{Quotientenkriterium}
Es sei eine Reihe $	\sum_{i=1}^{\infty} a_i$
vorgelegt. Existiert ein Grenzwert
\begin{equation*}
	q = \lim_{i \rightarrow \infty} \left| \frac{a_{i + 1}}{a_i} \right|
\end{equation*}
und ist $q < 1$, so konvergiert die Reihe. Ist $q > 1$, so divergiert die Reihe.
Ist $q = 1$ kann keine Aussage gemacht werden.

\subsubsection{Wurzelkriterium}
Es sei eine Reihe $\sum_{i=1}^{\infty} a_i$
\begin{equation*}
	q = \lim_{i \rightarrow \infty} \sqrt[i]{|a_i|}
\end{equation*}
und ist $q < 1$, so konvergiert die Reihe. Ist $q > 1$, so divergiert die Reihe.
Ist $q = 1$ kann keine Aussage gemacht werden.

\subsubsection{Leibniz-Kriterium}
Sei $\{u_i\}$ eine Folge von Zahlen, die entweder alle positiv oder negativ sind,
dann nennt man die Reihe
\begin{equation*}
	\sum_{i=1}^{\infty} (-1)^i u_i	
\end{equation*}
ein alternierende Reihe.\\

Für alternierende Reihen gilt das Leibniz-Kriterium: Konvergiert die Folge $\{u_i\}$ streng monoton gegen 0,
so konvergiert die Reihe ($u_1 > u_2 > \dots > u_i$)

\subsection{Wichtige Reihen}

\begin{align*}
	\sum_{i=1}^{\infty} \frac{1}{i}& = \frac{1}{1} + \frac{1}{2} + \dots + \frac{1}{i} \mbox{ (harmonische Reihe, divergiert)}\\
	\sum_{i=1}^{\infty} (-1)^{i-1}\frac{1}{i} & = 1 - \frac{1}{2} + \frac{1}{3} - \frac{1}{4} + \dots - \frac{1}{i} = \ln 2\\
	\sum_{i=1}^{\infty} a q^{i-1}& = a + aq + aq^2 + \dots + aq^i \mbox{ geometrische Reihe} (q > 1)\\
	\sum_{i=1}^{\infty} a q^{i-1}& = a + aq + aq^2 + \dots + aq^i = \frac{a}{1 - q} \mbox{ für } (|q| < 1) \\
	\sum_{i=0}^{\infty} \frac{1}{i!}& = \frac{1}{0!} + \frac{1}{1!} + \frac{1}{2!} + \dots + \frac{1}{i!} = \mathrm e \\
	\sum_{i=1}^{\infty} (-1)^{i-1}\frac{1}{2i - 1}& = 1 - \frac{1}{3} + \frac{1}{5} + \dots = \frac{\pi}{4}\\
	\sum_{i=1}^{\infty} \frac{1}{i^2}& = 1 + \frac{1}{2^2} + \frac{1}{3^2} + \dots  = \frac{\pi^2}{6}\\
	\sum_{i=1}^{\infty} (-1)^{i-1}\frac{1}{i^2}& = 1 - \frac{1}{2^2} + \frac{1}{3^2} - \dots = \frac{\pi^2}{12}\\	
	\sum_{i=1}^{\infty} \frac{1}{i \cdot (i + 1)}& = \frac{1}{2} + \frac{1}{6} + \frac{1}{12} + \dots = 1	
\end{align*}
Für die Eulersche Zahl gilt, das $0! = 1$

\subsection{Potenzreihen}
Unter einer Potenzreihe versteht man eine unendliche Reihe vom Typ
\begin{equation*}
	P(x) = \sum_{n=0}^{\infty} a_n \cdot (x - x_0)^n = a_0 + a_1 \cdot (x - x_0)^1 + a_2 \cdot (x - x_0)^2 + \dots + a_n \cdot (x - x_0)^n
\end{equation*}
Die Stelle $x_0$ heisst Entwicklungspunkt oder auch Entwicklungszentrum.
Die reellen Zahlen $a_0, a_1, a_2, \dots$ heissen Koeffizienten der Potenzreihe.

\subsubsection{Konvergenzbereich}
Die Menge aller $x$-Werte, für eine Potenzreihe konvergiert heisst Konvergenzbereich.

Zu jeder Potenzreihe gibt es ene positive Zahl $r$, Konvergenzradius genannt, mit folgenden Eigenschaften:
\begin{enumerate}\itemsep0em
	\item Die Potenzreihe konvergiert überall im Intervall $|x| < r$
	\item Die Potenzreihe divergiert dagegen für $|x| > r$.
	\item Über das Verhalten in $|x| = r$ lassen sich keine allgemeinen Aussagen machen $\Rightarrow$ explizit betrachten.
\end{enumerate}

Falls für alle Koeffizienten gilt $a_n \neq 0$ und der ein Grenzwert für $a_n$ vorhanden ist, lässt sich
der Konvergenzradius $r$ wie folgt berechnen:
\begin{align*}
	r& = \lim_{n \rightarrow \infty} \left|\frac{a_n}{a_{n+1}}\right| &
	r& = \frac{1}{\lim_{n \rightarrow \infty} \sqrt[n]{|a_n|}}
\end{align*}

\begin{itemize}\itemsep0em
	\item Für $x = 0$ konvergiert jede Potenzreihe und besitzt dort den Summenwert $P(0) = a_0$\\
	\item Es gibt Potenzreihen, die nur für $x = 0$ konvergieren
	\item Es gibt Potenzreihen, die für jedes $x \in \mathbb{R}$ konvergieren
	\item Allgemein konvergiert eine Potenzreihe in einem zum Nullpunkt symmetrischen Intervall $r$
\end{itemize}



	\section{Potenzreihenentwicklung}
\subsection{Taylorsche Reihe}
Die Taylorsche Reihe ist hilfreich um komplexe Funktionen in Polynome zu verwandeln. Je höher der Grad des Polynoms, desto stärker wird die Funktion angenähert.
\begin{align*}
	f(x)& = \frac{f(x_0)}{0!} + \frac{f'(x_0)}{1!}(x - x_0)^1 + \frac{f''(x_0)}{2!} (x - x_0)^2 +  \dots\\
	& = \sum_{n=0}^\infty \frac{f^{(n)} (x_0)}{n!} (x - x_0)^n
\end{align*}

Wobei $x_0$ als Entwicklungspunkt bzw. als Entwicklungszentrum betrachtet wird.

\subsection{Mac Laurinsche Reihe}
Die Mac Laurinsche Reihe ist ein Spezialfall der Taylor Reihe im Entwicklungspunkt $x_0 = 0$:
\begin{equation*}
	f(x) = f(0) + \frac{f'(0)}{1!} x + \frac{f''(0)}{2!} x^2 + \dots = \sum_{n=0}^\infty \frac{f^{(n)}(0)}{n!} x^n
\end{equation*}



	\section{Grenzwertregel Bernoulli/de L'Hospital}
\begin{equation*}
	\lim_{x \to x_0} \frac{f(x)}{g(x)} = \lim_{x \to x_0} \frac{f'(x)}{g'(x)}
\end{equation*}
\begin{itemize}\itemsep0em
	\item Voraussetzung: $f(x)$ und $g(x)$ sind in der Umgebung von $x_0$ stetig differenzierbar
	\item Gilt auch für Grenzübergänge $x \to \pm \infty$
	\item Manchmal muss die Regel mehrfach angewendet werden
	\item Es gibt Fälle, in denen die Regel versagt
\end{itemize}

\subsection{Umformungen}
\subsubsection{Typ A: $u(x) \cdot v(x)$ für $0 \cdot \infty$}
\begin{align*}
	u(x) \cdot v(x)& = \frac{u(x)}{\frac{1}{v(x)}} & u(x) \cdot v(x)& = \frac{v(x)}{\frac{1}{u(x)}} 
\end{align*}

\subsubsection{Typ B: $u(x) - v(x)$ für $\infty - \infty$}
\begin{equation*}
	u(x) - v(x) = \frac{\frac{1}{v(x)} - \frac{1}{u(x)}}{\frac{1}{u(x) \cdot v(x)}}
\end{equation*}

\subsubsection{Typ C: $u(x)^{v(x)}$ für $0^0, \infty^0, 1^\infty$}
\begin{equation*}
	u(x)^{v(x)} = \mathrm e^{v(x) \cdot \ln u(x)}
\end{equation*}


	\section{Komplexe Zahlen $\mathbb{C}$}
Eine komplexen Zahl $z$ ist ein geordnetes Paar $(x; y)$ aus zwei reellen Zahlen $x$ und $y$: $z = x + \mathrm j y$.
$x$ ist der Realteil von $z$, $y$ heisst Imaginärteil von $z$. Die imaginäre Einheit heisst $j$. Es gilt:
\begin{equation*}
	\mathrm j^2 = -1
\end{equation*}

\subsection{Darstellungsformen}
\settowidth{\MyLenA}{Trigonometrische Form~~}
\begin{tabular}{@{}p{\the\MyLenA}%
				@{}p{\linewidth - \the\MyLenA}}
Normalform & $z = x + \mathrm j y$ \\
Trigonometrische Form & $z = r \cdot (\cos \varphi + \mathrm j \sin \varphi)$\\
Exponentialform & $z = r \cdot \mathrm e^{\mathrm j \varphi}$
\end{tabular}

\begin{center}
\begin{tikzpicture}[scale=2,cap=round]
  \def\costhirty{0.8660256}
  \def\sinthirty{0.5}
  % Colors
  \colorlet{lightgrey}{white!80!black}

  % Styles
  \tikzstyle{axes}=[]
  \tikzstyle{important line}=[very thick]
  \tikzstyle{vector}=[-latex]

  % The graphic
  \draw[style=help lines,step=0.5cm] (-1.1,-1.1);

  \begin{scope}[style=axes]
    \draw[->] (-1.5,0) -- (1.5,0) node[right] {$Re(z)$};
    \draw[->] (0,-1.1) -- (0,1.1) node[above] {$Im(z)$};

    \foreach \x/\xtext in {-1, -.5/-\frac{1}{2}, 1}
      \draw[xshift=\x cm] (0pt,1pt) -- (0pt,-1pt) node[below,fill=white]
            {$\xtext$};

    \foreach \y/\ytext in {-1, -.5/-\frac{1}{2}, .5/\frac{1}{2}, 1}
      \draw[yshift=\y cm] (1pt,0pt) -- (-1pt,0pt) node[left,fill=white]
            {$\ytext$};
  \end{scope}

  \draw[style=important line]
    (0,0) -- node[left=1pt,fill=white] {$\mathrm j y$} +(0,\sinthirty);

  \draw[style=important line]
    (0,0) -- node[below=2pt,fill=white] {$x$} (\costhirty,0);

  \draw[vector] (0,0) -- (\costhirty, \sinthirty) node [right,fill=white] {$z = x + \mathrm j y$};

  \draw[dotted] (0,\sinthirty) -- (\costhirty, \sinthirty);

  \draw[dotted] (\costhirty,0) -- (\costhirty, \sinthirty);

  \draw[dotted] (\costhirty,0) -- (\costhirty, -\sinthirty);

  \draw[dotted] (0,-\sinthirty) -- (\costhirty, -\sinthirty);

  \draw[vector, dashed] (0,0) -- (\costhirty, -\sinthirty) node [right,fill=white] {$z^* = x - \mathrm j y$};

  \node (q1) at (1,1) {Q1: $\omega = 0$};
  \node (q2) at (-1, 1) {Q2: $\omega = \pi$};
  \node (q3) at (-1, -1) {Q3: $\omega = \pi$};
  \node (q4) at (1, -1) {Q4: $\omega = 2\pi$};
  \node (z) at (0.32, 0.3) {$r$};
  \node (phi) at (0.32, 0.1) {$\varphi$};
  \draw (0.5,0) arc (0:30:5mm);
\end{tikzpicture}
\end{center}

\subsection{Umrechnungen}
\subsubsection{Trigonometrisch/Exponential Form $\rightarrow$ Normalform}
\begin{align*}
	x& = r \cdot \cos \varphi\\
	y& = r \cdot \sin \varphi
\end{align*}

\subsubsection{Normalform $\rightarrow$ Trigonometrisch/Exponentialform}
\begin{align*}
	r = |z|& = \sqrt{x^2 + y^2}\\
	\varphi& = \arctan \left(\frac{y}{x}\right) + \omega
\end{align*}
Dabei heissen $r$ der Betrag und $\varphi$ Argument/Winkel/Phase von $z$.\\
$\omega$ ist abhängig vom Quadranten.

\subsection{Anmerkungen}
\begin{itemize}\itemsep0em
	\item $\mathbb{C} = \{z | z = x + \mathrm j y$ mit $x, y \in \mathbb{R}\}$
	\item $z_1 = x_1 + \mathrm j y_1  = z_2 = x_2 + \mathrm j y_2 \Rightarrow (x_1 = x_2) \wedge (y_1 = y_2)$
	\item Die konjugiert komplexe Zahl $z^* = (x + \mathrm j y)^* = x - \mathrm j y$.
	\item $e^{\mathrm j \pi} = -1$
\end{itemize}


	\section{Komplexe Rechnung}
\begin{itemize}\itemsep0em
	\item Addition und Subtraktion nur in Normalform möglich.
	\item Ungleichungen machen für komplexe Zahlen keinen Sinn.
\end{itemize}

\subsection{Addition/Subtraktion}
\begin{equation*}
	z_1 \pm z_2 = (x_1 \pm x_2) + \mathrm j (y_1 \pm y_2)
\end{equation*}

\subsection{Multiplikation}
\subsubsection{Normalform}
Das Produkt $z_1 \cdot z_2 = (x_1 + \mathrm j y_1) \cdot (x_2 + \mathrm j y_2)$ 
wird im Reellen durch Ausmultiplizieren der Klammern unter Beachtung der Beziehung $\mathrm j^2 = -1$ berechnet.
\subsubsection{Polarform}
Zwei komplexe Zahlen werden multipliziert, indem man ihre Beträge multipliziert und die Argumente addiert.
\begin{equation*}
	z_1 \cdot z_2 = r_1 \cdot \mathrm e^{\mathrm j \varphi_1} \cdot r_2 \cdot \mathrm e^{\mathrm j \varphi_2} = r_1 \cdot r_2 \cdot \mathrm e^{\mathrm j \varphi_1 + \varphi_2}
\end{equation*}

\subsubsection{Division}
\subsubsection{Normalform}
Der Quotient $\frac{z_1}{z_2}$ in der Normalform lässt sich wie folgt berechnen:
\begin{enumerate}\itemsep0em
	\item Der Bruch wird mit $z_2^*$, dem konjugiert komplexen Nenner erweitert:
	\begin{equation*}
		\frac{z_1}{z_2} = \frac{z_1 \cdot z_2^*}{z_2 \cdot z_2^*} = \frac{(x_1 + \mathrm j y_1) \cdot (x_2 - \mathrm j y_2)}{(x_2 + \mathrm j y_2) \cdot (x_2 - \mathrm j y_2)}
	\end{equation*}
	\item Zähler und Nenner werden unter Berücksichtigung von $j^2 = - 1$ ausmultipliziert ($\rightarrow$ der Nenner wird reell)
	\item Die im Zähler stehende komplexe Zahl wird gliedweise durch den Nenner dividiert.
\end{enumerate}
Die Division durch Null bleibt verboten.
\subsubsection{Polarform}
Zwei komplexe Zahlen werden dividiert, indem man ihre Beträge dividiert und die Argumente subtrahiert.
\begin{equation*}
	\frac{z_1}{z_2} = \frac{r_1 \cdot \mathrm e^{\mathrm j \varphi_1}}{r_2 \cdot \mathrm e^{\mathrm j \varphi_2}} = \frac{r_1}{r_2} \cdot \mathrm e^{\mathrm j (\varphi_1 - \varphi_2)}
\end{equation*}

Multiplikation und Division können als Drehstreckung bzw. Drehstauchung geometrisch interpretiert werden.

\subsection{Potenzieren}
Geht am einfachsten in der Polarform:
\begin{align*}
	z^n& = \left(r \cdot \mathrm e^{\mathrm j \varphi} \right)^n = r^n \cdot \mathrm e^{\mathrm j n \cdot \varphi}\\
	z^n& = \left(r \cdot \cos \varphi + \mathrm j \sin \varphi \right)^n = r^n \cdot (\cos n \cdot \varphi + \mathrm j \sin n \cdot \varphi)
\end{align*}

\subsection{Radizieren}
Geht am einfachsten in der Polarform:
\begin{align*}
	\sqrt[n]z& = \sqrt[n]{r \cdot \mathrm e^{\mathrm j \varphi}} = \sqrt[n]{r} \cdot \mathrm e^{\mathrm j \frac{\varphi + k \cdot 2 \pi}{n}}\\
	\sqrt[n]z& = \sqrt[n]{r \cdot \cos \varphi + \mathrm j \sin \varphi} = \sqrt[n]{r} \cdot (\cos \frac{\varphi + k \cdot 2 \pi}{n} + \mathrm j \sin \frac{\varphi + k\cdot 2 \pi}{n})
\end{align*}
Mit $k = 0, 1, 2, \dots, n - 1 \rightarrow$ eine $n$te Wurzel hat $n$ Lösungen.

\subsection{Eigenschaften der Grundrechenarten}
\begin{itemize}\itemsep0em
	\item Addition und Multiplikation sind kommutativ: $z_1 + z_2 = z_2 + z_1$
	\item Addition und Multiplikation sind assoziativ: $z_1 \cdot (z_2 \cdot z_3) = (z_1 \cdot z_2) \cdot z_3$
	\item Addition und Multiplikation sind über das Distributivgesetz verbunden: $z_1 \cdot (z_2 + z_3) = z_1 \cdot z_2 + z_1 \cdot z_3$
\end{itemize}


	% Teil 4
	\section{Differential mit mehreren Variablen}
Eine Funktion von $n$ unabhängigen Variablen ist eine Vorschrift, die jedem geordneten Zahlenpaar $(x_1; x_2; \dots; x_n)$ aus einer Definitionsmenge $D$ genau ein Element $y$ aus einem Wertebereich $W$ zuordnet: $y = f(x_1; x_2; \dots; x_n)$.
\subsection{Darstellungen}
\begin{description}\itemsep0em
	\item [Explizit] $y = f(x_1; x_2; \dots; x_n)$
	\item [Implizit] $F(x_1; x_2; \dots; x_n; y) = 0$
	\item Funktionstabelle: Bei zwei unabhängigen gibt es eine Matrix ($x_1$-Werte in den Zeilen, $x_2$-Werte in den Spalten). Mit mehreren unabhängigen kommen weiteren Tabellen
	für $x_3$ usw. dazu.
	\item [Graphisch]
		\begin{itemize}\itemsep0em
			\item Fläche (3d) Geht nur mit zwei unabhängigen Variablen.
			\item Schnittkurvendiagramm (2d) (mit zwei unabhängigen Variablen): $f(x_1; x_2) = $ Konstant
		\end{itemize}
\end{description}
 
\subsection{Grenzwert}
Eine Funktion zweier Variablen hat an der Stelle $(x_0, y_0)$ den Grenzwert $g$, wenn sich die Funktionswerte von $f(x, y)$ beim Grenzübergang $(x, y) \rightarrow (x_0, y_0)$ unabhängig vom Weg dem Grenzwert $g$ beliebig annähern.
\begin{equation*}
	g = \lim_{(x, y) \rightarrow (x_0, y_0)} f(x, y)
\end{equation*}
Lösungswege: Man nähere sich $(x_0, y_0)$ entlang einer geraden $y = mx$. So kann man in die Funktionsgleichung für jedes $y$ den Wert $mx$ einsetzen. Ergibt sich dann ein konstanter Wert, hat die Funktion einen Grenzwert. Falls das Ergebnis noch von $m$ abhängt, hat sie keinen Grenzwert.

\subsection{Stetigkeit}
Eine Funktion ist an einer Stelle stetig, wenn der Grenzwert vorhanden ist und mit dem Funktionswert übereinstimmt.

Funktionen, die an jeder Stelle des Definitionsbereich stetig sind, heissen stetige Funktionen.


	\section{Partielle Ableitungen}
Eine Funktion mit mehreren Variablen wird nach nur einer der Variablen abgeleitet. Die übrigen werden als Konstant angenommen.
\begin{equation*}
	\frac{\partial f}{\partial x} (x; y)= f_x(x; y) = m_x = \lim_{\Delta x \rightarrow 0} \frac{f(x + \Delta x, y) - f(x, y)}{\Delta x}
\end{equation*}
(Analog für alle weiteren).
$f_x$ entspricht dem Anstieg der Flächentangente in positiver $x$-Richtung im Punkt $(x, y)$.
Oft ist es nützlich eine oder mehrere Hilfsvariablen einzuführen:
\begin{align*}
	z& = f(x; 0) = xy^2 \cdot (\sin x + \sin y)\\
	u& = xy^2 \rightarrow u_x = y^2, u_y = 2xy\\
	v& = \sin x + \sin y \rightarrow v_x = \cos x, v_y = \cos y\\
	z& = u \cdot v\\
	z_x& = u_xv + uv_x = y^2(\sin x + \sin y) + xy^2 \cos x\\
	z_y& = u_yv + uv_y = 2xy (\sin x + \sin y) + xy^2 \cos y
\end{align*}

\subsection{Partielle Ableitungen höherer Ordnung}
Die partiellen Ableitungen erster Ordnung werden erneut abgeleitet.
Es ergeben sich dann $f_{xx}, f_{xy}, f_{yx}, f_{yy}$ usw. Wobei die Reihenfolge des Ableitens keine Rolle spielt: $f_{xy} = f_{yx}$, wenn die partiellen Ableitungen stetige Funktionen sind.


\subsection{Tangentialebene}
Die Tangentialebene $z = t(x, y)$ besitzt im Punkt $P = (x_0, y_0, z_0)$ die gleiche Steigung (aka gleiche partielle Ableitungen) wie die gegebene Funktion $z = f(x, y)$ ($z_0 = f(x_0, y_0)$).
% \begin{align*}
% 	a& = t_x(x_0, y_0) = f_x(x_0, y_0) & b&= t_y(x_0, y_0) = f_y(x_0, y_0)\\
% 	c& = z_0 - ax_0 - by_0 & z_0 = f(x_0, y_0)
% \end{align*}
\begin{equation*}
	z = f_x(x_0; y_0) \cdot (x - x_0) + f_y(x_0; y_0) \cdot (y - y_0) + z_0
\end{equation*}
Beispiel:
\begin{align*}
	z& = f(x; y) = x^2 + y^2, P = (1; 2; 5)\\
	f_x(x; y)& = 2x \Rightarrow f_x(1, 2) = 2\\
	f_y(x; y)& = 2y \Rightarrow f_y(1, 2) = 4\\
	z - 5& = 2 (x - 1) + 4 (y - 2)\\
	z & = 2x + 4y - 5
\end{align*}

\subsection{Das totale Differential}
Unter dem totalen (vollständigen) Differential einer Funktion $z = f(x; y)$ von zwei unabhängigen Variablen wird der linerare Ausdruck
\begin{equation*}
	\mathrm dz = f_x \mathrm dx + f_y \mathrm dy = \frac{\partial f}{\partial x} \mathrm dx + \frac{\partial f}{\partial y} \mathrm dy
\end{equation*}
verstanden. Es beschreibt die Änderung der Höhenkoordinate auf der im Berührungspunkt $P(x_0, y_0, z_0)$ errichteten Tangentialebene. d$x$, d$y$ und d$z$ sind dann Koordinaten auf der Tangentialebene bezogen auf $P$.

Mit weiteren unabhängigen Variablen würden diese einfach linear hinzugefügt:
\begin{equation*}
	\mathrm dy = f_{x_1} \mathrm dx_1 + f_{x_2} \mathrm dx_2 + \dots + f_{x_n} \mathrm dx_n
\end{equation*}

\subsection{Anwendungen}
\subsubsection{Implizite Differentiation}
Der Anstieg der implizit dargestellten Funktion $F(x; y) = 0$ im Punkt $P = (x; y)$ lässt sich wie folgt bestimmen:
\begin{equation*}
	y'(x; y) = - \frac{F_x(x; y}{F_y(x; y)}
\end{equation*}
Erst ableiten, dann einsetzen!

\subsubsection{Linearisierung}
In der Umgebung eines Flächenpunktes $P = (x_0; y_0; z_0)$ kann die nichtlineare Funktion $z = f(x; y)$ näherungsweise durch die Tangentialebene ersetzt werden:
\begin{equation*}
	\Delta z = f_x(x_0; y_0) \, \Delta x + f_y(x_0, y_0) \, \Delta y
\end{equation*}

\subsection{Extremwerte}
Eine Funktion $z = f(x; y)$ besitzt an der Stelle $(x_0; y_0)$ einen Extremwert, wenn gilt:
\begin{align*}
	f_x(x_0; y_0)& = 0 & f_y(x_0; y_0)& = 0\\
	\Delta & = f_{xx}(x_0; y_0) \cdot f_{yy}(x_0; y_0) - f_{xy}^2(x_0; y_0) > 0
\end{align*}
$f_xx(x_0; y_0) > 0 \Rightarrow $ Minimum, $f_xx(x_0; y_0) < 0 \Rightarrow $ Maximum.
Falls $\Delta = 0$ ist keine Aussage möglich. Falls $\Delta < 0$ handelt es sich um einen Sattelpunkt. $\Delta$ wird auch als Diskriminante bezeichnet. 

	\section{Doppelintegrale}
Der Grenzwert $\lim_{n \rightarrow \infty, \Delta A_k \rightarrow 0} \sum_{k=1}^n f(x_k; y_k) \Delta A_k$ wird (falls vorhanden) als Doppelintegral bezeichnet und als $\iint_A f(x;y) \mathrm dA$ geschrieben. Dabei ist $\mathrm dA = \mathrm dx \cdot \mathrm dy$.

\subsection{Berechnung}
\subsubsection{$x$ konstant, $y$ zwischen Funktionen}
\begin{equation*}
	\iint_A f(x; y) \mathrm dA = \int_{x=a}^b \int_{y = f_u(x)}^{f_o(x)} f(x; y) \mathrm dy \mathrm dx
\end{equation*}
Dabei sind $f_u(x)$ und $f_o(x)$ die untere bzw. die obere einschliessende Funktion.
\begin{enumerate}\itemsep0em
	\item Innere Integration nach $y$
	\item Äussere Integration nach $x$
\end{enumerate}


\subsection{$y$ konstant, $x$ zwischen Funktionen}
\begin{equation*}
	\iint_A f(x; y) \mathrm dA = \int_{y=a}^b \int_{x = g_l(y)}^{g_r(y)}
\end{equation*}
Dabei sind $g_l(y)$ und $g_r(y)$ die linke bzw. rechte einschliessende Funktion.
\begin{enumerate}\itemsep0em
	\item Innere Integration nach $x$
	\item Äussere Integration nach $y$
\end{enumerate}

\subsection{Doppelintegral in Polarkoordinaten}
($x = r \cos \varphi, y = r \sin \varphi, \mathrm dA = r\mathrm dr \mathrm d\varphi$)
Transformation Doppelintegral:
\begin{equation*}
	\iint_A f(x; y)\, \mathrm dA = \int_{\varphi = \varphi_1}^\varphi \int_{r=r_i(\varphi)}^{r_a(\varphi)} f(r \cdot \cos \varphi; r \cdot \sin \varphi) \cdot r\,\mathrm dr \,\mathrm d\varphi
\end{equation*}

\subsection{Flächenberechnungen}
Das Doppelintegral lässt beliebige Flächen berechnen. Dabei wird die Funktionsgleichung $f(x; y) = 1$ gesetzt:
\begin{align*}
	A& = \iint_a \mathrm dA\\
	A& = \int_{x=a}^b \int_{y=f_u(x)}^{f_o(x)} \mathrm dy\, \mathrm dx\\
	A& = \int_{\varphi = \varphi_1}^{\varphi_2} \int_{r=r_i(\varphi)}^{r_a(\varphi)} r\, \mathrm dr\, \mathrm d\varphi
\end{align*}
Beispiel:
\begin{align*}
	r(\varphi)& = 1 + \cos \varphi\mbox{ im Intervall } 0 \leq \varphi < 2 \pi\\
 	A & = \int_{\varphi = 0}^{2\pi}\int_{r=0}^{1 + \cos \varphi}\\
 	& = \int_{\varphi = 0}^{2\pi} \left[\frac{1}{2}r^2 \right]_0^{1 + \cos \varphi} \mathrm d\varphi\\
 	& = \int_{\varphi = 0}^{2\pi} \frac{(1 + \cos \varphi)^2}{2} \mathrm d\varphi\\
	& = \frac{1}{2} \int_{\varphi = 0}^{2\pi} (1 + 2 \cdot \cos \varphi + \cos^2 \varphi) \, \mathrm d\varphi\\
 	& = \frac{1}{2}\left[\varphi + 2 \cdot \sin \varphi + \frac{1}{2} \varphi + \frac{1}{4} \sin 2\varphi \right]_0^{2\pi} = \frac{3}{2}\pi
\end{align*}

	
	\section{Häufige Werte}
\subsection{Winkelmasse}
\begin{center}
\begin{tabular}{c|c|c|c|c}
\textbf{Rad}& \textbf{Grad} & \textbf{sin} 						& \textbf{cos} 						& \textbf{tan} \\\hline
$0$ 		& $0^\circ$ 	& $0$ 								& $1$ 								& $0$\\
$\pi/12$ 	& $15^\circ$ 	& $\frac{1}{4}(\sqrt{6}-\sqrt{2})$ 	& $\frac{1}{4}(\sqrt{6}+\sqrt{2})$ 	& $2 - \sqrt{3}$\\
$\pi/6$ 	& $30^\circ$ 	& $\frac{1}{2}$ 					& $\frac{1}{2}\sqrt{3}$ 			& $\frac{1}{3}\sqrt{3}$\\
$\pi/4$ 	& $45^\circ$ 	& $\frac{1}{2}\sqrt{2}$ 			& $\frac{1}{2}\sqrt{2}$				& $1$\\
$\pi/3$ 	& $60^\circ$ 	& $\frac{1}{2}\sqrt{3}$ 			& $\frac{1}{2}$						& $\sqrt{3}$\\
$\pi/2$ 	& $90^\circ$ 	& $1$ 								& $0$ 								& $\pm \infty$\\
\end{tabular}
\end{center}

\subsection{Pascalsches Dreieck}
\begin{tabular}{rccccccccccc} 
$n=0$:& & & & & & 1\\%{\smallskip\smallskip} 
$n=1$:& & & & & 1 & & 1\\%{\smallskip\smallskip} 
$n=2$:& & & & 1 & & 2 & & 1\\%{\smallskip\smallskip} 
$n=3$:& & & 1 & & 3 & & 3 & & 1\\%{\smallskip\smallskip} 
$n=4$:& & 1 & & 4 & & 6 & & 4 & & 1\\%{\smallskip\smallskip} 
$n=5$:& 1 & & 5 & & 10 & & 10 & & 5 & & 1\\
\end{tabular}

	

	\rule{0.3\linewidth}{0.25pt}\\
	\scriptsize
	Copyright \copyright\ 2013 Constantin Lazari\\
	% Should change this to be date of file, not current date.
	Revision: 1.0, Datum: \today\\
	\end{multicols}
\end{document}