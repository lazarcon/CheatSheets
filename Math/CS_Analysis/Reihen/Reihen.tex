\vfill
\section{Reihen}
Informell: Eine Reihe ist eine Folge, die dadurch entsteht, dass man die Glieder einer anderen Folge aufsummiert und die entstanden Partialsummen als neue Folge interpretiert.

Sei $\{a_i\}$ ein Folge von Zahlen und $p$ eine natürliche Zahl. Dann betrachtet man die Summe $\sum_{i=1}^{p} a_i$
der ersten $p$ Zahlen einer Folge. Gibt es eine Zahl $S$, so dass
\begin{equation*}
	\lim_{p \rightarrow \infty} \sum_{i = 1}^{p} a_i = S
\end{equation*}
ist, konvergiert also die bis ins unendliche fortgesetzte Summation der Folgeglieder $a_i$ gegen einen festen Wert, so sagt man, die
Reihe konvergiert gegen $S$ und schreibt in symbolischer Notation
\begin{equation*}
	\sum_{i = 1}^{\infty} a_i = S
\end{equation*}
Die Zahl $S$ bezeichnet den Summenwert der Reihe (oder auch den Reihenwert). Liegt keine konvergenz vor, so sagt man, die Reihe divergiert.

\subsection{Konvergenzkriterien}
Damit eine Reihe $\sum_{i = 1}^{\infty} a_i$ konvergieren kann ist es notwendig, dass
\begin{equation*}
	\lim_{i \rightarrow \infty} a_i = 0
\end{equation*}


\subsubsection{Quotientenkriterium}
Es sei eine Reihe $	\sum_{i=1}^{\infty} a_i$
vorgelegt. Existiert ein Grenzwert
\begin{equation*}
	q = \lim_{i \rightarrow \infty} \left| \frac{a_{i + 1}}{a_i} \right|
\end{equation*}
und ist $q < 1$, so konvergiert die Reihe. Ist $q > 1$, so divergiert die Reihe.
Ist $q = 1$ kann keine Aussage gemacht werden.

\subsubsection{Wurzelkriterium}
Es sei eine Reihe $\sum_{i=1}^{\infty} a_i$
\begin{equation*}
	q = \lim_{i \rightarrow \infty} \sqrt[i]{|a_i|}
\end{equation*}
und ist $q < 1$, so konvergiert die Reihe. Ist $q > 1$, so divergiert die Reihe.
Ist $q = 1$ kann keine Aussage gemacht werden.

\subsubsection{Leibniz-Kriterium}
Sei $\{u_i\}$ eine Folge von Zahlen, die entweder alle positiv oder negativ sind,
dann nennt man die Reihe
\begin{equation*}
	\sum_{i=1}^{\infty} (-1)^i u_i	
\end{equation*}
ein alternierende Reihe.\\

Für alternierende Reihen gilt das Leibniz-Kriterium: Konvergiert die Folge $\{u_i\}$ streng monoton gegen 0,
so konvergiert die Reihe ($u_1 > u_2 > \dots > u_i$)

\subsection{Wichtige Reihen}

\begin{align*}
	\sum_{i=1}^{\infty} \frac{1}{i}& = \frac{1}{1} + \frac{1}{2} + \dots + \frac{1}{i} \mbox{ (harmonische Reihe, divergiert)}\\
	\sum_{i=1}^{\infty} (-1)^{i-1}\frac{1}{i} & = 1 - \frac{1}{2} + \frac{1}{3} - \frac{1}{4} + \dots - \frac{1}{i} = \ln 2\\
	\sum_{i=1}^{\infty} a q^{i-1}& = a + aq + aq^2 + \dots + aq^i \mbox{ geometrische Reihe} (q > 1)\\
	\sum_{i=1}^{\infty} a q^{i-1}& = a + aq + aq^2 + \dots + aq^i = \frac{a}{1 - q} \mbox{ für } (|q| < 1) \\
	\sum_{i=0}^{\infty} \frac{1}{i!}& = \frac{1}{0!} + \frac{1}{1!} + \frac{1}{2!} + \dots + \frac{1}{i!} = \mathrm e \\
	\sum_{i=1}^{\infty} (-1)^{i-1}\frac{1}{2i - 1}& = 1 - \frac{1}{3} + \frac{1}{5} + \dots = \frac{\pi}{4}\\
	\sum_{i=1}^{\infty} \frac{1}{i^2}& = 1 + \frac{1}{2^2} + \frac{1}{3^2} + \dots  = \frac{\pi^2}{6}\\
	\sum_{i=1}^{\infty} (-1)^{i-1}\frac{1}{i^2}& = 1 - \frac{1}{2^2} + \frac{1}{3^2} - \dots = \frac{\pi^2}{12}\\	
	\sum_{i=1}^{\infty} \frac{1}{i \cdot (i + 1)}& = \frac{1}{2} + \frac{1}{6} + \frac{1}{12} + \dots = 1	
\end{align*}
Für die Eulersche Zahl gilt, das $0! = 1$

\subsection{Potenzreihen}
Unter einer Potenzreihe versteht man eine unendliche Reihe vom Typ
\begin{equation*}
	P(x) = \sum_{n=0}^{\infty} a_n \cdot (x - x_0)^n = a_0 + a_1 \cdot (x - x_0)^1 + a_2 \cdot (x - x_0)^2 + \dots + a_n \cdot (x - x_0)^n
\end{equation*}
Die Stelle $x_0$ heisst Entwicklungspunkt oder auch Entwicklungszentrum.
Die reellen Zahlen $a_0, a_1, a_2, \dots$ heissen Koeffizienten der Potenzreihe.

\subsubsection{Konvergenzbereich}
Die Menge aller $x$-Werte, für eine Potenzreihe konvergiert heisst Konvergenzbereich.

Zu jeder Potenzreihe gibt es ene positive Zahl $r$, Konvergenzradius genannt, mit folgenden Eigenschaften:
\begin{enumerate}\itemsep0em
	\item Die Potenzreihe konvergiert überall im Intervall $|x| < r$
	\item Die Potenzreihe divergiert dagegen für $|x| > r$.
	\item Über das Verhalten in $|x| = r$ lassen sich keine allgemeinen Aussagen machen $\Rightarrow$ explizit betrachten.
\end{enumerate}

Falls für alle Koeffizienten gilt $a_n \neq 0$ und der ein Grenzwert für $a_n$ vorhanden ist, lässt sich
der Konvergenzradius $r$ wie folgt berechnen:
\begin{align*}
	r& = \lim_{n \rightarrow \infty} \left|\frac{a_n}{a_{n+1}}\right| &
	r& = \frac{1}{\lim_{n \rightarrow \infty} \sqrt[n]{|a_n|}}
\end{align*}

\begin{itemize}\itemsep0em
	\item Für $x = 0$ konvergiert jede Potenzreihe und besitzt dort den Summenwert $P(0) = a_0$\\
	\item Es gibt Potenzreihen, die nur für $x = 0$ konvergieren
	\item Es gibt Potenzreihen, die für jedes $x \in \mathbb{R}$ konvergieren
	\item Allgemein konvergiert eine Potenzreihe in einem zum Nullpunkt symmetrischen Intervall $r$
\end{itemize}

