\section{Potenzreihenentwicklung}
\subsection{Taylorsche Reihe}
Die Taylorsche Reihe ist hilfreich um komplexe Funktionen in Polynome zu verwandeln. Je höher der Grad des Polynoms, desto stärker wird die Funktion angenähert.
\begin{align*}
	f(x)& = \frac{f(x_0)}{0!} + \frac{f'(x_0)}{1!}(x - x_0)^1 + \frac{f''(x_0)}{2!} (x - x_0)^2 +  \dots\\
	& = \sum_{n=0}^\infty \frac{f^{(n)} (x_0)}{n!} (x - x_0)^n
\end{align*}

Wobei $x_0$ als Entwicklungspunkt bzw. als Entwicklungszentrum betrachtet wird.

\subsection{Mac Laurinsche Reihe}
Die Mac Laurinsche Reihe ist ein Spezialfall der Taylor Reihe im Entwicklungspunkt $x_0 = 0$:
\begin{equation*}
	f(x) = f(0) + \frac{f'(0)}{1!} x + \frac{f''(0)}{2!} x^2 + \dots = \sum_{n=0}^\infty \frac{f^{(n)}(0)}{n!} x^n
\end{equation*}

