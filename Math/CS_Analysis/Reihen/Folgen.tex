\section{Folgen}
\begin{description}\itemsep0em
	\item [Folge] Sei $\N$ die Menge der natürlichen Zahlen und $A$ eine nicht leere Menge.
	Ein Folge entsteht, indem man jedem Element $n \in \N$ ein Element $a$ von $A$ zuordnet; man schreibt
	dann für diese Zuordnung:
	\begin{equation*}
		n \mapsto a_n
	\end{equation*}
	Die entstande Folge wird selbst mit
	\begin{equation*}
		\{a_n\}_{n \in \N} \mbox{ oder einfach mit } \{a_n\} \mbox {bezeichnet}
	\end{equation*}

	\item[Obere Schranke] Gibt es eine reele Zahl $K_O$ so, dass
	\begin{equation*}
		a_n \leq K_O \mbox{ für alle } n \in \N
	\end{equation*}
	gilt, so ist die Folge $\{a_n\}$ nach oben beschränkt. Man nennt $K_O$ die obere Schranke der Folge.

	\item[Untere Schranke] Gibt es eine reele Zahl $K_U$ so, dass
	\begin{equation*}
		a_n \geq K_U \mbox{ für alle } n \in \N
	\end{equation*}
	gilt, so ist die Folge $\{a_n\}$ nach unten beschränkt. Man nennt $K_U$ die untere Schranke der Folge.

	\item[Beschränkt] falls eine Folge sowohl nach oben, wie auch nach unten beschränkt ist.
\end{description}

\subsection{Monotonie}
\settowidth{\MyLenA}{Streng monoton wachsend~~}
\begin{tabular}{@{}p{\the\MyLenA}%
				@{}p{\linewidth - \the\MyLenA}}
Monoton steigend & $a_n \leq a_{n + 1}$ für alle $n \in \N$ \\
Streng monoton steigend & $a_n < a_{n + 1}$ für alle $n \in \N$\\
Monoton fallend & $a_n \geq a_{n + 1}$ für alle $n \in \N$ \\
Streng monoton fallend & $a_n > a_{n + 1}$ für alle $n \in \N$\\
\end{tabular}

Eine monoton steigende Folge mit der Indexmenge $\N$ ist immer nach unten beschränkt.
Die untere Schranke ist $a_1$.\\
Eine monoton fallende Folge mit der Indexmenge $\N$ ist immer nach oben beschränkt.
Die obere Schranke ist $a_1$

\subsection{Konvergenz}
Es sei $\{a_n\}_{n \in \N}$ eine reelle Folge und $a$ eine reelle Zahl. Man sagt,
die Folge konvergiert gegen den Grenzwert $a$, wenn für jede beliebige reelle Zahl
$\epsilon > 0$ ein Index $n_0$ existiert, so dass gilt:
\begin{equation*}
	|a_n - a| < \epsilon \mbox{ für alle } n \geq n_0
\end{equation*}
Man schreibt dann
\begin{equation*}
	a = \lim_{n \rightarrow \infty} a_n
\end{equation*}
oder auch
\begin{equation*}
	a_n \rightarrow a \mbox{ für } n \rightarrow \infty
\end{equation*}

\subsection{Rechenregeln}
Es seien $\{a_n\}$ eine konvergierende Folge mit dem Grenzwert $a$ und 
$\{b_n\}$ eine konvergierende Folge mit dem Grenzwert $b$. Dann gilt:

\settowidth{\MyLenA}{Multiplikation~~}
\begin{tabular}{@{}p{\the\MyLenA}%
				@{}p{\linewidth - \the\MyLenA}}
Addition & Die Folge $\{a_n + b_n\}$ konvergiert gegen $a + b$ \\
Subtraktion & Die Folge $\{a_n - b_n\}$ konvergiert gegen $a - b$ \\
Multiplikation & Die Folge $\{a_n \cdot b_n\}$ konvergiert gegen $a \cdot b$ \\
Division & Die Folge $\{\frac{a_n}{b_n}\}$ konvergiert gegen $\frac{a}{b}$ \\
\end{tabular}

Nach oben beschränkte, monoton steigende Folgen konvergieren.\\
Nach unten beschränkte, monoton fallende Folgen konvergieren.\\
Jede konvergente Folge ist beschränkt.\\
Der Grenzwert einer konvergenten Folge ist eindeutig bestimmt:
Jede Folge hat höchstens einen Grenzwert.