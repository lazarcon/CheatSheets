\section{Integrationsmethoden}
\subsection{Substitution}
\begin{equation*}
	\int f(x) \dop{x} = ?
\end{equation*}

\begin{enumerate}\itemsep0em
	\item Aufstellen der Substitutionsgleichungen:
	\begin{equation*}
		u = g(x) \rightarrow \frac{\dop{u}}{\dop{x}} = g'(x) \rightarrow \dop{x} = \frac{\dop{u}}{g'(x)}
	\end{equation*}
	\item Durchführen der Integralsubstitution durch Einsetzen der Substitutionsgleichungen in das vorgegebene Integral:
	\begin{equation*}
		\int f(x) \dop{x} = \int \varphi(u) \dop{u}
	\end{equation*}
	Das neue Integral enthält nur noch die Hilfsvariable $u$ und deren Differential $\dop{u}$. Der Integrand
	ist nur noch eine von $u$ abhängige Funktion $\varphi(u)$
	\item Integration (Berechnung des neuen Integrals)
	\begin{equation*}
		\int \varphi(u) \dop{u} = \Phi(u)
	\end{equation*}
	\item Rücksubsitution (mittels $u = g(x)$)
	\begin{equation*}
		\int f(x) \dop{x} = \Phi(u) = \Phi(g(x)) = F(x)
	\end{equation*}
\end{enumerate}
\begin{itemize}
	\item Die Funktion muss stetig differenzier- und umkehrbar sein.
	\item Die Substitution muss zu einer Vereinfachung führen
	\item Nach einsetzen der Substitutionsgleichung darf $x$ im Integral nicht mehr vorkommen
	\item Bei Wurzelausdrücken ist eine Substitutionsgleichung vom Typ $x = h(u)$ günstiger
	\item Bei bestimmten Integralen kann auf die Rücksubsitution verzichtet werden. 
	Dafür sind die Integrationsgrenzen mit $u = g(x)$ bzw. $x = h(u)$ zu berechnen.
\end{itemize}

\subsubsection{Beispiel mit $u = g(x)$}
\begin{align*}
	\lint{0}{1} x \cdot \sqrt{1 + x^2} \dop{x}& = ?\\
	u &= 1 + x^2 \rightarrow \frac{\dop{u}}{\dop{x}} = 2\cdot x \rightarrow \dop{x} = \frac{\dop{u}}{2 \cdot x}\\
	\mbox{Untergrenze: }& x = 0 \Rightarrow u = 1 + (0)^2 = 1\\
	\mbox{Obergrenze: }& x = 1 \Rightarrow u = 1 + (1)^2 = 2\\
	\lint{0}{1} x \cdot \sqrt{1 + x^2} \dop{x}& = \lint{u=1}{u=2} x \sqrt{u} \frac{\dop{u}}{2 \cdot x}\\
	& = \frac{1}{2} \cdot \lint{1}{2} \sqrt{u} \dop{u} = \frac{1}{2} \lint{1}{2} u^{\frac{1}{2}} \dop{u}\\
	& = \frac{1}{2} \left[ \frac{u^{\frac{3}{2}}}{\frac{3}{2}} \right]_1^2 = \frac{1}{3} \left[ \sqrt{u^3} \right]_1^2\\
	& = \frac{1}{3} (\sqrt{8} - \sqrt{1}) \approx 0{,}6095
\end{align*}

\subsection{Integralsubstitutionen}
\subsubsection{Typ A}
\begin{equation*}
	\int f(a \cdot x + b) \dop{x} = \frac{1}{a} \int f(u) \dop{u}
\end{equation*}
Subsitution: $u = a \dot x + b \rightarrow \dop{x} = \frac{\dop{u}}{a}$\\
Beispiel: $\int \sqrt{4x + 5} \dop{x}; u = 4x + 5$

\subsubsection{Typ B}
\begin{equation*}
	\int f(x) \cdot f'(x) \dop{x} = \frac{1}{2} (f(x))^2 + C
\end{equation*}
Substitution: $u = f(x) \rightarrow \dop{x} = \frac{\dop{u}}{f'(x)}$\\
Beispiel: $\int \sin x \cdot \cos x \dop{x}; u = \sin x$

\subsubsection{Typ C}
\begin{equation*}
	\int (f(x))^n \cdot f'(x) \dop{x} = \frac{1}{n + 1} (f(x))^{n + 1} + C
\end{equation*}
Substitution: $u = f(x) \rightarrow \dop{x} = \frac{\dop{u}}{f'(x)}$\\
Beispiel: $\int (\ln x)^2 \cdot \frac{1}{x} \dop{x}; u = \ln x$

\subsubsection{Typ D}
\begin{equation*}
	\int f(g(x)) \cdot g'(x) \dop{x} = \int f(u) \dop{u}
\end{equation*}
Substitution: $u = g(x) \rightarrow \dop{x} = \frac{\dop{u}}{g'(x)}$\\
Beispiel: $\int x \cdot \mathrm e^{x^2} \dop{x}; u = x^2$

\subsubsection{Typ E}
\begin{equation*}
	\int \frac{f'(x)}{f(x)} \dop{x} = \ln |f(x)| + C
\end{equation*}
Substitution: $u = f(x) \rightarrow \dop{x} = \frac{\dop{u}}{f'(x)}$\\
Beispiel: $\int \frac{2x - 3}{x^2 - 3x + 1} \dop{x}; u = x^2 - 3x + 1$

\subsection{Partielle (Produkt-)Integration}
\begin{equation*}
	\int f(x) \dop{x} = \int u \cdot v' \dop{x} = u \cdot v - \int u' \cdot v \dop{x}
\end{equation*}

Eine Funktion muss geschickt nach $u \cdot v'$ zerlegt werden. Die Stammfunktion von $v'$ muss sich ohne Schwierigkeiten ergeben.
Häufig muss erneut integriert oder substituiert werden.

\subsubsection{Beispiel}
\begin{align*}
	\int \overset{\downarrow}{x} \cdot \overset{\uparrow}{\mathrm e^x} \dop{x}& = ?\\
	u& = x \rightarrow u' = 1\\
	v'& = e^x \rightarrow v = e^x\\
	\int x \cdot \mathrm e^x \dop{x}& = x \cdot e^x - \int 1 \cdot e^x \dop x\\
	& = x  \cdot e^x - e^x + C = (x - 1) \cdot e^x + C
\end{align*}
