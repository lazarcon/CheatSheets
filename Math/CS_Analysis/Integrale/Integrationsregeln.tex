\section{Integrationsregeln}
\subsection{Faktorregel}
Ein konstanter Faktor darf vor das Integral gezogen werden.
\begin{equation*}
	\lint{a}{b} C \cdot f(x) \dop{x} = C \cdot \lint{a}{b} f(x) \dop{x}
\end{equation*}

\subsection{Summenregel}
Eine endliche Summe von Funktionen darf gliedweise integriert werden.
\begin{equation*}
	\lint{a}{b} (f_1(x) + \dots + f_n(x)) \dop{x} = \lint{a}{b} f_1(x) \dop{x} + \dots + \lint{a}{b} f_n(x) \dop{x}
\end{equation*}

\subsection{Vertauschungsregel}
Vertauschen der Integrationsgrenzen bewirkt einen Vorzeichenwechsel.
\begin{equation*}
	\lint{a}{b} f(x) \dop{x} = - \lint{b}{a} f(x) \dop{x}
\end{equation*}

\subsection{Gleiche Intervallgrenzen}
Fallen die Integrationsgrenzen zusammen ($a = b$), so ist der Integralwert gleich Null.
\begin{equation*}
	\lint{a}{a} f(x) \dop{x} = 0
\end{equation*}

\subsection{Zerlegen des Integrationsintervalls}
Für jede Stelle $c$ aus dem Integrationsinterval $a \leq c \leq b$ gilt:
\begin{equation*}
	\lint{a}{b} f(x) \dop{x} = \lint{a}{c} f(x) \dop{x} + \lint{c}{b} f(x) \dop{x}
\end{equation*}
