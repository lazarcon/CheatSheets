\section{Allgemein}
Die Integration ist die Umkehrung der Ableitung.
\begin{equation*}
	y = f(x) \xrightarrow{\mbox{Differentiation}} y' = f'(x) \xrightarrow{\mbox{Integration}} y = f(x) 
\end{equation*}

\subsection{Stammfunktion}
Es sei $f(x)$ eine auf dem Intervall $[a, b]$ definierte Funktion. Eine Funktion $F(x)$ heisst Stammfunktion
von $f(x)$ falls für alle $x \in [a, b]$ gilt: $F'(x) = f(x)$.\\
Eigenschaften:
\begin{enumerate}\itemsep0em
	\item Hat eine stetige Funktion $f(x)$ mindestens eine Stammfunktion, so hat sie unendliche viele Stammfunktionen.
	\item Zwei beliebige Stammfunktione $F_1(x) und F_2(x)$ unterscheiden sich durch eine additive Konstante $C$. ($F_1(x) - F_2(x) = $ konstant.
	\item Ist $F_1(x)$ eine beliebige Stammfunktion von $f(x)$, so ist auch $F_1(x) + C$ eine Stammfunktion von $f(x)$.
	Die allgemeine Stammfunktion ist: $F(x) = F_1(x) + C$, wobei $C$ eine beliebige reelle Konstante ist.
\end{enumerate}

\subsection{Flächeninhalt (bestimmtes Integral)}
Um die Fläche $A$ unterhalb einer Funktion $f(x)$ zu berechnen gilt folgendes Vorgehen:
\begin{enumerate}\itemsep0em
	\item Fläche in $n$ Streifen teilen
	\item Alle Streifenflächen berechnen
	\item Flächen aufsummieren
\end{enumerate}

In der Theorie wird eine Fläche in Rechtecke zerlegt, Untersumme ($U_n$) und Obersumme ($O_n$) berechnet. Die Fläche liegt zwischen diesen beiden Werten.
\begin{align*}
	U_n& = \sum_{k=1}^n f(x_{k-1}) \cdot \Delta x_k &   
	O_n& = \sum_{k=1}^n f(x_{k}) \cdot \Delta x_k\\
\end{align*}
\begin{equation*}
	A = \lim_{n \rightarrow \infty} U_n = \lim_{n \rightarrow \infty} O_n = \lim_{n \rightarrow \infty} \sum_{k=1}^n f(x_k) \cdot \Delta x_k = \int\limits_a^b f(x)\dop{x} = \int\limits_{x=a}^{x=b} \dop{A}
\end{equation*}

Das bestimmte Integral ist eine Zahl, die der Fläche entspricht.

\subsection{Flächenfunktion (unbestimmtes Integral)}
\begin{equation*}
	I(x) = \int\limits_a^x f(t)\dop{t}
\end{equation*}
Die obere Intervall Grenze wird offengelassen. Das unbestimmte Integral ist eine Funktion. Eigenschaften:
\begin{enumerate}\itemsep0em
	\item Das unbestimmte Integral $I(x) = \int_a^x f(t)\dop{t}$ repräsentiert den Flächeninhalt zwischen $y = f(t)$ und der $t$-Achse im Intervall $a \leq t \leq x$ in Abhängigkeit
	von der oberen Grenze $x$.
	\item Zu jeder stetigen Funktion $f(t)$ gibt es unendliche viele unbestimmte Integrale, die sich in ihrer unteren Grenze voneinander unterscheiden.
	\item Die Differenz zweier unbestimmter Integrale $I_1(x)$ und $I_2(x)$ von $f(t)$ ist eine Konstante.
\end{enumerate}
