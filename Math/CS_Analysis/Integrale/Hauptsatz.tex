\subsection{Fundamentalsatz}
Jedes unbestimmte Integral $I(x) = \int\limits_{a}^{x} f(x) \dop{x}$ der stetigen Funktion $f(x)$ ist eine Stammfunktion von $f(x)$:
\begin{equation*}
	I(x) = \int\limits_{a}^{x} f(x) \dop{x} \Rightarrow I'(x) = f(x)
\end{equation*}
heisst: Die Ableitung jedes unbestimmten Integrals ergibt die Integrandfunktion. Jeds unbestimmte Integral einer Funktion ist die Menge
aller Stammfunktionen.

\begin{itemize}\itemsep0em
	\item $I(x)$ ist eine stetig differenzierbare Funktion.
	\item Jedes unbestimmte Integral lässt sich schreiben als:
	\begin{equation*}
		I(x) = \lint{a}{x} f(x) \dop{x} = F(x) + C
	\end{equation*}
	\item Die Funktionenschar aller unbestimmter Integrale eine Funktion $f(x)$ schreibt man als
	\begin{equation*}
		\int f(x) \dop{x} = F(x) + C
	\end{equation*}
	wobei $F(x)$ eine beliebige Stammfunktion ist.
	\item Für stetige Funktionen sind die Begriffe \enquote{unbestimmtes Integral} und \enquote{Stammfunktion} synonym.
\end{itemize}



