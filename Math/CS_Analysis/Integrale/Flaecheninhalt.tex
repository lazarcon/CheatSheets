\section{Flächeninhalt}
Die Fläche ist immer ein positiver Wert $\rightarrow$ mit Beträgen arbeiten.

\subsection{Allgemeiner Fall}
Flächen, die teils oberhalb, teils unterhalb der $x$-Achse verlaufen, müssen in Teilflächen zerlegt werden,
die entweder oberhalb oder unterhalb der $x$-Achse verlaufen:
\begin{enumerate}\itemsep0em
	\item Nullstellen im Interval $a \leq x \leq b$ bestimmen
	\item Teilflächen aufsummieren (ggf. Skizze erstellen)
\end{enumerate}

\subsection{Fläche zw. zwei Kurven (ohne Schnittpunkte)}
Gegeben seien zwei Kurven $f_1(x)$ und $f_2(x)$
\begin{equation*}
	A = \left|\lint{a}{b} (f_1(x) - f_2(x)) \dop{x}\right|
\end{equation*}

\subsection{Fläche zw. zwei Kurven (mit Schnittpunkten)}
Erst die Schnittpunkte berechnen,
dann wie ohne Schnittpunkt bis zum Schnittpunkt berechnen
und aufsummieren.