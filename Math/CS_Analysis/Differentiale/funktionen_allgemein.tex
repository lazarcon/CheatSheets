\section{Funktionen}
Eine Funktion ist eine Vorschrift, die jedem Element $x$ aus eine Menge $D$ genau ein Element $y$ aus einer Menge $W$ zuordnet.
\begin{equation*}
	f\colon\, D\to W,\; x\mapsto y.	
\end{equation*}
Darstellungen:\\
1. Analytisch ($y = f(x)$ (explizit), $F(x;y) = 0$ (implizit)),\\ 2. Wertetabelle, 3. Graphisch, 
4. Parametrisch ($x = x(t)$, $y = y(t)$, Wertetabelle beginnt mit $t$)

\subsection{Funktionseigenschaften}
\subsubsection{Symmetrie}
\begin{tabular}{lcclc}
gerade:	& $f(-x) = f(x)$ & ~~~ & ungerade: & $f(-x) = - f(x)$\\
\end{tabular}

\subsubsection{Monotonie}
\settowidth{\MyLenA}{Streng monoton wachsend~~}
\begin{tabular}{@{}p{\the\MyLenA}%
				@{}p{\linewidth - \the\MyLenA}}
Monoton wachsend & $f(x_1) \leq f(x_2)$ ($x_1 < x_2$) \\
Streng monoton wachsend & $f(x_1) < f(x_2)$ ($x_1 < x_2$)\\
Monoton fallend & $f(x_1) \geq f(x_2)$ ($x_1 < x_2$) \\
Streng monoton fallend & $f(x_1) > f(x_2)$ ($x_1 < x_2$)\\
\end{tabular}

\subsubsection{Umkehrbarkeit}
Umkehrbar: $x_1 \neq x_2 \Rightarrow f(x_1) \neq f(x_2)$ (streng monton)\\
Bestimmen der Umkehrfunktion (Spiegelung an $y=x$):\\
1. $y = f(x)$ nach $x$ auflösen. Ergebnis: $x = f^{-1}(y)$.\\
2. Vertauschen von $x$ und $y$ im Ergebnis: $y = f^{-1}(x)$.\\
Definitions- und Wertebereich sind vertauscht.
\begin{equation*}
	x\mathop{\rightleftarrows}^{f}_{f^{-1}}f(x)
\end{equation*}

\subsubsection{Periodizität}
Periodisch mit Periode: $p: f(x \pm p) = f(x)$

\subsubsection{Stetigkeit}
Eine Funktion $f(x)$ heisst an der Stelle $x_0$ stetig, wenn der Grenzwert vorhanden ist und mit dem Funktionswert übereinstimmt:
\begin{equation*}
	\lim_{x\to x_0}{f(x)} = f(x_0)
\end{equation*}

Eine Funktion ist an der Stelle $x_0$ unstetig, wenn:\\
1. $f(x)$ an der Stelle $x_0$ nicht definiert ist (Definitionslücke).\\
2. An der Stelle $x_0$ kein Grenzwert vorhanden ist.\\
3. Funktions- und Grenzwert zwar vorhanden, aber verschieden sind.\\

\subsection{Grenzwert}
% Reelle Zahlenfolge: $\langle{a}\rangle = f(n)$ mit $n \in \mathbb{N}$*. $f(n)$ heisst Bildungsgesetz, die Zahlen: $a_1, a_2, \dots a_n$ heissen Glieder der Folge.\\
% Eine reelle Zahl $g$ heisst Grenzwert der Zahlenfolge $\langle{A}\rangle$, wenn es zu jedem $\epsilon > 0$ eine
% natürliche Zahl $n_0 > 0$ gibt, so dass für all $n \geq n_0$ stets gilt: $\left|a_n - g\right| < \epsilon$.

Die Funktion $f(x)$ hat an der Stelle $x_0$ einen Grenzwert $g$, wenn gilt
\begin{equation*}
	\lim_{x\to x_0\atop x < x_0}f(x) = \lim_{x\to x_0\atop x > x_0}f(x) = \lim_{x\to x_0}f(x) = g 
\end{equation*}
\textit{konvergent} = hat Grenzwert, \textit{divergent} hat keinen Grenzwert.

Lösungsschema zur Bestimmung des Grenzwerts $g = \lim_{x\to x_0} f(x)$:\\
1. Grundsätzlich $x_0$ in $f(x)$ einsetzen. Wenn $f(x_0)$ definiert ist: $g = \lim_{x\to x_0} f(x) = f(x_0)$.\\
2. Falls $f(x_0)$ nicht definiert ist, $f(x)$ vereinfachen.\\
3. Falls das nicht geht, den links und rechtsseitigen Grenzwert durch annähern von links und rechts ermitteln.\\
Polstelle: Der Grenzwert ist $+\infty$ oder $-\infty$.\\

\subsubsection{Rechenregeln}
\begin{align*}
\lim_{x\to x_0}{(k \cdot f(x))}& = k (\lim_{x\to x_0}{f(x)})\\
\lim_{x\to x_0}{(f(x) \pm g(x))}& = (\lim_{x\to x_0}{f(x)}) \pm (\lim_{x\to x_0}{g(x)})\\
\lim_{x\to x_0}{(f(x) \cdot g(x))}& = (\lim_{x\to x_0}{f(x)}) \cdot (\lim_{x\to x_0}{g(x)})\\
\lim_{x\to x_0}{(\frac{f(x)}{g(x)})}& = \frac{\lim_{x\to x_0}f(x)}{\lim_{x\to x_0}g(x)}\\
\end{align*}

