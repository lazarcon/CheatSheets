\section{Polynomfunktionen}
Allgemein: $f(x) = a_n \cdot x^n + a_{n-1} \cdot x^{n-1} + \dots + a_1 \cdot x^1 + a_0$\\
Der Grad des Polynoms ist $n$. Es gibt $n$ Nullstellen.

\subsection{Nullstellen-Formeln}
\settowidth{\MyLenA}{Biquadratisch~~}
\settowidth{\MyLenB}{$ax^3 + bx^2 + cx = 0$~~}
\begin{tabular}{@{}p{\the\MyLenA}%
				@{}p{\the\MyLenB}
				@{}p{\linewidth - \the\MyLenA - \the\MyLenB}}
Linear & $ax + b = 0$ & $x = - \frac{b}{a}$\\
Quadratisch & $ax^2 + bx + c = 0$ & $x_{1,2} = \frac{-b \pm \sqrt{b^{2} - 4ac}}{2a}$\\
Kubisch & $ax^3 + bx^2 + cx = 0$ & $x(ax^2 + bx + c) = 0$ mit $x_1 = 0$\\ 
Biquadratisch & $ax^4 + bx^2 + c = 0$ & $y = x^2 \Rightarrow ay^2 + by + c = 0$\\ 
\end{tabular}

\subsection{Geraden (erster Grad)}
Es sei $m$ die Steigung, $a$ der x- und $b$ der y-Achsenabschnitt. 
\settowidth{\MyLenA}{Achsenabschnittsform~~}
\settowidth{\MyLenB}{$\frac{y - y_1}{x - x_1} = \frac{y_2 - y_1}{x_2 - x_1}$~~}
\begin{tabular}{@{}p{\the\MyLenA}%
				@{}p{\the\MyLenB}
				@{}p{\linewidth - \the\MyLenA - \the\MyLenB}}
y-Achse, Steigung & $y = mx + b$ & \\
Achsenabschnittsform & $\frac{x}{a} + \frac{y}{b} = 1$ & \\
Punkt-Steigung & $\frac{y - y_1}{x - x_1} = m$ & Durch $P(x_1;y_1)$\\
Zwei-Punkte-Form & $\frac{y - y_1}{x - x_1} = \frac{y_2 - y_1}{x_2 - x_1}$ & Durch $P_1(x_1;y_1)$, $P_2(x_2;y_2)$\\
\end{tabular}

\subsection{Parabeln (zweiter Grad)}
Es sei $S$ der Scheitelpunkt.

\settowidth{\MyLenA}{Scheitelpunktsform~~}
\settowidth{\MyLenB}{$y = a(x - x_1)(x - x_2)$~~}
\begin{tabular}{@{}p{\the\MyLenA}%
				@{}p{\the\MyLenB}
				@{}p{\linewidth - \the\MyLenA - \the\MyLenB}}
Hauptform & $y = ax^2 + bx + c$ & $S = (-\frac{b}{2a}; \frac{4ac - b^2}{4a})$\\
Produktform & $y = a(x - x_1)(x - x_2)$ & $x_1$, $x_2$ sind Nullstellen\\
Scheitelpunktsform & $y - y_0 = a(x - x_0)^2$ & $S = (x_0;y_0)$\\
\end{tabular}

\subsection{Höhere Grade}
Besitzt eine Polynomfunktion $f(x)$ vom Grad $n$ an der Stelle $x_n$ eine Nullstelle,
so lässt sie sich schreiben als: $f(x) = (x - x_n) \cdot f_1(x)$.\\
$(x - x_n)$ heisst Linearfaktor, $f_1(x)$ heisst reduziertes Polynom vom Grad $n - 1$.\\
Besitzt eine Polynom vom Grad $n$ genau $n$ Nullstellen, so lässt es sich schreiben als:
\begin{equation*}
	f(x) = a_n(x - x_1)(x - x_2) \dots (x - x_n)
\end{equation*}

1. Das reduzierte Polynom erhält man durch das Horner-Schema.\\
2. Polynome solange reduzieren (raten weiterer Nullstellen) bis man auf eine Polynomfunktion zweiten Grades stösst, 
deren Nullstellen sich durch lösen der quadratischen Gleichung ergeben.

\subsubsection{Horner-Schema}
% 1. Koeffizienten in Reihenfolge notieren ($a_n, a_{n-1}, \dots, a_0$).\\
% 2. Denn ersten Koeffizienten ($a_n$) in die zweite Spalte der ersten Zeile schreiben.\\
% 3. Denn Wert der dritten Zeile mit der Nullstelle multiplizieren und in die nächste Spalte der zweiten Zeile schreiben.\\
% 4. Die Summe der ersten beiden Zeilen der nächsten Spalten bilden und in der dritten Zeile notieren.\\
% 5. Schritte 3 und 4 wiederholen, bis am Ende der Tabelle angelangt. \\

Gegeben: $y = 3 x^3 + 18 x^2 + 9 x - 30 = 3 (x^3 + 6 x^2 + 3 x - 10)$\\
Durch raten findet man eine Nullstelle bei $x = 1$ ($1 + 6 + 3 - 10 = 0$). 
\begin{tabular}{c|c|c|c|c}
			& $a_3 = 1$ 	& $a_2 = 6$ 			& $a_1 = 3$ 			& $a_0 = - 10$ 			\\ \hline
$x_0 = 1$ 	& 				& $a_3 \cdot x_0 = 1$	& $7 \cdot x_0 = 7$ 	& $10 \cdot x_0 = 15$	\\ \hline
			& $a_3 = 1$		& $6 + 1 = 7$			& $3 + 7 = 10$			& $-10 + 10 = 0$		\\ \hline
\end{tabular}

Umgeformt: $y = 3 (x - 1)(x^2 + 7 x + 10) \Rightarrow y = 3 (x - 1)(x + 2)(x + 5)$.\\

\subsection{Gebrochenrationale Funktionen}
Funktionen, die sich als Quotient zweier Polynomfunktionen $g(x)$ und $h(x)$ darstellen lassen heissen gebrochenrationale Funktionen: $f(x) = \frac{g(x)}{h(x)}$
Diese Funktionen sind echt gebrochen, wenn der Grad von $g(x)$ kleiner ist als der Grad von $h(x)$.
Sie werden mit Hilfe der Polynom-Division gelöst.

Nullstellen: $x_0: g(x_0) = 0$ und $h(x_0) \neq 0$.\\
Definitionslücke: Alle Stellen wo $h(x_0) = 0$.\\
Bestimmen der Null- und Polstellen:\\
1. Zähler- und Nennerpolynom in Linearfaktoren zerlegen.\\
2. die Zähler Linearfaktoren sind die Nullstellen,\\ 
3. die Nenner Linearfaktoren sind die Polstellen.\\

\subsection{Kreis und Ellipse}
Kreisgleichung (Mittelpunkt $M = (x_0; y_0)$, Radius $r$):\\
$(x - x_0)^2 + (y - y_0)^2 = r^2 $ oder $y = y_0 \pm \sqrt{r^2 - (x - x_0)^2}$\\

Ellipsengleichung (Mittelunkt $M = (x_0; y_0)$, x-Halbachse $a$, y-Halbachse $b$):\\ 
%$M$: Mittelpunkt; $F_1, F_2$: Brennpunkte, $a$: grosse, $b$: kleine Halbachse, $e$: Brennweite\\
$\frac{(x - x_0)^2}{a^2} + \frac{(y - y_0)^2}{b^2} = 1$ oder $y = y_0 \pm \frac{b}{a} \sqrt{a^2 - (x - x_0)^2}$
