\section{Differentialrechnung}
Berechnet die Steigung der Kurventangente an der Stelle $x_0$.\\
Voraussetzungen:
\begin{equation*}
	\lim_{\Delta x \to 0} \frac{\Delta y}{\Delta x} = \lim_{\Delta x \to 0} \frac{f(x_0 + \Delta x) - f(x_0)}{\Delta x}
\end{equation*}
und linksseitiger Grenzwert = rechtsseitiger Grenzwert.
Dann:
\begin{align*}
	m& = \tan \alpha = \frac{\Delta y}{\Delta x} = \frac{y_2 - y_1}{x_2 - x_1}\\
	\alpha& = \arctan m = \arctan \frac{\Delta y}{\Delta x}
\end{align*}
Eine Funktion ist differenzierbar wenn:
Stetigkeit $\nRightarrow$ diff.-bar, diff.-bar $\Rightarrow$ Stetigkeit, unstetig $\Rightarrow$ undiff.-bar

\subsection{Ableitungsregeln}
Ableitungen zusammengesetzter Funktionen, z.B. $y = \sin(2x)$ oder $ y = x^2 \cdot \mathrm e^{-x^2} $ auf elementare  
Ableitungen zurückführen.

Seien $f(x), g(x)$  und $h(x)$ (im Definitionsbereich) differenzierbare, reelle Funktionen,  und $a, b$ reelle Zahlen, dann gelten:
 
\settowidth{\MyLenA}{Konstante Funktion~~}
\begin{tabular}
	{
		@{}p{\the\MyLenA}%
		@{}p{\linewidth - \the\MyLenA}%
	}
	Konstante Funktion 	& $(a)' = 0$\\
	Faktorregel		& $(a \cdot f(x))' = a \cdot f'(x)$\\
	Summenregel		& $(f(x) \pm g(x))' = f'(x) \pm g'(x)$\\
	Produktregel		& $(f(x) \cdot g(x))' = f'(x) \cdot g(x) + f(x) \cdot g'(x)$\\
	Quotientenregel		& $(\frac{f(x)}{g(x)})' = \frac{f'(x)\cdot g(x) - f(x) \cdot g'(x)}{(g(x))^2}$\\
	Potenzregel		& $(x^n)' = nx^{n-1}$\\
	Kettenregel		& $(f(g(x)))' = (f \circ g)'(x) = f'(g(x)) \cdot g'(x)$\\
	Logarithmisch		& $f'(x) = (g(x)^{h(x)}) = f(x) \cdot (h'(x) \cdot \ln (g(x)) + h(x) \cdot \frac{g'(x)}{g(x)})$\\
\end{tabular}
Die Kettenregel ist im wesentlichen äussere Ableitung mal innere Ableitung. Beispiel:
\begin{align*}
	f:x \rightarrow f(x)& = (x^2 + 4)^3\\
	u:x \rightarrow u(x)& = x^2 + 4 \rightarrow u'(x) = 2x\\
	v:u \rightarrow v(u)& = u^3 \rightarrow v'(u) = 3u^2\\
	f(x)& = (v \circ u)(x) = v(u(x)) \rightarrow f'(x) = 3(x^2 + 4)^2 \cdot 2x
\end{align*}

\subsubsection{Ableitung Umkehrfunktion}
1. Umkehrfunktion bestimmen: $y = f(x) \Rightarrow x = g(y)$\\
2. $g'(y) = \frac{1}{f'(x)}$\\
3. Mit Hilfe von $y = f(x)$ $g'(y)$ als Funktion von $y$ schreiben\\
4. $x$ und $y$ in $g'(y)$ vertauschhen\\

\subsubsection{Ableitung in Parameterform}
$(x = x(t), y = y(t))' \Rightarrow y' = \frac{y'(t)}{x'(t)} = \frac{\dot{y}}{\dot{x}}$

\subsection{Differential}
$\mathrm dy = \mathrm df = f'(x_0) \cdot \mathrm dx$: Zuwachs der Ordinate an der Stelle $x_0$ bei Änderung von $x$ um $\mathrm dx$.

\subsection{Tangente und Normale}
\begin{align*}
	y_T& = f'(x_0)(x - x_0) + y_0 & \mbox{Tangente}\\
	y_N& = \frac{1}{f'(x)} \cdot (x - x_0) + y_0 & \mbox{Normale}
\end{align*}

\subsection{Linearisierung}
In der Umgebung von $P(x_0, y_0)$ gilt $\Delta y = f'(x_0) \Delta x$.

\subsection{Monotonie}
$y' = f'(x) > 0 \Rightarrow \mbox{streng monoton wachsend}$\\
$y' = f'(x) < 0 \Rightarrow \mbox{streng monoton fallend}$\\

\subsubsection{Krümmung}
Linkskrümmung: $y'' = f''(x_0) > 0$\\
Rechtskrümmung: $y'' = f''(x_0) < 0$\\
 
