\section{Differential mit mehreren Variablen}
Eine Funktion von $n$ unabhängigen Variablen ist eine Vorschrift, die jedem geordneten Zahlenpaar $(x_1; x_2; \dots; x_n)$ aus einer Definitionsmenge $D$ genau ein Element $y$ aus einem Wertebereich $W$ zuordnet: $y = f(x_1; x_2; \dots; x_n)$.
\subsection{Darstellungen}
\begin{description}\itemsep0em
	\item [Explizit] $y = f(x_1; x_2; \dots; x_n)$
	\item [Implizit] $F(x_1; x_2; \dots; x_n; y) = 0$
	\item Funktionstabelle: Bei zwei unabhängigen gibt es eine Matrix ($x_1$-Werte in den Zeilen, $x_2$-Werte in den Spalten). Mit mehreren unabhängigen kommen weiteren Tabellen
	für $x_3$ usw. dazu.
	\item [Graphisch]
		\begin{itemize}\itemsep0em
			\item Fläche (3d) Geht nur mit zwei unabhängigen Variablen.
			\item Schnittkurvendiagramm (2d) (mit zwei unabhängigen Variablen): $f(x_1; x_2) = $ Konstant
		\end{itemize}
\end{description}
 
\subsection{Grenzwert}
Eine Funktion zweier Variablen hat an der Stelle $(x_0, y_0)$ den Grenzwert $g$, wenn sich die Funktionswerte von $f(x, y)$ beim Grenzübergang $(x, y) \rightarrow (x_0, y_0)$ unabhängig vom Weg dem Grenzwert $g$ beliebig annähern.
\begin{equation*}
	g = \lim_{(x, y) \rightarrow (x_0, y_0)} f(x, y)
\end{equation*}
Lösungswege: Man nähere sich $(x_0, y_0)$ entlang einer geraden $y = mx$. So kann man in die Funktionsgleichung für jedes $y$ den Wert $mx$ einsetzen. Ergibt sich dann ein konstanter Wert, hat die Funktion einen Grenzwert. Falls das Ergebnis noch von $m$ abhängt, hat sie keinen Grenzwert.

\subsection{Stetigkeit}
Eine Funktion ist an einer Stelle stetig, wenn der Grenzwert vorhanden ist und mit dem Funktionswert übereinstimmt.

Funktionen, die an jeder Stelle des Definitionsbereich stetig sind, heissen stetige Funktionen.
