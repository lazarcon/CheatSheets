\section{Datenmodellierung}
Daten- und funktionale Anforderungen definieren. Anschliessend auf die Daten Abstraktionskonzepte anwenden:
Klassifikation (gleiche oder ähnliche \enquote{Dinge}), Aggregation (Zusammenfassen von \enquote{Dingen}, Name, Strasse, \dots zu Adresse),
Generalisierung und Spezialisierung (Verallgemeinerung, Teilmengenbeziehungen, Zerlegung in disjunkte Untermengen).

Bei der Modellierung gilt es sich auf den Anwendungsbereich zu beschränken. Die Modellierung muss unabhängig von der 
späteren Implementierung sein.

\subsection{Entity-Relationship-Model (P. P. S. Chen, 1976)}
\begin{description}
	\item [Entität] Ein  konkretes  oder  abstraktes  Objekt  der  realen  oder  einer  
Vorstellungswelt, welches eindeutig identifiziert werden kann. Eine Entität hat Attribute (Bsp. Herr Aebi). 
	\item [Entitätsmenge] Gleichartige Entitäten, d.h. Entitäten mit denselben Eigenschaften (Bsp. Personen).
	\item [Identifikationsschlüssel] Ein Attribut oder eine Menge von Attributen, welche eine Entität eines
Entitätstyps innerhalb dieses Typs eindeutig identifiziert (Bsp. Personalnummer)
	\item [Beziehung] Eine Beziehung zwischen Entitäten $e_1, e_2, \dots, e_n (n \geq 2)$ ist ein $n$-
Tupel, wobei jedes $e_i$ eine Entität eines Entitätstyps $E_i (1 \leq i \leq n)$ ist. Wenn derselbe Entitätstyp mehrfach 
vorkommt ($E_i$ und $E_j$ mit $i \neq j$), dann müssen deren verschiedene Rollen mit Hilfe zusätzlicher
Rollennamen bezeichnet werden.\\
	Gleichartige Beziehungen werden zu einem Beziehungstyp zusammengefasst (Bsp. Person als Autor eines Buches)
	Identifikationsattribute einer Entität, die an einer Beziehung beteiligt ist, gehört implizit zur Beziehung.
	Eine Beziehung kann darüber hinaus weitere Attribute haben. Der Identifikationschlüssel einer Beziehung wird
	aus Identifikationsattributen, der an der Beziehung beteiligten Entitätstypen gebildet.
	\item[Generalisierung (is-a)] Eine Generalisierung ist ein Enitätstyp $G$ der die gemeinsamen Attribute der
	spezialisierten Entitäten $E_1, E_2, \dots, E_n$ umfasst.
	Eine Spezialisierung ist disjunkt, wenn $\bigcap (E_1, E_2, \dots, E_n) = \emptyset$. Ansonsten ist er überdeckend.
\end{description}

\begin{center}
\scalebox{.87}{
\begin{tikzpicture}[node distance=1cm, every edge/.style={link}]
	\node[entity] (entity) {Entität};
	\node[attribute] (key) [above left=of entity] {\key{Schlüssel}} edge (entity);
	\node[attribute] (attribute) [left=of entity] {Attribut} edge (entity);
	\node[multi attribute] (mattribute) [below left=of entity] {Multiattribut} edge (entity);
	\node[relationship] (rel) [right=of entity] {Beziehung} edge (entity);
	\node[attribute] (relAttr) [below=of rel] {Bez.-Attribut} edge (rel);
	\node[isa] (isa) [below=of entity] {ISA} edge (entity);
\end{tikzpicture}
}
\end{center}

\subsection{Beziehungstypen (Kardinalität)}
\begin{description}
	\item [1:1] Jedem Element aus $E_1$ ist höchstens 1 Element aus $E_2$ zugeordnet
	\item [1:M] Jedem Element aus $E_1$ können beliebig viele Elementa aus $E_2$ zugeordnet sein,
	jedem Element aus $E_2$ aber höchstens ein Element aus $E_1$
	\item [M:M] Jedem Element aus $E_1$ können beliebig viele Elemente aus $E_2$ zugeordnet sein
	und umgekehrt. Diese Beziehungen müssen für Datenbanken aufgelöst werden.
\end{description}

Attribute werden aus Gründen der Übersichtlichkeit im ERM oft weggelassen und in ergänzenden Dokumenten festgehalten.
Die Details entstehen in der Regel beim logischen Entwurf. Der Entwurf ist ein iterativer Prozess an dem mehrere
Personen beteiligt sein sollten.  

\section{Logischer Entwurf}
Grundsätzlich wird aus einer Entität eine Relation, ebenso wird aus einer Beziehung eine Relation. Mehrwertige Attribute
werden aufgeteilt (zusätzliche Relationen), M:M Beziehungen werden mit Zwischentabellen aufgelöst (zusätzliche Relationen).

Am Ende steht ein logisches Schema, das in SQL (DDL) formuliert ist.

\subsection{Normalisierung}
Unsachgemässe Entwürfen können zu Mutationsanomalien (unerwünschte Auswirkung von Änderungen) führen. Durch
Normalisierung wird dies verhindert.

Anwendungswissen für die Normalisierung: Repetitionsgruppen (was kommt wie oft vor?), Schlüssel, funktionale Abhängigkeiten,
Codierungen.

\begin{description}
	\item [1. Normalform] Alle Attribute enthalten nur atomare Werte (mehrwertige Attribute werden in Tupel umgewandelt)
	\item [2. Normalform] Eine Relation ist in der zweiten Normalform, wenn die erste Normalform vorliegt 
	und jedes Nichtschlüsselattribut vom Primärschlüssel voll funktional abhängig ist. (Mehrere Tabellen entstehen)
	\item [3. Normalform] Die dritte Normalform ist erreicht, wenn sich das Relationenschema in 2NF befindet, und jedes 
	Nichtschlüsselattribut von keinem Schlüsselattribut transitiv abhängt.
\end{description}

\subsubsection{1NF $\rightarrow$ 2NF}
\begin{enumerate}
	\item Bestimme alle Nichtschlüsselmerkmale, die bereits von einem Teilschlüssel funktional abhängig sind.
	\item Bilde aus den Teilschlüsseln und allen von ihnen funktional abhängigen Nichtschlüsselmerkmalen eigene Tabellen.
	\item Entferne aus der ursprünglichen Tabelle alle nicht voll funktional abhängigen Nichtschlüsselmerkmale.
\end{enumerate}

\subsubsection{2NF $\rightarrow$ 3NF}
\begin{enumerate}
	\item Bestimme alle vom Schlüssel transitiv abhängigen Nichtschlüsselmerkmale.
	\item Bilde aus diesen transitiv abhängigen Nichtschlüsselmerkmalen und den Nichtschlüsselmerkmalen, 
	von denen sie funktional abhängig sind, eigene Tabellen.
	\item Entferne aus der ursprünglichen Tabelle alle transitiv abhängigen Nichtschlüsselmerkmale.
\end{enumerate}

Durch Normalisierung werden Redundanzen vermindert und Anomalien vermieden. Allerdings entstehen so 
viele kleine Relationen, die einen hohen Verarbeitungsaufwand nach sich ziehen (Update-Query-Tradeoff).