\section{Funktionale Abhängigkeit}
Das Attribut bzw. die Attributskombination $Y$ ist funktional abhängig vom
Attribut bzw. von der Attributskombination $X$ derselben Relation $R$, wenn
zu einem bestimmten Wert von $X$ höchstens ein Wert von $Y$ möglich ist ($X \rightarrow Y$)

\subsection{Regeln}
\settowidth{\MyLenA}{Pseudotransitivität~~}
\begin{tabular}{
	@{}p{\the\MyLenA}%
	@{}p{\linewidth-\the\MyLenA}}
	Projektivität & $(\{Y\} \subseteq \{X\}) \Rightarrow (X \rightarrow Y)$\\
	Distributivität & $(X \rightarrow YZ) \Rightarrow (X \rightarrow Y, X \rightarrow Z)$\\
	Additivität & $(X \rightarrow Y, X \rightarrow Z) \Rightarrow (X \rightarrow YZ)$\\
	Transitivität & $(X \rightarrow Y, Y \rightarrow Z) \Rightarrow (X \rightarrow Z)$\\
	Pseudotransitivität & $(X \rightarrow Y, YW \rightarrow Z) \Rightarrow (XW \rightarrow Z)$\\
	Akkumulation & $(X \rightarrow YZ, Z \rightarrow W) \Rightarrow (X \rightarrow YZW)$\\
	Erweiterung & $(X \rightarrow Y, W \rightarrow Z) \Rightarrow (XW \rightarrow YZ)$\\
\end{tabular}
