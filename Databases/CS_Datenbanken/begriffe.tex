\section{Begriffe}
\begin{description}
 \item [Information] (lat. \enquote{informatio}, Form oder Gestalt geben) ist nicht einheitlich
 definiert. Im allgemeinen wird darunter \enquote{übertragenes Wissen} verstanden.
 \item [Daten] (lat. \enquote{datum}, gegebenes) ebenfalls nicht einheitlich definiert. Grundsätzlich
 codierbare Angaben über Dinge oder Sachverhalte, die gespeichert und verarbeitet werden können.
 Die jährlich produzierte Datenmenge liegt bei ca. 10 Zettabytes ($10^{21}$ Bytes)
 \item [strukturierte Daten] haben eine explizite Struktur und sind in der Minderheit (Bsp. Tabellen)
 \item [unstruktierte Daten] haben keine explizite Struktur (ggf. implizite Struktur aufgrund einer
 zugrunde liegenden Grammatik. (Bsp. Text, Bilder, Filme, \dots)
 \item [semistrukturierte Daten] sind nur zum Teil strukturiert (Bsp. XML)
\end{description}

\subsection{Dateisystem vs. DBMS}
\subsubsection{Dateisystem}
Daten werden in Dateien vom Betriebssystem verwaltet. Anwendungen lesen/schreiben die Daten direkt. (Office-Anwendungen,
Buchhaltungssoftware, \dots)


% \begin{tabular}{
% 	@{}p{\linewidth/2}|
% 	@{}p{\linewidth/2}}
% 	\multicolumn{1}{c}{Vorteile} & \multicolumn{1}{c}{Nachteile}\\
% 	\begin{minipage}{\linewidth}
		\begin{itemize}
			\item [$+$] einfach, angepasst, effizient
			\item [$+$] keine Rücksicht auf andere
			\item [$+$] proprietäre Formate möglich
% 		\end{itemize}
% 	\end{minipage}
% 	&
% 	\begin{minipage}{\linewidth}
% 		\begin{itemize}
			\item [$-$] Verwendung für unterschiedliche Zwecke problematisch
			\item [$-$] i.d.R. Strukturänderung $\Rightarrow$ Programmänderung 
			\item [$-$] Synchroner Zugriff aufwändig zu realisieren
			\item [$-$] Abgestufte Zugriffsrechte aufwänding zu realisieren
			\item [$-$] Daten häufig mehrfach gespeichert
			\item [$-$] Datenaustausch, -integration komplex
		\end{itemize}
% 	\end{minipage}	
% \end{tabular}

\subsubsection{Datenbanksysteme}
Verwalten und nutzen (sehr) grosse Datenmengen. Der Zugriff ist deklarativ und mengenorientiert.
Die Daten werden einmalig und zentral definiert. Wichtige Aufgaben (Integritätskontrolle, Redundanzverwaltung,
Zugriffskontrolle und -optimierung, synchroner Zugriff, zentrale Datensicherung und Wiederherstellung) 
können so automatisiert werden. Allerdings sind Aufbau und Betrieb eines Datenbanksystems anspruchsvoll und teuer.


\subsection{Datenunabhängigkeit (DU)}
\begin{description}
	\item [Logische DU] Anwendungsprogramme müssen die logische Gesamtstruktur nicht kennen,
	um spezifische Verarbeitungen vorzunehmen $\Rightarrow$ sie sind von Datenbankschema-Änderungen
	nicht betroffen.
	\item [Physische DU] Anwendungsprogramme müssen die interne Organisation der Daten und Zugriffs- 
	und Speicherungsmöglichkeiten nicht kennen $\Rightarrow$ sie sind von Speicher- und Zugriffstrukturänderungen
	nicht betroffen
\end{description}

\subsection{Datenbank Management Systeme}
\textbf{DB-Typen:} hierarchisch, relational, objekt-relational, objektorientiert, deduktiv, netzwerk, \dots

Ein Datenbanksystem (DBS) besteht aus einem Datenbankverwaltungssystem (Software, DBMS) und einer Datenbank (Daten, DB).
Anwendungsprogramme kommunizieren mit dem DBMS. Anwendungsprogramme und DBS bilden ein Informationssystem (IS).

\subsubsection{Aufgaben DBMS}
Syntaxprüfung, Objekte und Zugriffsrechte prüfen, Metadaten lesen, Zugriffsmodul generieren, Transaktionskontrolle,
Daten lesen, Daten zusammenstellen, Daten ausgeben.

% \begin{tabular}{@{}p{\linewidth/2}|@{}p{\linewidth/2}}
%  \multicolumn{1}{c}{Dateisystem} & \multicolumn{1}{c}{Datenbank} \\\hline
%  + einfach, anpassbar, effizient & + langfristige Aufbewahrung\\
%  + Proprietäre Formate möglich & + grosse Mengen\\
%  + Keine Rücksicht auf ander & + viele Benutzer\\
%  - Struktur = Programmänderung & + einfache Rechteverwaltung\\
%  - Simultaner Zugriff schwierig & + logische Unabhängigkeit\\
%  - abgest. Zugriffsrechte schwierig & + physische Unabhängigkeit\\
%  - Mehrfachverwendung heikel &  + Automatisierung\\
%  - Logische \& physische abhängig & - teuer und aufwändig\\
% \end{tabular}

\subsubsection{ANSI-SPARC-3-Ebenen-Architektur (1975)}
\begin{description}
 \item [Extern] Sicht einzelner Anwendungen oder Benutzergruppen
 \item [Konzeptionell] Logische Gesamtsicht
 \item [Intern] Speicherung, Datenorganisation, Zugriffsstrukturen
\end{description}

\subsubsection{Aufbau und Betrieb}
Auf- und Ausbau bezeichnet das (iterative) erstellen verschiedener Datenbank Schemas, 
häufig im laufenden Betrieb. Betrieb, Nutzung und Verwaltung (Anzeigen, Einfügen, Ändern, Löschen, Backup, Restore)
mit Hilfe der Daten-Manipulationssprache (DML)