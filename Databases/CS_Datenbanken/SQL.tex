\section{SQL (IBM 1974)}
Zwar gibt es einen Standard, doch eingesetzte Systeme sind proprietär. Einerseits aus kommerziellen, andererseits aus
Trägheitsgründen - das SQL-Komitee hingt den Bedürfnissen des Marktes hinterher.

Leider ist SQL für \enquote{reale} Anwendungen zu wenig mächtig. (Keine Benutzerschnittstellen, komplexe Berechnungen nicht möglich,
Zugriff auf Geräte nicht möglich, Vernetzung problematisch) $\Rightarrow$ SQL wird mit Programmiersprachen (Java/JPA) kombiniert.

\subsection{Data Definition Language (DDL)}
Erzeugen und Ändern von Datenbank-Objekten.

Als erstes Datenbanken erzeugen:
\begin{lstlisting}
	CREATE DATABASE IF NOT EXISTS database
\end{lstlisting}

\subsubsection{Datentypen}
\begin{itemize}\itemsep0em
	\item Zahlen (Ganze (\texttt{integer, bigint}), Fix- (\texttt{decimal(p,s), numeric(p,s)}) und Fliesskomma (\texttt{float(p), real(p)}))
	\item Zeichenketten (\texttt{char(n), varchar(n)})
	\item Kalenderdaten (\texttt{date, time, datetime})
	\item Sonstige (herstellerabhängig)
\end{itemize}

\begin{lstlisting}
CREATE TABLE Products(
	PNo INTEGER PRIMARY KEY,
	Descr CHAR(30) NOT NULL UNIQUE,
	Weight FLOAT CHECK(Weight > 0.0),
	...
);
CREATE TABLE Orders(
	OrdNo INTEGER,
	...,
	PNo INTEGER	NOT NULL REFERENCES Products(PNo),
	CNo INTEGER	NOT NULL REFERENCES Customers(CNo),
	Status CHAR(7) DEFAULT 'ordered',
	ValidDate DATETIME NOT NULL,
	CONSTRAINT PK_Orders PRIMARY KEY(OrdNo),
	UNIQUE (ValidDate,PNo,CNo)
);
ALTER TABLE table ADD COLUMN newColumn datatype;
DROP TABLE table;
\end{lstlisting}


\subsection{Data Manipulation Language (DML)}
\subsubsection{Insert}
\begin{lstlisting}
INSERT INTO table (columnA, columnB) VALUES ("value1", "value2");
INSERT INTO table (columnA, columnB) 
	SELECT field1, field2 FROM tableB WHERE ...;
\end{lstlisting}
Würden Integritätsbedingungen verletzt, wird nichts geändert. Fehlende Attribute werden mit dem Default Wert oder \texttt{NULL} gefüllt.

\subsubsection{Update}
\begin{lstlisting}
UPDATE table SET field = "value" WHERE ...;
UPDATE table SET field = field + 1;
\end{lstlisting}
Würden Integritätsbedingungen verletzt, wird nichts geändert.

\subsubsection{Delete}
\begin{lstlisting}
DELETE FROM table WHERE ...;
\end{lstlisting}
Würden Integritätsbedingungen verletzt, wird nichts geändert.

\subsubsection{Select}
\begin{lstlisting}
SELECT * | [ALL | DISTINCT] expression [[AS] column-alias][,...]
FROM table [[AS] alias]
[,|CROSS JOIN table [[AS] alias] |
	NATURAL [INNER|OUTER]{LEFT|RIGHT|FULL}]
	JOIN table [[AS] alias]
	|[INNER|OUTER]{LEFT|RIGHT|FULL}]
	JOIN table [[AS] alias]
	{ON condition | USING (col-list)}
|UNION join-table [[AS] alias] [,...]]
[WHERE col-condition]
[GROUP BY colname [,...]
[HAVING group-condition]]
[UNION|INTERSECT|EXCEPT ...]
[ORDER BY col [ASC|DESC][,...]];

SELECT Name
FROM Customers AS c,
	(SELECT CNo FROM Orders
	WHERE Status <> 'paid' AND
	Validdate < DATE '2003-10-01') AS o
WHERE o.CNo = c.CNo

/* Join mit ON */
SELECT Name FROM Customers c 
JOIN Orders o ON o.CNo = c.CNo
WHERE Status <> 'paid' AND Validdate < DATE '2003-10-01'

/* Join mit USING */
SELECT Name FROM Customers 
JOIN Orders USING(CNo)
WHERE Status <> 'paid' AND Validdate < DATE '2003-10-01'

/* NATURAL JOIN */
SELECT Name FROM Customers 
NATURAL JOIN Orders 
WHERE Status <> 'paid' AND Validdate < DATE '2003-10-01'
\end{lstlisting}

\textbf{Regeln für \texttt{NULL}}
\begin{align*}
	\neg \mbox{\texttt{NULL}}& = \mbox{\texttt{NULL}} & &\\
	\mbox{\texttt{NULL}} \vee \mbox{\texttt{NULL}}& = \mbox{\texttt{NULL}} &
	\mbox{\texttt{NULL}} \wedge \mbox{\texttt{NULL}}& = \mbox{\texttt{NULL}}\\
	\mbox{false} \vee \mbox{\texttt{NULL}} & = \mbox{\texttt{NULL}} & 
	\mbox{false} \wedge \mbox{\texttt{NULL}} & = \mbox{false}\\
	\mbox{true} \vee \mbox{\texttt{NULL}} & = \mbox{true} & 
	\mbox{true} \wedge \mbox{\texttt{NULL}} & = \mbox{\texttt{NULL}}\\	
\end{align*}

\textbf{Selektionsprädikate}
\settowidth{\MyLenA}{\texttt{BETWEEN}~~}
\begin{tabular}{
	@{}p{\the\MyLenA}%
	@{}p{\linewidth-\the\MyLenA}}
	\texttt{IN} & \lstinline$SELECT * FROM table WHERE field IN ('value1','value2');$\\
	\texttt{LIKE} & \lstinline$SELECT * FROM table WHERE field LIKE 'value%';$\\
	\texttt{BETWEEN} & \lstinline$SELECT * FROM table WHERE field BETWEEN lower AND upper;$\\
	\texttt{IS NULL} & \lstinline$SELECT * FROM table WHERE field IS NOT NULL;$\\
\end{tabular}

Aggregatfunktionen: \lstinline$[COUNT|SUM|AVG|MIN|MAX](field)$
\texttt{COUNT} eleminiert Duplikate nicht, zählt \texttt{NULL}-Wert mit. Die anderen Aggregatfunktionen ignorieren
\texttt{NULL}-Werte.

Gruppen sind eine Multimenge von Tupeln mit denselben Werten für das Gruppierungskriterium. \texttt{HAVING} entspricht
der \texttt{WHERE}-Klausel.

\subsection{Data Controll Language}
Rechte Vergabe (\texttt{GRANT, REVOKE}), Schutz vor Ausfällen. Job des Datenbankadministrators.

\subsection{Transaktionen}
Eine Transaktion ist eine Folge von Schreib- und Lese-Zugriffen, die eine geschlossene Arbeitseinheit bilden.

\subsubsection{Multiuser}
Ein DBMS simuliert gleichzeitigen Zugriff, weil nur ein Zugriff pro CPU pro Zeit möglich ist. Es behandelt Fehler
automatisch und startet nach schwerwiegenden Fehlern neu.

Nutzer erhalten alternierend kurze Prozessorzeit. Dadurch laufen während des Prozesses eines Benutzers auch die 
Prozesse anderer Nutzer ab. So können inkonsistente Datenbestände entstehen, was das DBMS mit Transaktionen verhindert.
Mögliche Probleme:
\begin{description}
	\item [Lost-Update] Überschreiben bereits getätigter Updates (ohne neuerliches Read) $\Rightarrow$ Für Dauer
	der Transaktion sperren.
	\item [Dirty-Read] Lesen von Veränderungen durch später abgebrochene Transaktionen $\Rightarrow$ Nur bestätigte
	Transaktionen berücksichtigen.
	\item [Phantom-Read] Lesen von zwischenzeitlich von anderen Transaktionen durchgeführten Veränderungen. $\Rightarrow$
	Nur Datenbankzustand zu Beginn einer Transaktion verwenden.
\end{description}
Gleichzeitige Zugriffe werden durch Ablaufpläne (Schedules) modelliert.

\subsubsection{ACID-Prinzip}
A - Atomarität - Zusammenhängende Aktionen ganz oder gar nicht ausführen.\\
C - Consistency - Alle Aktionen hinterlassen die DB in einem konsistenten Zustand.\\
I - Isolation - Transaktionen laufen unabhängig von einander ab.\\
D - Durability - Resultate bleiben nach Transaktion persistent.\\

Commit - Transaktion durchführen, Rollback = Abbruch:
\begin{lstlisting}[language=SQL]
BEGIN TRANSACTION
COMMIT TRANSACTION
ROLLBACK TRANSACTION
\end{lstlisting}

\subsection{Transaktionsverarbeitung}
Datenobjekte werden gesperrt. Problem: Deadlock - gegenseitiges Warten auf Freigabe von Sperren. Solche Zyklen werden
vom DBMS erkannt. 

\subsection{Integritätssicherung}
Sicherstellen, dass nur \enquote{richtige} Daten in die DB gelangen. Die Menge der konsistenten Zustände wird eingeschränkt.
Integritätsicherung ist eine zentrale Aufgabe eines DBMS.

\subsubsection{Möglichkeiten}
\begin{itemize}
	\item Attribut-Bedingungen (Datentyp, \texttt{NOT NULL, CHECK})
	\item Wertebereichsbedingungen (\texttt{CREATE DOMAIN \dots AS \dots CHECK ()})
	\item Tupel-Bedingungen (\texttt{CHECK})
	\item Relationen-Bedingungen (\texttt{PRIMARY KEY})
	\item Referenzielle Bedingungen (\texttt{FOREIGN KEY \dots REFERENCES \dots})
\end{itemize}

Einschränkungen können auch verzögert aktiviert werden:
\begin{lstlisting}[language=SQL]
CREATE TABLE ... CONSTRAINT field ... INITIALLY {DEFERRED|IMMEDIATE}
\end{lstlisting}