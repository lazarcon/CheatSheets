\section{Datenbank-Entwicklung}
\settowidth{\MyLenA}{Anforderungsanalyse~~}
\begin{tabular}{
	@{}p{\the\MyLenA}%
	@{}p{\linewidth-\the\MyLenA}}
	Anforderungsanalyse, Spezifikation & Datenanforderungen durch Interviews, Fragebögen ermitteln\\
	Konzeptioneller Entwurf & Konzeptionelles Model beschreiben (DBMS unabhängig) $\Rightarrow$ Konzeptionelles Schema\\
	Logischer Entwurf & Logisches Schema aus dem Datenmodell erstellen, optimieren, externe Schemas definieren (DBMS abhängig) $\Rightarrow$ Logisches Schema\\ 
	Physischer Entwurf & Internes Schema definieren (DBMS abhängig) $\Rightarrow$ Physisches Schema\\
	Deklaration Schemas & In der DDL des DBS\\
\end{tabular}

Kriterien: Vollständigkeit, Korrektheit, minimale Redundanzen, Lesbarkeit, Erweiterbarkeit, Normalisierung.

Refactoring eines Modelles ist sehr aufwändig und wird deswegen kaum gemacht. Daher am besten mit einem kleinen
Kernteam (Business Spezialist (Anforderungen/Konzept), Datenmodellierer(Konzept/Logik), Datenbank Designer(Logik/Physisches Model)) arbeiten, das Entwürfe macht, die von einem grösseren Team
(Datenbankspezialisten, Projektmanager, Auftraggeber) evaluiert werden.

\begin{description}
	\item [Konzeptionelles Modell] weitgehend technologie-unabhängige Spezifikation der Daten, die in der Datenbank
		gespeichert werden sollen
	\item [Logisches Modell] Übersetzung des konzeptionellen Schemas in Strukturen, 
		die mit einem konkreten DBMS implementiert werden können
	\item [Physisches Modell] alle Anpassungen, die nötig sind um eine befriedigende Leistung im Betrieb zu erreichen
		(Datenverteilung,  Indexierung,  ...)
\end{description}

